\section{Numerical Representations}\label{sec:background-numerical}
Before we dive into descriptions, let us revisit the situation of \bN{}, \AD{List} and \AD{Vec}. If it was not coincidence that gave us ornaments from \bN{} to \AD{List} and from \AD{List} to \AD{Vec}, then we can expect to find ornaments beforehand, instead of as a consequence of the definitions of \AD{List} and \AD{Vec}.

Rather than finding the properties of \AD{Vec} that were already there, let us view \AD{Vec} as a consequence of the definition of \bN{} and \AF{lookup}. From \bN{}, we obtain a trivial type of arrays by reading \AF{lookup} as a prescript:
\ExecuteMetaData[Tex/Descriptions/Numrep]{Lookup2}
For this definition, the lookup function is simply the identity function on \AD{Lookup}. As this is the prototypical array corresponding to natural numbers, any other array type we define should satisfy all the same properties and laws \AD{Lookup} does, and should in fact be equivalent.


We remark that without further assumptions, we cannot use the equality type from \autoref{sec:background-proving} for this notion of sameness of types: repeating the definition of a type gives two distinct types with no equality between them. Instead, we import another notion of sameness, known as isomorphisms:
\ExecuteMetaData[Tex/Descriptions/Numrep]{Iso}
An \AD{Iso} from \AV{A} to \AV{B} is a map from \AV{A} to \AV{B} with a (two-sided) inverse\footnote{Which is equivalent to the other notion of equivalence: there is a map $f : A \to B$, and for each \AV{b} in \AV{B} there is exactly one \AV{a} in \AV{A} for which $f(a) = b$.}. In terms of elements, this that elements of \AV{A} and \AV{B} are in one-to-one correspondence.

Now, rather than defining \AD{Vec} ``out of the blue'' and proving that it is correct or isomorphic to \AD{Lookup}, we can also turn the \AD{Iso} on its head: Starting from the equation that \AD{Vec} is equivalent to \AD{Lookup}, we derive a definition of \AD{Vec} as if solving that equation \cite{calcdata}. As a warm-up, we can also derive \AD{Fin} from the fact that \AD{Fin}\ \AV{n} should contain \AV{n} elements, and thus be isomorphic to \AV{Σ[ m ∈ ℕ ] m < n}.

To express such a definition by isomorphism, we define:
\ExecuteMetaData[Tex/Descriptions/Numrep]{Def}
using
\ExecuteMetaData[Tex/Descriptions/Numrep]{isigma}
The type \AD{Def}\ \AV{A} is deceptively simple, after all, there is (up to isomorphism) only one unique term in it! However, when using \AD{Def}initions, the implicit \AD{Σ'} extracts the right-hand side of a proof of an isomorphism, allowing us to reinterpret a proof as a definition.

To keep the resulting \AD{Iso}s readable, we construct them as chains of smaller \AD{Iso}s using a variant of ``equational reasoning'' \cite{agdastdlib, plfa}, which lets us compose \AD{Iso}s while displaying the intermediate steps. In the calculation of \AD{Fin}, we will use the following lemmas\footnote{Note! \AF{ua}.}
\ExecuteMetaData[Tex/Descriptions/Numrep]{Fin-lemmas}
In the terminology of \autoref{sec:background-proving}, \AF{⊥-strict} states that ``if A is false, then A \emph{is} false'', if we allow reading isomorphisms as ``\emph{is}'', while \AF{<-split} states that the set of numbers below $n+1$ is 1 greater than the set of numbers below $n$.

Using these, we can calculate
\ExecuteMetaData[Tex/Descriptions/Numrep]{Fin-def}
This gives a different (but equivalent) definition of \AD{Fin} compared to \AF{FinD}: the description \AF{FinD} describes \AD{Fin} as an inductive family, whereas \AF{Fin-def} gives the same definition as a type-computing function \cite{progorn}.

This \AD{Def} then extracts to a definition of \AD{Fin}
\ExecuteMetaData[Tex/Descriptions/Numrep]{Fin}
To derive \AD{Vec}, we will use the isomorphisms
\ExecuteMetaData[Tex/Descriptions/Numrep]{Vec-lemmas}
which one can compare to the familiar exponential laws. These compose to calculate
\ExecuteMetaData[Tex/Descriptions/Numrep]{Vec-def}
which yields us a definition of vectors
\ExecuteMetaData[Tex/Descriptions/Numrep]{Vec}
and the \AD{Iso} to \AF{Lookup} in one go.

This explains how we can compute a type of lists or arrays (a numerical representation, here, \AD{Vec}) from a number system (\bN{}).


\section{Augmented Descriptions}
To describe more general numerical representations, we must first describe more general number systems. We do so very loosely, however, allowing for tree-like number systems so long as the values of nodes are linear combinations of the values of subnodes. This generalizes positional number systems such as \bN{} and binary numbers, and allows for more exotic number systems, but for example does not include \bN{}\ \AD{×}\ \bN{} with the Cantor pairing function as a number system.

By requiring that nodes are interpreted as linear combinations of subnodes, we can implement a universe of number systems as a special case of earlier universes by baking the relevant multipliers into the type-formers. Descriptions in the universe of number systems can then both be interpreted to datatypes, and can evaluate their values to \bN{} using the multipliers in their structure.

For there to be an ornament between a number system and its numerical representation, the descriptions of both need to live in the same universe. Hence, we will generalize the type of descriptions over information such as multipliers later, rather than defining a new universe of number systems here. The information needed to describe a number system can be separated between the type-formers. Namely, a leaf \AIC{𝟙} requires a constant in \bN{}, a recursive field \AIC{ρ} requires a multiplier in \bN{}, while a field \AIC{σ} will need a function to convert values to \bN{}.

To facilitate marking type-formers with specific bits of information, we define 
\ExecuteMetaData[Ornament/Desc]{Info}
to record the type of information corresponding to each type-former. We can summarize the information which makes a description into a number system as the following \AD{Info}:
\ExecuteMetaData[Ornament/Numerical]{Number} 
which will then ensure that each \AIC{𝟙} and \AIC{ρ} both are assigned a number \bN{}, and each \AIC{σ} is assigned a function that converts values of the type of its field to \bN{}\footnote{More about \AIC{δ} later.}.

On the other hand, we can also declare that a description needs no further information by:
\ExecuteMetaData[Ornament/Desc]{Plain}
By making the fields of information implicit in the type of descriptions, we can ensure that descriptions from \AD{U-ix} can be imported into the generalized universe without change.

%In a more interesting example, we can define
%\ExecuteMetaData[Ornament/Desc]{Countable}
%so that to define a \AD{DescI}\ \AF{Countable}, one has to provide a proof that each field is countable, which could be used to prove that each type represented by a \AD{DescI}\ \AF{Countable} is in turn also countable.

In the descriptions, the \AIC{δ} type-former, which we will discuss in closer detail in the next section, represents the inclusion of one description in a larger description. When we include another description, this description will also be equipped with extra information, which we allow to be different from the kind of information in the description it is included in. When this happens, we ask that the information on both sides is related by a transformation:
\ExecuteMetaData[Ornament/Desc]{InfoF}
which makes it possible to downcast (or upcast) between different types of information. This, for example, allows the inclusion of a number system \AD{DescI}\ \AF{Number} into an ordinary datatype \AD{Desc} without rewriting the former as a \AD{Desc} first.


\section{The Universe}\label{ssec:desc}
We also need to take care that the numerical representations we will construct indeed fit in our universe. The final universe \AD{U-ix} of \autoref{ssec:background-ix}, while already quite general, still excludes many interesting datastructures. In particular, the encoding of parameters forces recursive type occurrences to have the same applied parameters, ruling out nested datatypes such as (binary) random-access lists \cite{calcdata,purelyfunctional}:
\changed{Redo (or check) the Agda snippets below here.}
\ExecuteMetaData[Tex/DescOrn]{random-access}
and finger trees \cite{fingertrees}:
\ExecuteMetaData[Tex/DescOrn]{finger-tree}
Even if we could represent nested types in \AD{U-ix} we would find it still struggles with finger trees: Because adding non-recursive fields modifies the variable telescope, it becomes hard to reuse parts of a description in different places. Apart from that, the number of constructors needed to describe finger trees and similar types also grows quickly when adding fields like \AD{Digit}.

We will resolve these issues as follows. We can describe nested types by allowing parameters to be transformed before they are passed to recursive fields \cite{?}. By transforming variables before they are passed to subsequent fields, it becomes possible to share constructor descriptions\footnote{Downsides upsides.}. Finally, by adding a variant of \AIC{σ} specialized to descriptions, we can describe composite datatypes more succinctly.

Combining these changes, we define the following universe:
\ExecuteMetaData[Ornament/Desc]{Desc}
where the constructors are defined as:
\ExecuteMetaData[Ornament/Desc]{Con}
From this definition, we can recover the ordinary descriptions as
\ExecuteMetaData[Ornament/Desc]{Plain-synonyms}
%and first explain the differences in type-formers compared to \AD{U-ix}, and how these 
Let us explain this universe by discussing some of the old and new datatypes we can describe using it. Some of these datatypes do not make use of the full generality of this universe, so we define some shorthands to emulate the simpler descriptions. Using 
\ExecuteMetaData[Ornament/Desc]{sigma-pm}
(and the analogues for \AIC{δ}) we emulate unbound and bound fields respectively, and with 
\ExecuteMetaData[Ornament/Desc]{rho-zero}
we emulate an ordinary (as opposed to nested) recursive field. We can then describe \bN{} 
\ExecuteMetaData[Ornament/Desc]{NatD}
and \AD{List} as before
\ExecuteMetaData[Ornament/Desc]{ListD}
by replacing \AIC{σ} with \AF{σ-} and \AIC{ρ} with \AIC{ρ0}.

On the other hand, we bind the length of a vector as a field when defining vectors, so there we use \AF{σ+} instead:
\ExecuteMetaData[Ornament/Desc]{VecD}
With the nested recursive field \AIC{ρ}, we can define the type of binary random-access arrays. Recall that for random-access arrays, we have that an array with parameter \AV{A} contains zero, one, or two values of \AV{A}, but the recursive field must contain an array of twice the weight. Hence, the parameter passed to the recursive field is \AV{A times A}, for which we define
\ExecuteMetaData[Ornament/Desc]{Pair}
Passing \AF{Pair} to \AIC{rho} we can define random access lists:  
\ExecuteMetaData[Ornament/Desc]{RandomD}
To represent finger trees, we first represent the type of digits \AD{Digit}:
\ExecuteMetaData[Ornament/Desc]{DigitD}
We can then define finger trees as a composite type from \AD{Digit}:
\ExecuteMetaData[Ornament/Desc]{FingerD}
Here, the fact that the first \AIC{δ-} drops its field from the telescope makes it possible to reuse of \AD{Digit} in the second \AIC{δ-}.

These descriptions can be instantiated as before by taking the fixpoint
\ExecuteMetaData[Ornament/Desc]{fpoint}
of their interpretations as functors\todo{Also note that this (obviously?) completely ignores the Info of a description.}
\ExecuteMetaData[Ornament/Desc]{interpretation}
In this universe, we also need to insert the transformations of parameters \AV{f} in \AIC{ρ} and the transformations of variables \AV{h} in \AIC{σ} and \AIC{δ}.

Like for \AD{U-ix}, we can give the generic \AF{fold} for \AD{DescI}
\ExecuteMetaData[Ornament/Desc]{fold}

