\subsection{Ornamental descriptions}
The ornamental descriptions for \AD{DescI} take the same shape as those in \autoref{sec:background-ornamental-descriptions}, generalized to handle nested types, variable transformations, and composite types. These ornamental descriptions are defined such that a \AD{OrnDesc}\ \AV{Me′ Δ re-par J re-index D} represents a patch from a base description \AV{D} to a description with metadata \AV{Me′}, parameters \AV{Δ} and indices \AV{J}.

Note that metadata, as a non-structural property, has no direct influence on ornaments. So, we simply generalize over the metadata on \AV{D}, querying the metadata for the new description without imposing constraints.

As always, we start off by defining ornamental descriptions as lists of constructor ornaments :
\ExecuteMetaData[Ornament/OrnDesc]{OrnDesc}
Most of the modifications in \AD{DescI} are reflected in the constructor ornaments, and as a consequence this is also where we pay the price for the flexibility we built into \AD{ConI}. For example, because \AD{ConI} allows us to transform variables, \AD{ConOrnDesc} has to relate the transformations on both sides in order for \AF{ornForget} to exist. We (have to) dedicate a lot of lines to such commutativity squares of variables, but these squares involving \AD{Vxf} can generally be ignored; this is witnessed by the \AF{Oσ+} and \AF{Oσ-} variants of the \AIC{σ} ornament, automatically filling those squares in the usual cases of binding or ignoring fields.

The structure-preserving ornaments are defined as usual
\ExecuteMetaData[Ornament/OrnDesc]{ConOrn-preserve}
where \AIC{ρ} has a new field relating the old and new nesting transforms \AV{g} and \AV{d}. Likewise, \AIC{σ} now has a field relating the old and new variable transforms, which for example prevents us from unbinding a field in the new description which was used in the old description. The ornament \AIC{δ} now represents the direct copying of a \AIC{δ} in descriptions (up to \AV{re-par} and \AV{re-var}).

Where only \AIC{Δσ} could add fields before, we can now also add fields described by \AIC{δ} using \AIC{Δδ}:
\ExecuteMetaData[Ornament/OrnDesc]{ConOrn-extend}
Again, \AIC{Δσ} requires the relation of old and new variables.

Now, if we have a description \AV{D'} with a composite field \AIC{δ}\ \AV{R}\ \AV{d}\ \AV{j}\ \AV{R}\ \AV{D} referencing \AV{R}, then we expect that a patch on \AV{R} also induces a patch on \AV{D'}. We generalize this by defining a kind of sequential composition of ornaments\footnote{As opposed to Ko's parallel composition \cite{kophd}, which composes two ornaments on the same description \AV{D}, producing something that incorporates changes from both.}, taking two ornamental descriptions, one on \AV{R} and one on \AV{D}, and producing an ornamental description on \AV{D'}:
\ExecuteMetaData[Ornament/OrnDesc]{ConOrn-compose}
If we try to forget \AIC{∙δ}, the parameters to \AV{R} can be computed in two ways. Namely, we can first convert back to the context of \AV{CD} according to \AV{DE} and compute the parameter for \AV{R} there with the original \AV{fΘ}, or we can first compute the parameter in the new context using the new \AV{fΛ} and then convert this back to the parameter for \AV{R} according to \AV{RR′}. To avoid any ambiguity that arises from this, we require that both ways around this square are equal:
% https://q.uiver.app/#q=WzAsNCxbMCwwLCJXIFxcJiBcXERlbHRhIl0sWzAsMSwiVlxcJiBcXEdhbW1hIl0sWzEsMCwiXFxMYW1iZGEiXSxbMSwxLCJcXFRoZXRhIl0sWzAsMSwiXFxtYXRocm17cmUtdmFyfVxcdGltZXNcXG1hdGhybXtyZS1pbmRleH0iLDJdLFswLDIsImZcXExhbWJkYSJdLFsxLDMsImZcXFRoZXRhIiwyXSxbMiwzLCJjJyJdXQ==
\[\begin{tikzcd}
	{W \& \Delta} & \Lambda \\
	{V\& \Gamma} & \Theta
	\arrow["{\mathrm{re-var}\times\mathrm{re-index}}"', from=1-1, to=2-1]
	\arrow["f\Lambda", from=1-1, to=1-2]
	\arrow["f\Theta"', from=2-1, to=2-2]
	\arrow["{c'}", from=1-2, to=2-2]
\end{tikzcd}\]
Using these and the other new commutativity squares, we can again define \AF{ornForget} from an ornamental algebra analogous to the one for \AD{U-ix}:
\ExecuteMetaData[Ornament/OrnDesc]{ornForget}
The precise meaning of ornamental descriptions as descriptions is given by the conversion
\ExecuteMetaData[Ornament/OrnDesc]{toDesc}
which makes use of the implicit metadata fields in the constructor ornaments to reconstruct the metadata on the target description.

Like \AD{DescI}, the ornaments support variable transformations and nesting, of which we rarely utilize the full potential. In the common use-cases the commutativity squares the ornaments require are trivial, such as copying or adding (non-)dependent fields, and copying a uniformly recursive field. This means that we will mostly rely on the following shorthands to hide those trivial proofs:

\begin{center}
\begin{tabular}{ll|l}
    \hline
    \AF{Oσ+}\ \AV{S CO}   &= \AIC{σ}\ \AV{S}\ \AF{id}\ \AV{\_ (λ \_ →}\ \AIC{refl}\AV{) CO}               & copy dependent field \\
    \AF{Oσ-}\ \AV{S CO}   &= \AIC{σ}\ \AV{S}\ \ARF{fst}\ \AV{re-var (λ \_ →}\ \AIC{refl}\AV{) CO}          & copy non-dependent "  \\
    \AF{OΔσ+}\ \AV{S CO}  &= \AIC{Δσ}\ \AV{S}\ \AF{id}\ \AV{(re-var}\ \AF{∘}\ \ARF{fst}\AV{) (λ \_ →}\ \AIC{refl}\AV{) CO}  & insert dependent " \\
    \AF{OΔσ-}\ \AV{S CO}  &= \AIC{Δσ}\ \AV{S}\ \ARF{fst}\ \AV{re-var (λ \_ →}\ \AIC{refl}\AV{) CO}         & insert non-dependent " \\
    \AF{Oρ0}\ \AV{j q CO} &= \AIC{ρ}\ \AF{id}\ \AV{j (λ \_ →}\ \AIC{refl}\AV{) q CO}                & uniformly recursive " \\
\end{tabular}
\end{center}

With \AD{OrnDesc} we can reproduce the examples of the ornamental descriptions for \AD{U-ix}, such as \AD{Vec} from \AD{List}:
\ExecuteMetaData[Ornament/OrnDesc]{VecOD}
Rather than defining \AD{Random} on its own, we can use the new flexibility in \AIC{ρ} and describe random access lists as an ornament from binary numbers:
\ExecuteMetaData[Ornament/OrnDesc]{RandomOD}
Likewise, we can give an ornament turning phalanges into digits
\ExecuteMetaData[Ornament/OrnDesc]{DigitOD}
and assemble these into finger trees with \AIC{δ•}
\ExecuteMetaData[Ornament/OrnDesc]{FingerOD}
