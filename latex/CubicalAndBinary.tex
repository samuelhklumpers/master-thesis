\documentclass[Main.tex]{subfiles}


\begin{document}
Let us quickly review the small set of features in Cubical Agda that we will be using extensively throughout this article.\footnote{\cite{cuagda} gives a comprehensive introduction to cubical agda.} We note that there are some downsides to cubical, such as that
\ExecuteMetaData[Tex/CubicalAndBinary]{cubical}\todo{Not sure if it would be helpful to have a more extensive introduction covering all features used.}
also implies the negation of axiom K, which in turn complicates both some termination checking and some universe levels.\todo{To be precise, the mapFold in \cite{progorn} gets painted red.} And, more obviously, we get saddled with the proof obligation that our types are sets should we use certain constructions.

Of course, this downside is more than offset by the benefits of changing our primitive notion of equality, which among other things, lets us access univalence, which drastically cut down the investment required to both show more complex structures to be equivalent (at least when compared to non-cubical). Here, equality arises not (directly) from the indexed inductive definition we are used to, but rather from the presence of the interval type \AgdaPrimitiveType{I}. This type represents a set of two points \AgdaInductiveConstructor{i0} and \AgdaInductiveConstructor{i1}, which are considered ``identified'' in the sense that they are connected by a path. To define a function out of this type, we also have to define the function on all the intermediate points, which is why we call such a function a ``path''. Terms of other types are then considered identified when there is a path between them.

As an added benefit, this different perspective gives intuitive interpretations to some proofs of equality, like
\ExecuteMetaData[Tex/CubicalAndBinary]{sym}
where \AgdaFunction{∼\_} is the interval reversal, swapping \AgdaInductiveConstructor{i0} and \AgdaInductiveConstructor{i1}, so that \AgdaFunction{sym} simply reverses the given path.

Furthermore, because we can now interpret paths in records and function differently, we get a host of ``extensionality'' for free. For example, a path in $A \to B$ is indeed a function which takes each $i$ in \AgdaPrimitiveType{I} to a function $A \to B$. Using this, function extensionality becomes tautological 
\ExecuteMetaData[Tex/CubicalAndBinary]{funExt}

Finally, the \AgdaPrimitiveType{Glue} type tells us that equivalent types fit together in a new type, in a way that guarantees univalence
\ExecuteMetaData[Tex/CubicalAndBinary]{ua}
For our purposes, we can interpret univalence as ``equivalent types are identified'', and, we can treat equivalences as the ``HoTT-compatible'' generalization of bijections. In particular, type isomorphisms like $1 \to A \simeq A$ actually become paths $1 \to A \equiv A$, such that we can transport proofs along them. We will demonstrate this by a slightly more practical example.


\subsection{Binary numbers}\label{ssec:binary}
Let us motivate the cubical method by showing the equivalence of the ``Peano'' naturals and the ``Leibniz'' naturals. Recall that the Peano naturals are defined as 
\ExecuteMetaData[Tex/CubicalAndBinary]{Peano}
This definition enjoys a simple induction principle and has many proofs of its properties in standard libraries. However, it is too slow to be of practical use: most arithmetic operations defined on \bN{} have time complexity in the order of the value of the result.

Of course, the alternative are the more performant binary numbers: the time complexities for binary numbers are usually logarithmic in the resultant values, but these are typically less well-covered in terms of proofs. This does not have to be a problem, because the \bN{} naturals and the binary numbers should be equivalent after all!

Let us make this formal. We define the Leibniz naturals as follows:
\todo{is this too much code and too little explanation at once?}
\ExecuteMetaData[Leibniz/Base.tex]{Leibniz}
Here, the \AgdaInductiveConstructor{0b} constructor encodes 0, while the \AgdaInductiveConstructor{\_1b} and \AgdaInductiveConstructor{\_2b} constructors respectively add a 1 and a 2 bit, under the usual interpretation of binary numbers:
\ExecuteMetaData[Leibniz/Base.tex]{toN}
Let us construct the equivalence from \bN{} to \bL{}. First, we can also interpret a number in \bN{} as a binary number by repeating the successor operation on binary numbers:
\ExecuteMetaData[Leibniz/Base.tex]{bsuc}
\ExecuteMetaData[Leibniz/Base.tex]{fromN}
To show that the operations are inverses, we observe that the interpretation respects successors
\ExecuteMetaData[Leibniz/Properties.tex]{toN-suc}
and that the inverse respects even and odd numbers
\ExecuteMetaData[Leibniz/Properties.tex]{fromN-1}
The wanted statement follows
\ExecuteMetaData[Leibniz/Properties.tex]{N-iso-L}
but since we now have a bijection, we also get an equivalence
\ExecuteMetaData[Leibniz/Properties.tex]{N-equiv-L}
Finally, by univalence, we can identify \bN{} and \bL{} naturals
\ExecuteMetaData[Leibniz/Properties.tex]{N-is-L}

Using the path \AgdaFunction{ℕ≡L} we can already prove some otherwise difficult properties, e.g.,
\ExecuteMetaData[Leibniz/Properties.tex]{isSetL}
Let us define an operation on \bL{} and demonstrate how we can also transport proofs about operations from \bN{} to \bL{}. 

\subsection{Use as definition: functions from specifications}\label{ssec:useas}
As an example, we will define addition of binary numbers. We could take
\ExecuteMetaData[Tex/CubicalAndBinary]{badplus}
But this would be rather inefficient, incurring an $O(n + m)$ overhead when adding $n + m$, so we could better define addition directly. We would prefer to give a definition which makes use of the binary nature of \bL{}, while agreeing with the addition on \bN{}.

Such a definition can be derived from the specification ``agrees with \AgdaFunction{\_+\_}'', so we implement the following syntax for giving definitions by equational reasoning, inspired by the ``use-as-definition'' notation from \cite{calcdata}:
\ExecuteMetaData[Prelude/UseAs.tex]{Def}
which infers the definition from the right endpoint of a path using an implicit pair type
\ExecuteMetaData[Prelude/UseAs.tex]{isigma}
\investigate{As of now, I am unsure if this reduces the effort of implementing a coherent function, or whether it is more typically possible to give a smarter or shorter proof by just giving a definition and proving an easier property of it\footnote{I will put the alternative in the appendix for now}}

With this we can define addition on \bL{} and show it agrees with addition on \bN{} in one motion
\ExecuteMetaData[Leibniz/Properties.tex]{plus-def}
\ExecuteMetaData[Leibniz/Properties.tex]{plus-good}

\subsection{Structure Identity Principle}
We see that as a consequence (modulo some \AgdaPrimitiveType{PathP} lemmas), we get a path from (\bN{}, \AgdaFunction{N.+}) to (\bL{}, \AgdaFunction{plus}). More generally, we can view a type $X$ combined with a function $f: X \to X \to X$ as a kind of structure, which in fact coincides with a magma. We can see that paths between magmas correspond to paths between the underlying types $X$ and paths over this between their operations $f$. This observation is further generalized by the Structure Identity Principle (SIP), formalized in \cite{iri}. Given a structure, which in our case is just a binary operation\todo{Replace with BinOp}
\ExecuteMetaData[Extra/Algebra.tex]{MagmaStr}
this principle produces an appropriate definition ``structured equivalence'' $\iota$. The $\iota$ is such that if structures $X, Y$ are $\iota$-equivalent, then they are identified. In this case, the $\iota$ asks us to provide \AgdaFunction{plus-coherent}, so we have just shown that the \AgdaFunction{plus} magma on \bL{}
\ExecuteMetaData[Leibniz/Properties.tex]{magmaL}
and the \AgdaFunction{\_+\_} magma on \bN{} and are identical
\ExecuteMetaData[Leibniz/Properties.tex]{magma-equal}
As a consequence, properties of \AgdaFunction{\_+\_} directly yield corresponding properties of \AgdaFunction{plus}. For example,
\ExecuteMetaData[Leibniz/Properties.tex]{assoc-transport}
\end{document}
