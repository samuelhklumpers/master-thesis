In conclusion, we formulated a universe encoding \AD{DescI} with which we can describe number systems in \AD{DescI}\ \AF{Number}. We adapted the language of ornamental descriptions \AD{OrnDesc} to \AD{DescI}, such that the numerical representations of the number systems could be seen as ornaments on top of the number systems. These ornaments and descriptions let us implement the generic programs \AF{TreeOD} and \AF{TrieOD} which, from a number system compute the associated (un)indexed numerical representation, of which we informally outlined proofs of correctness. 

While it is possible to write down a direct proof of correctness for \AF{TrieOD} by comparing it to \AF{Lookup} via \AF{value}, and from this extract a proof of correctness for \AF{TreeOD}, one might expect there to be a more useful and less laborious angle of attack. 

Namely, we expect that if we define \AF{PathOD} as a generic ornament from a \AD{DescI}\ \AF{Number} to the corresponding finite type (such that \AF{PathOD}\ \AV{ND}\ \AV{n} is equivalent to \AD{Fin}\ (\AF{value}\ \AV{n})), then we can show that \AF{TrieOD}\ \AV{ND}\ \AV{n} has a \AF{tabulate}/\AF{lookup} pair for \AF{PathOD}\ \AV{ND}\ \AV{n}, from which it follows that \AF{TrieOD}\ \AV{ND}\ \AV{n}\ \AV{A} is equivalent to \AF{PathOD}\ \AV{ND}\ \AV{n}\ \AV{→}\ \AV{A}, and in consequence \AF{TrieOD}\ \AV{ND} corresponds to \AD{Vec}.

Due to the \AF{remember}-\AF{forget} isomorphism \cite{algorn}, we have that \AF{TreeOD}\ \AV{ND} is equivalent to \AV{Σ}\ (\AD{μ} \AV{ND})\ (\AF{TrieOD}\ \AV{ND}), and in turn we also find that \AF{TreeOD}\ \AV{ND} is a normal functor (also referred to as Traversable). This yields traversability of \AF{TreeOD}\ \AV{ND}, with as corollaries \AF{toList}\footnote{Note that the foldable structure we get from the generic \AF{fold} is significantly harder to work with for this purpose.} and properties such as that \AF{toList} is a lifting of \AF{value} (again in the sense of \cite{orntrans}).

However, it turns out that \AF{PathOD} is not so easy to define, as we can see by the following.

\subsection{Σ-descriptions are more natural for expressing finite types}\label{sec:closed-universe}
Due to our representation of types as sums of products, representing the finite types of larger number systems quickly becomes much more complex. Consider the binary numbers from before:
\ExecuteMetaData[Tex/Discussion/Sigma-Desc]{Leibniz}
The finite type associated to \AD{Leibniz} has more constructors than \AD{Leibniz}:
\ExecuteMetaData[Tex/Discussion/Sigma-Desc]{FinB}
In general, given a description of a number system \AV{N}, the number of constructors of the finite type \AD{FinN} of \AV{N} depends directly on the interpretation of \AV{N}, preventing the construction \AD{FinN} by simple recursion on \AD{DescI} (that is, without passing around lists of constructors instead). Furthermore, since our definition of ornaments insists ornaments preserve the number of constructors, there cannot be an ornament from an arbitrary number system to its finite type. 

The apparent asymmetry between number systems and finite types stems from the definition of \AIC{σ} in \AD{DescI}. In \AD{DescI} and similar sums-of-products universes \cite{practgen,sijsling}, the remainder of a constructor \AV{C} after a \AIC{σ}\ \AV{S} simply has its context extended by \AV{S}. In contrast, a universe with \emph{Σ-descriptions} \cite{effectfully,progorn,algorn} (in the terminology of \cite{sijsling}) encodes a dependent field \AV{(s : S)} by asking for a function \AV{C} assigning values \AV{s} to descriptions.

Compared to Σ-descriptions, a sums-of-products universe keeps out some more exotic descriptions which do not have an obvious associated Agda datatype\footnote{Consider the constructor \AIC{σ}\ \bN{}\ \AV{λ}\ \AV{n}\ \AV{→}\ \AF{power}\ \AIC{ρ}\ \AV{n}\ \AIC{𝟙} which takes a number \AV{n} and asks for \AV{n} recursive fields (where \AF{power}\ \AV{f}\ \AV{n}\ \AV{x} applies \AV{f} \AV{n} times to \AV{x}). This description, resembling a rose tree, does not (trivially) lie in a sums-of-products universe.}.

However, this also prevents us from writing down the simpler form of finite types. If we instead started from Σ-descriptions, taking functions into \AD{DescI} to encode dependent fields, we could compute a ``type of paths'' in a number system by adding and deleting the appropriate fields. Consider the universe:
\ExecuteMetaData[Tex/Discussion/Sigma-Desc]{Sigma-Desc}
In this universe we can present the binary numbers as:
\ExecuteMetaData[Tex/Discussion/Sigma-Desc]{LeibnizD}
The finite type for these numbers can be described by:
\ExecuteMetaData[Tex/Discussion/Sigma-Desc]{FinBD}
Since this description of \AF{FinB} largely has the same structure as \AF{Leibniz}, and as a consequence also the numerical representation associated to \AF{Leibniz}, this would simplify proving that the indexed numerical representation is indeed equivalent to the representable representation (the maps out of \AF{FinB}). In a framework of ornaments for Σ-descriptions \cite{progorn,algorn}, we can even describe the finite type as an ornament on the number system.


\subsection{Branching numerical representations}
A numerical representation constructed by \AF{TrieOD} looks like a finger tree: the structure typically has a central chain, which rather than directly storing elements directly in nodes, stores the elements in trees of which the depth increases with the level of the node.

In contrast, structures like Braun trees, as Hinze and Swierstra \cite{calcdata} compute from binary numbers, represent the weight of a node by branching themselves. Because this kind of branching is uniform, i.e., each branch looks the same, we can still give an equivalent construction. By combining \AF{TreeOD} and \AF{TrieOD}, and using 
\ExecuteMetaData[Tex/Discussion]{power}
to apply \AIC{ρ} \AV{k}-fold in the case of \AIC{ρ}\ \AV{\{if = k\}}, rather than over \AV{k}-element vectors, we can replicate the structure of a Braun tree from \AF{BinND}. However, if we use the Σ-descriptions we discussed above, we can more elegantly present these structures by adding an internal branch over \AD{Fin}\ \AV{k}.

\subsection{Indices do not depend on parameters}\label{sec:no-dep-ix}
In \AD{DescI}, we represent the indices of a description as a single constant type, as opposed to an extension of the parameter telescope \cite{practgen}. This simplification keeps the treatment of ornaments and numerical representations more to the point, but rules out types like the identity type \AD{≡}. 

By replacing index computing functions \AV{Γ}\ \AF{\&}\ \AV{V}\ \AF{⊢}\ \AV{I} with dependent functions
\ExecuteMetaData[Tex/Discussion]{index-interpretation}
we can allow indices to depend on parameters in our framework. As a consequence, we have to modify nested recursive fields to ask for the index type \AF{⟦}\ \AV{I}\ \AF{⟧tel} precomposed with \AV{g :}\ \AF{Cxf}\ \AV{Γ Γ}, and we have to replace the square like \AV{i}\ \AF{∘}\ \AV{j′}\ \AF{∼}\ \AV{i′}\ \AF{∘}\ \AF{over}\ \AV{v} in the definition of ornaments with heterogeneous squares.

Other consequence of allowing indices to depend on parameters is that this allows \emph{algebraic ornaments} \cite{algorn} to be formulated in \AD{OrnDesc} in their fully general form, and also lets us describe \emph{singleton types}, which can, roughly speaking, be used to compute the additional information needed to invert \AF{ornForget}.


\subsection{No RoseTrees}
In \AD{DescI}, we encode nested types by allowing nesting over a function of parameters \AF{Cxf}\ \AV{Γ}\ \AV{Γ}. This is less expressive than full nested types, which may also nest a recursive field under a strictly positive functor. For example, rose trees
\ExecuteMetaData[Tex/Discussion]{RoseTree}
cannot be directly expressed as a \AD{DescI}\footnote{And, since \AD{DescI} does not allow for higher-order inductive arguments like Escot and Cockx \cite{practgen}, we can also not give an essentially equivalent definition.}.

If we were to describe full nested types, allowing applications of functors in the types of recursive arguments, we would have to convince Agda that these functors are indeed positive, possibly by using polarity annotations\footnote{See \url{https://github.com/agda/agda/pull/6385}.}. Alternatively, we could encode strictly positive functors in a separate universe, which only allows using parameters in strictly positive contexts \cite{sijsling}. Finally, we could modify \AD{DescI} in such a way that we can decide if a description uses a parameter strictly positively, for which we would modify \AIC{ρ} and \AIC{σ}, or add variants of \AIC{ρ} and \AIC{σ} restricted to strictly positive usage of parameters.


\subsection{No levitation}
Since our encoding does not support higher-order inductive arguments, let alone definitions by induction-recursion, there is no code for \AD{DescI} in itself. Such self-describing universes have been described by Chapman et al. \cite{levitation}, and we expect that the additional features of \AD{DescI}, i.e., parameters, nesting, and composition, would not obstruct a similar levitating variant of \AD{DescI}. Due to the work of Dagand and McBride \cite{orntrans}, ornaments might even be generalized to inductive-recursive descriptions.

If that is the case, then modifications of universes like \AD{Meta} could be expressed internally. In particular, rather than defining \AD{DescI} such that it can describe datatypes with the information of, e.g., number systems, \AD{DescI} should be expressible as an ornamental description on \AD{Desc}, in contrast to how \AD{Desc} is an instance of \AD{DescI} in our framework. This would allow treating information explicitly in \AD{DescI}, and not at all in \AD{Desc}.

Furthermore, constructions like \AF{TrieOD}, which have the recursive structure of a \AF{fold} over \AD{DescI}, could be expressed by instantiating \AF{fold} to \AD{DescI}.


\subsection{Metadata more tasteful externally than internally}
On the other hand, while incorporating general metadata into \AD{DescI} works out neatly in our case, and in general seems to work out if we think about one use-case at a time, it might not work so nicely in other situations. For example, if we are working with \AF{Number}, but we are given a \AD{DescI}\ \AF{Plain} (i.e., \AF{Desc}), then we would have to duplicate that description in \AD{DescI}\ \AV{Number} before we could use it. Even worse, if we want to give the constructors of a number system nice names using \AF{Names}, we would have to rewrite our code and descriptions to use something like the product of \AF{Number} and \AF{Names}. 

It might be more portable to take the same approach in handling metadata as True sums of products \cite{truesop}, where metadata is described externally to the universe and only combined again if needed by a generic function. From this point of view, a type of metadata can simply be a convenient function from \AD{Desc} to \AD{Type}, and if \AF{Number} was presented in this way, then \AF{TreeOD} would not have to ask for \AD{DescI}\ \AF{Number}, but rather for a \AV{D} of \AD{Desc} with \AF{Number}\ \AV{D}.


\subsection{δ is conservative}\label{sec:redundant-delta}
We define our universe \AD{DescI} with \AIC{δ} as a former of fields with known descriptions, and this makes it easier to write down \AF{TreeOD}, even though \AIC{δ} is redundant. If more concise universes and ornaments are preferable, we can actually get all the features of \AIC{δ} and ornaments like \AIC{∙δ} by describing them using \AIC{σ}, annotations, and other ornaments.

Indeed, rather than using \AIC{δ} to add a field from a description \AV{R}, we can simply use \AIC{σ} to add \AV{S}\ \AV{=}\ \AD{μ}\ \AV{R}, and remember that \AV{S} came from \AV{R} in the information:
\ExecuteMetaData[Tex/Discussion]{Delta-Meta}
We can then define \AIC{δ} as a pattern synonym matching on the \AIC{just} case, and \AIC{σ} matching on the \AIC{nothing} case.

Recall that, leaving out some details, the ornament \AIC{∙δ} lets us compose an ornament from \AV{D} to \AV{D'} with an ornament from \AV{R} to \AV{R'}, yielding an ornament from \AIC{δ}\ \AV{D}\ \AV{R} to \AIC{δ}\ \AV{D'}\ \AV{R'}. This ornament can equivalently be modelled by first adding a new field \AD{μ}\ \AV{R'}, and then deleting the original \AD{μ}\ \AV{R} field. The ornament \AIC{∇} \cite{kophd} allows one to provide a default value for a field, deleting it from the description. Hence, we can model \AIC{∙δ} by binding a value \AV{r'} of \AD{μ}\ \AV{R'} with \AF{OΔσ+} and deleting the field \AD{μ}\ \AV{R} using a default value computed by \AF{ornForget}.

This also partially explains why we did not refer to algebraic ornaments at all in our construction of \AF{TrieOD}; We can see that \AF{TrieOD} looks very similar to the algebraic ornament over \AF{TreeOD}, which sends ornaments from \AV{D} to \AV{E} to an ornament to a \AV{D}-indexed variant of \AV{E}. However, the case of \AIC{δ} requires \AF{TrieOD} to step in and re-index the subdescription. In contrast, the algebraic ornament would simply treat a \AIC{δ} like its equivalent \AIC{σ}, and even though this would produce a correct numerical representation, this amounts to presenting a \AD{Vec} as a tuple of a length \AV{n}, a \AD{List} \AV{xs}, and a proof that \AV{n} is equal to \AF{length}\ \AV{xs}.

Thus, while it would be possible to present \AF{TrieOD} as a kind of algebraic ornament, this would require redefining algebraic ornaments from algebras that are rather specific about how they treat a \AIC{σ}.


\subsection{No sparse numerical representations}\label{sec:discussion-no-sparse}
The encoding of number systems in a universe we explained in \autoref{ssec:numbers} corresponds to a generalization of dense number systems. Consequently, this excludes the skew binary numbers \cite{oka95b} in their useful sparse representation.

Representations of sparse number systems can be regained by allowing addition \emph{and} variable multiplication in a \AIC{σ}. In such a setup, skew binary numbers and other sparse representations could be described by adding their gaps as fields, and computing the appropriate multiplier from there. While not worked out in this thesis, this extension is compatible with the construction of numerical representations.

Another notable extension of \AF{Number} is to let some recursive and composite fields be interpreted by multiplication, with which we could equip \AF{U-fin} with its obvious interpretation into \bN{}. This can be compared to the last exponential law we did not use in \autoref{sec:desc-numrep}, which is that $A^{BC} = (A^B)^C$. Furthermore, any indexed numerical representation acts as a representable functor \AV{F}. If \AV{F} and \AV{G} are numerical representations corresponding to number systems \AV{N} and \AV{M}, then the multiplication of \AV{N} and \AV{M} just corresponds to composition \AV{F}\ \AF{∘}\ \AV{G}.
