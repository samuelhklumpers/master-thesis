\documentclass{article}

\usepackage[style=alphabetic]{biblatex}
\addbibresource{refs.bib}

\usepackage{comment}

\setlength{\marginparwidth}{2cm} % remove when done

\usepackage{todonotes}
\usepackage{xcolor}
\usepackage[hidelinks]{hyperref}

%\hypersetup{
%    colorlinks=true,
%    linkcolor=cyan
%    }

\usepackage{catchfilebetweentags}
\usepackage{quiver} 
\usepackage{tabularx}
\usepackage{amsthm}


\theoremstyle{plain}% default
\newtheorem{theorem}{Theorem}[section]
\newtheorem{lemma}[theorem]{Lemma}
\newtheorem{prop}[theorem]{Proposition}
\newtheorem*{cor}{Corollary}

\theoremstyle{definition}
\newtheorem{defn}{Definition}[section]
\newtheorem{remark}{Remark}[section]
\newtheorem{claim}{Claim}[section]

\renewcommand{\partautorefname}{Part}%
\renewcommand{\sectionautorefname}{Section}%
\renewcommand{\subsectionautorefname}{Subsection}%

\providecommand{\theoremautorefname}{Theorem}%
\providecommand{\lemmaautorefname}{Lemma}%
\providecommand{\propautorefname}{Proposition}%
\providecommand{\corautorefname}{Corollary}%

\providecommand{\defnautorefname}{Definition}%
\providecommand{\remarkautorefname}{Remark}%
\providecommand{\claimautorefname}{Claim}%


\usepackage[links]{agda}
\AgdaNoSpaceAroundCode{}

% from: https://agda.readthedocs.io/en/v2.6.3/_downloads/59877ce886494c991a213f09e29b712c/article-luaxelatex-different-fonts.lagda.tex
\usepackage{fontspec}

\usepackage{luaotfload}
\directlua{luaotfload.add_fallback
  ("mycustomfallback",
    { "JuliaMono:style=Regular;"
    , "NotoSansMono:style=Regular;"
    , "NotoSansMath:style=Regular;"
    }
  )}
\defaultfontfeatures{RawFeature={fallback=mycustomfallback}}

\newfontfamily{\AgdaSerifFont}{Linux Libertine O}
\newfontfamily{\AgdaSansSerifFont}{Linux Biolinum O}
\newfontfamily{\AgdaTypewriterFont}{inconsolata}
\renewcommand{\AgdaFontStyle}[1]{{\AgdaSansSerifFont{}#1}}
\renewcommand{\AgdaKeywordFontStyle}[1]{{\AgdaSansSerifFont{}#1}}
\renewcommand{\AgdaStringFontStyle}[1]{{\AgdaTypewriterFont{}#1}}
\renewcommand{\AgdaCommentFontStyle}[1]{{\AgdaTypewriterFont{}#1}}
\renewcommand{\AgdaBoundFontStyle}[1]{\textit{\AgdaSerifFont{}#1}}

\newcommand{\towrite}[1]{\par\textcolor{blue}{Write here about: ``#1''}\par}
\newcommand{\toremove}[1]{\textcolor{red}{This is going to be (re)moved: ``#1''}}


% macros
\newcommand{\investigate}[1]{\par\vspace{1\baselineskip}\textcolor{gray}{\textit{#1}}\vspace{1\baselineskip}\par}

% symbols
\newcommand{\bN}{\AgdaDatatype{ℕ}}
\newcommand{\bL}{\AgdaDatatype{Leibniz}}


\title{Generic Proofs and Constructions in Agda}
\author{Samuel Klumpers\\6057314}

% previous (provisional) titles:
% Restoring (part of) the friendship between recursion schemes and without-K (provisional)
% The gentle art of smashing things to bits and pieces
% Running in circles in Agda

\begin{document}
\maketitle
\tableofcontents

%This document is generated from a literate agda file!
\newpage

\begin{comment}
\begin{abstract} %provisional
    This thesis introduces the concepts of the structure identity principle, numerical representations, and ornamentations, and aims to combine these to simplify the presentation and verification of finger trees, demonstrating the generalizability and improved compactness and security of the resulting code. 
\end{abstract}
\end{comment}

\part{Proposal}
\section{Outline}
In this document I propose a master thesis project, which will investigate and attempt to counter the obstacles one can encounter when replacing one datastructure with a more complicated one, while keeping the end-result verifiable.

\autoref{sec:intro} describes some challenges that can arise when replacing structures, along with known ways to tackle them.

\autoref{sec:research-question} lists some remaining and new questions regarding this kind of replacement, and proposes methods and ideas to tackle and answer the open challenges and questions.

\autoref{sec:resources} lists literature and frameworks relevant to these questions.

\autoref{sec:planning} proposes a planning for the described project.

\autoref{part:preliminary} summarizes and contains the preliminary work done for this project.

\section{Program Vivisection}\label{sec:intro}
\todo[inline, color=red]{Spend some more time on the context}
Agda \cite{agda} is a functional programming language and a proof assistant, taking inspiration from languages like Haskell and other proof assistants such as Coq. We can write programs as we would in Haskell, and then express and prove their properties all inside Agda. This allows us to demonstrate the correctness of programs by formal proof rather than by testing. 

However, this level of formality also trades-off the uncertainty of testing for a time-investment to produce these proofs. In this thesis, we will explore a variety of methods of proving properties of our programs, focussing on the problems that one may encounter, presenting solutions as they arise. Let us sketch some of these problems.

First, merely adapting a program to Agda may already require changes to the datatypes used in it; for example, if a program manipulating a \AgdaDatatype{List} uses the unsafe \AgdaFunction{head} function, then one is forced to replace the \AgdaDatatype{List} by a datatype that ensures non-emptiness, such as a \AgdaDatatype{NonEmpty} list or a length-aware vector \AgdaDatatype{Vec}. On the other hand, there might be sections of a program where the concrete length is not relevant for correctness and only gets in the way. As a result, one might find themselves duplicating common functions like concatenation \AgdaFunction{\_++\_} to only alter their signatures.

However, the ``new'' datatype (\AgdaDatatype{Vec}) is typically a simple variation on the old datatype (\AgdaDatatype{List}) making small adjustments to the existing constructors; in this case, we decorate the nil and cons constructors with natural numbers representing the length. This kind of modification of types falls in the framework of ornamentation as described by Ko and Gibbons \cite{progorn}; if two types are reified to their \textit{descriptions}, then \textit{ornaments} express whether the types are ``similar'' by acting as a recipe to produce one type from the other. By restricting the operations to the copying of corresponding parts, and the introduction of fields or dropping of indices, the existence of such an ornament ensures that the types have the same recursive structure.

\towrite{Something about patches.}

\towrite{For each invariant a new datatype? Still ornaments}

Now that we know we can organize similar datatypes using ornaments, it is time to look at dissimilar datatypes. It is conventional to prototype a program using simpler types or implementations, and only replace these with more performant alternatives in critical places; knowing that this is eventually going to happen, one might as well prepare for it. While this may quickly turn into a refactoring nightmare in the general case, we can hope for a more satisfying transition if we restrict our attention to a narrower scope. As an example, we might start programming using \AgdaDatatype{List}s, but replace this with a \AgdaDatatype{Tree} if we notice that the program spends most of its time in \AgdaFunction{lookup} operations. To gain a speedup, we will have to reimplement the operations on \AgdaDatatype{Tree}. This would also double the number of necessary proofs; however, we have two ways to avoid this problem. 

We will look at the more specific solution first. This solution is guided by the realization that \AgdaDatatype{List} and \AgdaDatatype{Tree}, like most other containers, still have similarities if their recursive structure is very different. That is, both resemble a number system, and, Okasaki \cite{purelyfunctional} notes that this resemblance to number systems is ``surprisingly common''. In the case of lists and Braun trees\footnote{Braun trees are a kind of binary tree, of which its shape is determined by its size.}, one can present both by deriving them from unary and binary numbers respectively, as is made formal by Hinze and Swierstra \cite{calcdata}. One can then apply this \textit{numerical representation} to simplify or make trivial the proofs of the properties we hesitated to duplicate before.

\towrite{If we instead hide our datatypes behind interfaces, we can use proof transport as an alternative.}

\towrite{Something about fingertrees, leading into the research question and proposed work}


\section{\toremove{Introduction}}
%The dependently typed functional programming language Agda \cite{agda} can, when restricted to its reasonable parts, be translated into readable and safe Haskell \cite{agda2hs}. However, the intrinsic safety of languages like Agda can also lead to code duplication by encouraging the use of multiple variants of the same datatype. As an example, the coverage check forces the \AgdaFunction{head} function on \AgdaDatatype{List} to return a \AgdaDatatype{Maybe}. This \AgdaDatatype{Maybe} can be avoided by moving to the length-indexed list type \AgdaDatatype{Vec}, at the cost of duplicating functions like \AgdaFunction{\_++\_}, which we need at both types.

Something similar happens when replacing an implementation with a more efficient one. For example, when implementing binary trees as a more efficient alternative to lists, the proofs of the same properties will differ between list and tree, and tend to be more difficult for the latter. Switching between implementations of an interface not only duplicates code, but also (and sometimes more than) doubles the effort required to verify both.\todo[inline, color=red]{concrete example?}

%There is plenty of prior work dealing with problems like these. The work in \cite{orntrans} and \cite{progorn} provides the means to relate similar datatypes, such as lists and vectors, using the mechanism of ornamentation, letting us organize variants of the same datatype in a rigid framework.  %This leads them to define the concept of patches, which can aid us when defining \AgdaFunction{\_++\_} for the second time by forcing the new version to be coherent.
%In fact, the algebraic nature of ornaments yields the definition of the vector type for free, provided we relate lists to natural numbers \cite{algorn}. %Such constructions rely heavily on descriptions of datastructures and often come with limitations in their expressiveness. These descriptions in turn impose additional ballast on the programmer, leading us to investigate reflection like in \cite{practgen} as a means to bring datatypes and descriptions closer when possible.

Other work like \cite{calcdata}\todo[inline, color=red]{don't use \textbackslash cite as noun} simplifies the proofs relating to certain containers directly, formally executing the way of though of numerical representations as noted in \cite{purelyfunctional}.
%From another point of view, lists and trees are not so different at all, provided we look at them through the interface of one-sided flexible arrays; this idea noted in \cite{purelyfunctional} and formalized in \cite{calcdata} where both are shown to be instances of numerical representations by calculating them from a numeral system. 

When two types are isomorphic and equivalent under an interface, proofs of properties of these implementations should be interconvertible. By using structured equivalences and univalence, \cite{iri} characterizes equivalences under interfaces.
%While this is achievable through meta-programming, substituting conversions to and from into the proof terms, this is internally expressible in Cubical Agda.

%We can liken the situation to movement on a plane, where ornamentation moves us vertically by modifying constructors or indices, and structured equivalences move us horizontally to and from equivalent but more equivalent implementations. In this paper, we will investigate a variety of means of moving around structures and proofs, and ways to make this more efficient or less intrusive.

In \autoref{sec:leibniz}, we will follow \cite{iri}, and look at how proofs on unary naturals can be transported to the binary naturals. Then in \autoref{sec:numrep} we recall how numeral systems in particular induce container types in \cite{calcdata}, which we attempt to reformulate in the language of ornaments in \autoref{ssec:ornaments}, using the framework of \cite{progorn}. In \autoref{sec:userfriendly} we investigate how we can make the earlier methods more easily accessible to the user, and, ourselves, when we give a description of finger trees in \autoref{sec:fingertrees}.\todo[inline, color=red]{Ok, but make the research question more concrete}



\section{Research Question and Contributions}\label{sec:research-question}
The research question of this project will be: \textit{can we describe finger trees \cite{fingertrees} in the frameworks of numerical representations and ornamentation \cite{progorn}, simplifying the verification of their properties as flexible two-sided arrays?} We expect to encounter a variety of obstacles along the way, and so the project will focus on finding existing or developing new theory to deal with these. For example, an obstacle can be that the corresponding number system has many representations for the same number, which we expect to describe using quotients \cite{cuagda} and reason about using representation independence \cite{iri}. If this is accomplished or deemed infeasible at an early stage, we can generalize the results we have to other related problems; for example, we may view the problem of generating arbitrary values for testing as an instance of an enumeration problem, but then through the lens of ornaments.


\section{Related work}\label{sec:resources}
\towrite{Organize this into a human text}

\begin{tabularx}{\textwidth}{l | X}
    \cite{iri}      & Equivalences from relations, transport over structures \\ 
    \hline 
    \cite{calcdata} & Prototyping datatypes with properties deriving from the underlying numbers, enjoy the definitional equalities of representables at the inductive types \\  
    \hline 
    \cite{progorn}  & Using simple types as stepping stones to more impressive types, generalization of numerical representations? Are ``ornament-computing functions'' powerful, or are they nice party-tricks and is heterogeneization a coincidence? \\  
    \hline 
    \cite{purelyfunctional}   & Not so much work in the same direction, but definitely a source of inspiration, and a good reference for which datastructure is good at what \\
    \hline 
    \cite{hetbin}        & A predecessor/inspiration for heterogeneization \\
    \hline 
    \cite{glookup}       & Perhaps an inspiration to do more with the numerical representations/ornaments experiment \\
    \hline 
    \cite{orntrans}      & How many functions do we keep when we turn our naturals into lists or vectors? \\
    \hline 
    \cite{hsip}          & Can we push SIP to eat indexed structures? (I still have to attempt to digest this) \\
    \hline 
    \cite{practgen}      & Source of inspiration for some generic constructions (and perhaps a list of things we don't need to try again) \\
    \hline 
    \cite{tgalois}  & It claims to achieve what univalence does, without the Cubical/HoTT overhead? (Haven't fully read, even if it doesn't do what this makes me think it does, it references a wealth of other papers that might) \\
    \hline 
    \cite{coveringbases} & Inspiration/application for enumeration \\
    \hline 
    \cite{recschemes} & Like mapFold, but more probably \\
    %\hline 
    %InitFinal?    & The least fixpoint of $X \mapsto \top + A \times X$ is a list, the largest is a stream. What happens in between the two? (This might not be relevant at all) \\
    \hline 
    \cite{amortcoind}    & Could we derive the amortized bounds on (our or Ralf's) fingertrees using this? (Not sure if I want to...)
\end{tabularx}


\section{Planning}\label{sec:planning}
\towrite{This will be done}


\part{Preliminary work}\label{part:preliminary}
\towrite{This has been done}

\section{Proof Transport via the Structure Identity Principle}\label{sec:leibniz}
Let us quickly review the small set of features in Cubical Agda that we will be using extensively throughout this article.\footnote{\cite{cuagda} gives a comprehensive introduction to cubical agda.}

%Of course, this downside is more than offset by the benefits of changing our primitive notion of equality, which we will see makes it easier to show that ``equivalent'' structures behave identically. 
In Cubical Agda, the primitive notion of equality arises not (directly) from the indexed inductive definition we are used to, but rather from the presence of the interval type \AgdaPrimitiveType{I}. This type represents a set of two points \AgdaInductiveConstructor{i0} and \AgdaInductiveConstructor{i1}, which are considered ``identified'' in the sense that they are connected by a path. To define a function out of this type, we also have to define the function on all the intermediate points, which is why we call such a function a ``path''. Terms of other types are then considered identified when there is a path between them.

While the benefits are overwhelming for us, this is not completely without downsides, such as that
%\ExecuteMetaData[Tex/CubicalAndBinary]{cubical}% \todo{Not sure if it would be helpful to have a more extensive introduction covering all features used.} % at this moment, probably not, as the cubical usage is rather tame, so I'll probably stick to introducing stuff as it becomes necessary. % TODO then write that somewhere
the negation of axiom K complicates both some termination checking and some universe levels.\footnote{In particular, this prompts rather far-reaching (but not fundamental) changes to the code of previous work, such as to that of \cite{progorn} in \autoref{sec:userfriendly}.} Furthermore, if we use certain homotopical constructions, like set quotients, we will also have to prove that our types are sets, before we can use them.

On the positive side, this different perspective gives intuitive interpretations to some proofs of equality, like
\ExecuteMetaData[Tex/CubicalAndBinary]{sym}
where \AgdaFunction{∼\_} is the interval reversal, swapping \AgdaInductiveConstructor{i0} and \AgdaInductiveConstructor{i1}, so that \AgdaFunction{sym} simply reverses the given path.

Furthermore, because we can now interpret paths in records and function differently, we get a host of ``extensionality'' for free. For example, a path in $A \to B$ is indeed a function which takes each $i$ in \AgdaPrimitiveType{I} to a function $A \to B$. Using this, function extensionality becomes tautological 
\ExecuteMetaData[Tex/CubicalAndBinary]{funExt}

Finally, while in ``non-univalent'' Agda bijections or isomorphisms do not play such a central role, much of our work will rest on equivalences, as the ``HoTT-compatible'' generalization of bijections. This is because the \AgdaPrimitiveType{Glue} type tells us that equivalent types fit together in a new type, in a way that guarantees univalence
\ExecuteMetaData[Tex/CubicalAndBinary]{ua}
This essentially states that ``equivalent types are identified'', such that type isomorphisms like $1 \to A \simeq A$ actually become paths $1 \to A \equiv A$, making it so that we can transport proofs along them. We will demonstrate this by a slightly more practical example in the next section.


\subsection{Binary numbers}\label{ssec:binary}
Let us demonstrate an application of univalence by exploiting the equivalence of the ``Peano'' naturals and the ``Leibniz'' naturals. Recall that the Peano naturals are defined as 
\ExecuteMetaData[Tex/CubicalAndBinary]{Peano}
This definition enjoys a simple induction principle and has many proofs of its properties in standard libraries. However, it is too slow to be of practical use: most arithmetic operations defined on \bN{} have time complexity in the order of the value of the result.

Of course, the alternative are the more performant binary numbers: the time complexities for binary numbers are usually logarithmic in the resultant values. However, the number of cases for a proof about binary numbers also grows quicker than it would for unary numbers. This does not have to be a problem, because the \bN{} naturals and the binary numbers should be equivalent after all!

Let us make this formal. We define the Leibniz naturals as follows:
\ExecuteMetaData[Leibniz/Base.tex]{Leibniz}
Here, the \AgdaInductiveConstructor{0b} constructor encodes 0, while the \AgdaInductiveConstructor{\_1b} and \AgdaInductiveConstructor{\_2b} constructors respectively add a 1 and a 2 bit, under the usual interpretation of binary numbers:
\ExecuteMetaData[Leibniz/Base.tex]{toN}
This defines one direction of the equivalence from \bN{} to \bL{}, for the other direction, we can interpret a number in \bN{} as a binary number by repeating the successor operation on binary numbers:
\ExecuteMetaData[Leibniz/Base.tex]{bsuc}
\ExecuteMetaData[Leibniz/Base.tex]{fromN}
To show that \AgdaFunction{toℕ} is an isomorphism, we have to show that it is the inverse of \AgdaFunction{fromℕ}. By induction on \bL{} and basic arithmetic on \bN{} we see that
\ExecuteMetaData[Leibniz/Properties.tex]{toN-suc}
so \AgdaFunction{toℕ} respects successors. Similarly, by induction on \bN{} we get
\ExecuteMetaData[Leibniz/Properties.tex]{fromN-1}
and % I can't get the code blocks to stick together lol
\ExecuteMetaData[Leibniz/Properties.tex]{fromN-2}
so that \AgdaFunction{fromℕ} respects even and odd numbers. We can then prove that applying \AgdaFunction{toℕ} and \AgdaFunction{fromℕ} after each other is the identity by repeating these lemmas
\ExecuteMetaData[Leibniz/Properties.tex]{N-iso-L}
This isomorphism can be promoted to an equivalence
\ExecuteMetaData[Leibniz/Properties.tex]{N-equiv-L}
which, finally, lets us identify \bN{} and \bL{} by univalence
\ExecuteMetaData[Leibniz/Properties.tex]{N-is-L}
The path \AgdaFunction{ℕ≡L} then allows us to transport properties from \bN{} directly to \bL{}, e.g.,
\ExecuteMetaData[Leibniz/Properties.tex]{isSetL}
This can be generalized even further to transport proofs about operations from \bN{} to \bL{}. 

\subsection{Use as definition: functions from specifications}\label{ssec:useas}
As an example, we will define addition of binary numbers. We could transport binary operations
\ExecuteMetaData[Extra/Algebra]{BinOp}
to get
\ExecuteMetaData[Tex/CubicalAndBinary]{badplus}
but this would be rather inefficient, incurring an $O(n + m)$ overhead when adding $n + m$. It is more efficient to define addition on \bL{} directly, making use of the binary nature of \bL{}, while agreeing with the addition on \bN{}. Such a definition can be derived from the specification ``agrees with \AgdaFunction{\_+\_}'', so we implement the following syntax for giving definitions by equational reasoning, inspired by the ``use-as-definition'' notation from \cite{calcdata}:
\ExecuteMetaData[Prelude/UseAs.tex]{Def}
which infers the definition from the right endpoint of a path using an implicit pair type
\ExecuteMetaData[Prelude/UseAs.tex]{isigma}
% \investigate{As of now, I am unsure if this reduces the effort of implementing a coherent function, or whether it is more typically possible to give a smarter or shorter proof by just giving a definition and proving an easier property of it\footnote{I will put the alternative in the appendix for now}}

With this we can define addition on \bL{} and show it agrees with addition on \bN{} in one motion
\ExecuteMetaData[Leibniz/Properties.tex]{plus-def}
Now we can easily extract the definition of \AgdaFunction{plus} and its correctness with respect to \AgdaFunction{\_+\_} 
\ExecuteMetaData[Leibniz/Properties.tex]{plus-good}

We remark \AgdaFunction{Def} is close in concept to refinement types\footnote{À la \href{https://agda.github.io/agda-stdlib/Data.Refinement.html}{Data.Refinement}.}, but importantly, the equality proof is relevant for us, and the value is inferred rather than given. \footnote{Unfortunately, normalizing an application of a \AgdaFunction{defined-by} function also causes a lot of unnecessary wrapping and unwrapping, so \AgdaFunction{Def} is mostly only useful for presentation.} %for now..


\subsection{Structure Identity Principle}
Now \bN{} with \AgdaFunction{N.+} form, in particular, a magma. The same goes for \bL{} and \AgdaFunction{plus}, but notice that a path in a \AgdaDatatype{Σ} type is just a \AgdaDatatype{Σ} of paths! This means that we get a path from (\bN{}, \AgdaFunction{N.+}) to (\bL{}, \AgdaFunction{plus}). More generally, a magma is simply a type $X$ with some structure, which is a function $f: X \to X \to X$ in the case of a magma. We can see that paths between magmas correspond to paths $p$ between the underlying types $X$ and paths over $p$ between their operations $f$. This observation is further generalized by the Structure Identity Principle (SIP), formalized in \cite{iri}. Given a structure, which in our case is just a binary operation
\ExecuteMetaData[Extra/Algebra.tex]{MagmaStr}
this principle produces an appropriate definition ``structured equivalence'' $\iota$. The $\iota$ is such that if structures $X, Y$ are $\iota$-equivalent, then they are identified. In the case of \AgdaFunction{MagmaStr}, the $\iota$ asks us to provide something with the same type as \AgdaFunction{plus-coherent}, so we have just shown that the \AgdaFunction{plus} magma on \bL{}
\ExecuteMetaData[Leibniz/Properties.tex]{magmaL}
and the \AgdaFunction{\_+\_} magma on \bN{} and are identical
\ExecuteMetaData[Leibniz/Properties.tex]{magma-equal}
As a consequence, properties of \AgdaFunction{\_+\_} directly yield corresponding properties of \AgdaFunction{plus}. For example,
\ExecuteMetaData[Leibniz/Properties.tex]{assoc-transport}

\section{Types from Specifications: Ornamentation and Calculation}\label{sec:numrep}
\documentclass[Main.tex]{subfiles}

\begin{document}
While the practical applications of the last example do not stretch very far\footnote{Considering that \AgdaDatatype{ℕ} is a candidate to be replaced by a more suitable unsigned integer type when compiling to Haskell anyway.}, the approach generalizes to the more relevant containers and their associated laws.

In the same vein as the last section, we could define a simple but inefficient array type, and a more efficient implementation using trees. Then we can show that these are equivalent, such that when the simple type satisfies a set of laws, trees will satisfy them as well. We could then start developing all sorts of complex implementations fine-tuned to each operation and figure out how these are equivalent to some simpler type, but let us first take a step back, and investigate how we can make this approach run smoothly in an even simpler example.

Rather than defining inductively defining a container and then showing that it is represented by a lookup function, we can go the other way and define a type by insisting that it is equivalent to such a function. This approach, in particular the case in which one calculates a container with the same shape as a numeral system, was dubbed numerical representations in \cite{purelyfunctional}, and has some formalized examples in, e.g., \cite{calcdata} and \cite{progorn}. Numerical representations form the starting point for defining more complex datastructures based off of simpler basic structures, so let us run through an example.

\subsection{Numerical representations: from numbers to containers}
We can compute the type of vectors starting from \bN{}.\footnote{This is adapted (and fairly abridged) from \cite{calcdata}} For simplicity, we define them as a type computing function via the ``use-as-definition`` notation from before. We expect vectors to be represented by 
\[ lookup \]
where we use the finite type \AgdaDatatype{Fin} as an index into vector. We define this as
\[ finfromsigma \]
The computation of vectors proceeds as follows
\[ vectors \]

\investigate{SIP doesn't mesh very well with indexed stuff, does HSIP help?}

Arrays are made to be indexed, but let us list some expectations
\todo{do this}

The implementation of vectors as functions is very straightforward
\[ \]
and clearly satisfies our interface
\[ \]
Again these proofs transport to vectors.\todo{If one was determined to cobble together the path over path over path we need now.}

(This computation can of course be generalized to any arity zeroless numeral system; unfortunately beyond this set of base types, this ``straightforward'' computation from numeral system to container loses its efficacy. In a sense, the n-ary natural numbers are exactly the base types for which the required steps are convenient type equivalences like $(A + B) \to C = (A \to C) \times (B \to C)$?)

%\subsection{Relating types by structure: Ornamentation (unfinished)}\label{sec:ornament}
\subsection{Numerical representations as ornaments}
We could peform the same computation for \bL{}, which would yield the type of binary trees, but we note that these computations proceed with roughly the same pattern: each constructor of the numeral system gets assigned a value, and is amended with a field holding a number of elements and subnodes using this value as a ``weight''. But wait! Such modifications of constructors are already made formal by the concept of ornamentation!

Ornamentation, as exposed in \cite{algorn} and \cite{progorn}, lets us formulate what it means for two types to have a ``similar'' recursive structure. This is achieved by interpreting (indexed inductive) datatypes from descriptions, between which an ornament is seen as a certificate of similarity, describing which fields or indices need to be introduced or dropped. Furthermore, a one-sided ornament: an ornamental description, lets us describe new datatypes by recording the modifications to an existing description.
\todo{Again not sure if it helps to reiterate Desc, Orn, and OrnDesc.}

This links back to the construction in the previous section, since \bN{} and \AgdaDatatype{Vec} share the same recursive structure, so \AgdaDatatype{Vec} can be formed by introducing indices and adding a field holding an element at each node.\footnote{These and similar examples are also documented in \cite{progorn}} For this, we first have to give a description of \bN{} to work with\todo{Clearly this can use more explanation (the question is, how much?)}
\todo{include this}
Now we can write down the ornament which adds fields to the \AgdaFunction{suc} constructor
\[ include me \]
With the least fixpoint and description extraction from \cite{progorn}, this is sufficient to define \AgdaDatatype{Vec}.

Note that we cannot hope to give an unindexed ornament from \bL{} into trees, since trees have a very different recursive structure! Instead, we must keep track at what level we are in the tree so that we can ask for adequately many elements. 
\todo{include this}

In fact, this ``folding in'' technique seems to apply rather generally, let us digress.

\subsection{Folding in}
Let us describe this procedure of folding a complex recursive structure into a simpler structure more generally, and relate this to the construction of binary heaps in \cite{progorn}.
\todo{go}


\end{document}

\section{More equivalences for less effort}\label{sec:userfriendly}
% REPLACE X BY A?
The setup some approaches in earlier sections require makes them tedious or impractical to apply. In this section we will look at some ways how part of this problem could be alleviated through generics, or by alternative descriptions of concepts like equivalences through the lens of initial algebras. 

In later sections we will construct many more equivalences between more complicated types than before, so we will dive right into the latter. Reflecting upon \autoref{sec:leibniz}, we see that when one establishes an equivalence, most of the time is spent working out a series of tedious lemmas to show that the conversion functions are mutual inverses, which tend to be relatively easy to define. We take away two things from this; the first is that the conversion functions are perhaps too obvious, and the second is that we should really avoid talking about sections and retractions lest we incur tedium!\footnote{The latter perhaps less so, because it is useful to show a map to be monic.} We will reuse the machinery from \cite{progorn} to illustrate how the definitions in \autoref{sec:leibniz} were actually forced for a large part.

First, we remark that \AgdaDatatype{μ} is internalization of the representation of simple\footnote{Of course, indexed datatypes are indexed W-types, mutually recursive datatypes are represented yet differently\dots} datatypes as W-types. Thus, we will assume that one of the sides of the equivalence is always represented as an initial algebra of a polynomial functor, and hence the \AgdaDatatype{μ} of a \AgdaDatatype{Desc′}.

\subsection{Well-founded monic algebras are initial}\label{ssec:wellfounded}
Unfortunately, the machinery from \cite{progorn} relies on axiom K for a small but crucial part. To be precise, in a cubical setting, the type \AgdaDatatype{μ} as given stops being initial for its base functor! In this section, we will be working with a simplified and repaired version. Namely, we simplify \AgdaDatatype{Desc′} to 
\ExecuteMetaData[Extra/ProgOrn/Desc]{DescS}
To complete the definition of \AgdaDatatype{μ}
\ExecuteMetaData[Extra/ProgOrn/Desc]{mu}
we will need to implement \AgdaDatatype{Base}. We remark that in \cite{progorn}, the recursion of \AgdaFunction{mapFold} is a structural descent in \AgdaFunction{⟦ D' ⟧ (μ D)}. Because \AgdaFunction{⟦\_⟧} is a type computing function and not a datatype, this descent becomes invalid\footnote{Refer to the  \href{https://agda.readthedocs.io/en/latest/language/without-k.html\#restrictions-on-termination-checking}{without K} page.}, and \AgdaFunction{mapFold} fails the termination check. We resolve this by defining \AgdaDatatype{Base} as a datatype
\ExecuteMetaData[Extra/ProgOrn/Desc]{Base}
such that this descent is allowed by the termination checker without axiom K.\footnote{This has, again by the absence of axiom K, the consequence of pushing the universe levels up by one. However, this is not too troublesome, as equivalences can go between two levels, and indeed types are equivalent to their lifts.}

Recall that the \AgdaDatatype{Base} functors of descriptions are special polynomial functors, and the fixpoint of a base functor is its initial algebra. The situation so far is summarized by the diagram
% https://q.uiver.app/?q=WzAsMyxbMSwwLCJGXFxtdV9GIl0sWzEsMSwiXFxtdSBGIl0sWzAsMSwiWCJdLFswLDEsIlxcbWF0aHJte2Nvbn0iXSxbMiwxLCJlIiwyLHsic3R5bGUiOnsidGFpbCI6eyJuYW1lIjoiYXJyb3doZWFkIn0sImJvZHkiOnsibmFtZSI6ImRhc2hlZCJ9fX1dXQ==
\[\begin{tikzcd}[ampersand replacement=\&]
	\& {F\mu_F} \\
	X \& {\mu F}
	\arrow["{\mathrm{con}}", from=1-2, to=2-2]
	\arrow["e"', dashed, tail reversed, from=2-1, to=2-2]
\end{tikzcd}\]
so, we are looking for sufficient conditions on $X$ to get the equivalence $e: X \cong \mu F$. Note that when $X \cong \mu F$, then there necessarily is an initial algebra $F X \to X$. Conversely, if the algebra $(X, f)$ is isomorphic to $(\mu F, \mathrm{con})$, then $X \cong \mu F$ would follow immediately, so it is equivalent to ask for the algebras to be isomorphic instead.

\subsubsection{Datatypes as initial algebras}
To characterize when such algebras are isomorphic, we reiterate some basic category theory, simultaneously rephrasing it in Agda terms.\footnote{We are not reusing a pre-existing category theory library for the simple reasons that it is not that much work to write out the machinery explicitly, and that such libraries tend to phrase initial objects in the correct way, which is too restrictive for us.}\todo{Maybe category theory reference}

Let $C$ be a category, and let $a, b, c$ be objects of $C$, so that in particular we have identity arrows $1_a : a \to a$ and for arrows $g : b \to c, f : a \to b$ composite arrows $gf : a \to c$ subject to associativity. In our case, $C$ is the category of types, with ordinary functions as arrows.

Recall that an endofunctor, which is simply a functor $F$ from $C$ to itself, assigns objects to objects and sends arrows to arrows
\ExecuteMetaData[Extra/Category]{RawFunctor}
These assignments are subject to the identity and composition laws
\ExecuteMetaData[Extra/Category]{Functor}
An $F$-algebra is just a pair of an object $a$ and an arrow $Fa \to a$
\ExecuteMetaData[Extra/Category]{Algebra}
Algebras themselves again form a category $C^F$. The arrows of $C^F$ are the arrows $f$ of $C$ such that the following square commutes% https://q.uiver.app/?q=WzAsNCxbMCwwLCJGYSJdLFsxLDAsIkZiIl0sWzAsMSwiYSJdLFsxLDEsImIiXSxbMiwzLCJmIiwyXSxbMCwyLCJVX2EiLDJdLFsxLDMsIlVfYiJdLFswLDEsIkZmIl1d
\[\begin{tikzcd}[ampersand replacement=\&]
	Fa \& Fb \\
	a \& b
	\arrow["f"', from=2-1, to=2-2]
	\arrow["{U_a}"', from=1-1, to=2-1]
	\arrow["{U_b}", from=1-2, to=2-2]
	\arrow["Ff", from=1-1, to=1-2]
\end{tikzcd}\]
So we define
\ExecuteMetaData[Extra/Category]{AlgSqr}
and
\ExecuteMetaData[Extra/Category]{AlgMap}
Note that we take the propositional truncation of the square, such that algebra maps with the same underlying morphism become propositionally equal
\ExecuteMetaData[Extra/Category]{AlgPath}
The identity and composition in $C^F$ arise directly from those of the underlying arrows in $C$.

Recall that an object $\emptyset$ is initial when for each other object $a$, there is an unique arrow $!: \emptyset \to a$. By reversing the proofs of initiality of \AgdaDatatype{μ} and the main result of this section, we obtain a slight variation upon the usual definition. Namely, unicity is often expressed as contractability of a type
\ExecuteMetaData[Tex/Snippets]{isContr}
Instead, we again use a truncation
\ExecuteMetaData[Extra/Category]{weakContr}
but note that this also, crucially, slightly stronger than connectedness. We define initiality for arbitrary relations
\ExecuteMetaData[Extra/Category]{Initial}
such that it closely resembles the definition of least element. Then, $A$ is an initial algebra when
\ExecuteMetaData[Extra/Category]{InitAlg}

By basic category theory (using the usual definition of initial objects), two initial objects $a$ and $b$ are always isomorphic;
namely, initiality guarantees that there are arrows $f : a \to b$ and $g : b \to a$, which by initiality must compose to the identities again.

Similarly, we get that
\ExecuteMetaData[Extra/Category]{InitAlg-equiv}
However, we only have the equalities from the isomorphism inside a propositional truncation. But fortunately, being an equivalence is a property, so we can eliminate from the truncations to get the wanted result.

Note that even though we warned ourselves, we are still talking about sections and retractions to establish that $f$ is an equivalence! However, this result also makes sure we will not have to speak of them again.\footnote{For now...}

\subsubsection{Accessibility}
As a consequence, we get that $X$ is isomorphic to $\mu D$ when $X$ is an initial algebra for the base functor of $D$; $\mu D$ is initial by its fold, and by induction on $\mu D$ using the squares of algebra maps. 

\begin{remark}
    We need (in general) not hope $\mu D$ is a strict initial object in the category of algebras. For a strict initial object, having a map $a \to \emptyset$ implies $a \cong \emptyset$. This is not the case here: strict initial objects satisfy $a \times \emptyset \cong \emptyset$, but for the $X \mapsto 1 + X$-algebras $\mathbb{N}$ and $2^\mathbb{N}$ clearly $2^\mathbb{N} \times \mathbb{N} \cong \mathbb{N}$ does not hold. On the other hand, the ``obvious'' sufficient condition to let $C^F$ have strict initial objects is that $F$ is a left adjoint, but then the carrier of the initial algebra is simply $\bot$.
\end{remark}

Looking back at \autoref{sec:leibniz}, we see that \AgdaDatatype{Leibniz} is an initial $F: X \mapsto 1 + X$ algebra because for any other algebra, the image of \AgdaFunction{0b} is fixed, and by \AgdaFunction{bsuc} all other values are determined by chasing around the square. Thus, we are looking for a similar structure on $f : FX \to X$ that supports recursion.

Clearly we will need something stronger than $FX \cong X$, as in general a functor can have many fixpoints. For this, we define what it means for an element $x$ to be accessible by $f$. This definition uses a mutually recursive datatype as follows:
We state that an element $x$ of $X$ is accessible when there is an accessible $y$ in its fiber over $f$
\ExecuteMetaData[Extra/Category/Poly]{Acc}
Accessibility of an element $x$ of \AgdaFunction{Base A E} is defined by cases on $E$; if $E$ is \AgdaFunction{ṿ n} and $x$ is a \AgdaFunction{Vec A n}, then $x$ is accessible if all its elements are; if $x$ is \AgdaFunction{σ S E'}, then $x$ is accessible if \AgdaFunction{snd x} is
\ExecuteMetaData[Extra/Category/Poly]{Acc'}
Consequently, $X$ is well-founded for an algebra when all its elements are accessible
\ExecuteMetaData[Extra/Category/Poly]{Wf}

We can see well-foundedness as an upper bound on the size of $X$, if it were larger than $\mu D$, some of its elements would inevitably get out of reach of an algebra. \textit{Now} having $FX \cong X$ also gives us a lower bound, but remark that having a well-founded injection $f: FX \to X$ is already sufficient, as accessibility gives a section of $f$, making it an iso. This leads us to claim
\begin{claim}\label{claim:wf-inj-init}
    If there is a mono $f : FX \to X$ and $X$ is well-founded for $f$, then $X$ is an initial $F$-algebra.
\end{claim}

\subsubsection{Proof sketch of \autoref{claim:wf-inj-init}}
Let us be on our way. Suppose $X$ is well-founded for the mono $f : FX \to X$. To show that $(X, f)$ is initial, let us take another algebra $(Y, g)$, and show that there is a unique arrow $(X, f) \to (Y, g)$.\todo{This section is about as digestable as a brick.}

By \AgdaDatatype{Acc}-recursion and because all $x$ are accessible, we can define a plain map into $Y$
\ExecuteMetaData[Extra/Category/WellFounded]{Wf-rec}
This construction is an instance of the concept of ``well-founded recursion''\footnote{This is formalized in the \href{https://agda.github.io/agda-stdlib/Induction.WellFounded.html}{standard-library} with many other examples.}, so we let ourselves be inspired by these methods. In particular, we prove an irrelevance lemma
\ExecuteMetaData[Extra/Category/WellFounded]{Wf-rec-irr}
which implies the unfolding lemma
\ExecuteMetaData[Extra/Category/WellFounded]{Wf-rec-unfold}
The unfolding lemma ensures that the map we defined by \AgdaFunction{Wf-rec} is a map of algebras. The proof that this map is unique proceeds analogously to that in the proof that $\mu D$ is initial, but here we instead use \AgdaDatatype{Acc}-recursion
\ExecuteMetaData[Extra/Category/WellFounded]{Wf+inj=Init}
Thus, we conclude that $X$ is initial. The main result is then a corollary of initiality of $X$ and the isomorphism of initial objects
\ExecuteMetaData[Extra/Category/WellFounded]{Wf+inj=mu}


\subsubsection{Example}
Let us redo the proof in \autoref{sec:leibniz}, now using this result. 

\begin{comment}
\section{FingerTrees}\label{sec:fingertrees}
Fingertrees are often (rightfully so) referred to as ``the fastest persistent datastructure for most purposes'', but while simpler than implementations achieving the same bounds, they are still challenging to reason about; in this section, we will investigate how we can fit the description and analysis of fingertrees, or variants upon them, into the frameworks of calculating datastructures and ornamental programming.

We compare the work in calculating datastructures to solving associativity equations in groups by shifting to the Cayley representation, such as in [..]


%\section{Discussion and Future Work}\label{sec:discussion}


\newpage
\section{Temporary}\label{sec:temp}
\listoftodos
%\subfile{Scratch.tex}
\end{comment}




\printbibliography
\end{document}
