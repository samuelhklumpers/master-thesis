\section{Ornamental descriptions}
These ornamental descriptions take the same shape as those in \autoref{sec:background-ornamental-descriptions}, generalized to handle nested types, variable transformations, and composite types. Like the interpretation of a description \AD{DescI}, ornaments also completely ignore the \AD{Info} of a \AD{DescI}.

We will define \AD{OrnDesc}\ \AV{If′ Δ c J i D} to represent the ornaments building on top of a base description \AV{D}, yielding descriptions with information \AV{If′}, parameters \AV{Δ}, and indices \AV{J}:
\ExecuteMetaData[Ornament/OrnDesc]{OrnDesc}
We use \AF{∼} to write down pointwise equality of functions, which in this case all are commutativity squares. Since \AD{ConI} allows the transformation of variable telescopes, we have to dedicate a lot of lines to writing down commutativity squares for variables, which along with the generally high number of arguments and implicits\footnote{Of which even more are hidden!} makes the definition rather dry and long. However, these squares involving \AD{Vxf} can generally ignored, as witnessed by the \AF{Oσ+} and \AF{Oσ-} variants of the constructors, which automatically fill those squares in the common cases of binding or ignoring fields.

The constructor ornaments can be split into three segments: structure-preserving ornaments, extensions, and composition. The structure-preserving ornaments are
\ExecuteMetaData[Ornament/OrnDesc]{ConOrn-preserve}
These represent the ornaments in which the base description and the target description share the same field, up to conversions of parameters, variables, and indices.

The ornaments extending structure are
\ExecuteMetaData[Ornament/OrnDesc]{ConOrn-extend}
representing the insertion of fields in the target which are not present in the base description.

Finally, the ornament
\ExecuteMetaData[Ornament/OrnDesc]{ConOrn-compose}
makes it possible compose an ornament onto a \AIC{δ} in the base description.

Compared to the previous ornaments, we have the new constructors \AIC{δ}, \AIC{Δδ} and \AIC{δ•}, where the first two are analogues of \AIC{σ} and \AIC{Δσ}. The \AIC{δ•} constructor states that an ornamental description from a description \AV{R} and a (constructor) ornamental description from \AV{CD} can be composed to form an ornamental description from the composition (in the sense of the \AV{δ} type-former) of \AV{CD} with \AV{R}. The new commutativity squares in all the constructors both ensure the existence of functions such as 
\ExecuteMetaData[Ornament/OrnDesc]{ornForget-type}
like for the simpler ornaments, and that these ornamental descriptions indeed still induce ornaments.

The precise meaning of ornamental descriptions as descriptions is given by the conversion:
\ExecuteMetaData[Ornament/OrnDesc]{toDesc}
which makes use of the implicit \AV{If′} fields in the constructor ornaments to reconstruct the information on the target description.

But let us make the uses of \AD{OrnDesc} more clear by means of examples, where we make use of the variants of some ornaments specialized to binding or ignoring fields:
\ExecuteMetaData[Ornament/OrnDesc]{O-sigma-pm}
With these we can give the familiar ornamental description from \AD{List} to \AD{Vec}:
\ExecuteMetaData[Ornament/OrnDesc]{VecOD}
Using the new flexibility in \AIC{ρ}, we can now start from a description of binary numbers:
\ExecuteMetaData[Ornament/OrnDesc]{LeibnizD}
and describe random access lists as an ornament from binary numbers:
\ExecuteMetaData[Ornament/OrnDesc]{RandomOD}
Likewise, we can define phalanges as
\ExecuteMetaData[Ornament/OrnDesc]{PhalanxD}
By giving an ornament turning \AD{Three} into \AD{Digits}
\ExecuteMetaData[Ornament/OrnDesc]{DigitOD}
we can then use \AIC{δ•} to compose \AD{Digits} into phalanges, making binary fingertrees
\ExecuteMetaData[Ornament/OrnDesc]{FingerOD}
