
\begin{outline}
    :warning: The other way of parameters-indices
\end{outline}

\appendix

\section{Without K but with universe hierarchies}\label{app:withoutk}
See \cite{practgen} and the small blurb rewriting interpretations as datatypes.

\section{Big sigma}\label{app:large-sigma}

\section{gfold}\label{app:gfold}

\section{ornForget}\label{app:ornforget}

\section{Finger trees}
%\towrite{Can we prove that the time complexity of head is always less than cons, similarly for lookup and insert?}
\towrite{Cite everything}

We know that some datastructures can be presented as non-redundant numerical representations, for example lists by unary numbers, random access lists by binary numbers [calculating], and, skew binary heaps by skew binary numbers [progorn]. So far, some of these examples do support amortized constant time consing, but they have at best logarithmic time snocing. This is reflected by their number systems, for which either the natural successor operation is constant time, but can only act at the front, or is logarithmic time to begin with. We will instead look at more redundant number systems, and refine these step-by-step to produce structures similar to finger trees, giving us datastructures with fast access to both ends, and some of their properties for free.

\subsection{Binary finger trees}
For any numerical representation, we see that the operations on the represented datatype must be coherent with the corresponding operations on the number system. Hence, if we want to have constant time cons and snoc, we must first have constant time suc anc cus. By starting from a more symmetric number system, we can ensure good performance for both. Note that such a system is necessarily redundant, as there must be cases where neither suc nor cus recurses, yet both must clearly yield different values!

The obvious first candidate are symmetric unary numbers
\[ ... \]
but we can also see that subtraction has linear time
\[ ... \]
which gives a linear lower bound on the time complexity of lookup. By using a binary backbone for the numbers, we can also get a good lookup
\[ ... \]
However, this shape is still not ideal. We can see that for values like
\[ ... \]
The pair of suc
\[ ... \]
and pred
\[ ... \]
can compose to always take logarithmic time [fingertrees anew]. To avoid this, we can give the numbers bigger fingers
\[ ... \]
Now applying pred to the pathological case produces a value for which suc and pred both are constant time
\[ ... \]
More formally, we can use a three-colour scheme [purely functional/tarjan] to prove that any sequence of suc/cus pred/derp will amortize to constant time. Again, the interpretation of this number system is given by
\[ ... \]

To extract the datastructure, we must find a suitable index type for these numbers. Since the numbers are redundant, we can also get trees of different shapes with the same size, each having a different and incompatible index type. Still, the trees of a fixed shape are represented by functions, and the isomorphisms will still hold. The computation of the index type from the interpretation of the numbers is straightforward
\[ ... \]
\dots
This lets us define the tree type as
\[ ... \]
and gives definitions of the basic array operations
\[ ... \]

We can again trieify this to get a concrete datastructure
\[ ... \]
Consequently, since the representable arrays obey head (cons x xs) = x, the concrete arrays obey this as well.
On the other hand, the redundancy of the numbers, for which suc cus = cus suc does not hold, also causes cons snoc = snoc cons to not hold either; it seems that binary finger trees are not a very nice array type. We would like to quotient the redundancy of the number type away, which would also alleviate our issues related to indexing.

We can do this by imposing the following relation on the numbers
\[ ... \]
and turning the number system into a setoid. Eliminating from this setoid should then respect this relation, so for example, we have to prove
\[ ... \]

Similarly, we can keep the implementation of the binary finger trees, but put this under an appropriate relation as well
\[ ... \]
which ensures that the construction of the trees respects the relation on the number as well..
















\section{Heterogenization}
%The situation in which one wants to collect a variety of types is not uncommon, and is typically handled by tuples. However, if e.g., you are making a game in Haskell, you might feel the need to maintain a list of ``Drawables'', which may be of different types. Such a list would have to be a kind of ``heterogeneous list''. In Haskell, this can be resolved by using an existentially quantified list, which, informally speaking, can contain any type implementing a given constraint, but can only be inspected as if it contains the intersection of all types implementing this constraint. 

This ports directly to Agda, but becomes cumbersome quickly, and impractical if we want to be able to inspect the elements. The alternative is to split our heterogeneous list into two parts; one tracking the types, and one tracking the values. In practice, this means that we implement a heterogeneous list as a list of values indexed over a list of types. This approach and mainly its specialization to binary trees is investigated by Swierstra \cite{hetbin}.

We will demonstrate that we can express this ``lift a type over itself'' operation as an ornament. For this, we make a small adjustment to \AgdaDatatype{RDesc} to track a type parameter separately from the fields. Using this we define an ornament-computing function, which given a description computes an ornamental description on top of it:
\ExecuteMetaData[Tex/Heterogenize]{HetO}
This ornament relates the original unindexed type to a type indexed over it; we see that this ornament largely keeps all fields and structure identical, only performing the necessary bookkeeping in the index, and adding extra fields before parameters.

As an example, we adapt the list description
\ExecuteMetaData[Tex/Heterogenize]{List}
which is easily heterogenized to an \AgdaDatatype{HList}. In fact, \AgdaFunction{HetO} seems to act functorially; if we lift \AgdaDatatype{Maybe} like
\ExecuteMetaData[Tex/Heterogenize]{HMaybe}
then we can lift functions like \AgdaFunction{head} as
\ExecuteMetaData[Tex/Heterogenize]{hhead}



\section{More equivalences for less effort}\label{sec:userfriendly}
Noting that constructing equivalences directly or from isomorphisms as in \autoref{ssec:leibniz} can quickly become challenging when one of the sides is complicated, we work out a different approach making use of the initial semantics of W-types instead. We claim that the functions in the isomorphism of \autoref{ssec:leibniz} were partially forced, but this fact was unused there.

First, we explain that if we assume that one of the two sides of the equivalence is a fixpoint or initial algebra of a polynomial functor (that is, the \AgdaDatatype{μ} of a \AgdaDatatype{Desc′}), this simplifies giving an equivalence to showing that the other side is also initial.

We describe how we altered the original ornaments \cite{progorn} to ensure that \AgdaDatatype{μ} remains initial for its base functor in Cubical Agda, explaining why this fails otherwise, and how defining base functors as datatypes avoids this issue.

In a subsection focussing on the categorical point of view, we show how we can describe initial algebras (and truncate the appropriate parts) in such a way that the construction both applies to general types (rather than only sets), and still produces an equivalence at the end. We explain how this definition, like the usual definition, makes sure that a pair of initial objects always induces a pair of conversion functions, which automatically become inverses. Finally, we explain that we can escape our earlier truncation by appealing to the fact that ``being an equivalence'' is a proposition.

Next, we describe some theory, using which other types can be shown to be initial for a given algebra. This is compared to the construction in \autoref{ssec:leibniz}, observing that intuitively, initiality follows because the interpretation of the zero constructor is forced by the square defining algebra maps, and the other values are forced by repeatedly applying similar squares. This is clarified as an instance of recursion over a polynomial functor.

To characterize when this recursion is allowed, we define accessibility with respect to polynomial functors as a mutually recursive datatype as follows. This datatype is constructed using the fibers of the algebra map, defining accessibility of elements of these fibers by cases over the description of the algebra. Then we remark that this construction is an atypical instance of well-founded recursion, and define a type as well-founded for an algebra when all its elements are accessible.

We interpret well-foundedness as an upper bound on the size of a type, leading us to claim that injectivity of the algebra map gives a lower bound, which is sufficient to induce the isomorphism. We sketch the proof of the theorem, relating part of this construction to similar concepts in the formalization of well-founded recursion in the Standard Library. In particular, we prove an irrelevance and an unfolding lemma, which lets us show that the map into any other algebra induced by recursion is indeed an algebra map. By showing that it is also unique, we conclude initiality, and get the isomorphism as a corollary. 

The theorem is applied and demonstrated to the example of binary naturals. We remark that the construction of well-foundedness looks similar to view-patterns. After this, we conclude that this example takes more lines that the direct derivation in \autoref{ssec:leibniz}, but we argue that most of this code can likely be automated.

\towrite{Merge}

%% REPLACE X BY A?
The setup some approaches in earlier sections require makes them tedious or impractical to apply. In this section we will look at some ways how part of this problem could be alleviated through generics, or by alternative descriptions of concepts like equivalences through the lens of initial algebras. 

In later sections we will construct many more equivalences between more complicated types than before, so we will dive right into the latter. Reflecting upon \autoref{sec:leibniz}, we see that when one establishes an equivalence, most of the time is spent working out a series of tedious lemmas to show that the conversion functions are mutual inverses, which tend to be relatively easy to define. We take away two things from this; the first is that the conversion functions are perhaps too obvious, and the second is that we should really avoid talking about sections and retractions lest we incur tedium!\footnote{The latter perhaps less so, because it is useful to show a map to be monic.} We will reuse the machinery of Ko and Gibbons \cite{progorn} to illustrate how the definitions in \autoref{sec:leibniz} were actually forced for a large part.

First, we remark that \AgdaDatatype{μ} is internalization of the representation of simple\footnote{Of course, indexed datatypes are indexed W-types, mutually recursive datatypes are represented yet differently\dots} datatypes as W-types. Thus, we will assume that one of the sides of the equivalence is always represented as an initial algebra of a polynomial functor, and hence the \AgdaDatatype{μ} of a \AgdaDatatype{Desc′}.

\subsection{Well-founded monic algebras are initial}\label{ssec:wellfounded}
Unfortunately, the machinery developed by Ko and Gibbons \cite{progorn} relies on axiom K for a small but crucial part. To be precise, in a cubical setting, the type \AgdaDatatype{μ} as given stops being initial for its base functor! In this section, we will be working with a simplified and repaired version. Namely, we simplify \AgdaDatatype{Desc′} to 
\ExecuteMetaData[Extra/ProgOrn/Desc]{DescS}
To complete the definition of \AgdaDatatype{μ}
\ExecuteMetaData[Extra/ProgOrn/Desc]{mu}
we will need to implement \AgdaDatatype{Base}. We remark that in the original setup, the recursion of \AgdaFunction{mapFold} is a structural descent in \AgdaFunction{⟦ D' ⟧ (μ D)}. Because \AgdaFunction{⟦\_⟧} is a type computing function and not a datatype, this descent becomes invalid\footnote{Refer to the  \href{https://agda.readthedocs.io/en/latest/language/without-k.html\#restrictions-on-termination-checking}{without K} page.}, and \AgdaFunction{mapFold} fails the termination check. We resolve this by defining \AgdaDatatype{Base} as a datatype
\ExecuteMetaData[Extra/ProgOrn/Desc]{Base}
such that this descent is allowed by the termination checker without axiom K.\footnote{This has, again by the absence of axiom K, the consequence of pushing the universe levels up by one. However, this is not too troublesome, as equivalences can go between two levels, and indeed types are equivalent to their lifts.}

Recall that the \AgdaDatatype{Base} functors of descriptions are special polynomial functors, and the fixpoint of a base functor is its initial algebra. We are looking for sufficient conditions on $X$ to get the equivalence $e: X \cong \mu F$. Note that when $X \cong \mu F$, then there necessarily is an initial algebra $F X \to X$. Conversely, if the algebra $(X, f)$ is isomorphic to $(\mu F, \mathrm{con})$, then $X \cong \mu F$ would follow immediately, so it is equivalent to ask for the algebras to be isomorphic instead.

\begin{comment}
The situation so far is summarized by the diagram
% https://q.uiver.app/?q=WzAsMyxbMSwwLCJGXFxtdV9GIl0sWzEsMSwiXFxtdSBGIl0sWzAsMSwiWCJdLFswLDEsIlxcbWF0aHJte2Nvbn0iXSxbMiwxLCJlIiwyLHsic3R5bGUiOnsidGFpbCI6eyJuYW1lIjoiYXJyb3doZWFkIn0sImJvZHkiOnsibmFtZSI6ImRhc2hlZCJ9fX1dXQ==
\[\begin{tikzcd}[ampersand replacement=\&]
	\& {F\mu_F} \\
	X \& {\mu F}
	\arrow["{\mathrm{con}}", from=1-2, to=2-2]
	\arrow["e"', dashed, tail reversed, from=2-1, to=2-2]
\end{tikzcd}\]
\end{comment}
\subsubsection{Datatypes as initial algebras}
To characterize when such algebras are isomorphic, we reiterate some basic category theory, simultaneously rephrasing it in Agda terms.\footnote{We are not reusing a pre-existing category theory library for the simple reasons that it is not that much work to write out the machinery explicitly, and that such libraries tend to phrase initial objects in the correct way, which is too restrictive for us.}

Let $C$ be a category, and let $a, b, c$ be objects of $C$, so that in particular we have identity arrows $1_a : a \to a$ and for arrows $g : b \to c, f : a \to b$ composite arrows $gf : a \to c$ subject to associativity. In our case, $C$ is the category of types, with ordinary functions as arrows.

Recall that an endofunctor, which is simply a functor $F$ from $C$ to itself, assigns objects to objects and sends arrows to arrows
\ExecuteMetaData[Extra/Category]{RawFunctor}
These assignments are subject to the identity and composition laws
\ExecuteMetaData[Extra/Category]{Functor}
An $F$-algebra is just a pair of an object $a$ and an arrow $Fa \to a$
\ExecuteMetaData[Extra/Category]{Algebra}
Algebras themselves again form a category $C^F$. The arrows of $C^F$ are the arrows $f$ of $C$ such that the following square commutes% https://q.uiver.app/?q=WzAsNCxbMCwwLCJGYSJdLFsxLDAsIkZiIl0sWzAsMSwiYSJdLFsxLDEsImIiXSxbMiwzLCJmIiwyXSxbMCwyLCJVX2EiLDJdLFsxLDMsIlVfYiJdLFswLDEsIkZmIl1d
\[\begin{tikzcd}[ampersand replacement=\&]
	Fa \& Fb \\
	a \& b
	\arrow["f"', from=2-1, to=2-2]
	\arrow["{U_a}"', from=1-1, to=2-1]
	\arrow["{U_b}", from=1-2, to=2-2]
	\arrow["Ff", from=1-1, to=1-2]
\end{tikzcd}\]
So we define
\ExecuteMetaData[Extra/Category]{AlgSqr}
and
\ExecuteMetaData[Extra/Category]{AlgMap}
Note that we take the propositional truncation of the square, such that algebra maps with the same underlying morphism become propositionally equal
\ExecuteMetaData[Extra/Category]{AlgPath}
The identity and composition in $C^F$ arise directly from those of the underlying arrows in $C$.

Recall that an object $\emptyset$ is initial when for each other object $a$, there is a unique arrow $!: \emptyset \to a$. By reversing the proofs of initiality of \AgdaDatatype{μ} and the main result of this section, we obtain a slight variation upon the usual definition. Namely, unicity is often expressed as contractability of a type
\ExecuteMetaData[Tex/Snippets]{isContr}
Instead, we again use a truncation
\ExecuteMetaData[Extra/Category]{weakContr}
but note that this also, crucially, slightly stronger than connectedness. We define initiality for arbitrary relations
\ExecuteMetaData[Extra/Category]{Initial}
such that it closely resembles the definition of least element. Then, $A$ is an initial algebra when
\ExecuteMetaData[Extra/Category]{InitAlg}

By basic category theory (using the usual definition of initial objects), two initial objects $a$ and $b$ are always isomorphic;
namely, initiality guarantees that there are arrows $f : a \to b$ and $g : b \to a$, which by initiality must compose to the identities again.

Similarly, we get that
\ExecuteMetaData[Extra/Category]{InitAlg-equiv}
However, we only have the equalities from the isomorphism inside a propositional truncation. But fortunately, being an equivalence is a property, so we can eliminate from the truncations to get the wanted result.

%Note that even though we warned ourselves, we are still talking about sections and retractions to establish that $f$ is an equivalence! However, this result also makes sure we will not have to speak of them again.

\subsubsection{Accessibility}
As a consequence, we get that $X$ is isomorphic to $\mu D$ when $X$ is an initial algebra for the base functor of $D$; $\mu D$ is initial by its fold, and by induction on $\mu D$ using the squares of algebra maps. 

\begin{remark}
    We need (in general) not hope $\mu D$ is a strict initial object in the category of algebras. For a strict initial object, having a map $a \to \emptyset$ implies $a \cong \emptyset$. This is not the case here: strict initial objects satisfy $a \times \emptyset \cong \emptyset$, but for the $X \mapsto 1 + X$-algebras $\mathbb{N}$ and $2^\mathbb{N}$ clearly $2^\mathbb{N} \times \mathbb{N} \cong \mathbb{N}$ does not hold. On the other hand, the ``obvious'' sufficient condition to let $C^F$ have strict initial objects is that $F$ is a left adjoint, but then the carrier of the initial algebra is simply $\bot$.
\end{remark}

Looking back at \autoref{sec:leibniz}, we see that \AgdaDatatype{Leibniz} is an initial $F: X \mapsto 1 + X$ algebra because for any other algebra, the image of \AgdaFunction{0b} is fixed, and by \AgdaFunction{bsuc} all other values are determined by chasing around the square. Thus, we are looking for a similar structure on $f : FX \to X$ that supports recursion.

Clearly we will need something stronger than $FX \cong X$, as in general a functor can have many fixpoints. For this, we define what it means for an element $x$ to be accessible by $f$. This definition uses a mutually recursive datatype as follows:
We state that an element $x$ of $X$ is accessible when there is an accessible $y$ in its fiber over $f$
\ExecuteMetaData[Extra/Category/Poly]{Acc}
Accessibility of an element $x$ of \AgdaFunction{Base A E} is defined by cases on $E$; if $E$ is \AgdaFunction{ṿ n} and $x$ is a \AgdaFunction{Vec A n}, then $x$ is accessible if all its elements are; if $x$ is \AgdaFunction{σ S E'}, then $x$ is accessible if \AgdaFunction{snd x} is
\ExecuteMetaData[Extra/Category/Poly]{Acc'}
Consequently, $X$ is well-founded for an algebra when all its elements are accessible
\ExecuteMetaData[Extra/Category/Poly]{Wf}

We can see well-foundedness as an upper bound on the size of $X$, if it were larger than $\mu D$, some of its elements would inevitably get out of reach of an algebra. \textit{Now} having $FX \cong X$ also gives us a lower bound, but remark that having a well-founded injection $f: FX \to X$ is already sufficient, as accessibility gives a section of $f$, making it an iso. This leads us to claim
\begin{claim}\label{claim:wf-inj-init}
    If there is a mono $f : FX \to X$ and $X$ is well-founded for $f$, then $X$ is an initial $F$-algebra.
\end{claim}

\subsubsection{Proof sketch of \autoref{claim:wf-inj-init}}
Let us be on our way. Suppose $X$ is well-founded for the mono $f : FX \to X$. To show that $(X, f)$ is initial, let us take another algebra $(Y, g)$, and show that there is a unique arrow $(X, f) \to (Y, g)$.\todo[inline]{This section is about as digestable as a brick.}

By \AgdaDatatype{Acc}-recursion and because all $x$ are accessible, we can define a plain map into $Y$
\ExecuteMetaData[Extra/Category/WellFounded]{Wf-rec}
This construction is an instance of the concept of ``well-founded recursion''\footnote{This is formalized in the \href{https://agda.github.io/agda-stdlib/Induction.WellFounded.html}{standard-library} with many other examples.}, so we let ourselves be inspired by these methods. In particular, we prove an irrelevance lemma
\ExecuteMetaData[Extra/Category/WellFounded]{Wf-rec-irr}
which implies the unfolding lemma
\ExecuteMetaData[Extra/Category/WellFounded]{Wf-rec-unfold}
The unfolding lemma ensures that the map we defined by \AgdaFunction{Wf-rec} is a map of algebras. The proof that this map is unique proceeds analogously to that in the proof that $\mu D$ is initial, but here we instead use \AgdaDatatype{Acc}-recursion
\ExecuteMetaData[Extra/Category/WellFounded]{Wf+inj=Init}
Thus, we conclude that $X$ is initial. The main result is then a corollary of initiality of $X$ and the isomorphism of initial objects
\ExecuteMetaData[Extra/Category/WellFounded]{Wf+inj=mu}


\subsubsection{Example}
Let us redo the proof in \autoref{sec:leibniz}, now using this result. Recall the description of naturals \AgdaFunction{NatD}. To show that \AgdaFunction{Leibniz} is isomorphic to \AgdaFunction{Nat}, we will need a \AgdaFunction{NatD}-algebra and a proof of its well-foundedness. We define the algebra
\ExecuteMetaData[Tex/Leibniz2]{bsuc'}

For well-foundedness, we use something similar to view-patterns %[mcbride]
(the main difference being that we look through the entire structure, instead of a single layer)
\ExecuteMetaData[Tex/Leibniz2]{Peano-View}
where the mutually recursive proof of \AgdaFunction{view} is ``almost trivial''. Well-foundedness follows fairly immediately
\ExecuteMetaData[Tex/Leibniz2]{Wf-bsuc}

Injectivity of \AgdaFunction{bsuc\_1} happens to be harder to prove from retractions than directly, so we prove it directly, from which the wanted statement follows
\ExecuteMetaData[Tex/Leibniz2]{L-is-mu-N}

Note that in this case it took us more code to prove the same statement! However, we stress that the code that we did write became more forced, and might be more amenable to automation.

