\documentclass{article}

\usepackage[style=alphabetic]{biblatex}
\addbibresource{refs.bib}

\usepackage{comment}

\setlength{\marginparwidth}{2cm} % remove when done

\usepackage{todonotes}
\usepackage{xcolor}
\usepackage[hidelinks]{hyperref}

%\hypersetup{
%    colorlinks=true,
%    linkcolor=cyan
%    }

\usepackage{catchfilebetweentags}
\usepackage{quiver} 
\usepackage{tabularx}
\usepackage{adjustbox}
\usepackage{longtable}
\usepackage{amsthm}
\usepackage{amsmath}


\theoremstyle{plain}% default
\newtheorem{theorem}{Theorem}[section]
\newtheorem{lemma}[theorem]{Lemma}
\newtheorem{prop}[theorem]{Proposition}
\newtheorem*{cor}{Corollary}

\theoremstyle{definition}
\newtheorem{defn}{Definition}[section]
\newtheorem{remark}{Remark}[section]
\newtheorem{claim}{Claim}[section]

\renewcommand{\partautorefname}{Part}%
\renewcommand{\sectionautorefname}{Section}%
\renewcommand{\subsectionautorefname}{Subsection}%

\providecommand{\theoremautorefname}{Theorem}%
\providecommand{\lemmaautorefname}{Lemma}%
\providecommand{\propautorefname}{Proposition}%
\providecommand{\corautorefname}{Corollary}%

\providecommand{\defnautorefname}{Definition}%
\providecommand{\remarkautorefname}{Remark}%
\providecommand{\claimautorefname}{Claim}%


\usepackage[links]{agda}
\AgdaNoSpaceAroundCode{}

% from: https://agda.readthedocs.io/en/v2.6.3/_downloads/59877ce886494c991a213f09e29b712c/article-luaxelatex-different-fonts.lagda.tex
\usepackage{fontspec}

\usepackage{luaotfload}

\directlua{luaotfload.add_fallback
  ("myfallback",
    { "JuliaMono:style=Regular;"
    , "NotoSansMono:style=Regular;"
    , "NotoSansMath:style=Regular;"
    , "Segoe UI Emoji:mode=harf;"
    }
  )}
\defaultfontfeatures{RawFeature={fallback=myfallback}}

\setmainfont{Latin Modern Roman}

\newfontfamily{\AgdaSerifFont}{Linux Libertine O}
\newfontfamily{\AgdaSansSerifFont}{Linux Biolinum O}
\newfontfamily{\AgdaTypewriterFont}{inconsolata}
\renewcommand{\AgdaFontStyle}[1]{{\AgdaSansSerifFont{}#1}}
\renewcommand{\AgdaKeywordFontStyle}[1]{{\AgdaSansSerifFont{}#1}}
\renewcommand{\AgdaStringFontStyle}[1]{{\AgdaTypewriterFont{}#1}}
\renewcommand{\AgdaCommentFontStyle}[1]{{\AgdaTypewriterFont{}#1}}
\renewcommand{\AgdaBoundFontStyle}[1]{\textit{\AgdaSerifFont{}#1}}

\newcommand{\towrite}[1]{\par\textcolor{blue}{Write here about: ``#1''}\par}
\newcommand{\toremove}[1]{\textcolor{red}{This is going to be (re)moved: ``#1''}}


% macros
\newcommand{\investigate}[1]{\par\vspace{1\baselineskip}\textcolor{gray}{\textit{#1}}\vspace{1\baselineskip}\par}

% symbols
\newcommand{\bN}{\AgdaDatatype{ℕ}}
\newcommand{\bL}{\AgdaDatatype{Leibniz}}


\title{Ornaments and Proof Transport applied to Numerical Representations}
\author{Samuel Klumpers\\6057314}

% previous (provisional) titles:
% Restoring (part of) the friendship between recursion schemes and without-K (provisional)
% The gentle art of smashing things to bits and pieces
% Running in circles in Agda

\begin{document}
\maketitle

%This document is generated from a literate agda file!
%\newpage

\begin{abstract}
\towrite{Provisional}
This thesis explains the concepts of the structure identity principle, numerical representations, and ornaments, and aims to combine these to simplify the presentation and verification of finger trees, demonstrating the generalizability and improved compactness and security of the resulting code. Consequently, we also investigate to which extent ornaments, and other generic programs relying on axiom K, remain applicable in the cubical setting required for the structure identity principle.
\end{abstract}

\tableofcontents

\begin{comment}
\section*{Outline}
In this document I propose a master thesis project, in which I will investigate and attempt to counter the obstacles one can encounter when replacing one datastructure with a more complicated one, focussing on how we can retain or reuse properties and proofs from the simpler datastructure.

We introduce the topics of proof transport, ornamentation, and numerical representation by presenting problems and explaining how these topics can be applied to them. Following up on these problems, we make our research question more precise, list some related questions, and propose methods by which we may answer these questions. We then overview related research and existing theory, highlighting the problems they originally solved, and how we may apply them to our research question. Next, we summarize preliminary work done for this project. Finally, we propose a planning, explaining more precisely how we split the research question into parts and what subproblems we intend to solve working towards answering our research question.
\end{comment}


\section{Introduction}\label{sec:intro}
\section{Introduction}\label{sec:intro}
Agda \cite{agda} is a functional programming language and a proof assistant, taking inspiration from languages like Haskell and other proof assistants like Coq. We can write programs like we would in Haskell, and then express and prove their properties all inside Agda. This allows us to demonstrate the correctness of programs by formal proof%, which is sound,
rather than by testing%, which is complete
; of course, producing an often tedious proof typically demands more effort than covering the relevant code with testcases.

In this thesis, we will explore some methods of proving properties of our programs, focussing on the problems or inconveniences that may arise, and how to deal with them. Let us sketch some problems and their remedies to get an idea of what awaits us, before we dive into the nitty-gritty details.

First, merely adapting a program to Agda may already require changes to the datatypes used in it; for example, if a program manipulating a \AgdaDatatype{List} uses the unsafe \AgdaFunction{head} function, then one is forced to replace the \AgdaDatatype{List} by a datatype that ensures non-emptyness, such as a \AgdaDatatype{NonEmpty} list or a length-aware vector \AgdaDatatype{Vec}. On the other hand, there might be sections of a program where the concrete length is not relevant for correctness and only gets in the way. As a result, one might find themselves duplicating common functions like concatenation \AgdaFunction{\_++\_} and only altering their signatures.

Clearly we would not be writing this if there was no way out, which there is! Often, the ``new'' datatype (\AgdaDatatype{Vec}) is simply a variation on the old datatype (\AgdaDatatype{List}) making small adjustments to the existing constructors; in this case, we decorate the nil and cons constructors with natural numbers representing the length. This kind of modification of types falls in the framework of ornamentation as described by Ko and Gibbons \cite{progorn}; if two types are reified to their \textit{descriptions}, then \textit{ornaments} express whether the types are ``similar'' by acting as a recipe to produce one type from the other. By restricting the operations to the copying of corresponding parts, and the introduction of fields or dropping of indices, the existence of such an ornament ensures that the types have the same recursive structure.

\towrite{Something about patches.}

\towrite{For each invariant a new datatype? Still ornaments}

Now that we know we can decently satisfyingly organize similar datatypes, it is time to look at dissimilar datatypes. It is certainly not foolish to prototype a program using simpler types or implementations, and only replace these with more performant alternatives in critical places; knowing that this is eventually going to happen, one might as well prepare for it. While this may quickly turn into a refactoring nightmare in the general case, we can hope for a more satisfying transition if we restrict our attention to a more narrow scope. As an example, we might start programming something with a \AgdaDatatype{List}, but replace this with a \AgdaDatatype{Tree} if we notice that the program spends most of its time in \AgdaFunction{lookup} operations. We are of course reimplementing the operations on \AgdaDatatype{Tree} utilizing their binary nature to gain a speedup, but it also seems as though we are about to double the number of necessary proofs; however, we have two ways to avoid this problem. 

We will look at the more specific solution first. This solution is guided by the realization that \AgdaDatatype{List} and \AgdaDatatype{Tree}, like most other containers, still have something in common even if their recursive structure is very different. That is, both resemble a number system, and, Okasaki \cite{purelyfunctional} notes that this resemblance to number systems is ``surprisingly common''. In the case of lists and Braun trees\footnote{Braun trees are a kind of binary tree, of which its shape is determined by its size.}, one can present both by deriving them from unary and binary numbers respectively, as is made formal by Hinze and Swierstra in \cite{calcdata}. One can then apply this \textit{numerical representation} to simplify or make trivial the proofs of the properties we hesitated to duplicate before.

\towrite{If we instead hide our datatypes behind interfaces, we can use proof transport as an alternative.}


\section{\toremove{Introduction (old)}}
%The dependently typed functional programming language Agda \cite{agda} can, when restricted to its reasonable parts, be translated into readable and safe Haskell \cite{agda2hs}. However, the intrinsic safety of languages like Agda can also lead to code duplication by encouraging the use of multiple variants of the same datatype. As an example, the coverage check forces the \AgdaFunction{head} function on \AgdaDatatype{List} to return a \AgdaDatatype{Maybe}. This \AgdaDatatype{Maybe} can be avoided by moving to the length-indexed list type \AgdaDatatype{Vec}, at the cost of duplicating functions like \AgdaFunction{\_++\_}, which we need at both types.

Something similar happens when replacing an implementation with a more efficient one. For example, when implementing binary trees as a more efficient alternative to lists, the proofs of the same properties will differ between list and tree, and tend to be more difficult for the latter. Switching between implementations of an interface not only duplicates code, but also (and sometimes more than) doubles the effort required to verify both.

%There is plenty of prior work dealing with problems like these. The work in \cite{orntrans} and \cite{progorn} provides the means to relate similar datatypes, such as lists and vectors, using the mechanism of ornamentation, letting us organize variants of the same datatype in a rigid framework.  %This leads them to define the concept of patches, which can aid us when defining \AgdaFunction{\_++\_} for the second time by forcing the new version to be coherent.
%In fact, the algebraic nature of ornaments yields the definition of the vector type for free, provided we relate lists to natural numbers \cite{algorn}. %Such constructions rely heavily on descriptions of datastructures and often come with limitations in their expressiveness. These descriptions in turn impose additional ballast on the programmer, leading us to investigate reflection like in \cite{practgen} as a means to bring datatypes and descriptions closer when possible.

Other work like \cite{calcdata} simplifies the proofs relating to certain containers directly, formally executing the way of though of numerical representations as noted in \cite{purelyfunctional}.
%From another point of view, lists and trees are not so different at all, provided we look at them through the interface of one-sided flexible arrays; this idea noted in \cite{purelyfunctional} and formalized in \cite{calcdata} where both are shown to be instances of numerical representations by calculating them from a numeral system. 

When two types are isomorphic and equivalent under an interface, proofs of properties of these implementations should be interconvertible. By using structured equivalences and univalence, \cite{iri} characterizes equivalences under interfaces.
%While this is achievable through meta-programming, substituting conversions to and from into the proof terms, this is internally expressible in Cubical Agda.

%We can liken the situation to movement on a plane, where ornamentation moves us vertically by modifying constructors or indices, and structured equivalences move us horizontally to and from equivalent but more equivalent implementations. In this paper, we will investigate a variety of means of moving around structures and proofs, and ways to make this more efficient or less intrusive.

In \autoref{sec:leibniz}, we will follow \cite{iri}, and look at how proofs on unary naturals can be transported to the binary naturals. Then in \autoref{sec:numrep} we recall how numeral systems in particular induce container types in \cite{calcdata}, which we attempt to reformulate in the language of ornaments in \autoref{ssec:ornaments}, using the framework of \cite{progorn}. In \autoref{sec:userfriendly} we investigate how we can make the earlier methods more easily accessible to the user, and, ourselves, when we give a description of finger trees in \autoref{sec:fingertrees}.



\subsection{The Problem}

\subsection{Contributions}


\section{Background}
\section{Agda}\label{sec:background-agda}
We formalize our work in the programming language Agda \cite{agda}. While we will only occasionally reference Haskell, those more familiar with Haskell might understand (the reasonable part of) Agda as the subset of total Haskell programs \cite{agda2hs}.

Agda is a total functional programming language with dependent types. Here, totality means that functions of a given type always terminate in a value of that type, ruling out non-terminating (and not obviously terminating) programs. Using dependent types we can use Agda as a proof assistant, allowing us to state and prove theorems about our datastructures and programs. 

In this section, we will explain and highlight some parts of Agda which we use in the later sections. 
Many of the types we use in this section are also described and explained in most Agda tutorials (\cite{ulftutorial}, \cite{plfa}, etc.), and can be imported from the standard library \cite{agdastdlib}.

%Note that we use \texttt{--type-in-type} to keep the explanations more readable. 
%and \texttt{--with-K}
% Even though the former makes Agda inconsistent, and the latter is not strictly necessary, we know that our work can be ported to a setting with neither option \autoref{app:withoutK}.


\section{Data in Agda}\label{sec:background-data}
At the level of generalized algebraic datatypes Agda is close to Haskell. In both languages, one can define objects using data declarations, and interact with them using function declarations. For example, we can define the type of \emph{booleans}:
\ExecuteMetaData[Tex/Background]{Bool}
The constructors of this type state that we can make values of \AD{Bool} in exactly two ways: \AIC{false} and \AIC{true}. We can then define functions on \AD{Bool} by pattern matching. As an example, we can define the conditional operator as
\ExecuteMetaData[Tex/Background]{conditional}
When \emph{pattern matching}, the coverage checker ensures we define the function on all cases of the type matched on, and thus the function is completely defined. % mark: shuffle

We can also define a type representing the natural numbers
\ExecuteMetaData[Tex/Background]{Nat}
Here, \bN{} always has a \AF{zero} element, and for each element $n$ the constructor \AIC{suc} expresses that there is also an element representing $n + 1$. Hence, \bN{} represents the \textit{naturals} by encoding the existential axioms of the Peano axioms. By pattern matching and recursion on \bN{}, we define the less-than operator:
\ExecuteMetaData[Tex/Background]{lt}
One of the cases contains a recursive instance of \bN{}, so termination checker also verifies that this recursion indeed terminates, ensuring that we still define \AV{n}\ \AF{<?} \AV{m} for all possible combinations of \AV{n} and \AV{m}. %Essentially, the coverage and termination checker make sure that any valid definition by pattern matching corresponds to a valid proof by cases and induction.
In this case the recursion is valid, since both arguments decrease before the recursive call, meaning that at some point \AV{n} or \AV{m} hits \AIC{zero} and the recursion terminates.

Like in Haskell, we can \emph{parametrize} a datatype over other types to make \emph{polymorphic} type, which we can use to define lists of values for all types:
\ExecuteMetaData[Tex/Background]{List} 
A list of \AV{A} can either be empty \AIC{[]}, or contain an element of \AV{A} and another list via \AIC{\_∷\_}. In other words, \AD{List} is a type of \emph{finite sequences} in \AV{A} (in the sense of sequences as an abstract type \cite{purelyfunctional}).

Using polymorphic functions, we can manipulate and inspect lists by inserting or extracting elements. For example, we can define a function to look up the value at some position \AV{n} in a list
\ExecuteMetaData[Tex/Background]{lookup-list}
However, this function \emph{partial}, as we are relying on the type
\ExecuteMetaData[Tex/Background]{Maybe}
to handle the case where the position falls outside the list and we cannot return an element. 
If we know the length of the list \AV{xs}, then we also know for which positions \AF{lookup} will succeed, and for which it will not. We define 
\ExecuteMetaData[Tex/Background]{length}
so that we can test whether the position \AV{n} lies inside the list by checking \AV{n}\ \AF{<?}\ \AF{length}\ \AV{xs}. If we declare \AF{lookup} as a dependent function consuming a proof of \AV{n}\ \AF{<?}\ \AF{length}\ \AV{xs}, then \AF{lookup} always succeeds. However, this actually only moves the burden of checking whether the output was \AIC{nothing} afterwards to proving that \AV{n}\ \AF{<?}\ \AF{length}\ \AV{xs} beforehand.

We can avoid both by defining an \emph{indexed type} representing numbers below an upper bound
\ExecuteMetaData[Tex/Background]{Fin}
Like parameters, indices add a variable to the context of a datatype, but unlike parameters, indices can influence the availability of constructors. The type \AD{Fin} is defined such that a variable of type \AD{Fin}\ \AV{n} represents a number less than \AV{n}. Since both constructors \AIC{zero} and \AIC{suc} dictate that the index is the \AIC{suc} of some natural \AV{n}, we see that \AD{Fin}\ \AIC{zero} has no values. On the other hand, \AIC{suc} gives a value of \AD{Fin}\ (\AIC{suc}\ \AV{n}) for each value of \AD{Fin}\ \AV{n}, and \AIC{zero} gives exactly one additional value of \AD{Fin}\ (\AIC{suc}\ \AV{n}) for each \AV{n}. By induction (externally), we find that \AD{Fin}\ \AV{n} has exactly \AV{n} closed terms, each representing a number less than \AV{n}.

To complement \AD{Fin}, we define another indexed type representing lists of a known length, also known as vectors:
\ExecuteMetaData[Tex/Background]{Vec}
The \AIC{[]} constructor of this type produces the only term of type \AD{Vec}\ \AV{A}\ \AIC{zero}. The \AIC{\_∷\_} constructor ensures that a \AD{Vec}\ \AV{A}\ (\AIC{suc}\ \AV{n}) always consists of an element of \AV{A} and a \AD{Vec}\ \AV{A}\ \AV{n}. By induction, we find that a \AD{Vec}\ \AV{A}\ \AV{n} contains exactly \AV{n} elements of \AV{A}. Thus, we conclude that \AD{Fin}\ \AV{n} is exactly the type of positions in a \AD{Vec}\ \AV{A}\ \AV{n}. In comparison to \AD{List}, we can say that \AD{Vec} is a type of arrays (in the sense of arrays as the abstract type of sequences of a fixed length). Furthermore, knowing the index of a term \AV{xs} of type \AV{Vec}\ \AV{A}\ \AV{n} uniquely determines the the constructor it was formed by. Namely, if \AV{n} is \AIC{zero}, then \AV{xs} is \AIC{[]}, and if \AV{n} is \AIC{suc} of \AV{m}, then \AV{xs} is formed by \AIC{\_∷\_}. 

Using this, we define a variant of \AF{lookup} for \AD{Fin} and \AD{Vec}, taking a vector of length \AV{n} and a position below \AV{n}:
\ExecuteMetaData[Tex/Background]{lookup}
The case in which we would return \AIC{nothing} for lists, which is when \AV{xs} is \AIC{[]}, is omitted. This happens because \AV{x} of type \AD{Fin}\ \AV{n} is either \AIC{zero} or \AIC{suc}\ \AV{i}, and both cases imply that \AV{n} is \AIC{suc}\ \AV{m} for some \AV{m}. As we saw above, a \AD{Vec}\ \AV{A}\ (\AIC{suc} \AV{m}) is always formed by \AIC{\_∷\_}, making the case in which \AV{xs} is \AIC{[]} impossible. Consequently, lookup always succeeds for vectors,
% demonstrating that vectors are correct-by-construction. 
however, this does not yet prove that \AF{lookup} necessarily returns the right element, we will need some more logic to verify this.

\section{Proving in Agda}\label{sec:background-proving}
To describe equality of terms we define a new type
\ExecuteMetaData[Tex/Background]{equiv}
If we have a value \AV{x} of \AV{a}\ \AD{≡}\ \AV{b}, then, as the only constructor of \AD{\_≡\_} is \AIC{refl}, we must have that \AV{a} is equal to \AV{b}. We can use this type to describe the behaviour of functions like \AF{lookup}: If we insert elements into a vector with
\ExecuteMetaData[Tex/Background]{insert}
we can express the correctness of \AF{lookup} as
\ExecuteMetaData[Tex/Background]{lookup-insert-type}
stating that we expect to find an element where we insert it.

% When we use pattern matching in a function, the coverage and termination checker ensure that the resulting function is total and defined by well-founded recursion\cite{?}. If we are proving some statement by constructing a function as a proof, this means that we can interpret a function definition by (dependent) pattern matching and well-founded recursion as a proof by well-founded induction\cite{?}.

%So, to 
To prove the statement, we proceed as when defining any other function. 
By simultaneous induction on the position and vector, we prove
\ExecuteMetaData[Tex/Background]{lookup-insert}
In the first two cases, where we \AF{lookup} the first position, \AF{insert}\ \AV{xs}\ \AIC{zero}\ \AV{y} simplifies to \AV{y}\ \AF{∷}\ \AF{xs}, so the lookup immediately returns \AV{y} as wanted. In the last case, we have to prove that \AF{lookup} is correct for \AV{x}\ \AF{∷}\ \AF{xs}, so we use that the \AF{lookup} ignores the term \AV{x} and we appeal to the correctness of \AF{lookup} on the smaller list \AV{xs} to complete the proof.

Like \AD{\_≡\_}, we can encode many other logical operations into datatypes, which establishes a correspondence between types and formulas, known as the Curry-Howard isomorphism. For example, we can encode disjunctions (the logical `or' operation) as
\ExecuteMetaData[Tex/Background]{uplus}

The other components of the isomorphism are as follows. Conjunction (logical `and') can be represented by\footnote{We use a record here, rather than a datatype with a constructor \AV{A → B →}\ \AV{A}\ \AD{×}\ \AV{B}. The advantage of using a record is that this directly gives us projections like \ARF{fst}\ \AV{:}\ \AV{A}\ \AD{×}\ \AV{B}\ \AV{→ A}, and lets us use eta equality, making $(a, b) = (c , d) \iff a = c \land b = d$ holds automatically.}
\ExecuteMetaData[Tex/Background]{product}
True and false are respectively represented by
\ExecuteMetaData[Tex/Background]{true}
so that always \AIC{tt}\ \AV{:}\ \AD{⊤}, and 
\ExecuteMetaData[Tex/Background]{false}
The body of \AD{⊥} is not accidentally left out: because \AD{⊥} has no constructors, there is no proof of false\footnote{If we did not use \AV{--type-in-type}, and even in that case I can only hope.}.

Because we identify function types with logical implications, we can also define the negation of a formula \AV{A} as ``\AV{A} implies false'':
\ExecuteMetaData[Tex/Background]{not}
The logical quantifiers $\forall$ and $\exists$ act on formulas with a free variable in a specific domain of discourse. We represent closed formulas by types, so we can represent a formula with a free variable of type \AV{A} by a function values of \AV{A} to types \AV{A}\ \AV{→}\ \AD{Type}, also known as a predicate. The universal quantifier $\forall a P(a)$ is true when for all $a$ the formula $P(a)$ is true, so we represent the universal quantification of a predicate \AV{P} as a dependent function type \AV{(a : A) → P a}, producing for each \AV{a} of type \AV{A} a proof of \AV{P}\ \AV{a}. The existential quantifier $\exists a P(a)$ is true when there is some $a$ such that $P(a)$ is true, so we represent the existential quantification as
\ExecuteMetaData[Tex/Background]{exists}
so that we have \AD{Σ}\ \AV{A}\ \AV{P} iff we have an element \AV{fst} of \AV{A} and a proof \AV{snd} of \AV{P}\ \AV{a}. To avoid the need for lambda abstractions in existentials, we define the syntax
\ExecuteMetaData[Tex/Background]{sigma-syntax}
letting us write \AD{Σ[}\ \AV{a}\ \AD{∈}\ \AV{A}\ \AD{]}\ \AV{P a} for $\exists a P(a)$.

\section{Descriptions}\label{sec:background-descriptions}
In the previous sections we completed a quadruple of types (\bN{}, \AD{List}, \AD{Vec}, \AD{Fin}), 
%, even computing the latter two from \bN{}.
which have nice interactions (\AF{length}, \AF{lookup}). Similar to the type of \AF{length}\ \AV{:}\ \AD{List}\ \AV{A}\ \AV{→}\ \bN{}, we can define
\ExecuteMetaData[Tex/Background]{toList}
converting vectors back to lists. In the other direction, we can also promote a list to a vector by recomputing its index:
\ExecuteMetaData[Tex/Background]{toVec}
We claim that is not a coincidence, but rather happens because \bN{}, \AD{List}, and \AD{Vec} have the same ``shape''.

But what is the shape of a datatype? In this section, we will explain a framework of datatype descriptions and ornaments, allowing us to describe the shapes of datatypes and use these for generic programming \cite{ulftutorial, genericsamm, effectfully, practgen}. Recall that while polymorphism allows us to write one program for many types at once, those programs act parametrically \cite{reynolds1983types, wadlerfree}: polymorphic functions must work for all types, thus they cannot inspect values of their type argument. Generic programs, by design, do use the structure of a datatype, allowing for more complex functions that do inspect values\footnote{Think of JSON encoding types with encodable fields \cite{truesop}, or deriving functor instances for a broad class of types \cite{haskellderiving}.}.

Using datatype descriptions we can then relate \bN{}, \AD{List} and \AD{Vec}, explaining how \AF{length} and \AF{toList} are instances of a generic construction. Let us walk through some ways of defining descriptions. We will start from simpler descriptions, building our way up to more general types, until we reach a framework in which we can describe \bN{}, \AD{List}, \AD{Vec} and \AD{Fin}. 
%, which, as a bonus, gives some insight into the meaning of datatypes.


\subsection{Finite types}\label{ssec:background-fin}
A datatype description, which are datatypes of which each value again represents a datatype, consist of two components. Namely, a type of descriptions \AV{U}, also referred to as codes, and an interpretation \AV{U}\ \AV{→}\ \AD{Type}, decoding descriptions to the represented types. In the terminology of Martin-L{\"{o}}f type theory (MLTT)\cite{levitation}, %\todo{No citation for MLTT? Agda is a rather loose extension, none of the original papers really match.}
where types of types like \AD{Type} are called universes, we can think of a type of descriptions as an internal universe.

As a start, we define a basic universe with two codes \AIC{𝟘} and \AIC{𝟙}, respectively representing the types \AD{⊥} and \AD{⊤}, and the requirement that the universe is closed under sums and products:
\ExecuteMetaData[Tex/Background]{U-fin}
The meaning of the codes in this universe is then assigned by the interpretation
\ExecuteMetaData[Tex/Background]{int-fin}
which indeed sends \AIC{𝟘} to \AD{⊥}, \AIC{𝟙} to \AD{⊤}, sums to sums and products to products\footnote{One might recognize that \AF{⟦\_⟧fin} is a morphism between the rings (\AD{U-fin}, \AIC{⊕}, \AIC{⊗}) and (\AD{Type}, \AD{⊎}, \AD{×}). Similarly, \AD{Fin} also gives a ring morphism from \bN{} with \AF{+} and \AF{×} to \AD{Type}, and in fact \AF{⟦\_⟧fin} factors through \AD{Fin} via the map sending the expressions in \AD{U-fin} to their value in \bN{}.}.

In this universe, we can encode the type of booleans simply as 
\ExecuteMetaData[Tex/Background]{BoolD}
The types \AIC{𝟘} and \AIC{𝟙} are finite, and sums and products of finite types are also finite, which is why we call \AD{U-fin} the universe of finite types. Consequently, the type of naturals \bN{} cannot fit in \AD{U-fin}.

\subsection{Recursive types}\label{ssec:background-rec}
To accommodate \bN{}, we need to be able to express recursive types. By adding a code \AIC{ρ} to \AD{U-fin} representing recursive type occurrences, we can express those types: 
\ExecuteMetaData[Tex/Background]{U-rec}
However, the interpretation cannot be defined like in the previous example: when interpreting \AIC{𝟙}\ \AIC{⊕}\ \AIC{ρ}, we need to know that the whole type was \AIC{𝟙}\ \AIC{⊕}\ \AIC{ρ} while processing \AIC{ρ}. As a consequence, we have to split the interpretation in two phases. First, we interpret the descriptions into polynomial functors
\ExecuteMetaData[Tex/Background]{int-rec}
Then, by viewing such a functor as a type with a free type variable, the functor can model a recursive type by setting the variable to the type itself:
\ExecuteMetaData[Tex/Background]{mu-rec}
Recall the definition of \bN{}, which can be read as the declaration that \AD{ℕ} is a fixpoint: \AD{ℕ}\ \AD{≡}\ \AV{F}\ \AD{ℕ} for \AV{F X = ⊤ ⊎ X}. This makes representing \bN{} as simple as:
\ExecuteMetaData[Tex/Background]{NatD}

\subsection{Sums of products}\label{ssec:background-sop}
A downside of \AD{U-rho} is that the definitions of types do not mirror their equivalent definitions in user-written Agda. We can define a similar universe using that polynomials can always be canonically written as sums of products. For this, we split the descriptions into a stage in which we can form sums, on top of a stage where we can form products.
\ExecuteMetaData[Tex/Background]{U-sop}
When doing this, we can also let the left-hand side of a product be any type, allowing us to represent ordinary fields:
\ExecuteMetaData[Tex/Background]{Con-sop}
The interpretation of this universe, while analogous to the one in the previous section, is also split into two parts:
\ExecuteMetaData[Tex/Background]{int-sop}
In this universe, we can define the type of lists as a description quantified over a type:
\ExecuteMetaData[Tex/Background]{ListD-bad}
Using this universe requires us to split functions on descriptions into multiple parts, but makes interconversion between representations and concrete types straightforward.

\subsection{Parametrized types}\label{ssec:background-par}
The encoding of fields in \AD{U-sop} makes the descriptions large in the following sense: by letting \AV{S} in \AIC{σ} be an infinite type, we can get a description referencing infinitely many other descriptions. As a consequence, we cannot inspect an arbitrary description in its entirety. We will introduce parameters in such a way that we recover the finiteness of descriptions as a bonus.

In the last section, we saw that we could define the parametrized type \AD{List} by quantifying over a type. However, in some cases, we will want to be able to inspect or modify the parameters belonging to a type. % mark: why
%footnote{For example, deriving Traversable for parametrized types as functions would not be possible (without macros), as one could not decide whether the signature of a type in a field is compatible.}
To represent the parameters of a type, we will need a new gadget.

In a naive attempt, we can represent the parameters of a type as \AD{List}\ \AD{Type}. However, this cannot represent many useful types, of which the parameters depend on each other. For example, in the existential quantifier \AD{Σ\_}, the type \AV{A}\ \AV{→}\ \AD{Type} of second parameter \AV{B} references back to the first parameter \AV{A}.

In a general parametrized type, parameters can refer to the values of all preceding parameters. The parameters of a type are thus a sequence of types depending on each other, which we call telescopes \cite{practgen, sijsling, telescopes} (also known as contexts in MLTT). We define telescopes using induction-recursion:
\ExecuteMetaData[Tex/Background]{Tel-simple}
A telescope can either be empty, or be formed from a telescope and a type in the context of that telescope. Here, we used the meaning of a telescope \AF{⟦\_⟧tel} to define types in the context of a telescope. This meaning represents the valid assignment of values to parameters:
\ExecuteMetaData[Tex/Background]{int-simple}
interpreting a telescope into the dependent product of all the parameter types.

This definition of telescopes would let us write down the type of \AD{Σ}:
\ExecuteMetaData[Tex/Background]{sigma-tel}
but is not sufficient to define \AD{Σ}, as we need to be able to bind a value \AV{a} of \AV{A} and reference it in the field \AV{P}\ \AV{a}. By quantifying telescopes over a type \cite{practgen}, we can represent bound arguments using almost the same setup:
\ExecuteMetaData[Tex/Background]{Tel-type}
A \AD{Tel}\ \AV{P} then represents a telescope for each value of \AV{P}, which we can view as a telescope in the context of \AV{P}. For readability, we redefine values in the context of a telescope as:
\ExecuteMetaData[Tex/Background]{entails}
so we can define telescopes and their interpretations as:
\ExecuteMetaData[Tex/Background]{Tel-def}
By setting \AV{P}\ \AV{=}\ \AD{⊤}, we recover the previous definition of parameter-telescopes. We can then define an extension of a telescope as a telescope in the context of a parameter telescope:
\ExecuteMetaData[Tex/Background]{ExTel}
representing a telescope of variables over the fixed parameter-telescope \AV{Γ}, which can be extended independently of \AV{Γ}. Extensions can be interpreted by interpreting the variable part given the interpretation of the parameter part:
\ExecuteMetaData[Tex/Background]{int-ExTel}
We will name maps \AV{Δ → Γ} of telescopes \AF{Cxf}\ \AV{Δ}\ \AV{Γ}. Given such a map \AV{g}, name maps \AV{W → V} between extensions \AF{Vxf}\ \AV{g}\ \AV{W}\ \AV{V}:
\ExecuteMetaData[Tex/Background]{tele-helpers} %mark: map-var
We also defined two functions we will use extensively later: \AF{var→par} states that a map of extensions extend to a map of the whole telescope, and \AF{Vxf-▷} lets us extend a map of extensions by acting as the identity on a new variable. 

In the descriptions directly relay the parameter telescope to the constructors, resetting the variable telescope to \AIC{∅} for each constructor:
\ExecuteMetaData[Tex/Background]{U-par}
Of the constructors we only modify the \AIC{σ} to request a type \AV{S} in the context of \AV{V}, and to extend the context for the subsequent fields by \AV{S}:
\ExecuteMetaData[Tex/Background]{Con-par}
Replacing the function \AV{S →}\ \AD{U-sop} by \AD{Con-par}\ (\AV{V}\ \AIC{▷}\ \AV{S}) allows us to bind the value of \AV{S} while avoiding the higher order argument. The interpretation of the universe is then:
\ExecuteMetaData[Tex/Background]{int-par}
In particular, we provide \AV{X} the parameters and variables in the \AIC{σ} case, and extend context by \AV{s} before passing to the rest of the interpretation.

In this universe, we can describe lists using a one-type telescope:
\ExecuteMetaData[Tex/Background]{ListD}
This description declares that \AD{List} has two constructors, one with no fields, corresponding to \AIC{[]}, and the second with one field and a recursive field, representing \AIC{\_∷\_}. In the second constructor, we used pattern lambdas to deconstruct the telescope\footnote{Due to a quirk in the interpretation of telescopes, the \AIC{∅} part always contributes a value \ARF{tt} we explicitly ignore, which also explicitly needs to be provided when passing parameters and variables.} and extract the type \AV{A}.
Using the variable bound in \AIC{σ}, we can also define the existential quantifier:
\ExecuteMetaData[Tex/Background]{SigmaD}
having one constructor with two fields. Here, the first field of type \AV{A} adds a value \AV{a} to the variable telescope, which we recover in the second field by pattern matching, before passing it to \AV{B}.


\subsection{Indexed types}\label{ssec:background-ix}
Lastly, we can integrate indexed types \cite{iir} into the universe by abstracting over indices
\ExecuteMetaData[Tex/Background]{U-ix}
Recall that in native Agda datatypes, a choice of constructor can fix the indices of the recursive fields and the resultant type, so we encode:
\ExecuteMetaData[Tex/Background]{Con-ix}
%In most cases, the index is simply threaded through the interpretation, allowing for a choice in the relevant codes.
If we are constructing a term of some indexed type, then the previous choices of constructors and arguments build up the actual index of this term. This actual index must then match the index we expected in the declaration of this term. This means that in the case of a leaf, we have to replace the unit type with the necessary equality between the expected and actual indices \cite{algorn}:
\ExecuteMetaData[Tex/Background]{int-ix}
In a recursive field, the expected index can be chosen based on parameters and variables. % mark: wording

In this universe, we can define finite types and vectors as:
\ExecuteMetaData[Tex/Background]{FinD}
and
\ExecuteMetaData[Tex/Background]{VecD}
These are equivalent, but since we do not model implicit fields, they are slightly different in use compared to \AD{Fin} and \AD{Vec}. In the first constructor of \AF{VecD} we report an actual index of \AIC{zero}. In the second, we have a field \bN{} to bring the index \AV{n} into scope, which is used to request a recursive field with index \AV{n}, and report the actual index of \AIC{suc}\ \AV{n}. 

Let us also show how the definitions of naturals and lists from earlier sections can be replicated in \AD{U-ix}
\ExecuteMetaData[Tex/Background]{new-Nat-List}
Writing the descriptions \AF{NatD}, \AF{ListD} and \AF{VecD} next to each other makes it easy to see the similarities: \AF{ListD} is the same as \AF{NatD} with a type parameter and one more \AIC{σ}. Likewise, \AF{VecD} is the same as \AF{ListD}, but now indexing over \bN{} and with yet one more \AIC{σ} of \bN{}. This kind of analysis is the focus of \autoref{sec:background-ornaments}.

\subsubsection{Generic Programming}
As a bonus, we can also use \AD{U-ix} for generic programming. For example, by a long construction which can be found in \autoref{app:gfold}, we can define the generic \AF{fold} operation:
\ExecuteMetaData[Tex/Background]{fold-type}
Let us describe how \AF{fold} works intuitively. We can interpret a term of \AF{⟦}\ \AV{D}\ \AF{⟧D}\ \AV{X} as a term of \AF{μ-ix}\ \AV{D}, where the recursive positions hold values of \AV{X} rather than values of \AF{μ-ix}\ \AV{D}. Then \AF{fold} states that a function collapsing such terms into values of \AV{X} extends to a function collapsing \AF{μ-ix}\ \AV{D} into \AV{X}, recursively collapsing applications of \AIC{con} from the bottom up.

As a more concrete example, when instantiating \AF{fold} to \AF{ListD}, the type \AF{⟦}\ \AV{ListD}\ \AF{⟧D}\ \AV{X} reduces (up to equivalence) to \AD{⊤}\ \AD{⊎}\ (\AV{A}\ \AD{×}\ \AV{X}\ \AV{A})\ \AF{→}\ \AV{X}\ \AV{A}, and \AF{fold} becomes
\ExecuteMetaData[Tex/Background]{foldr-type}
which, much like the familiar \AF{foldr} operation lets us consume a \AD{List}\ \AV{A} to produce a value \AV{X A}, provided a value \AV{X A} in the empty case, and a means to convert a pair (\AV{A}, \AV{X A}) to \AV{X A}.

Do note that this version takes a polymorphic function as an argument, as opposed to the usual fold which has the quantifiers on the outside:
\ExecuteMetaData[Tex/Background]{usual-fold}
Like a couple of constructions we will encounter in later sections, we can recover the usual fold into a type \AV{C} by generalizing \AV{C} to some kind of maps into \AV{C}. For example, by letting \AV{X} be continuation-passing computations into \bN{}, we can recover
\ExecuteMetaData[Tex/Background]{foldr-sum}


\section{Ornaments}\label{sec:background-ornaments}
In this section we will introduce a simplified definition of ornaments, which we will use to compare descriptions. Purely looking at their descriptions, \bN{} and \AD{List} are rather similar, except that \AD{List} has a parameter and an extra field \bN{} does not have. We could say that we can form the type of lists by starting from \bN{} and adding this parameter and field, while keeping everything else the same. In the other direction, we see that each list corresponds to a natural by stripping this information. Likewise, the type of vectors is almost identical to \AD{List}, can be formed from it by adding indices, and each vector corresponds to a list by dropping the indices.

Observations like these can be generalized using ornaments \cite{algorn, progorn, sijsling}, which define a binary relation describing which datatypes can be formed by ``decorating'' others. Conceptually, a type can be decorated by adding or modifying fields, extending its parameters, or refining its indices.

Essential to the concept of ornaments is the ability to convert back, forgetting the extra structure. After all, if there is an ornament from \AV{A} to \AV{B}, then \AV{B} is \AV{A} with extra fields and parameters, and more specific indices. In that case, we should also be able to discard those extra fields, parameters, and more specific indices, obtaining a conversion from \AV{B} to \AV{A}. If \AV{A} is a \AD{U-ix}\ \AV{Γ}\ \AV{I} and \AV{B} is a \AD{U-ix}\ \AV{Δ}\ \AV{J}, then a conversion from \AV{B} to \AV{A} presupposes a function \AV{re-par :}\ \AF{Cxf}\ \AV{Δ}\ \AV{Γ} for re-parametrization, and a function \AV{re-index :}\ \AV{J}\ \AV{→}\ \AV{I} for re-indexing.

In the same way that descriptions in \AD{U-ix} are lists of constructor descriptions, ornaments are lists of constructor ornaments. We define the type of ornaments reparametrizing with \AV{re-par} and reindexing with \AV{re-index} as a type indexed over \AD{U-ix}:
\ExecuteMetaData[Tex/Background]{Orn}
The conversion between types induced by an ornament is then embodied by the forgetful map
\ExecuteMetaData[Tex/Background]{bimap}
\ExecuteMetaData[Tex/Background]{ornForget-type}
which will revert the modifications made by the constructor ornaments, and restores the original indices and parameters.

The allowed modifications are controlled by the definition of constructor ornaments \AD{ConOrn}. We must keep in mind that each constructor of \AD{ConOrn} also has to be reverted by \AF{ornForget}, accordingly, some modifications have preconditions, which are in this case always pointwise equalities:
\ExecuteMetaData[Tex/Background]{htpy}
Since constructors exist in the context of variables, we let constructor ornaments transform variables with \AV{re-var}, in addition to parameters and indices.

The first three constructors of \AD{ConOrn} represent the operations which copy the corresponding constructors of \AD{Con-ix}\footnote{Viewing \AD{ConOrn} as a binary relation on \AD{Con-ix}, these represent the preservation of \AD{ConOrn} by \AIC{𝟙}, \AIC{ρ}, and \AIC{σ}, up to parameters, variables, and indices.}. The \AIC{Δσ} constructors allows one to add fields which are not present on the original datatype.
\ExecuteMetaData[Tex/Background]{ConOrn}
% yes re-par can be implicit most of the time
% when you actually start using ornaments generically, it will come back to bite you though
The commuting square \AF{re-index}\ \AF{∘}\ \AV{j}\ \AF{∼}\ \AV{i}\ \AF{∘}\ \AF{var→par}\ \AV{re-var} in the first two constructors ensures that the indices on both sides are indeed related, up to \AV{re-index} and \AV{re-var}.

Now, we can show that lists are indeed naturals decorated with fields:
\ExecuteMetaData[Tex/Background]{NatD-ListD}
This ornament preserves most structure of \bN{}, only adding a field using \AIC{∆σ}\footnote{Note that \AV{S}, and some later arguments we provide to ornaments, are implicit argument: Agda would happily infer them from \AF{ListD} and later \AF{VecD} had we omitted them.}. As \bN{} has no parameters or indices, \AD{List} has more specific parameters, namely a single type parameter. Consequently, all commuting squares factor through the unit type and can be satisfied with \AV{λ}\ \AV{\_}\ \AV{→}\ \AIC{refl}. 

We can also ornament lists to get vectors by reindexing them over \bN{}
\ExecuteMetaData[Tex/Background]{ListD-VecD}
We bind a new field of \bN{} with \AIC{∆σ}, extracting it in \AIC{𝟙} and \AIC{ρ} to declare that the constructor corresponding to \AIC{\_∷\_} takes a vector of length \AV{n} and returns a vector of length \AIC{suc}\ \AV{n}. 

The conversions from lists to naturals, and from vectors to lists are given by \AF{ornForget}. We define \AF{ornForget} as a \AF{fold} over an algebra that erases a single layer of decorations
\ExecuteMetaData[Tex/Background]{ornForget}
Recursively applying this algebra, which reinterprets values of \AV{E} as values of \AV{D}, lets us take apart a value in the fixpoint \AD{μ-ix}\ \AV{E} and rebuild it to a value of \AF{μ-ix}\ \AV{D}. This algebra
\ExecuteMetaData[Tex/Background]{ornAlg}
is a special case of the erasing function, which undecorates interpretations of arbitrary types \AV{X}:
\ExecuteMetaData[Tex/Background]{ornErase}
Reading off the ornament, we see which bits of \AV{CE} are new and which are copied from \AV{CD}, and consequently which parts of a term \AV{x} under an interpretation of \AV{CE} need to be forgotten, and which needs to be copied or translated. Specifically, the first three cases of \AF{conOrnErase} correspond to the structure-preserving ornaments, and merely translate equivalent structures from \AV{CE} to \AV{CD}.

For example, in the first case the ornament \AIC{𝟙}\ \AV{sq} copies leaves, telling us that \AV{CD} is \AIC{𝟙 i'} and \AV{CE} is \AIC{𝟙 j'}. The interpretation \AV{⟦ 𝟙 j' ⟧C \_ p j} of a leaf \AV{𝟙 j'} at parameters \AV{p} and index \AV{j} is simply the equality of expected and actual indices \AV{j ≡ (j' p)}. The term \AV{x} of \AV{j ≡ (j' p)}, then only has to be converted to the corresponding proof of equality on the \AV{CD} side: \AV{re-index j ≡ (i' (var→par re-var p))}. This is precisely accomplished by applying \AF{re-index} to both sides and composing with the square \AV{sq} at \AV{p}.

Likewise, in the case of \AIC{ρ} we only have to show that \AV{x} can be converted from one \AIC{ρ} to the other \AIC{ρ} by translating its parameters, and in the \AIC{σ} case the field is directly copied. The only other ornament \AIC{Δσ} adding fields, is easily undone by removing those fields. 

Thus, \AF{ornForget} establishes that \AV{E} in an ornament \AD{Orn}\ \AV{g}\ \AV{i}\ \AV{D}\ \AV{E} is an adorned version of \AV{D} by associating to each value of \AV{E} its an underlying value in \AV{D}. Additionally, \AF{ornForget} makes it simple to relate functions between related types. For example, instantiating \AF{ornForget} for \AF{NatD-ListD} yields \AF{length}. Hence, the statement that \AF{length} sends concatenation \AF{\_++\_} to addition \AF{\_+\_}, i.e. \AV{length (xs ++ ys) ≡ length xs + length ys}, is equivalent to the statement that \AF{\_++\_} and \AF{\_+\_} are related, or that \AF{\_++\_} is a lifting of \AF{\_+\_} \cite{orntrans}. %\marker{Ik hoop dat dit minder wazig is en de mental typechecking load wat reduceert.}

% remark, ornForget is not epi in general because of ∆σ ⊥

\section{Ornamental Descriptions}\label{sec:background-ornamental-descriptions}
By defining the ornaments \AF{NatD-ListD} and \AF{ListD-VecD} we could show that lists are numbers with fields and vectors are lists with fixed lengths. Even though we had to give \AF{ListD} before we could define \AF{NatD-ListD}, the value of \AF{NatD-ListD} actually forces the right-hand side to be \AF{ListD}.

This means we can also use an ornament to represent a description as a patch on top of another description, if we leave out the right-hand side of the ornament. Ornamental descriptions are precisely defined as ornaments without the right-hand side, and effectively bundle a description and an ornament to it\footnote{Consequently, \AD{OrnDesc}\ \AV{Δ}\ \AV{J}\ \AV{g}\ \AV{i}\ \AV{D} must simply be a convenient representation of \AD{Σ}\ (\AD{U-ix}\ \AV{Δ}\ \AV{J})\ (\AD{Orn}\ \AV{g}\ \AV{i}\ \AV{D}).}. Their definition is analogous to that of ornaments, making the arguments which would only appear in the new description explicit:
\ExecuteMetaData[Tex/Background]{OrnDesc}
\ExecuteMetaData[Tex/Background]{ConOrnDesc}
Using \AD{OrnDesc} we can describe lists as the patch on \AF{NatD} which inserts a \AIC{σ} in the constructor corresponding to \AIC{suc}:
\ExecuteMetaData[Tex/Background]{NatOD}
To extract \AF{ListD} from \AF{NatOD}, we can use the projection applying the patch in an ornamental description:
\ExecuteMetaData[Tex/Background]{toDesc}
The other projection reconstructs the ornament \AF{NatD-ListD} from \AF{NatOD}:
\ExecuteMetaData[Tex/Background]{toOrn}
As a consequence, \AD{OrnDesc} enjoys the features of both \AD{Desc} and \AD{Orn}, such as interpretation into a datatype by \AF{μ} and the conversion to the underlying type by \AF{ornForget}, by factoring through these projections.

In later sections, %mark: precisely?
we will routinely use \AD{OrnDesc} to view triples like (\AF{NatD}, \AF{ListD}, \AF{VecD}) as a base type equipped with two patches in sequence.


% exercise to reader: show OrnDesc AD ~ Exist[ BD in Desc ] Orn AD BD  

\section{Preliminary work}\label{sec:preliminary}
\towrite{Adapt and split into background and actual work}

\section{Proof Transport via the Structure Identity Principle}\label{sec:leibniz}
To give an understanding of the basics of Cubical Agda \cite{cuagda} and the Structure Identity Principle (SIP), we walk through the steps to transport proofs about addition on Peano naturals to Leibniz naturals. We give an overview of some features of Cubical Agda, such as that paths give the primitive notion of equality, until the simplified statement of univalence. We do note that Cubical Agda has two downsides relating to termination checking and universe levels, which we encounter in later sections.

Starting by defining the unary Peano naturals and the binary Leibniz naturals, we prove that they are isomorphic by interpreting them into eachother. We explain that these interpretations are easily seen to be mutual inverses by proving lemmas stating that both interpretations ``respect the constructors'' of the types. Next, we demonstrate how this isomorphism can be promoted into an equivalence or an equality, and remark that this is sufficient to transport intrinsic properties, such as having decidable equality, from one natural to the other.

Noting that transporting unary addition to binary addition is possible but not efficient, we define binary addition while ensuring that it corresponds to unary addition. We present a variant on refinement types as a syntax to recover definition from chains of equality reasoning, allowing one to rewrite definitions while preserving equalities.

We clarify that to transport proofs referring to addition from unary to binary naturals, we indeed require that these are meaningfully related. Then, we observe that in this instance, the pairs of ``type and operation'' are actually equated as magmas, and explain that this is an instance of the SIP.

Finally, we describe the use case of the SIP, how it generalizes our observation about magmas, and how it can calculate the minimal requirements to equate to implementations of an interface. This is demonstrated by transporting associativity from unary addition to binary addition, noting that this would save many lines of code provided there is much to be transported.

\towrite{Merge}

Let us quickly review some features of Cubical Agda \cite{cuagda} that we will use in this section.

%Of course, this downside is more than offset by the benefits of changing our primitive notion of equality, which we will see makes it easier to show that ``equivalent'' structures behave identically. 
In Cubical Agda, the primitive notion of equality arises not (directly) from the indexed inductive definition we are used to, but rather from the presence of the interval type \AgdaPrimitiveType{I}. This type represents a set of two points \AgdaInductiveConstructor{i0} and \AgdaInductiveConstructor{i1}, which are considered ``identified'' in the sense that they are connected by a path. To define a function out of this type, we also have to define the function on all the intermediate points, which is why we call such a function a ``path''. Terms of other types are then considered identified when there is a path between them.

While the benefits are overwhelming for us\todo[inline, color=red]{Which?}, this is not completely without downsides, such as that
%\ExecuteMetaData[Tex/CubicalAndBinary]{cubical}% \todo[inline]{Not sure if it would be helpful to have a more extensive introduction covering all features used.} % at this moment, probably not, as the cubical usage is rather tame, so I'll probably stick to introducing stuff as it becomes necessary. % TODO then write that somewhere
the negation of axiom K complicates both some termination checking and some universe levels.\footnote{In particular, this prompts rather far-reaching (but not fundamental) changes to the code of previous work, such as to the machinery of ornaments \cite{progorn} in \autoref{sec:userfriendly}.} Furthermore, if we use certain homotopical constructions, and we wish to eliminate from our types as if they were sets, then we will also have to prove that they are indeed sets.

On the positive side, this different perspective gives intuitive interpretations to some proofs of equality, like
\ExecuteMetaData[Tex/CubicalAndBinary]{sym}
where \AgdaFunction{∼\_} is the interval reversal, swapping \AgdaInductiveConstructor{i0} and \AgdaInductiveConstructor{i1}, so that \AgdaFunction{sym} simply reverses the given path.

Furthermore, because we can now interpret paths in record and function types in a new way, we get a host of ``extensionality'' for free. For example, a path in $A \to B$ is indeed a function which takes each $i$ in \AgdaPrimitiveType{I} to a function $A \to B$. Using this, function extensionality becomes tautological 
\ExecuteMetaData[Tex/CubicalAndBinary]{funExt}

Finally, %while in ``non-univalent'' Agda bijections or isomorphisms do not play such a central role,
much of our work will rest on equivalences, as the ``HoTT-compatible'' generalization of bijections. This is because in Cubical Agda, we have the univalence theorem 
%the \AgdaPrimitiveType{Glue} type tells us that equivalent types fit together in a new type, in a way that guarantees univalence
\ExecuteMetaData[Tex/CubicalAndBinary]{ua}
stating that ``equivalent types are identified'', such that type isomorphisms like $1 \to A \simeq A$ become paths $1 \to A \equiv A$, making it so that we can transport proofs along them. We will demonstrate this by a more practical example in the next section.


\subsection{Unary numbers are binary numbers}\label{ssec:binary}
Let us demonstrate an application of univalence by exploiting the equivalence of the ``Peano'' naturals and the ``Leibniz'' naturals. Recall that the Peano naturals are defined as 
\ExecuteMetaData[Tex/CubicalAndBinary]{Peano}
This definition enjoys a simple induction principle and is well-covered in most libraries. However, the definition is also impractically slow, since most arithmetic operations defined on \bN{} have time complexity in the order of the value of the result.

As an alternative we can use binary numbers, for which for example addition has logarithmic time complexity. Standard libraries tend to contain few proofs about binary number properties, but this does not have to be a problem: the \bN{} naturals and the binary numbers should be equivalent after all!

Let us make this formal. We define the Leibniz naturals as follows:
\ExecuteMetaData[Leibniz/Base.tex]{Leibniz}
Here, the \AgdaInductiveConstructor{0b} constructor encodes 0, while the \AgdaInductiveConstructor{\_1b} and \AgdaInductiveConstructor{\_2b} constructors respectively add a 1 and a 2 bit, under the usual interpretation of binary numbers:
\ExecuteMetaData[Leibniz/Base.tex]{toN}
\ExecuteMetaData[Leibniz/Base.tex]{toN-2}
This defines one direction of the equivalence from \bN{} to \bL{}, for the other direction, we can interpret a number in \bN{} as a binary number by repeating the successor operation on binary numbers:
\ExecuteMetaData[Leibniz/Base.tex]{bsuc}
\ExecuteMetaData[Leibniz/Base.tex]{fromN}
To show that \AgdaFunction{toℕ} is an isomorphism, we have to show that it is the inverse of \AgdaFunction{fromℕ}. By induction on \bL{} and basic arithmetic on \bN{} we see that
\ExecuteMetaData[Leibniz/Properties.tex]{toN-suc}
so \AgdaFunction{toℕ} respects successors. Similarly, by induction on \bN{} we get
\ExecuteMetaData[Leibniz/Properties.tex]{fromN-1}
and % I can't get the code blocks to stick together lol
\ExecuteMetaData[Leibniz/Properties.tex]{fromN-2}
so that \AgdaFunction{fromℕ} respects even and odd numbers. We can then prove that applying \AgdaFunction{toℕ} and \AgdaFunction{fromℕ} after each other is the identity by repeating these lemmas
\ExecuteMetaData[Leibniz/Properties.tex]{N-iso-L}
This isomorphism can be promoted to an equivalence
\ExecuteMetaData[Leibniz/Properties.tex]{N-equiv-L}
which, finally, lets us identify \bN{} and \bL{} by univalence
\ExecuteMetaData[Leibniz/Properties.tex]{N-is-L}
The path \AgdaFunction{ℕ≡L} then allows us to transport properties from \bN{} directly to \bL{}; as an example, we have not yet shown that \bL{} is discrete, i.e., has decidable equality. Using substitution, we can quickly derive this\footnote{Of course, this gives a rather inefficient equality test, but for the homotopical consequences this is not a problem.}
\ExecuteMetaData[Leibniz/Properties.tex]{DiscreteL}
This can be generalized even further to transport proofs about operations from \bN{} to \bL{}.

\subsection{Functions from specifications}\label{ssec:useas}
As an example, we will define addition of binary numbers. We could transport \AgdaFunction{\_+\_} as a binary operation
\ExecuteMetaData[Extra/Algebra]{BinOp}
from \bN to \bL to get
\ExecuteMetaData[Tex/CubicalAndBinary]{badplus}
But this is inefficient, incurring an $O(n + m)$ overhead when adding $n$ and $m$. It is more efficient to define addition on \bL{} directly, making use of the binary nature of \bL{}, while agreeing with the addition on \bN{}. Such a definition can be derived from the specification ``agrees with \AgdaFunction{\_+\_}'', so we implement a syntax for giving definitions by equational reasoning, inspired by the ``use-as-definition'' notation used by Hinze and Swierstra \cite{calcdata}: Using an implicit pair type
\ExecuteMetaData[Prelude/UseAs.tex]{isigma}
we define
\ExecuteMetaData[Prelude/UseAs.tex]{Def}
which extracts a definition as the right endpoint of a given path.
% \investigate{As of now, I am unsure if this reduces the effort of implementing a coherent function, or whether it is more typically possible to give a smarter or shorter proof by just giving a definition and proving an easier property of it\footnote{I will put the alternative in the appendix for now}}

With this we can define addition on \bL{} and show it agrees with addition on \bN{} in one motion
\ExecuteMetaData[Leibniz/Properties.tex]{plus-def}
Now we can easily extract the definition of \AgdaFunction{plus} and its correctness with respect to \AgdaFunction{\_+\_} 
\ExecuteMetaData[Leibniz/Properties.tex]{plus-good}

We remark that \AgdaFunction{Def} is close in concept to refinement types\footnote{À la \href{https://agda.github.io/agda-stdlib/Data.Refinement.html}{Data.Refinement}.}, but extracts the value from the proof, rather than requiring it before. \footnote{Unfortunately, normalizing an application of a \AgdaFunction{defined-by} function also causes a lot of unnecessary wrapping and unwrapping, so \AgdaFunction{Def} is mostly only useful for presentation.} %for now..


\subsection{The Structure Identity Principle}
We point out that \bN{} with \AgdaFunction{N.+} and \bL{} with \AgdaFunction{plus} form magmas, that is, inhabitants of
\ExecuteMetaData[Extra/Algebra.tex]{Magma'}
Using that a path in a dependent pair corresponds to a dependent pair of paths, we get a path from (\bN{}, \AgdaFunction{N.+}) to (\bL{}, \AgdaFunction{plus}). %More generally, a magma is simply a type $X$ with some structure, which is a function $f: X \to X \to X$ in the case of a magma. We can see that paths between magmas correspond to paths $p$ between the underlying types $X$ and paths over $p$ between their operations $f$.
This observation is further generalized by the Structure Identity Principle (SIP) as a form of representation independence \cite{iri}. Given a structure, which in our case is just a binary operation
\ExecuteMetaData[Extra/Algebra.tex]{MagmaStr}
this principle produces an appropriate definition ``structured equivalence'' $\iota$. The $\iota$ is such that if structures $X, Y$ are $\iota$-equivalent, then they are identified. In the case of \AgdaFunction{MagmaStr}, the $\iota$ asks us to provide something with the same type as \AgdaFunction{plus-coherent}, so we have just shown that the \AgdaFunction{plus} magma on \bL{}
\ExecuteMetaData[Leibniz/Properties.tex]{magmaL}
and the \AgdaFunction{\_+\_} magma on \bN{} and are identical
\ExecuteMetaData[Leibniz/Properties.tex]{magma-equal}
As a consequence, properties of \AgdaFunction{\_+\_} directly yield corresponding properties of \AgdaFunction{plus}. For example,
\ExecuteMetaData[Leibniz/Properties.tex]{assoc-transport}\todo[inline, color=red]{Express what this accomplishes, and why this is impressive compared to without univalence}


\section{Types from Specifications: Ornamentation and Calculation}\label{sec:numrep}
Using an example where we try to safely refactor a piece of code to use trees rather than lists, we motivate the need for a framework to organize different container types under a similar description. We explain that for indexed types, we can use representability, e.g., vectors correspond to functions out of finite types.

We describe how we can also derive these datastructures from functions \cite{calcdata}, starting from a number system, yielding a numerical representation \cite{purelyfunctional}; this is demonstrated by an example deriving of vectors from unary naturals \cite{calcdata}. The vector type is computed by chains of equality reasoning like in the previous section, giving the correspondence to functions out of the finite type.

We illustrate how both (the functions out of a type and the concrete vectors) can implement an array interface, such as two-sided flexible arrays. We remark that the laws for such interfaces can be more easily proven on the function-based implementations, so that they can be transported to the concrete implementation.

Reflecting on this derivation, we note that the computation for binary naturals would be analogous, amending constructors constructors with fields holding appropriate number of elements. We relate this to ornamentation \cite{progorn}, which lets us relate types by recursive structure. After that, we give a short overview of the capabilities of descriptions and ornaments, and demonstrate these by deriving the list datatype from the unary natural type using an ornamental description.

We remark that this approach needs to be adjusted to use indices before it can be applied to binary naturals; we clarify how ``metaphorical'' this construction of binary trees is to binary numbers by letting the weight of the ``digits'' control the numbers of elements in each constructor. We explain that in doing so, the shape of the binary tree seen as a binary number then corresponds to the number of elements it contains.

After that, we give a completely different application of ornaments; we recall that the construction of heterogeneous lists, lists which contain elements with different types, is rather mechanical. Observing that a ``heterogeneous X'' is expressed as an ``X-indexed X'', we assert that this self-indexing can be captured as an ornament.

To define this ornament, we needed to include a parameter field in the definition of descriptions and ornaments. Then we define ``heterogeneization'' as an ``ornament-computing function'', which takes a description and produces an ornamental description. We demonstrate how we can heterogeneize lists and maybes, allowing us to produce a natural implementation of the heterogeneous head operation, and relate this to earlier work deriving heterogeneous random access lists \cite{hetbin}.

\towrite{Merge}

While the practical applications of the last example do not stretch very far\footnote{Considering that \AgdaDatatype{ℕ} is a candidate to be replaced by a more suitable unsigned integer type when compiling to Haskell anyway.}, the approach generalizes to the more relevant containers and their associated laws.

In the same vein as the last section, we could define a simple but inefficient array type, and a more efficient implementation using trees. Then we can show that these are equivalent, such that when the simple type satisfies a set of laws, trees will satisfy them as well. We could then start developing all sorts of complex implementations fine-tuned to each operation and figure out how these are equivalent to some simpler type, but let us first take a step back, and investigate how we can make this approach run smoothly in a simpler example.

Rather than inductively defining a container and then showing that it is represented by a lookup function, we can go the other way around and define a type by insisting that it is equivalent to such a function. This approach, in particular the case in which one calculates a container with the same shape as a numeral system, was dubbed numerical representations in \cite{purelyfunctional}, and has some formalized examples in, e.g., \cite{calcdata} and \cite{progorn}. Numerical representations form the starting point for defining more complex datastructures based on simpler ones, so let us demonstrate such a calculation. 

\subsection{Numerical representations: from numbers to containers}\label{ssec:numrep}
We can compute the type of vectors starting from \bN{}.\footnote{This is adapted (and fairly abridged) from \cite{calcdata}} For simplicity, we define them as a type computing function via the ``use-as-definition`` notation from before. We expect vectors to be represented by 
\ExecuteMetaData[Tex/NumRep]{Lookup}
where we use the finite type \AgdaDatatype{Fin} as an index into vector. Using this representation as a specification, we can compute both \AgdaDatatype{Fin} and a type of vectors. The finite type can be computed from the evident definition
\ExecuteMetaData[Tex/NumRep]{Fin-def}
using
\ExecuteMetaData[Tex/NumRep]{leq-split}
Likewise, vectors can be computed by applying a sequence of type isomorphisms
\ExecuteMetaData[Tex/NumRep]{Vec}
\investigate{SIP doesn't mesh very well with indexed stuff, does HSIP help?}
Of course, a container would not be of much use without lookup functions, so we define an interface
\ExecuteMetaData[Tex/NumRep]{Array}
which at the very least has to satisfy laws like
\ExecuteMetaData[Tex/NumRep]{Laws}
We could directly show that \AgdaDatatype{Vec} satisfies this, but now that we defined \AgdaDatatype{Vec} from \AgdaDatatype{Lookup} we might as well use this fact.

The implementation of arrays as functions is very straightforward
\ExecuteMetaData[Tex/NumRep]{FunArray}
and clearly satisfies our interface
\ExecuteMetaData[Tex/NumRep]{FunLaw}
We can implement arrays based on \AgdaDatatype{Vec} as well
\ExecuteMetaData[Tex/NumRep]{VecArray}
and again, we can transport the proofs from \AgdaDatatype{Lookup} to \AgdaDatatype{Vec}.\footnote{Except in this oversimplified case the laws are trivial for \AgdaDatatype{Vec} as well.}\todo{If one was determined to cobble together the path over path over path we need now.}
\investigate{As you can see, taking ``use-as-definition'' too literally prevents Agda from solving a lot of metavariables.}

\investigate{This computation can of course be generalized to any arity zeroless numeral system; unfortunately beyond this set of base types, this ``straightforward'' computation from numeral system to container loses its efficacy. In a sense, the n-ary natural numbers are exactly the base types for which the required steps are convenient type equivalences like $(A + B) \to C = (A \to C) \times (B \to C)$?}

%\subsection{Relating types by structure: Ornamentation (unfinished)}\label{sec:ornament}
\subsection{Numerical representations as ornaments}\label{ssec:ornaments}
We could peform the same computation for \bL{}, which would yield the type of binary trees, but we note that these computations proceed with roughly the same pattern: each constructor of the numeral system gets assigned a value, and is amended with a field holding a number of elements and subnodes using this value as a ``weight''. But wait! Such modifications of constructors are already made formal by the concept of ornamentation!\todo{It seems like Agda forgets that we defined Leibniz if we move between different .tex files, so I'll have to float all AgdaTargets to the top level at some point...}

Ornamentation, as exposed in \cite{algorn} and \cite{progorn}, lets us formulate what it means for two types to have a ``similar'' recursive structure. This is achieved by interpreting (indexed inductive) datatypes from descriptions, between which an ornament is seen as a certificate of similarity, describing which fields or indices need to be introduced or dropped. Furthermore, a one-sided ornament: an ornamental description, lets us describe new datatypes by recording the modifications to an existing description.
\todo{Again not sure how much space I should use to reiterate Desc, Orn, and OrnDesc.}

This links back to the construction in the previous section, since \bN{} and \AgdaDatatype{Vec} share the same recursive structure, so \AgdaDatatype{Vec} can be formed by introducing indices and adding a field holding an element at each node.\footnote{These and similar examples are also documented in \cite{progorn}} 

However, instead deriving \AgdaDatatype{List} from \bN{} generalizes to \bL{} with less notational overhead, so lets tackle that case first. For this, we have to give a description of \bN{} to work with\todo{Clearly this can use more explanation (the question is, how much?)}
\ExecuteMetaData[Tex/NumRepOrn]{NatD}
Recall that \AgdaInductiveConstructor{σ} adds a field, upon which the rest of the description may vary, and \AgdaInductiveConstructor{ṿ} lists the recursive fields and their indices (which can only be \AgdaInductiveConstructor{tt}).
We can now write down the ornament which adds fields to the \AgdaFunction{suc} constructor
\ExecuteMetaData[Tex/NumRepOrn]{ListO}
Here, the \AgdaInductiveConstructor{σ} and \AgdaInductiveConstructor{ṿ} are forced to match those of \AgdaDatatype{NatD},
but the \AgdaInductiveConstructor{Δ} adds a new field. With the least fixpoint and description extraction from \cite{progorn}, this is sufficient to define \AgdaDatatype{List}. Note that we cannot hope to give an unindexed ornament from \bL{}
\ExecuteMetaData[Tex/NumRepOrn]{LeibnizD}
into trees, since trees have a very different recursive structure! Instead, we must keep track at what level we are in the tree so that we can ask for adequately many elements:
\ExecuteMetaData[Tex/NumRepOrn]{TreeO}
We use the \AgdaFunction{power} combinator to ensure that the digit at position $n$, which has weight $2^n$ in the interpretation of a binary number, also holds its value times $2^n$ elements. This makes sure that the number of elements in the tree shaped after a given binary number also is the value of that  binary number.

This ``folding in'' technique using the indices to keep track of structure seems to apply more generally. Let us explore this a bit further, and return later to the generalization of the pattern from numeral systems to datastructures.
% i.e. why did this even work?

\subsection{Folding in}\label{ssec:flattening}
Let us describe this procedure of folding a complex recursive structure into a simpler structure more generally. In particular, we will demonstrate that for linear datatypes, such as \bN{} and \bL{}, and for a given unindexed datatype, there is always an indexed datatype isomorphic to it at some index, and an ornament from the linear type to the indexed type. 

Suppose we are given a description, the first thing we can do to simplify it is collect all fields in one place
\ExecuteMetaData[Tex/Flatten]{RField}
Next, we will certainly have to count the number of recursive occurrences we are tracking, so we define
\ExecuteMetaData[Tex/Flatten]{Number}
where \AgdaInductiveConstructor{𝟙} records that we are at the top level, and \AgdaInductiveConstructor{ṿ} denotes that we are below a constructor with some number of recursive fields. This simplifies our task to implementing the types in
\ExecuteMetaData[Tex/Flatten]{nested}
such a way that we get an isomorphism 
\ExecuteMetaData[Tex/Flatten]{wish}
Thus, \AgdaDatatype{Fields} is forced to have a \AgdaInductiveConstructor{leaf} constructor like 
\ExecuteMetaData[Tex/Flatten]{Fields}
if \AgdaFunction{nested} is to work at \AgdaInductiveConstructor{𝟙}. The \AgdaInductiveConstructor{node} constructor makes sure that if we have collection of \AgdaDatatype{Fields}, then we can gather them in a field at a higher level. We can then count the subnodes of a given \AgdaDatatype{Fields} as
\ExecuteMetaData[Tex/Flatten]{subnodes}
where \AgdaFunction{RSize} counts the number of recursive fields of a particular branch
\ExecuteMetaData[Tex/Flatten]{RSize}
Note that \AgdaFunction{subnodes} effectively keeps the shape of the previous field, but unfolds the recursive fields at the bottom of the tree by one level.

\investigate{I then tried and realized how unpleasant even the functions from the original type to the nested type are to write.}

As a trivialty, we get that any type, interpreted as a container, always decomposes as an ornament over a ``numerical'' base type.\todo{Or at least, that was where I was trying to go with this, but I notice that this still is a bit further away.} This links to the construction of binary heaps in \cite{progorn}, as in hindsight, starting from the usual binary heaps would yield binary numbers and their binary heap ornament (in a much less useful package).

\section{More equivalences for less effort}\label{sec:userfriendly}
Noting that constructing equivalences directly or from isomorphisms as in \autoref{sec:leibniz} can quickly become challenging when one of the sides is complicated, we work out a different approach making use of the initial semantics of W-types instead. We claim that the functions in the isomorphism of \autoref{sec:leibniz} were partially forced, but this fact was not made use of.

First, we explain that if we assume that one of the two sides of the equivalence is a fixpoint or initial algebra of a polynomial functor (that is, the \AgdaDatatype{μ} of a \AgdaDatatype{Desc′}), this simplifies giving an equivalence to showing that the other side is also initial.

We describe how we altered the original ornaments \cite{progorn} to ensure that \AgdaDatatype{μ} remains initial for its base functor in Cubical Agda, explaining why this fails otherwise, and how defining base functors as datatypes avoids this issue.

In a subsection focussing on the categorical point of view, we show how we can describe initial algebras (and truncate the appropriate parts) in such a way that the construction both applies to general types (rather than only sets), and still produces an equivalence at the end. We explain how this definition, like the usual definition, makes sure that a pair of initial objects always induces a pair of conversion functions, which automatically become inverses. Finally, we explain that we can escape our earlier truncation by appealing to the fact that ``being an equivalence'' is a proposition.

Next, we describe some theory, using which other types can be shown to be initial for a given algebra. This is compared to the construction in \autoref{sec:leibniz}, observing that intuitively, initiality follows because the interpretation of the zero constructor is forced by the square defining algebra maps, and the other values are forced by repeatedly applying similar squares. This is clarified as an instance of recursion over a polynomial functor.

To characterize when this recursion is allowed, we define accessibility with respect to polynomial functors as a mutually recursive datatype as follows. This datatype is constructed using the fibers of the algebra map, defining accessibility of elements of these fibers by cases over the description of the algebra. Then we remark that this construction is an atypical instance of well-founded recursion, and define a type as well-founded for an algebra when all its elements are accessible.

We interpret well-foundedness as an upper bound on the size of a type, leading us to claim that injectivity of the algebra map gives a lower bound, which is sufficient to induce the isomorphism. We sketch the proof of the theorem, relating part of this construction to similar concepts in the formalization of well-founded recursion in the Standard Library. In particular, we prove an irrelevance and an unfolding lemma, which lets us show that the map into any other algebra induced by recursion is indeed an algebra map. By showing that it is also unique, we conclude initiality, and get the isomorphism as a corrolary. 

The theorem is applied and demonstrated to the example of binary naturals. We remark that the construction of well-foundedness looks similar to view-patterns. After this, we conclude that this example takes more lines that the direct deriviation in \autoref{sec:leibniz}, but we argue that most of this code can likely be automated.

\towrite{Merge}

% REPLACE X BY A?
The setup some approaches in earlier sections require makes them tedious or impractical to apply. In this section we will look at some ways how part of this problem could be alleviated through generics, or by alternative descriptions of concepts like equivalences through the lens of initial algebras. 

In later sections we will construct many more equivalences between more complicated types than before, so we will dive right into the latter. Reflecting upon \autoref{sec:leibniz}, we see that when one establishes an equivalence, most of the time is spent working out a series of tedious lemmas to show that the conversion functions are mutual inverses, which tend to be relatively easy to define. We take away two things from this; the first is that the conversion functions are perhaps too obvious, and the second is that we should really avoid talking about sections and retractions lest we incur tedium!\footnote{The latter perhaps less so, because it is useful to show a map to be monic.} We will reuse the machinery of Ko and Gibbons \cite{progorn} to illustrate how the definitions in \autoref{sec:leibniz} were actually forced for a large part.

First, we remark that \AgdaDatatype{μ} is internalization of the representation of simple\footnote{Of course, indexed datatypes are indexed W-types, mutually recursive datatypes are represented yet differently\dots} datatypes as W-types. Thus, we will assume that one of the sides of the equivalence is always represented as an initial algebra of a polynomial functor, and hence the \AgdaDatatype{μ} of a \AgdaDatatype{Desc′}.

\subsection{Well-founded monic algebras are initial}\label{ssec:wellfounded}
Unfortunately, the machinery developed by Ko and Gibbons \cite{progorn} relies on axiom K for a small but crucial part. To be precise, in a cubical setting, the type \AgdaDatatype{μ} as given stops being initial for its base functor! In this section, we will be working with a simplified and repaired version. Namely, we simplify \AgdaDatatype{Desc′} to 
\ExecuteMetaData[Extra/ProgOrn/Desc]{DescS}
To complete the definition of \AgdaDatatype{μ}
\ExecuteMetaData[Extra/ProgOrn/Desc]{mu}
we will need to implement \AgdaDatatype{Base}. We remark that in the original setup, the recursion of \AgdaFunction{mapFold} is a structural descent in \AgdaFunction{⟦ D' ⟧ (μ D)}. Because \AgdaFunction{⟦\_⟧} is a type computing function and not a datatype, this descent becomes invalid\footnote{Refer to the  \href{https://agda.readthedocs.io/en/latest/language/without-k.html\#restrictions-on-termination-checking}{without K} page.}, and \AgdaFunction{mapFold} fails the termination check. We resolve this by defining \AgdaDatatype{Base} as a datatype
\ExecuteMetaData[Extra/ProgOrn/Desc]{Base}
such that this descent is allowed by the termination checker without axiom K.\footnote{This has, again by the absence of axiom K, the consequence of pushing the universe levels up by one. However, this is not too troublesome, as equivalences can go between two levels, and indeed types are equivalent to their lifts.}

Recall that the \AgdaDatatype{Base} functors of descriptions are special polynomial functors, and the fixpoint of a base functor is its initial algebra. We are looking for sufficient conditions on $X$ to get the equivalence $e: X \cong \mu F$. Note that when $X \cong \mu F$, then there necessarily is an initial algebra $F X \to X$. Conversely, if the algebra $(X, f)$ is isomorphic to $(\mu F, \mathrm{con})$, then $X \cong \mu F$ would follow immediately, so it is equivalent to ask for the algebras to be isomorphic instead.

\begin{comment}
The situation so far is summarized by the diagram
% https://q.uiver.app/?q=WzAsMyxbMSwwLCJGXFxtdV9GIl0sWzEsMSwiXFxtdSBGIl0sWzAsMSwiWCJdLFswLDEsIlxcbWF0aHJte2Nvbn0iXSxbMiwxLCJlIiwyLHsic3R5bGUiOnsidGFpbCI6eyJuYW1lIjoiYXJyb3doZWFkIn0sImJvZHkiOnsibmFtZSI6ImRhc2hlZCJ9fX1dXQ==
\[\begin{tikzcd}[ampersand replacement=\&]
	\& {F\mu_F} \\
	X \& {\mu F}
	\arrow["{\mathrm{con}}", from=1-2, to=2-2]
	\arrow["e"', dashed, tail reversed, from=2-1, to=2-2]
\end{tikzcd}\]
\end{comment}
\subsubsection{Datatypes as initial algebras}
To characterize when such algebras are isomorphic, we reiterate some basic category theory, simultaneously rephrasing it in Agda terms.\footnote{We are not reusing a pre-existing category theory library for the simple reasons that it is not that much work to write out the machinery explicitly, and that such libraries tend to phrase initial objects in the correct way, which is too restrictive for us.}

Let $C$ be a category, and let $a, b, c$ be objects of $C$, so that in particular we have identity arrows $1_a : a \to a$ and for arrows $g : b \to c, f : a \to b$ composite arrows $gf : a \to c$ subject to associativity. In our case, $C$ is the category of types, with ordinary functions as arrows.

Recall that an endofunctor, which is simply a functor $F$ from $C$ to itself, assigns objects to objects and sends arrows to arrows
\ExecuteMetaData[Extra/Category]{RawFunctor}
These assignments are subject to the identity and composition laws
\ExecuteMetaData[Extra/Category]{Functor}
An $F$-algebra is just a pair of an object $a$ and an arrow $Fa \to a$
\ExecuteMetaData[Extra/Category]{Algebra}
Algebras themselves again form a category $C^F$. The arrows of $C^F$ are the arrows $f$ of $C$ such that the following square commutes% https://q.uiver.app/?q=WzAsNCxbMCwwLCJGYSJdLFsxLDAsIkZiIl0sWzAsMSwiYSJdLFsxLDEsImIiXSxbMiwzLCJmIiwyXSxbMCwyLCJVX2EiLDJdLFsxLDMsIlVfYiJdLFswLDEsIkZmIl1d
\[\begin{tikzcd}[ampersand replacement=\&]
	Fa \& Fb \\
	a \& b
	\arrow["f"', from=2-1, to=2-2]
	\arrow["{U_a}"', from=1-1, to=2-1]
	\arrow["{U_b}", from=1-2, to=2-2]
	\arrow["Ff", from=1-1, to=1-2]
\end{tikzcd}\]
So we define
\ExecuteMetaData[Extra/Category]{AlgSqr}
and
\ExecuteMetaData[Extra/Category]{AlgMap}
Note that we take the propositional truncation of the square, such that algebra maps with the same underlying morphism become propositionally equal
\ExecuteMetaData[Extra/Category]{AlgPath}
The identity and composition in $C^F$ arise directly from those of the underlying arrows in $C$.

Recall that an object $\emptyset$ is initial when for each other object $a$, there is a unique arrow $!: \emptyset \to a$. By reversing the proofs of initiality of \AgdaDatatype{μ} and the main result of this section, we obtain a slight variation upon the usual definition. Namely, unicity is often expressed as contractability of a type
\ExecuteMetaData[Tex/Snippets]{isContr}
Instead, we again use a truncation
\ExecuteMetaData[Extra/Category]{weakContr}
but note that this also, crucially, slightly stronger than connectedness. We define initiality for arbitrary relations
\ExecuteMetaData[Extra/Category]{Initial}
such that it closely resembles the definition of least element. Then, $A$ is an initial algebra when
\ExecuteMetaData[Extra/Category]{InitAlg}

By basic category theory (using the usual definition of initial objects), two initial objects $a$ and $b$ are always isomorphic;
namely, initiality guarantees that there are arrows $f : a \to b$ and $g : b \to a$, which by initiality must compose to the identities again.

Similarly, we get that
\ExecuteMetaData[Extra/Category]{InitAlg-equiv}
However, we only have the equalities from the isomorphism inside a propositional truncation. But fortunately, being an equivalence is a property, so we can eliminate from the truncations to get the wanted result.

%Note that even though we warned ourselves, we are still talking about sections and retractions to establish that $f$ is an equivalence! However, this result also makes sure we will not have to speak of them again.

\subsubsection{Accessibility}
As a consequence, we get that $X$ is isomorphic to $\mu D$ when $X$ is an initial algebra for the base functor of $D$; $\mu D$ is initial by its fold, and by induction on $\mu D$ using the squares of algebra maps. 

\begin{remark}
    We need (in general) not hope $\mu D$ is a strict initial object in the category of algebras. For a strict initial object, having a map $a \to \emptyset$ implies $a \cong \emptyset$. This is not the case here: strict initial objects satisfy $a \times \emptyset \cong \emptyset$, but for the $X \mapsto 1 + X$-algebras $\mathbb{N}$ and $2^\mathbb{N}$ clearly $2^\mathbb{N} \times \mathbb{N} \cong \mathbb{N}$ does not hold. On the other hand, the ``obvious'' sufficient condition to let $C^F$ have strict initial objects is that $F$ is a left adjoint, but then the carrier of the initial algebra is simply $\bot$.
\end{remark}

Looking back at \autoref{sec:leibniz}, we see that \AgdaDatatype{Leibniz} is an initial $F: X \mapsto 1 + X$ algebra because for any other algebra, the image of \AgdaFunction{0b} is fixed, and by \AgdaFunction{bsuc} all other values are determined by chasing around the square. Thus, we are looking for a similar structure on $f : FX \to X$ that supports recursion.

Clearly we will need something stronger than $FX \cong X$, as in general a functor can have many fixpoints. For this, we define what it means for an element $x$ to be accessible by $f$. This definition uses a mutually recursive datatype as follows:
We state that an element $x$ of $X$ is accessible when there is an accessible $y$ in its fiber over $f$
\ExecuteMetaData[Extra/Category/Poly]{Acc}
Accessibility of an element $x$ of \AgdaFunction{Base A E} is defined by cases on $E$; if $E$ is \AgdaFunction{ṿ n} and $x$ is a \AgdaFunction{Vec A n}, then $x$ is accessible if all its elements are; if $x$ is \AgdaFunction{σ S E'}, then $x$ is accessible if \AgdaFunction{snd x} is
\ExecuteMetaData[Extra/Category/Poly]{Acc'}
Consequently, $X$ is well-founded for an algebra when all its elements are accessible
\ExecuteMetaData[Extra/Category/Poly]{Wf}

We can see well-foundedness as an upper bound on the size of $X$, if it were larger than $\mu D$, some of its elements would inevitably get out of reach of an algebra. \textit{Now} having $FX \cong X$ also gives us a lower bound, but remark that having a well-founded injection $f: FX \to X$ is already sufficient, as accessibility gives a section of $f$, making it an iso. This leads us to claim
\begin{claim}\label{claim:wf-inj-init}
    If there is a mono $f : FX \to X$ and $X$ is well-founded for $f$, then $X$ is an initial $F$-algebra.
\end{claim}

\subsubsection{Proof sketch of \autoref{claim:wf-inj-init}}
Let us be on our way. Suppose $X$ is well-founded for the mono $f : FX \to X$. To show that $(X, f)$ is initial, let us take another algebra $(Y, g)$, and show that there is a unique arrow $(X, f) \to (Y, g)$.\todo[inline]{This section is about as digestable as a brick.}

By \AgdaDatatype{Acc}-recursion and because all $x$ are accessible, we can define a plain map into $Y$
\ExecuteMetaData[Extra/Category/WellFounded]{Wf-rec}
This construction is an instance of the concept of ``well-founded recursion''\footnote{This is formalized in the \href{https://agda.github.io/agda-stdlib/Induction.WellFounded.html}{standard-library} with many other examples.}, so we let ourselves be inspired by these methods. In particular, we prove an irrelevance lemma
\ExecuteMetaData[Extra/Category/WellFounded]{Wf-rec-irr}
which implies the unfolding lemma
\ExecuteMetaData[Extra/Category/WellFounded]{Wf-rec-unfold}
The unfolding lemma ensures that the map we defined by \AgdaFunction{Wf-rec} is a map of algebras. The proof that this map is unique proceeds analogously to that in the proof that $\mu D$ is initial, but here we instead use \AgdaDatatype{Acc}-recursion
\ExecuteMetaData[Extra/Category/WellFounded]{Wf+inj=Init}
Thus, we conclude that $X$ is initial. The main result is then a corollary of initiality of $X$ and the isomorphism of initial objects
\ExecuteMetaData[Extra/Category/WellFounded]{Wf+inj=mu}


\subsubsection{Example}
Let us redo the proof in \autoref{sec:leibniz}, now using this result. Recall the description of naturals \AgdaFunction{NatD}. To show that \AgdaFunction{Leibniz} is isomorphic to \AgdaFunction{Nat}, we will need a \AgdaFunction{NatD}-algebra and a proof of its well-foundedness. We define the algebra
\ExecuteMetaData[Tex/Leibniz2]{bsuc'}

For well-foundedness, we use something similar to view-patterns %[mcbride]
(the main difference being that we look through the entire structure, instead of a single layer)
\ExecuteMetaData[Tex/Leibniz2]{Peano-View}
where the mutually recursive proof of \AgdaFunction{view} is ``almost trivial''. Well-foundedness follows fairly immediately
\ExecuteMetaData[Tex/Leibniz2]{Wf-bsuc}

Injectivity of \AgdaFunction{bsuc\_1} happens to be harder to prove from retractions than directly, so we prove it directly, from which the wanted statement follows
\ExecuteMetaData[Tex/Leibniz2]{L-is-mu-N}

Note that in this case it took us more code to prove the same statement! However, we stress that the code that we did write became more forced, and might be more amenable to automation.




\begin{comment}
\section{FingerTrees}\label{sec:fingertrees}
Finger trees are often (rightfully so) referred to as ``the fastest persistent datastructure for most purposes'', but while simpler than implementations achieving the same bounds, they are still challenging to reason about; in this section, we will investigate how we can fit the description and analysis of fingertrees, or variants upon them, into the frameworks of calculating datastructures and ornamental programming.

We compare the work in calculating datastructures to solving associativity equations in groups by shifting to the Cayley representation, such as in [..]


%\section{Discussion and Future Work}\label{sec:discussion}


\newpage
\section{Temporary}\label{sec:temp}
\listoftodos
%\subfile{Scratch.tex}
\end{comment}



\section{Related work}\label{sec:resources}
\towrite{Adapt this to the non-proposal form}

\subsection{The Structure Identity Principle}
If we write a program, and replace an expression by an equal one, then we can prove that the behaviour of the program can not change. Likewise, if we replace one implementation of an interface with another, in such a way that the correspondence respects all operations in the interface, then the implementations should be equal when viewed through the interface. Observations like these are instances of ``representation indepencence'', but even in languages with an internal notation of type equality, the applicability is usually exclusive to the metatheory.

In our case, moving from Agda's ``usual type theory'' to Cubical Agda, a cubical homotopy type theory, \textit{univalence} \cite{cuagda} lets us internalize a kind of representation independence known as the Structure Identity Principle \cite{iri}, and even generalize it from equivalences to quasi-equivalence relations. We will also be able to apply univalence to get a true ``equational reasoning'' for types when we are looking at numerical representations.

Still, representation independence in non-homotopical settings may be internalized in some cases \cite{tgalois}, and remains of interest in the context of generic constructions that conflict with cubical.

\subsection{Numerical Representations}
Rather than equating implementations after the fact, we can also ``compute'' datastructures by imposing equations. In the case of container types, one may observe similarities to number systems \cite{purelyfunctional} and call such containers numerical representations. One can then use these representations to prototype new datastructures that automatically inherit properties and equalities from their underlying number systems \cite{calcdata}.

From another perspective, numerical representations run by using representability as a kind of ``strictification'' of types, suggesting that we may be able to generalize the approach of numerical representations, using that any (non-indexed) infinitary inductive-recursive type supports a lookup operation \cite{glookup}.

% In the original setup \cite{calcdata}, the chains of equality reasoning over types had to be unfolded to transport a property from ``natural lookup tables'' to vectors. We expect that one might generalize the SIP to support indexed types, and use this to directly transport proofs from one side of the equality to the other.

\subsection{Ornamentation}
While we can derive datastructures from number systems by going through their index types \cite{calcdata}, we may also interpret numerical representations more literally as intstructions to rewrite a number system to a container type. We can record this transformation internally using ornaments, which can then be used to derive an indexed version of the container \cite{algorn}, or can be modified further to naturally integrate other constraints, e.g., ordering, into the resulting structure \cite{progorn}. Furthermore, we can also use the forgetful functions induced by ornaments to generate specifications for functions defined on the ornamented types \cite{orntrans}.

\subsection{Generic constructions}
Being able to define a datatype and reflect its structure in the same language opens doors to many more interesting constructions \cite{practgen}; a lot of ``recipes'' we recognize, such as defining the eliminators for a given datatype, can be formalized and automated using reflection and macros. We expect that other type transformations can also be interpreted as ornaments, like the extraction of heterogeneous binary trees from level-polymorphic binary trees \cite{hetbin}. 



% 
\section{Research Question and Contributions}\label{sec:research-question}
The research question of this project will be: \textit{can we describe finger trees \cite{fingertrees} in the frameworks of numerical representations and ornamentation \cite{progorn}, simplifying the verification of their properties as flexible two-sided arrays?} This question generates a number of interesting subproblems, such as that the number system corresponding to finger trees has many representations for the same number, which we expect to describe using quotients \cite{cuagda} and reason about using representation independence \cite{iri}. If this is accomplished or deemed infeasible at an early stage, we can generalize the results we have to other related problems; for example, we may view the problem of generating arbitrary values for testing as an instance of an enumeration problem, through the lens of ornaments.

\section{Planning}\label{sec:planning}
In the planning of the project, we identify four main topics.

\subsection{Finger trees}
In the context of numerical representations, we will define and test variants of finger trees. Due to the 2-3 tree structure of the original finger trees \cite{fingertrees}, finger trees are not readily rendered as numerical representations, leading us to the following subexperiments.

First, we will attempt to simplify the definition of finger trees, and test how this changes the performance bounds on their two-sided flexible array operations. Then, we can compute the ``trivial numerical representation'' on the original finger trees, and check to what extent the arising representation simplifies the proofs of the two-sided flexible array laws. Finally, we may try to forget about finger trees for a moment, and try to construct different numerical representations, achieving a subset of, or ideally all of, the performance bounds of finger trees.

Furthermore, the numerical representation of any ``symmetric array'' like finger trees seems to have a redundant associated number system. We know that for most operations, we can either simply ignore this, or place the type in a quotient (quotienting over the fibers of a map directly, or applying quasi-equivalence relations). However, indexing a quotiented type remains challenging; so  as further work, it may be interesting to find a non-redundant symmetric numerical representation, or investigate index types for quotient types further.

\subsection{Enumeration}
To characterize numerical representations, we first have to describe number systems; from one point of view, we can accept any type with a surjective interpretation into naturals as a number system. This description also allows for redundant number systems. The other point of view is that a number system must be countably infinite, which ensures that the system can be made non-redundant by enumerating it.

We can approach the problem from the other side, and look at enumerations first, investigating how large classes of W-types can always be equipped with an enumeration structure. As side-questions to this, we can look at applications of enumeration to random testing, where not only the existence of the enumeration matters, but also the ``fairness'' and the memory usage. We will investigate if and how enumerations can return unbalanced shapes, and how we can ameliorate this; also keeping in mind how the memory usage of enumerations can be reduced by avoiding the replication of identical subtrees.

\subsection{Ornaments}
In our preliminary work we apply ornaments to describe numerical representations, express heterogeneization, and we use descriptions to characterize equivalences to initial algebras. One downside is that each result uses a different definition of description or ornament, which all have their advantages and drawbacks. We identify the following interesting directions to further research ornaments:

Heterogeneization uses a variant of descriptions allowing parameter introduction, but this does not allow treating the parameter as a variable, nor supports nested types, which may be fruitful to generalize by changing descriptions to allow for higher order functors \cite{initenough}.

We can also restrict descriptions and ornaments to a closed universe, allowing us to avoid increasing the levels in the cubical compatible setup.  

Furthermore, we have not yet investigated the applicability of patches \cite{orntrans} to our experiments; these could be interesting when lifting flexible two-sided array operations from a number system to custom finger tree, but may also need adjustments to be able to deal with our modifications to descriptions.

Finally, we think that, like heterogeneization, there are more common and intuitive transformations of types which can be captured as ornament-computing functions.

% also descriptions make a mess, and even more so in a closed universe, but this should be resolvable using \cite{practgen}


\subsection{SIP}
The SIP as described earlier allows us to concisely prove the equivalences of implementations of structures. However, by definition of the SIP, this limited to structures over unindexed types, while in the context of vectors we may want to express a structure over an indexed type, in which case the indexes themselves may also be only equivalent rather than definitionally equal. Furthermore, the implementation we will use \cite{iri} restricts the basic structure formers; while in our scenarios we do not need much more complicated structures, we do expect the SIP to apply to structures containing most W-types with a free parameter. Solutions to both problems might be applicable to our research, so both may be interesting as further work.

% wouldn't having sigmas in structures be sufficient to emulate indexed types?
% > ah I guess that having a sigma-of-structures already means the ``snd structure'' is an ``indexed structure''
% but that should not be different from the current implemenation, shouldn't the structures always be applied before we have to prove them equivalent?


\begin{longtable}{l l}
Date & Target \\
\hline
2023-04-24 & Finger trees               \\
2023-05-01 & "              \\
2023-05-08 & "         \\
2023-05-15 & Enumerations                                        \\
2023-05-22 & "        \\
2023-05-29 & "                                                                  \\
2023-06-05 & "  \\
2023-06-12 & Ornamentation                                                                  \\
2023-06-19 & "                                           \\
2023-06-26 & "                                   \\
2023-07-03 & Holiday                                                            \\
2023-07-10 & ?                                                                  \\
2023-07-17 & "                                                                  \\
2023-07-24 & "                                                                  \\
2023-07-31 & "                                                                  \\
2023-08-07 & "                                                                  \\
2023-08-14 & "                                                                  \\
2023-08-21 & "                                                                  \\
2023-08-28 & ?                                                                  \\
2023-09-04 & ?                                                                  \\
2023-09-11 & SIP                                    \\
2023-09-18 & "                                                                  \\
2023-09-25 & TBD\footnote{This slot is flexible, and can be filled by one of the earlier experiments if I find that one of them requires more time, or may be filled by another experiment should I encounter new interesting and relevant questions}                                \\
2023-10-02 & "                                                              \\
2023-10-09 & Write                                                                  \\
2023-10-16 & "                                                                  \\
2023-10-23 & "                                                                  \\
2023-10-30 & "                                                                  \\
2023-11-06 & "                                                                  \\
2023-11-13 & "                                                                  \\
2023-11-20 & Prepare presentation                                               \\
2023-11-27 & "                                                                  \\
2023-12-04 & "                                                                  \\
2023-12-11 & Present thesis                                                     \\
2023-12-18 & -                                                                  \\
2023-12-22 & End date of research project                                       \\
\caption{The proposed planning for the research project.}
\end{longtable}

\begin{comment}
\begin{longtable}{l l}
Date & Target \\
\hline
2023-04-24 & Define and work out (better) simplified finger trees               \\
2023-05-01 & Force representability onto conventional finger trees              \\
2023-05-08 & Is there an ethical numrep with the bounds of finger trees?        \\
2023-05-15 & Experiment with enumeration                                        \\
2023-05-22 & How fair is enumeration/can we make better use of sharing?         \\
2023-05-29 & "                                                                  \\
2023-06-05 & Vectors are indexed, finger trees are not, SIP for indexed types?  \\
2023-06-12 & "                                                                  \\
2023-06-19 & Small universe ornaments                                           \\
2023-06-26 & Find out what HSIP means for us                                    \\
2023-07-03 & Holiday                                                            \\
2023-07-10 & ?                                                                  \\
2023-07-17 & "                                                                  \\
2023-07-24 & "                                                                  \\
2023-07-31 & "                                                                  \\
2023-08-07 & "                                                                  \\
2023-08-14 & "                                                                  \\
2023-08-21 & "                                                                  \\
2023-08-28 & ?                                                                  \\
2023-09-04 & ?                                                                  \\
2023-09-11 & Find more generic constructions                                    \\
2023-09-18 & "                                                                  \\
2023-09-25 & Can patches work better in C-c C-,                                 \\
2023-10-02 & Write                                                              \\
2023-10-09 & "                                                                  \\
2023-10-16 & "                                                                  \\
2023-10-23 & "                                                                  \\
2023-10-30 & "                                                                  \\
2023-11-06 & "                                                                  \\
2023-11-13 & "                                                                  \\
2023-11-20 & Prepare presentation                                               \\
2023-11-27 & "                                                                  \\
2023-12-04 & "                                                                  \\
2023-12-11 & Present thesis                                                     \\
2023-12-18 & -                                                                  \\
2023-12-22 & End date of research project                                       \\
\caption{The proposed planning for the research project.}
\end{longtable}
\end{comment}





\printbibliography
\end{document}
