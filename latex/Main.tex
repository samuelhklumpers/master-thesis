\documentclass[10pt]{article}

\usepackage[style=alphabetic]{biblatex}
\addbibresource{refs.bib}

\usepackage{comment}

\setlength{\marginparwidth}{2cm} % remove when done

\usepackage{todonotes}
\usepackage{xcolor}
\usepackage[hidelinks]{hyperref}


\usepackage{catchfilebetweentags}
\usepackage{quiver} 
\usepackage{tabularx}
%\usepackage{adjustbox}
%\usepackage{longtable}
\usepackage{amsthm}
\usepackage{amsmath}
\usepackage{listings}

\theoremstyle{plain}
\newtheorem{theorem}{Theorem}[section]
\newtheorem{lemma}[theorem]{Lemma}
\newtheorem{prop}[theorem]{Proposition}
\newtheorem{conjecture}{Conjecture}[section]
\newtheorem*{cor}{Corollary}

\theoremstyle{definition}
\newtheorem{defn}{Definition}[section]
\newtheorem{remark}{Remark}[section]
\newtheorem{claim}{Claim}[section]
\newtheorem{example}{Example}[section]

\renewcommand{\partautorefname}{Part}
\renewcommand{\sectionautorefname}{Section}
\renewcommand{\subsectionautorefname}{Subsection}

\providecommand{\theoremautorefname}{Theorem}
\providecommand{\lemmaautorefname}{Lemma}
\providecommand{\propautorefname}{Proposition}
\providecommand{\conjectureautorefname}{Conjecture}
\providecommand{\corautorefname}{Corollary}
\providecommand{\defnautorefname}{Definition}
\providecommand{\remarkautorefname}{Remark}
\providecommand{\exampleautorefname}{Example}
\providecommand{\claimautorefname}{Claim}


\usepackage[links]{agda}
%\AtBeginEnvironment{code}{\fontsize{8}{10}}
\AgdaNoSpaceAroundCode{}

\usepackage{fontspec}
\usepackage{luaotfload}

\directlua{luaotfload.add_fallback
  ("myfallback",
    { "JuliaMono:style=Regular;"
    , "NotoSansMono:style=Regular;"
    , "NotoSansMath:style=Regular;"
    , "Segoe UI Emoji:mode=harf;"
    }
  )}
\defaultfontfeatures{RawFeature={fallback=myfallback}}

\setmainfont{Latin Modern Roman}

\newfontfamily{\AgdaSerifFont}{JuliaMono Regular}[Scale=0.8]
\newfontfamily{\AgdaSansSerifFont}{JuliaMono Regular}[Scale=0.8]
\newfontfamily{\AgdaTypewriterFont}{JuliaMono Regular}[Scale=0.8]
\setmonofont{JuliaMono Regular}[Scale=0.8]
\renewcommand{\AgdaFontStyle}[1]{{\AgdaSansSerifFont{}#1}}
\renewcommand{\AgdaKeywordFontStyle}[1]{{\AgdaSansSerifFont{}#1}}
\renewcommand{\AgdaStringFontStyle}[1]{{\AgdaTypewriterFont{}#1}}
\renewcommand{\AgdaCommentFontStyle}[1]{{\AgdaTypewriterFont{}#1}}
\renewcommand{\AgdaBoundFontStyle}[1]{\textit{\AgdaSerifFont{}#1}}

\newcommand{\AF}[1]{\AgdaFunction{#1}}
\newcommand{\AD}[1]{\AgdaDatatype{#1}}
\newcommand{\AV}[1]{\AgdaBoundFontStyle{#1}}
\newcommand{\AIC}[1]{\AgdaInductiveConstructor{#1}}
\newcommand{\ACC}[1]{\AgdaCoinductiveConstructor{#1}}

\definecolor{git-green}{HTML}{13A10E}
\definecolor{git-orange}{HTML}{C69026}
\definecolor{nondescriptyellow}{HTML}{D6B656}

\newcommand{\added}[1]{\textcolor{git-green}{+#1}}
\newcommand{\changed}[1]{\textcolor{git-orange}{$\bullet$#1}}
\newcommand{\towrite}[1]{\todo[color=cyan]{#1}}
\newcommand{\toremove}[1]{\textcolor{red}{This is going to be (re)moved: ``#1''}}

  
\lstnewenvironment{semicomment}{\color{gray}\lstset{breaklines=true}}{}
\lstnewenvironment{outline}{\color{nondescriptyellow}\lstset{breaklines=true}}{}

\newcommand{\investigate}[1]{\par\vspace{1\baselineskip}\textcolor{gray}{#1}\vspace{1\baselineskip}\par}

\newcommand{\bN}{\AgdaDatatype{ℕ}}
\newcommand{\bL}{\AgdaDatatype{Leibniz}}


\title{Ornaments and Proof Transport applied to Numerical Representations}
\author{Samuel Klumpers\\6057314}


\begin{document}
\maketitle

\begin{abstract}
\begin{semicomment}
This thesis explains the concepts numerical representations and ornaments, and aims to combine these to simplify the presentation and verification of finger trees. We demonstrate the generalizability and easier verification of the resulting code. Further, we also investigate to which extent descriptions and ornaments, and generic programs built on top of these, remain effective in a setting without axiom K.
\end{semicomment}
\end{abstract}
    

\tableofcontents

\listoftodos

\section{Introduction}\label{sec:introduction}
\section{Introduction}\label{sec:intro}
Agda \cite{agda} is a functional programming language and a proof assistant, taking inspiration from languages like Haskell and other proof assistants like Coq. We can write programs like we would in Haskell, and then express and prove their properties all inside Agda. This allows us to demonstrate the correctness of programs by formal proof%, which is sound,
rather than by testing%, which is complete
; of course, producing an often tedious proof typically demands more effort than covering the relevant code with testcases.

In this thesis, we will explore some methods of proving properties of our programs, focussing on the problems or inconveniences that may arise, and how to deal with them. Let us sketch some problems and their remedies to get an idea of what awaits us, before we dive into the nitty-gritty details.

First, merely adapting a program to Agda may already require changes to the datatypes used in it; for example, if a program manipulating a \AgdaDatatype{List} uses the unsafe \AgdaFunction{head} function, then one is forced to replace the \AgdaDatatype{List} by a datatype that ensures non-emptyness, such as a \AgdaDatatype{NonEmpty} list or a length-aware vector \AgdaDatatype{Vec}. On the other hand, there might be sections of a program where the concrete length is not relevant for correctness and only gets in the way. As a result, one might find themselves duplicating common functions like concatenation \AgdaFunction{\_++\_} and only altering their signatures.

Clearly we would not be writing this if there was no way out, which there is! Often, the ``new'' datatype (\AgdaDatatype{Vec}) is simply a variation on the old datatype (\AgdaDatatype{List}) making small adjustments to the existing constructors; in this case, we decorate the nil and cons constructors with natural numbers representing the length. This kind of modification of types falls in the framework of ornamentation as described by Ko and Gibbons \cite{progorn}; if two types are reified to their \textit{descriptions}, then \textit{ornaments} express whether the types are ``similar'' by acting as a recipe to produce one type from the other. By restricting the operations to the copying of corresponding parts, and the introduction of fields or dropping of indices, the existence of such an ornament ensures that the types have the same recursive structure.

\towrite{Something about patches.}

\towrite{For each invariant a new datatype? Still ornaments}

Now that we know we can decently satisfyingly organize similar datatypes, it is time to look at dissimilar datatypes. It is certainly not foolish to prototype a program using simpler types or implementations, and only replace these with more performant alternatives in critical places; knowing that this is eventually going to happen, one might as well prepare for it. While this may quickly turn into a refactoring nightmare in the general case, we can hope for a more satisfying transition if we restrict our attention to a more narrow scope. As an example, we might start programming something with a \AgdaDatatype{List}, but replace this with a \AgdaDatatype{Tree} if we notice that the program spends most of its time in \AgdaFunction{lookup} operations. We are of course reimplementing the operations on \AgdaDatatype{Tree} utilizing their binary nature to gain a speedup, but it also seems as though we are about to double the number of necessary proofs; however, we have two ways to avoid this problem. 

We will look at the more specific solution first. This solution is guided by the realization that \AgdaDatatype{List} and \AgdaDatatype{Tree}, like most other containers, still have something in common even if their recursive structure is very different. That is, both resemble a number system, and, Okasaki \cite{purelyfunctional} notes that this resemblance to number systems is ``surprisingly common''. In the case of lists and Braun trees\footnote{Braun trees are a kind of binary tree, of which its shape is determined by its size.}, one can present both by deriving them from unary and binary numbers respectively, as is made formal by Hinze and Swierstra in \cite{calcdata}. One can then apply this \textit{numerical representation} to simplify or make trivial the proofs of the properties we hesitated to duplicate before.

\towrite{If we instead hide our datatypes behind interfaces, we can use proof transport as an alternative.}


\section{\toremove{Introduction (old)}}
%The dependently typed functional programming language Agda \cite{agda} can, when restricted to its reasonable parts, be translated into readable and safe Haskell \cite{agda2hs}. However, the intrinsic safety of languages like Agda can also lead to code duplication by encouraging the use of multiple variants of the same datatype. As an example, the coverage check forces the \AgdaFunction{head} function on \AgdaDatatype{List} to return a \AgdaDatatype{Maybe}. This \AgdaDatatype{Maybe} can be avoided by moving to the length-indexed list type \AgdaDatatype{Vec}, at the cost of duplicating functions like \AgdaFunction{\_++\_}, which we need at both types.

Something similar happens when replacing an implementation with a more efficient one. For example, when implementing binary trees as a more efficient alternative to lists, the proofs of the same properties will differ between list and tree, and tend to be more difficult for the latter. Switching between implementations of an interface not only duplicates code, but also (and sometimes more than) doubles the effort required to verify both.

%There is plenty of prior work dealing with problems like these. The work in \cite{orntrans} and \cite{progorn} provides the means to relate similar datatypes, such as lists and vectors, using the mechanism of ornamentation, letting us organize variants of the same datatype in a rigid framework.  %This leads them to define the concept of patches, which can aid us when defining \AgdaFunction{\_++\_} for the second time by forcing the new version to be coherent.
%In fact, the algebraic nature of ornaments yields the definition of the vector type for free, provided we relate lists to natural numbers \cite{algorn}. %Such constructions rely heavily on descriptions of datastructures and often come with limitations in their expressiveness. These descriptions in turn impose additional ballast on the programmer, leading us to investigate reflection like in \cite{practgen} as a means to bring datatypes and descriptions closer when possible.

Other work like \cite{calcdata} simplifies the proofs relating to certain containers directly, formally executing the way of though of numerical representations as noted in \cite{purelyfunctional}.
%From another point of view, lists and trees are not so different at all, provided we look at them through the interface of one-sided flexible arrays; this idea noted in \cite{purelyfunctional} and formalized in \cite{calcdata} where both are shown to be instances of numerical representations by calculating them from a numeral system. 

When two types are isomorphic and equivalent under an interface, proofs of properties of these implementations should be interconvertible. By using structured equivalences and univalence, \cite{iri} characterizes equivalences under interfaces.
%While this is achievable through meta-programming, substituting conversions to and from into the proof terms, this is internally expressible in Cubical Agda.

%We can liken the situation to movement on a plane, where ornamentation moves us vertically by modifying constructors or indices, and structured equivalences move us horizontally to and from equivalent but more equivalent implementations. In this paper, we will investigate a variety of means of moving around structures and proofs, and ways to make this more efficient or less intrusive.

In \autoref{sec:leibniz}, we will follow \cite{iri}, and look at how proofs on unary naturals can be transported to the binary naturals. Then in \autoref{sec:numrep} we recall how numeral systems in particular induce container types in \cite{calcdata}, which we attempt to reformulate in the language of ornaments in \autoref{ssec:ornaments}, using the framework of \cite{progorn}. In \autoref{sec:userfriendly} we investigate how we can make the earlier methods more easily accessible to the user, and, ourselves, when we give a description of finger trees in \autoref{sec:fingertrees}.




\part*{Background}\label{part:background}
\addcontentsline{toc}{part}{\nameref{part:background}}
We extend upon and make heavy use of existing work for generic programming and ornaments, so let us take a closer look at the nuts and bolts to see what all the concepts are about.

\section{Agda}\label{sec:background-agda}
We formalize our work in the programming language Agda \cite{agda}. While we will only occasionally reference Haskell, those more familiar with Haskell might understand (the reasonable part of) Agda as the subset of total Haskell programs \cite{agda2hs}.

Agda is a total functional programming language with dependent types. Here, totality means that functions of a given type always terminate in a value of that type, ruling out non-terminating (and not obviously terminating) programs. Using dependent types we can use Agda as a proof assistant, allowing us to state and prove theorems about our datastructures and programs. 

In this section, we will explain and highlight some parts of Agda which we use in the later sections. 
Many of the types we use in this section are also described and explained in most Agda tutorials (\cite{ulftutorial}, \cite{plfa}, etc.), and can be imported from the standard library \cite{agdastdlib}.

%Note that we use \texttt{--type-in-type} to keep the explanations more readable. 
%and \texttt{--with-K}
% Even though the former makes Agda inconsistent, and the latter is not strictly necessary, we know that our work can be ported to a setting with neither option \autoref{app:withoutK}.


\section{Data in Agda}\label{sec:background-data}
At the level of generalized algebraic datatypes Agda is close to Haskell. In both languages, one can define objects using data declarations, and interact with them using function declarations. For example, we can define the type of \emph{booleans}:
\ExecuteMetaData[Tex/Background]{Bool}
The constructors of this type state that we can make values of \AD{Bool} in exactly two ways: \AIC{false} and \AIC{true}. We can then define functions on \AD{Bool} by pattern matching. As an example, we can define the conditional operator as
\ExecuteMetaData[Tex/Background]{conditional}
When \emph{pattern matching}, the coverage checker ensures we define the function on all cases of the type matched on, and thus the function is completely defined. % mark: shuffle

We can also define a type representing the natural numbers
\ExecuteMetaData[Tex/Background]{Nat}
Here, \bN{} always has a \AF{zero} element, and for each element $n$ the constructor \AIC{suc} expresses that there is also an element representing $n + 1$. Hence, \bN{} represents the \textit{naturals} by encoding the existential axioms of the Peano axioms. By pattern matching and recursion on \bN{}, we define the less-than operator:
\ExecuteMetaData[Tex/Background]{lt}
One of the cases contains a recursive instance of \bN{}, so termination checker also verifies that this recursion indeed terminates, ensuring that we still define \AV{n}\ \AF{<?} \AV{m} for all possible combinations of \AV{n} and \AV{m}. %Essentially, the coverage and termination checker make sure that any valid definition by pattern matching corresponds to a valid proof by cases and induction.
In this case the recursion is valid, since both arguments decrease before the recursive call, meaning that at some point \AV{n} or \AV{m} hits \AIC{zero} and the recursion terminates.

Like in Haskell, we can \emph{parametrize} a datatype over other types to make \emph{polymorphic} type, which we can use to define lists of values for all types:
\ExecuteMetaData[Tex/Background]{List} 
A list of \AV{A} can either be empty \AIC{[]}, or contain an element of \AV{A} and another list via \AIC{\_∷\_}. In other words, \AD{List} is a type of \emph{finite sequences} in \AV{A} (in the sense of sequences as an abstract type \cite{purelyfunctional}).

Using polymorphic functions, we can manipulate and inspect lists by inserting or extracting elements. For example, we can define a function to look up the value at some position \AV{n} in a list
\ExecuteMetaData[Tex/Background]{lookup-list}
However, this function \emph{partial}, as we are relying on the type
\ExecuteMetaData[Tex/Background]{Maybe}
to handle the case where the position falls outside the list and we cannot return an element. 
If we know the length of the list \AV{xs}, then we also know for which positions \AF{lookup} will succeed, and for which it will not. We define 
\ExecuteMetaData[Tex/Background]{length}
so that we can test whether the position \AV{n} lies inside the list by checking \AV{n}\ \AF{<?}\ \AF{length}\ \AV{xs}. If we declare \AF{lookup} as a dependent function consuming a proof of \AV{n}\ \AF{<?}\ \AF{length}\ \AV{xs}, then \AF{lookup} always succeeds. However, this actually only moves the burden of checking whether the output was \AIC{nothing} afterwards to proving that \AV{n}\ \AF{<?}\ \AF{length}\ \AV{xs} beforehand.

We can avoid both by defining an \emph{indexed type} representing numbers below an upper bound
\ExecuteMetaData[Tex/Background]{Fin}
Like parameters, indices add a variable to the context of a datatype, but unlike parameters, indices can influence the availability of constructors. The type \AD{Fin} is defined such that a variable of type \AD{Fin}\ \AV{n} represents a number less than \AV{n}. Since both constructors \AIC{zero} and \AIC{suc} dictate that the index is the \AIC{suc} of some natural \AV{n}, we see that \AD{Fin}\ \AIC{zero} has no values. On the other hand, \AIC{suc} gives a value of \AD{Fin}\ (\AIC{suc}\ \AV{n}) for each value of \AD{Fin}\ \AV{n}, and \AIC{zero} gives exactly one additional value of \AD{Fin}\ (\AIC{suc}\ \AV{n}) for each \AV{n}. By induction (externally), we find that \AD{Fin}\ \AV{n} has exactly \AV{n} closed terms, each representing a number less than \AV{n}.

To complement \AD{Fin}, we define another indexed type representing lists of a known length, also known as vectors:
\ExecuteMetaData[Tex/Background]{Vec}
The \AIC{[]} constructor of this type produces the only term of type \AD{Vec}\ \AV{A}\ \AIC{zero}. The \AIC{\_∷\_} constructor ensures that a \AD{Vec}\ \AV{A}\ (\AIC{suc}\ \AV{n}) always consists of an element of \AV{A} and a \AD{Vec}\ \AV{A}\ \AV{n}. By induction, we find that a \AD{Vec}\ \AV{A}\ \AV{n} contains exactly \AV{n} elements of \AV{A}. Thus, we conclude that \AD{Fin}\ \AV{n} is exactly the type of positions in a \AD{Vec}\ \AV{A}\ \AV{n}. In comparison to \AD{List}, we can say that \AD{Vec} is a type of arrays (in the sense of arrays as the abstract type of sequences of a fixed length). Furthermore, knowing the index of a term \AV{xs} of type \AV{Vec}\ \AV{A}\ \AV{n} uniquely determines the the constructor it was formed by. Namely, if \AV{n} is \AIC{zero}, then \AV{xs} is \AIC{[]}, and if \AV{n} is \AIC{suc} of \AV{m}, then \AV{xs} is formed by \AIC{\_∷\_}. 

Using this, we define a variant of \AF{lookup} for \AD{Fin} and \AD{Vec}, taking a vector of length \AV{n} and a position below \AV{n}:
\ExecuteMetaData[Tex/Background]{lookup}
The case in which we would return \AIC{nothing} for lists, which is when \AV{xs} is \AIC{[]}, is omitted. This happens because \AV{x} of type \AD{Fin}\ \AV{n} is either \AIC{zero} or \AIC{suc}\ \AV{i}, and both cases imply that \AV{n} is \AIC{suc}\ \AV{m} for some \AV{m}. As we saw above, a \AD{Vec}\ \AV{A}\ (\AIC{suc} \AV{m}) is always formed by \AIC{\_∷\_}, making the case in which \AV{xs} is \AIC{[]} impossible. Consequently, lookup always succeeds for vectors,
% demonstrating that vectors are correct-by-construction. 
however, this does not yet prove that \AF{lookup} necessarily returns the right element, we will need some more logic to verify this.

\section{Proving in Agda}\label{sec:background-proving}
To describe equality of terms we define a new type
\ExecuteMetaData[Tex/Background]{equiv}
If we have a value \AV{x} of \AV{a}\ \AD{≡}\ \AV{b}, then, as the only constructor of \AD{\_≡\_} is \AIC{refl}, we must have that \AV{a} is equal to \AV{b}. We can use this type to describe the behaviour of functions like \AF{lookup}: If we insert elements into a vector with
\ExecuteMetaData[Tex/Background]{insert}
we can express the correctness of \AF{lookup} as
\ExecuteMetaData[Tex/Background]{lookup-insert-type}
stating that we expect to find an element where we insert it.

% When we use pattern matching in a function, the coverage and termination checker ensure that the resulting function is total and defined by well-founded recursion\cite{?}. If we are proving some statement by constructing a function as a proof, this means that we can interpret a function definition by (dependent) pattern matching and well-founded recursion as a proof by well-founded induction\cite{?}.

%So, to 
To prove the statement, we proceed as when defining any other function. 
By simultaneous induction on the position and vector, we prove
\ExecuteMetaData[Tex/Background]{lookup-insert}
In the first two cases, where we \AF{lookup} the first position, \AF{insert}\ \AV{xs}\ \AIC{zero}\ \AV{y} simplifies to \AV{y}\ \AF{∷}\ \AF{xs}, so the lookup immediately returns \AV{y} as wanted. In the last case, we have to prove that \AF{lookup} is correct for \AV{x}\ \AF{∷}\ \AF{xs}, so we use that the \AF{lookup} ignores the term \AV{x} and we appeal to the correctness of \AF{lookup} on the smaller list \AV{xs} to complete the proof.

Like \AD{\_≡\_}, we can encode many other logical operations into datatypes, which establishes a correspondence between types and formulas, known as the Curry-Howard isomorphism. For example, we can encode disjunctions (the logical `or' operation) as
\ExecuteMetaData[Tex/Background]{uplus}

The other components of the isomorphism are as follows. Conjunction (logical `and') can be represented by\footnote{We use a record here, rather than a datatype with a constructor \AV{A → B →}\ \AV{A}\ \AD{×}\ \AV{B}. The advantage of using a record is that this directly gives us projections like \ARF{fst}\ \AV{:}\ \AV{A}\ \AD{×}\ \AV{B}\ \AV{→ A}, and lets us use eta equality, making $(a, b) = (c , d) \iff a = c \land b = d$ holds automatically.}
\ExecuteMetaData[Tex/Background]{product}
True and false are respectively represented by
\ExecuteMetaData[Tex/Background]{true}
so that always \AIC{tt}\ \AV{:}\ \AD{⊤}, and 
\ExecuteMetaData[Tex/Background]{false}
The body of \AD{⊥} is not accidentally left out: because \AD{⊥} has no constructors, there is no proof of false\footnote{If we did not use \AV{--type-in-type}, and even in that case I can only hope.}.

Because we identify function types with logical implications, we can also define the negation of a formula \AV{A} as ``\AV{A} implies false'':
\ExecuteMetaData[Tex/Background]{not}
The logical quantifiers $\forall$ and $\exists$ act on formulas with a free variable in a specific domain of discourse. We represent closed formulas by types, so we can represent a formula with a free variable of type \AV{A} by a function values of \AV{A} to types \AV{A}\ \AV{→}\ \AD{Type}, also known as a predicate. The universal quantifier $\forall a P(a)$ is true when for all $a$ the formula $P(a)$ is true, so we represent the universal quantification of a predicate \AV{P} as a dependent function type \AV{(a : A) → P a}, producing for each \AV{a} of type \AV{A} a proof of \AV{P}\ \AV{a}. The existential quantifier $\exists a P(a)$ is true when there is some $a$ such that $P(a)$ is true, so we represent the existential quantification as
\ExecuteMetaData[Tex/Background]{exists}
so that we have \AD{Σ}\ \AV{A}\ \AV{P} iff we have an element \AV{fst} of \AV{A} and a proof \AV{snd} of \AV{P}\ \AV{a}. To avoid the need for lambda abstractions in existentials, we define the syntax
\ExecuteMetaData[Tex/Background]{sigma-syntax}
letting us write \AD{Σ[}\ \AV{a}\ \AD{∈}\ \AV{A}\ \AD{]}\ \AV{P a} for $\exists a P(a)$.

\section{Descriptions}\label{sec:background-descriptions}
In the previous sections we completed a quadruple of types (\bN{}, \AD{List}, \AD{Vec}, \AD{Fin}), 
%, even computing the latter two from \bN{}.
which have nice interactions (\AF{length}, \AF{lookup}). Similar to the type of \AF{length}\ \AV{:}\ \AD{List}\ \AV{A}\ \AV{→}\ \bN{}, we can define
\ExecuteMetaData[Tex/Background]{toList}
converting vectors back to lists. In the other direction, we can also promote a list to a vector by recomputing its index:
\ExecuteMetaData[Tex/Background]{toVec}
We claim that is not a coincidence, but rather happens because \bN{}, \AD{List}, and \AD{Vec} have the same ``shape''.

But what is the shape of a datatype? In this section, we will explain a framework of datatype descriptions and ornaments, allowing us to describe the shapes of datatypes and use these for generic programming \cite{ulftutorial, genericsamm, effectfully, practgen}. Recall that while polymorphism allows us to write one program for many types at once, those programs act parametrically \cite{reynolds1983types, wadlerfree}: polymorphic functions must work for all types, thus they cannot inspect values of their type argument. Generic programs, by design, do use the structure of a datatype, allowing for more complex functions that do inspect values\footnote{Think of JSON encoding types with encodable fields \cite{truesop}, or deriving functor instances for a broad class of types \cite{haskellderiving}.}.

Using datatype descriptions we can then relate \bN{}, \AD{List} and \AD{Vec}, explaining how \AF{length} and \AF{toList} are instances of a generic construction. Let us walk through some ways of defining descriptions. We will start from simpler descriptions, building our way up to more general types, until we reach a framework in which we can describe \bN{}, \AD{List}, \AD{Vec} and \AD{Fin}. 
%, which, as a bonus, gives some insight into the meaning of datatypes.


\subsection{Finite types}\label{ssec:background-fin}
A datatype description, which are datatypes of which each value again represents a datatype, consist of two components. Namely, a type of descriptions \AV{U}, also referred to as codes, and an interpretation \AV{U}\ \AV{→}\ \AD{Type}, decoding descriptions to the represented types. In the terminology of Martin-L{\"{o}}f type theory (MLTT)\cite{levitation}, %\todo{No citation for MLTT? Agda is a rather loose extension, none of the original papers really match.}
where types of types like \AD{Type} are called universes, we can think of a type of descriptions as an internal universe.

As a start, we define a basic universe with two codes \AIC{𝟘} and \AIC{𝟙}, respectively representing the types \AD{⊥} and \AD{⊤}, and the requirement that the universe is closed under sums and products:
\ExecuteMetaData[Tex/Background]{U-fin}
The meaning of the codes in this universe is then assigned by the interpretation
\ExecuteMetaData[Tex/Background]{int-fin}
which indeed sends \AIC{𝟘} to \AD{⊥}, \AIC{𝟙} to \AD{⊤}, sums to sums and products to products\footnote{One might recognize that \AF{⟦\_⟧fin} is a morphism between the rings (\AD{U-fin}, \AIC{⊕}, \AIC{⊗}) and (\AD{Type}, \AD{⊎}, \AD{×}). Similarly, \AD{Fin} also gives a ring morphism from \bN{} with \AF{+} and \AF{×} to \AD{Type}, and in fact \AF{⟦\_⟧fin} factors through \AD{Fin} via the map sending the expressions in \AD{U-fin} to their value in \bN{}.}.

In this universe, we can encode the type of booleans simply as 
\ExecuteMetaData[Tex/Background]{BoolD}
The types \AIC{𝟘} and \AIC{𝟙} are finite, and sums and products of finite types are also finite, which is why we call \AD{U-fin} the universe of finite types. Consequently, the type of naturals \bN{} cannot fit in \AD{U-fin}.

\subsection{Recursive types}\label{ssec:background-rec}
To accommodate \bN{}, we need to be able to express recursive types. By adding a code \AIC{ρ} to \AD{U-fin} representing recursive type occurrences, we can express those types: 
\ExecuteMetaData[Tex/Background]{U-rec}
However, the interpretation cannot be defined like in the previous example: when interpreting \AIC{𝟙}\ \AIC{⊕}\ \AIC{ρ}, we need to know that the whole type was \AIC{𝟙}\ \AIC{⊕}\ \AIC{ρ} while processing \AIC{ρ}. As a consequence, we have to split the interpretation in two phases. First, we interpret the descriptions into polynomial functors
\ExecuteMetaData[Tex/Background]{int-rec}
Then, by viewing such a functor as a type with a free type variable, the functor can model a recursive type by setting the variable to the type itself:
\ExecuteMetaData[Tex/Background]{mu-rec}
Recall the definition of \bN{}, which can be read as the declaration that \AD{ℕ} is a fixpoint: \AD{ℕ}\ \AD{≡}\ \AV{F}\ \AD{ℕ} for \AV{F X = ⊤ ⊎ X}. This makes representing \bN{} as simple as:
\ExecuteMetaData[Tex/Background]{NatD}

\subsection{Sums of products}\label{ssec:background-sop}
A downside of \AD{U-rho} is that the definitions of types do not mirror their equivalent definitions in user-written Agda. We can define a similar universe using that polynomials can always be canonically written as sums of products. For this, we split the descriptions into a stage in which we can form sums, on top of a stage where we can form products.
\ExecuteMetaData[Tex/Background]{U-sop}
When doing this, we can also let the left-hand side of a product be any type, allowing us to represent ordinary fields:
\ExecuteMetaData[Tex/Background]{Con-sop}
The interpretation of this universe, while analogous to the one in the previous section, is also split into two parts:
\ExecuteMetaData[Tex/Background]{int-sop}
In this universe, we can define the type of lists as a description quantified over a type:
\ExecuteMetaData[Tex/Background]{ListD-bad}
Using this universe requires us to split functions on descriptions into multiple parts, but makes interconversion between representations and concrete types straightforward.

\subsection{Parametrized types}\label{ssec:background-par}
The encoding of fields in \AD{U-sop} makes the descriptions large in the following sense: by letting \AV{S} in \AIC{σ} be an infinite type, we can get a description referencing infinitely many other descriptions. As a consequence, we cannot inspect an arbitrary description in its entirety. We will introduce parameters in such a way that we recover the finiteness of descriptions as a bonus.

In the last section, we saw that we could define the parametrized type \AD{List} by quantifying over a type. However, in some cases, we will want to be able to inspect or modify the parameters belonging to a type. % mark: why
%footnote{For example, deriving Traversable for parametrized types as functions would not be possible (without macros), as one could not decide whether the signature of a type in a field is compatible.}
To represent the parameters of a type, we will need a new gadget.

In a naive attempt, we can represent the parameters of a type as \AD{List}\ \AD{Type}. However, this cannot represent many useful types, of which the parameters depend on each other. For example, in the existential quantifier \AD{Σ\_}, the type \AV{A}\ \AV{→}\ \AD{Type} of second parameter \AV{B} references back to the first parameter \AV{A}.

In a general parametrized type, parameters can refer to the values of all preceding parameters. The parameters of a type are thus a sequence of types depending on each other, which we call telescopes \cite{practgen, sijsling, telescopes} (also known as contexts in MLTT). We define telescopes using induction-recursion:
\ExecuteMetaData[Tex/Background]{Tel-simple}
A telescope can either be empty, or be formed from a telescope and a type in the context of that telescope. Here, we used the meaning of a telescope \AF{⟦\_⟧tel} to define types in the context of a telescope. This meaning represents the valid assignment of values to parameters:
\ExecuteMetaData[Tex/Background]{int-simple}
interpreting a telescope into the dependent product of all the parameter types.

This definition of telescopes would let us write down the type of \AD{Σ}:
\ExecuteMetaData[Tex/Background]{sigma-tel}
but is not sufficient to define \AD{Σ}, as we need to be able to bind a value \AV{a} of \AV{A} and reference it in the field \AV{P}\ \AV{a}. By quantifying telescopes over a type \cite{practgen}, we can represent bound arguments using almost the same setup:
\ExecuteMetaData[Tex/Background]{Tel-type}
A \AD{Tel}\ \AV{P} then represents a telescope for each value of \AV{P}, which we can view as a telescope in the context of \AV{P}. For readability, we redefine values in the context of a telescope as:
\ExecuteMetaData[Tex/Background]{entails}
so we can define telescopes and their interpretations as:
\ExecuteMetaData[Tex/Background]{Tel-def}
By setting \AV{P}\ \AV{=}\ \AD{⊤}, we recover the previous definition of parameter-telescopes. We can then define an extension of a telescope as a telescope in the context of a parameter telescope:
\ExecuteMetaData[Tex/Background]{ExTel}
representing a telescope of variables over the fixed parameter-telescope \AV{Γ}, which can be extended independently of \AV{Γ}. Extensions can be interpreted by interpreting the variable part given the interpretation of the parameter part:
\ExecuteMetaData[Tex/Background]{int-ExTel}
We will name maps \AV{Δ → Γ} of telescopes \AF{Cxf}\ \AV{Δ}\ \AV{Γ}. Given such a map \AV{g}, name maps \AV{W → V} between extensions \AF{Vxf}\ \AV{g}\ \AV{W}\ \AV{V}:
\ExecuteMetaData[Tex/Background]{tele-helpers} %mark: map-var
We also defined two functions we will use extensively later: \AF{var→par} states that a map of extensions extend to a map of the whole telescope, and \AF{Vxf-▷} lets us extend a map of extensions by acting as the identity on a new variable. 

In the descriptions directly relay the parameter telescope to the constructors, resetting the variable telescope to \AIC{∅} for each constructor:
\ExecuteMetaData[Tex/Background]{U-par}
Of the constructors we only modify the \AIC{σ} to request a type \AV{S} in the context of \AV{V}, and to extend the context for the subsequent fields by \AV{S}:
\ExecuteMetaData[Tex/Background]{Con-par}
Replacing the function \AV{S →}\ \AD{U-sop} by \AD{Con-par}\ (\AV{V}\ \AIC{▷}\ \AV{S}) allows us to bind the value of \AV{S} while avoiding the higher order argument. The interpretation of the universe is then:
\ExecuteMetaData[Tex/Background]{int-par}
In particular, we provide \AV{X} the parameters and variables in the \AIC{σ} case, and extend context by \AV{s} before passing to the rest of the interpretation.

In this universe, we can describe lists using a one-type telescope:
\ExecuteMetaData[Tex/Background]{ListD}
This description declares that \AD{List} has two constructors, one with no fields, corresponding to \AIC{[]}, and the second with one field and a recursive field, representing \AIC{\_∷\_}. In the second constructor, we used pattern lambdas to deconstruct the telescope\footnote{Due to a quirk in the interpretation of telescopes, the \AIC{∅} part always contributes a value \ARF{tt} we explicitly ignore, which also explicitly needs to be provided when passing parameters and variables.} and extract the type \AV{A}.
Using the variable bound in \AIC{σ}, we can also define the existential quantifier:
\ExecuteMetaData[Tex/Background]{SigmaD}
having one constructor with two fields. Here, the first field of type \AV{A} adds a value \AV{a} to the variable telescope, which we recover in the second field by pattern matching, before passing it to \AV{B}.


\subsection{Indexed types}\label{ssec:background-ix}
Lastly, we can integrate indexed types \cite{iir} into the universe by abstracting over indices
\ExecuteMetaData[Tex/Background]{U-ix}
Recall that in native Agda datatypes, a choice of constructor can fix the indices of the recursive fields and the resultant type, so we encode:
\ExecuteMetaData[Tex/Background]{Con-ix}
%In most cases, the index is simply threaded through the interpretation, allowing for a choice in the relevant codes.
If we are constructing a term of some indexed type, then the previous choices of constructors and arguments build up the actual index of this term. This actual index must then match the index we expected in the declaration of this term. This means that in the case of a leaf, we have to replace the unit type with the necessary equality between the expected and actual indices \cite{algorn}:
\ExecuteMetaData[Tex/Background]{int-ix}
In a recursive field, the expected index can be chosen based on parameters and variables. % mark: wording

In this universe, we can define finite types and vectors as:
\ExecuteMetaData[Tex/Background]{FinD}
and
\ExecuteMetaData[Tex/Background]{VecD}
These are equivalent, but since we do not model implicit fields, they are slightly different in use compared to \AD{Fin} and \AD{Vec}. In the first constructor of \AF{VecD} we report an actual index of \AIC{zero}. In the second, we have a field \bN{} to bring the index \AV{n} into scope, which is used to request a recursive field with index \AV{n}, and report the actual index of \AIC{suc}\ \AV{n}. 

Let us also show how the definitions of naturals and lists from earlier sections can be replicated in \AD{U-ix}
\ExecuteMetaData[Tex/Background]{new-Nat-List}
Writing the descriptions \AF{NatD}, \AF{ListD} and \AF{VecD} next to each other makes it easy to see the similarities: \AF{ListD} is the same as \AF{NatD} with a type parameter and one more \AIC{σ}. Likewise, \AF{VecD} is the same as \AF{ListD}, but now indexing over \bN{} and with yet one more \AIC{σ} of \bN{}. This kind of analysis is the focus of \autoref{sec:background-ornaments}.

\subsubsection{Generic Programming}
As a bonus, we can also use \AD{U-ix} for generic programming. For example, by a long construction which can be found in \autoref{app:gfold}, we can define the generic \AF{fold} operation:
\ExecuteMetaData[Tex/Background]{fold-type}
Let us describe how \AF{fold} works intuitively. We can interpret a term of \AF{⟦}\ \AV{D}\ \AF{⟧D}\ \AV{X} as a term of \AF{μ-ix}\ \AV{D}, where the recursive positions hold values of \AV{X} rather than values of \AF{μ-ix}\ \AV{D}. Then \AF{fold} states that a function collapsing such terms into values of \AV{X} extends to a function collapsing \AF{μ-ix}\ \AV{D} into \AV{X}, recursively collapsing applications of \AIC{con} from the bottom up.

As a more concrete example, when instantiating \AF{fold} to \AF{ListD}, the type \AF{⟦}\ \AV{ListD}\ \AF{⟧D}\ \AV{X} reduces (up to equivalence) to \AD{⊤}\ \AD{⊎}\ (\AV{A}\ \AD{×}\ \AV{X}\ \AV{A})\ \AF{→}\ \AV{X}\ \AV{A}, and \AF{fold} becomes
\ExecuteMetaData[Tex/Background]{foldr-type}
which, much like the familiar \AF{foldr} operation lets us consume a \AD{List}\ \AV{A} to produce a value \AV{X A}, provided a value \AV{X A} in the empty case, and a means to convert a pair (\AV{A}, \AV{X A}) to \AV{X A}.

Do note that this version takes a polymorphic function as an argument, as opposed to the usual fold which has the quantifiers on the outside:
\ExecuteMetaData[Tex/Background]{usual-fold}
Like a couple of constructions we will encounter in later sections, we can recover the usual fold into a type \AV{C} by generalizing \AV{C} to some kind of maps into \AV{C}. For example, by letting \AV{X} be continuation-passing computations into \bN{}, we can recover
\ExecuteMetaData[Tex/Background]{foldr-sum}


\section{Ornaments}\label{sec:background-ornaments}
In this section we will introduce a simplified definition of ornaments, which we will use to compare descriptions. Purely looking at their descriptions, \bN{} and \AD{List} are rather similar, except that \AD{List} has a parameter and an extra field \bN{} does not have. We could say that we can form the type of lists by starting from \bN{} and adding this parameter and field, while keeping everything else the same. In the other direction, we see that each list corresponds to a natural by stripping this information. Likewise, the type of vectors is almost identical to \AD{List}, can be formed from it by adding indices, and each vector corresponds to a list by dropping the indices.

Observations like these can be generalized using ornaments \cite{algorn, progorn, sijsling}, which define a binary relation describing which datatypes can be formed by ``decorating'' others. Conceptually, a type can be decorated by adding or modifying fields, extending its parameters, or refining its indices.

Essential to the concept of ornaments is the ability to convert back, forgetting the extra structure. After all, if there is an ornament from \AV{A} to \AV{B}, then \AV{B} is \AV{A} with extra fields and parameters, and more specific indices. In that case, we should also be able to discard those extra fields, parameters, and more specific indices, obtaining a conversion from \AV{B} to \AV{A}. If \AV{A} is a \AD{U-ix}\ \AV{Γ}\ \AV{I} and \AV{B} is a \AD{U-ix}\ \AV{Δ}\ \AV{J}, then a conversion from \AV{B} to \AV{A} presupposes a function \AV{re-par :}\ \AF{Cxf}\ \AV{Δ}\ \AV{Γ} for re-parametrization, and a function \AV{re-index :}\ \AV{J}\ \AV{→}\ \AV{I} for re-indexing.

In the same way that descriptions in \AD{U-ix} are lists of constructor descriptions, ornaments are lists of constructor ornaments. We define the type of ornaments reparametrizing with \AV{re-par} and reindexing with \AV{re-index} as a type indexed over \AD{U-ix}:
\ExecuteMetaData[Tex/Background]{Orn}
The conversion between types induced by an ornament is then embodied by the forgetful map
\ExecuteMetaData[Tex/Background]{bimap}
\ExecuteMetaData[Tex/Background]{ornForget-type}
which will revert the modifications made by the constructor ornaments, and restores the original indices and parameters.

The allowed modifications are controlled by the definition of constructor ornaments \AD{ConOrn}. We must keep in mind that each constructor of \AD{ConOrn} also has to be reverted by \AF{ornForget}, accordingly, some modifications have preconditions, which are in this case always pointwise equalities:
\ExecuteMetaData[Tex/Background]{htpy}
Since constructors exist in the context of variables, we let constructor ornaments transform variables with \AV{re-var}, in addition to parameters and indices.

The first three constructors of \AD{ConOrn} represent the operations which copy the corresponding constructors of \AD{Con-ix}\footnote{Viewing \AD{ConOrn} as a binary relation on \AD{Con-ix}, these represent the preservation of \AD{ConOrn} by \AIC{𝟙}, \AIC{ρ}, and \AIC{σ}, up to parameters, variables, and indices.}. The \AIC{Δσ} constructors allows one to add fields which are not present on the original datatype.
\ExecuteMetaData[Tex/Background]{ConOrn}
% yes re-par can be implicit most of the time
% when you actually start using ornaments generically, it will come back to bite you though
The commuting square \AF{re-index}\ \AF{∘}\ \AV{j}\ \AF{∼}\ \AV{i}\ \AF{∘}\ \AF{var→par}\ \AV{re-var} in the first two constructors ensures that the indices on both sides are indeed related, up to \AV{re-index} and \AV{re-var}.

Now, we can show that lists are indeed naturals decorated with fields:
\ExecuteMetaData[Tex/Background]{NatD-ListD}
This ornament preserves most structure of \bN{}, only adding a field using \AIC{∆σ}\footnote{Note that \AV{S}, and some later arguments we provide to ornaments, are implicit argument: Agda would happily infer them from \AF{ListD} and later \AF{VecD} had we omitted them.}. As \bN{} has no parameters or indices, \AD{List} has more specific parameters, namely a single type parameter. Consequently, all commuting squares factor through the unit type and can be satisfied with \AV{λ}\ \AV{\_}\ \AV{→}\ \AIC{refl}. 

We can also ornament lists to get vectors by reindexing them over \bN{}
\ExecuteMetaData[Tex/Background]{ListD-VecD}
We bind a new field of \bN{} with \AIC{∆σ}, extracting it in \AIC{𝟙} and \AIC{ρ} to declare that the constructor corresponding to \AIC{\_∷\_} takes a vector of length \AV{n} and returns a vector of length \AIC{suc}\ \AV{n}. 

The conversions from lists to naturals, and from vectors to lists are given by \AF{ornForget}. We define \AF{ornForget} as a \AF{fold} over an algebra that erases a single layer of decorations
\ExecuteMetaData[Tex/Background]{ornForget}
Recursively applying this algebra, which reinterprets values of \AV{E} as values of \AV{D}, lets us take apart a value in the fixpoint \AD{μ-ix}\ \AV{E} and rebuild it to a value of \AF{μ-ix}\ \AV{D}. This algebra
\ExecuteMetaData[Tex/Background]{ornAlg}
is a special case of the erasing function, which undecorates interpretations of arbitrary types \AV{X}:
\ExecuteMetaData[Tex/Background]{ornErase}
Reading off the ornament, we see which bits of \AV{CE} are new and which are copied from \AV{CD}, and consequently which parts of a term \AV{x} under an interpretation of \AV{CE} need to be forgotten, and which needs to be copied or translated. Specifically, the first three cases of \AF{conOrnErase} correspond to the structure-preserving ornaments, and merely translate equivalent structures from \AV{CE} to \AV{CD}.

For example, in the first case the ornament \AIC{𝟙}\ \AV{sq} copies leaves, telling us that \AV{CD} is \AIC{𝟙 i'} and \AV{CE} is \AIC{𝟙 j'}. The interpretation \AV{⟦ 𝟙 j' ⟧C \_ p j} of a leaf \AV{𝟙 j'} at parameters \AV{p} and index \AV{j} is simply the equality of expected and actual indices \AV{j ≡ (j' p)}. The term \AV{x} of \AV{j ≡ (j' p)}, then only has to be converted to the corresponding proof of equality on the \AV{CD} side: \AV{re-index j ≡ (i' (var→par re-var p))}. This is precisely accomplished by applying \AF{re-index} to both sides and composing with the square \AV{sq} at \AV{p}.

Likewise, in the case of \AIC{ρ} we only have to show that \AV{x} can be converted from one \AIC{ρ} to the other \AIC{ρ} by translating its parameters, and in the \AIC{σ} case the field is directly copied. The only other ornament \AIC{Δσ} adding fields, is easily undone by removing those fields. 

Thus, \AF{ornForget} establishes that \AV{E} in an ornament \AD{Orn}\ \AV{g}\ \AV{i}\ \AV{D}\ \AV{E} is an adorned version of \AV{D} by associating to each value of \AV{E} its an underlying value in \AV{D}. Additionally, \AF{ornForget} makes it simple to relate functions between related types. For example, instantiating \AF{ornForget} for \AF{NatD-ListD} yields \AF{length}. Hence, the statement that \AF{length} sends concatenation \AF{\_++\_} to addition \AF{\_+\_}, i.e. \AV{length (xs ++ ys) ≡ length xs + length ys}, is equivalent to the statement that \AF{\_++\_} and \AF{\_+\_} are related, or that \AF{\_++\_} is a lifting of \AF{\_+\_} \cite{orntrans}. %\marker{Ik hoop dat dit minder wazig is en de mental typechecking load wat reduceert.}

% remark, ornForget is not epi in general because of ∆σ ⊥

\section{Ornamental Descriptions}\label{sec:background-ornamental-descriptions}
By defining the ornaments \AF{NatD-ListD} and \AF{ListD-VecD} we could show that lists are numbers with fields and vectors are lists with fixed lengths. Even though we had to give \AF{ListD} before we could define \AF{NatD-ListD}, the value of \AF{NatD-ListD} actually forces the right-hand side to be \AF{ListD}.

This means we can also use an ornament to represent a description as a patch on top of another description, if we leave out the right-hand side of the ornament. Ornamental descriptions are precisely defined as ornaments without the right-hand side, and effectively bundle a description and an ornament to it\footnote{Consequently, \AD{OrnDesc}\ \AV{Δ}\ \AV{J}\ \AV{g}\ \AV{i}\ \AV{D} must simply be a convenient representation of \AD{Σ}\ (\AD{U-ix}\ \AV{Δ}\ \AV{J})\ (\AD{Orn}\ \AV{g}\ \AV{i}\ \AV{D}).}. Their definition is analogous to that of ornaments, making the arguments which would only appear in the new description explicit:
\ExecuteMetaData[Tex/Background]{OrnDesc}
\ExecuteMetaData[Tex/Background]{ConOrnDesc}
Using \AD{OrnDesc} we can describe lists as the patch on \AF{NatD} which inserts a \AIC{σ} in the constructor corresponding to \AIC{suc}:
\ExecuteMetaData[Tex/Background]{NatOD}
To extract \AF{ListD} from \AF{NatOD}, we can use the projection applying the patch in an ornamental description:
\ExecuteMetaData[Tex/Background]{toDesc}
The other projection reconstructs the ornament \AF{NatD-ListD} from \AF{NatOD}:
\ExecuteMetaData[Tex/Background]{toOrn}
As a consequence, \AD{OrnDesc} enjoys the features of both \AD{Desc} and \AD{Orn}, such as interpretation into a datatype by \AF{μ} and the conversion to the underlying type by \AF{ornForget}, by factoring through these projections.

In later sections, %mark: precisely?
we will routinely use \AD{OrnDesc} to view triples like (\AF{NatD}, \AF{ListD}, \AF{VecD}) as a base type equipped with two patches in sequence.


% exercise to reader: show OrnDesc AD ~ Exist[ BD in Desc ] Orn AD BD  


\part{Descriptions and ornaments}\label{part:ornament}
\changed{Somewhat final version above, draft/notes/rough comments/outline below.}
\changed{Redo (or check) the Agda snippets below here.}
%outline:
%we explained why descriptions and ornaments are crucial to achieve our goals
%however, the descriptions we explained earlier are not powerful enough to house finger trees


%To capture finger trees as an ornament over a number system, we will need to describe ornaments over nested datatypes. In this section we will work out descriptions and ornaments suitable for nested datatypes.
If we are going to simplify working with complex containers %, such as finger trees,
by instantiating generic programs to them, we should first make sure that these types fit into the descriptions.

We construct descriptions for nested datatypes by extending the encoding of parametric and indexed datatypes from \autoref{ssec:bg-desc} with three features: information bundles, parameter transformation, and description composition. Also, to make sharing constructors easier, we introduce variable transformations. Transforming variables before they are passed to child descriptions allows both aggressively hiding variables and introducing values as if by let-constructs.

We base the encoding of off existing encodings \cite{sijsling,practgen}. The descriptions take shape as sums of products, enforce indices at leaf nodes, and have explicit parameter and variable telescopes. Unlike some encodings \cite{effectfully, practgen}, we do not allow higher-order inductive arguments. 

We use type-in-type and with-K to simplify the presentation, noting that these can be eliminated respectively by moving to Typeω and by implementing interpretations as datatypes.

\section{The descriptions}\label{ssec:desc}
We use telescopes identical to those in \autoref{ssec:bg-desc}:
\ExecuteMetaData[Ornament/Desc]{telescopes}
Recall that a \AgdaDatatype{Tel} represents a sequence of types, which can depend on the external type $P$. This lets us represent a telescope succeeding another using \AgdaDatatype{ExTel}. A term of the interpretation \AgdaFunction{⟦\_⟧tel} is then a sequence of terms of all the types in the telescope.

We use some shorthands
\ExecuteMetaData[Ornament/Desc]{tele-shorthands}
\ExecuteMetaData[Ornament/Desc]{shorthands}

As we will see in \autoref{sec:trieo}, some generics require descriptions augmented with more information. For example, a number system needs to describe both a datatype and its interpretation into naturals. This can be incorporated into a description by allowing description formers to query specific pieces of information. We will control where and when which pieces get queried by parametrizing descriptions over information bundles  
\ExecuteMetaData[Ornament/Desc]{Info}
Here a bundle declares for example that \AgdaField{𝟙i} is the type of information has to be provided at a \AgdaInductiveConstructor{𝟙} former. Remark that in \AgdaField{σi}, the bundle can ask for something depending on the type of the field. In \AgdaField{δi}, the bundle can ask something regarding the parameters and indices (e.g., it can force only unindexed subdescriptions.).

\begin{example}
    For example, we can encode a class of number systems using the information 
    \ExecuteMetaData[Ornament/Numerical]{Number}
    (refer to \autoref{sec:trieo}). If we then define the unit type, when viewed as a \AgdaFunction{Number}
    \ExecuteMetaData{Ornament/Numerical}{Unit}
    we have to provide the information that the only value of the unit type evaluates to 1.
\end{example}

We can recover the conventional descriptions by providing the plain bundle:
\ExecuteMetaData[Ornament/Desc]{Plain}
We define the ``down-casting'' of information as
\ExecuteMetaData[Ornament/Desc]{InfoF}
allowing us to reuse more specific descriptions in less specific ones, so that e.g., a number system can be used in a plain datatype.

We can now define the descriptions, which should represent a mapping between parametrized indexed functors
\ExecuteMetaData[Ornament/Desc]{PIType}
Recall that a description 
\ExecuteMetaData[Ornament/Desc]{DescI}
is simply a list of constructor descriptions
\ExecuteMetaData[Ornament/Desc]{Con}
The interpretations \hyperlink{desc-desc-interpretation}{\AgdaFunction{⟦\_⟧}} of the formers can be found below.

Leaves are formed by
\ExecuteMetaData[Ornament/Desc]{Con-1}
Here \AgdaBoundFontStyle{if} queries information according to \texttt{If}, and \AgdaBoundFontStyle{j} computes the index of the leaf from the parameters and variables.

A recursive field is formed by
\ExecuteMetaData[Ornament/Desc]{Con-rho}
where \AgdaBoundFontStyle{j} now determines the index of the recursive field. The function \AgdaBoundFontStyle{g} represents a parameter transform: the parameters of the recursive field can now changed at each recursive level, allowing us to describe nested datatypes. The remainder of the fields are described by \AgdaBoundFontStyle{C}. Note that a recursive field is intentionally not brought into scope: making use of it requires induction-recursion anyway!

A non-recursive field is formed similarly to a recursive field
\ExecuteMetaData[Ornament/Desc]{Con-sigma}
The type of the field is given by \AgdaBoundFontStyle{S}, which may depend on the values of the preceding fields. We bring the field into scope, so we continue the description in an extended context. However, we allow the remainder of the description to provide a conversion from \texttt{V ▷ S} into \texttt{W} to select a new context. This makes it possible to hide fields which are unused in the remainder.

Almost analogously, we make composition of descriptions internal by a variant of \AgdaInductiveConstructor{σ}
\ExecuteMetaData[Ornament/Desc]{Con-delta}
This takes a description \texttt{R}, and acts like the \AgdaInductiveConstructor{σ} of \texttt{μ R}, only with more ceremony. This will allow us to form descriptions by composing other descriptions, avoiding multiplying the number of constructors of composite datatypes.

Similar to \AgdaInductiveConstructor{ρ}, the functions \AgdaBoundFontStyle{j} and \AgdaBoundFontStyle{g} control indices and parameters, only now of the applied description. As we allow the description \AgdaBoundFontStyle{R} of the field to have a different kind of information bundle \AgdaBoundFontStyle{If′}, we must ask that we can down-cast it into \AgdaBoundFontStyle{If} via \AgdaBoundFontStyle{iff}. 

Descriptions and constructor descriptions can then be interpreted to appropriate kind of functor, constructor descriptions also taking variables
\hypertarget{desc-desc-interpretation}{}
\ExecuteMetaData[Ornament/Desc]{interpretation}
We see that a leaf becomes a constraint between expected index and the actual index. A recursive field passes down a transformation of the current parameters and the expected index computed from the variables, before interpreting the remainder of the description. Likewise, a non-recursive field adds a field with type depending on variables, but also adds this field to the variables, which are then transformed and passed on to the remainder. The composite field is analogous, only adding a field from a description rather than a type. Finally, the list of constructor descriptions are interpreted as alternatives.

The fixpoint can then be taken over the interpretation of a description
\ExecuteMetaData[Ornament/Desc]{fpoint}
giving the datatype represented by the description.

We can then give a generic fold for the represented datatypes
\ExecuteMetaData[Ornament/Desc]{fold}
which descends the description, mapping itself over all recursive fields before applying the folding function.
\begin{remark}
    The situation of \AgdaFunction{fold} is very common when dealing with different kinds of recursive interpretations: functions from the fixpoint are generally defined from functions out of the interpretation, generalizing over the inner description while pattern matching on the outer description. 
\end{remark}
Note that the fold requires a rather general function, limiting its usefulness: because of the parameter transformations, we cannot instantiate the fold to a single parameter. Defining, e.g., the vector sum, would require us to inspect the description, and ask that a vector of naturals can be converted into a vector of naturals, which is trivial in this case.

\todo{Sigma plus/minus}

Let's look at some examples. We can encode the naturals as a type parametrized by \AgdaInductiveConstructor{∅} and indexed by \AgdaDatatype{⊤}
\ExecuteMetaData[Ornament/Desc]{NatD}
Lists can be encoded similarly, but this time using the telescope
\ExecuteMetaData[Ornament/Desc]{ListTel}
declaring that lists have a single type parameter. Compared to the naturals, the description now also asks for a field in the second case
\ExecuteMetaData[Ornament/Desc]{ListD}
Since the type parameter is at the top of the parameter telescope, the type of the field is given as \AgdaBoundFontStyle{par top}.

Vectors are described using the same structure, but have indices in \bN{}.
\ExecuteMetaData[Ornament/Desc]{VecD}
In the first case, the index is fixed at 0. The second case declares that to construct a vector of length \AgdaBoundFontStyle{suc ∘ top}, the recursive field must have length \AgdaFunction{top}. Note that unlike index-first types, we cannot know the expected index from inside the description, so much like native indexed types, we must add a field choosing an index.

Recall the type of finger trees. Using parameter transformations and composition, we can give a description of full-fledges finger trees! First, we describe the digits
\ExecuteMetaData[Ornament/Desc]{DigitD}
and define the nodes\footnote{We could give the nodes as a description, but in this case we only use them in the recursive fields, so we would take the fixpoint without looking at their description anyway.}
\ExecuteMetaData[Ornament/Desc]{Node}
We encode finger trees as
\ExecuteMetaData[Ornament/Desc]{FingerD}
In the third case, we have digits which are passed the parameters on both sides in composite fields, and a recursive field in the middle. The recursive field has a parameter transformation, turning the type parameter \AgdaBoundFontStyle{A} into a \AgdaBoundFontStyle{Node A} in the recursive child.

%\investigate{Making \AgdaDatatype{Desc} coinductive would do a couple of things. First, recursion and composition become identical. Second, nesting of both types becomes easier to describe, but potentially impossible to prove strictly positive.}

%\investigate{We intentionally dodge having index telescopes (or having the index type depend on the parameters and values). Does this really change anything?}



\section{The ornaments}
\towrite{Put something that isn't yet in \autoref{ssec:bg-orn} here.}

\ExecuteMetaData[Ornament/Orn]{Orn-type}
\ExecuteMetaData[Ornament/Orn]{ornForget-type}
%Thus, the relation should be precise enough pairs of \AgdaBoundFontStyle{E} and \AgdaBoundFontStyle{D} for which we could not define \AgdaFunction{ornForget}.

We will walk through the constructor ornaments
\ExecuteMetaData[Ornament/Orn]{ConOrn-type}
again, an ornament between datatypes is just a list of ornaments between their constructors
\ExecuteMetaData[Ornament/Orn]{Orn}
Note that all ornaments completely ignore information bundles! They cannot affect the existence of \AgdaFunction{ornForget} after all.

Copying parts from one description to another, up to parameter and index refinement, corresponds to reflexivity. Preservation of leaves follows the rule
\ExecuteMetaData[Ornament/Orn]{Orn-1}
We can see that this commuting square (\texttt{e (k p) ≡ j (over f p)}) is necessary: take a value of \texttt{E} at \texttt{p, i}, where \texttt{i} is given as \texttt{k p}. Then \AgdaFunction{ornForget} has to convert this to a value of \texttt{D} at \texttt{f p , e i}, but since \texttt{e i} must match \texttt{j (f p)}, this is only possible if \texttt{e (k p) = j (f p)}.

Preserving a recursive field similarly requires a square of indices and conversions to commute
\ExecuteMetaData[Ornament/Orn]{Orn-rho}
additionally requiring the recursive parameters to commute with the conversion. \todo{Does adding the derivations for the squares everywhere make this section clearler?}

Preservation of non-recursive fields and description fields is analogous
\ExecuteMetaData[Ornament/Orn]{Orn-sigma-delta}
differing only in that non-recursive fields appears transformed on the right hand, while description fields have their conversions modified instead. For this rule, we need that the variable transformations fit into a commuting square with the parameter conversions. The condition on indices for descriptions, which is a commuting triangle, is encoded in the return type\footnote{Should this become a problem like with \AgdaInductiveConstructor{ρ}, modifying the rule to require a triangle is trivial.}.

Ornaments would not be very interesting if they only related identical structures. The left-hand side can also have more fields than the right-hand side, in which case \AgdaFunction{ornForget} will simply drop the fields which have no counterpart on the right-hand side. As a consequence, the description extending rules have fewer conditions than the description preserving rules: 
\ExecuteMetaData[Ornament/Orn]{Orn-+-rho}
Note that this extension\footnote{Kind of breaking the ``ornaments relate types with similar recursive structure'' interpretation.} with a recursive field has no conditions.

Extending by a non-recursive field or a description field again only requires the variable transform to interact well with the parameter conversion
\ExecuteMetaData[Ornament/Orn]{Orn-+-sigma-delta}

In the other direction, the left-hand side can also omit a field which appears on the right-hand side, provided we can produce a default value
\ExecuteMetaData[Ornament/Orn]{Orn---sigma-delta}
These rules let us describe the basic set of ornaments between datatypes.

Intuitively we also expect a conversion to exist when two constructors have description fields which are not equal, but are only related by an ornament. Such a composition of ornaments takes two ornaments, one between the field, and one between the outer descriptions. This composition rule reads:\todo{The implicits kind of get out of control here, but the rule is also unreadable without them. I might hide the rule altogether and only run an example with it.}
\ExecuteMetaData[Ornament/Orn]{Orn-comp}
We first require two commuting squares, one relating the parameters of the fields to the inner and outer parameter conversions, and one relating the indices of the fields to the inner index conversion and the outer parameter conversion. Then, the last square has a rather complicated equation, which merely states that the variable transforms for the remainder respect the outer parameter conversion.

We will construct \AgdaFunction{ornForget} as a \AgdaFunction{fold}. Using
\ExecuteMetaData[Ornament/Orn]{erase-type}
we can define the algebra which forgets the added structure of the outer layer
\ExecuteMetaData[Ornament/Orn]{ornAlg}
Folding over this algebra gives the wanted function
\ExecuteMetaData[Ornament/Orn]{ornForget}

\todo{NatD was removed here}

We can also relate lists and vectors
\ExecuteMetaData[Ornament/Orn]{ListD-VecD}
Now the parameter conversion is the identity, since both have a single type parameter. The index conversion is \AgdaFunction{!}, since lists have no indices. Again, most structure is preserved, we only note that vectors have an added field carrying the length.

Instantiating \AgdaFunction{ornForget} to these ornaments, we now get the functions \AgdaFunction{length} and \AgdaFunction{toList} for free!

%\investigate{Having a function of the same type as \AgdaFunction{ornForget} is not sufficient to deduce an ornament. An obstacle is that the usual empty type (no constructors) and the non-wellfounded empty type (only a recursive field) don't have an ornament. Also, while the leaf-preservation case spells itself out, the substitutions obviously don't give us a way to recover the equalities.}


\section{Ornamental descriptions}
A description can say ``this is how you make this datatype'', an ornament can say ``this is how you go between these types''. However, an ornament needs its left-hand side to be predefined before it can express the relation, while we might also interpret an ornament as a set of instructions to translate one description into another. A slight variation on ornaments can make this kind of usage possible: ornamental descriptions.

An ornamental description drops the left-hand side when compared to an ornament, and interprets the remaining right-hand side as the starting point of the new datatype:
\ExecuteMetaData[Ornament/OrnDesc]{ConOrnDesc-type}
The definition of ornamental descriptions can be derived in a straightforward manner from ornaments, removing all mentions of the LHS and making all fields which then no longer appear in the indices explicit\footnote{One might expect to need less equalities, alas, this is difficult because of \autoref{rem:orn-lift}.}. We will show the leaf-preserving rule as an example, the others are derived analogously:
\ExecuteMetaData[Ornament/OrnDesc]{OrnDesc-1}
As we can see, the only change we need to make is that \AgdaBoundFontStyle{k} becomes explicit and fully annotated.

Almost by construction, we have that an ornamental description can be decomposed into a description of the new datatype
\ExecuteMetaData[Ornament/OrnDesc]{toDesc}
and an ornament between the starting description and this new description
\ExecuteMetaData[Ornament/OrnDesc]{toOrn}


\section{Temporary: future work}
\begin{remark}
    Note that this allows us to express datatypes like finger trees, but not rose trees. Such datatypes would need a way to place a functor ``around the \AgdaInductiveConstructor{ρ}'', which then also requires a description of strictly positive functors. In our setup, this could only be encoded by separating general descriptions from strictly positive descriptions. The non-recursive fields of these strictly positive descriptions then need to be restricted to only allow compositions of strictly positive context functions. 
\end{remark} % \investigate{This setup does not allow nesting over recursive fields, which is necessary for structures like rose trees. This is actually kind of essential for enumeration. Nesting over a recursive field is problematic: we can incorporate it by adding ``this'' implicitly to a \AgdaInductiveConstructor{δ}, but then the \AgdaBoundFontStyle{R} needs to be strictly positive in its last argument, meaning we need to split \AgdaDatatype{Desc} into a strictly positive part and normal part. The strictly positive part should then only allow strictly positive parameter transforms in recursive and non-recursive fields, requiring an embedding of transforms.}

\begin{remark}
    Variable transforms are not essential in these descriptions, but there are a couple of reasons for keeping them. In particular, they make it possible to reuse a description in multiple contexts, and save us from writing complex expressions in the indices of our ornaments. On the other hand, the transforms still make defining ornaments harder (the majority of the commuting squares are from variables). Isolating them into a single constructor of \AgdaDatatype{Desc}, call it \AgdaInductiveConstructor{v}, seems like a good middle ground, but raises some odd questions, like ``why is there no ornament between \AgdaBoundFontStyle{v (g ∘ f) C} and \AgdaBoundFontStyle{v g (v f C)}''. (Furthermore, this also does not simplify the indices of ornaments).
\end{remark} %\investigate{Variable transforms are both less essential and less troublesome than I first thought. We can move variable transforms into a new former, and it probably simplifies the definition of ornaments a lot.}

\begin{remark}
    Rather, ornaments themselves could act as information bundles. If there was a description for \AgdaDatatype{Desc}, that is. Such a scheme of levitation would make it easier to switch between being able to actively manipulate information, and not having to interact with it at all. However, the complexity of our descriptions makes this a non-trivial task; since our \AgdaDatatype{Desc} is given by mutual recursion and induction-recursion, the descriptions, and the ornaments, would have to be amended to encode both forms of recursion as well.
\end{remark} % \investigate{If we levitate, then informed descriptions become ornaments over \AgdaDatatype{Desc}. This gives us the best of both worlds (modulo reflecting the description into a datatype): in plain descriptions, information does not even exist, and in informed descriptions, it is explicit. For levitation, we likely need induction-recursion and mutual recursion.}

\begin{remark}\label{rem:orn-lift}
    Rather than having the user provide two indices and show that the square commutes, we can ask for a ``lift'' $k$
    % https://q.uiver.app/#q=WzAsNCxbMCwwLCJcXGJ1bGxldCJdLFsxLDAsIlxcYnVsbGV0Il0sWzAsMSwiXFxidWxsZXQiXSxbMSwxLCJcXGJ1bGxldCJdLFswLDEsImUiXSxbMiwzLCJmIiwyXSxbMiwwLCJqIl0sWzMsMSwiaSIsMl0sWzMsMCwiayIsMV1d
    \[\begin{tikzcd}
        \bullet & \bullet \\
        \bullet & \bullet
        \arrow["e", from=1-1, to=1-2]
        \arrow["f"', from=2-1, to=2-2]
        \arrow["j", from=2-1, to=1-1]
        \arrow["i"', from=2-2, to=1-2]
        \arrow["k"{description}, from=2-2, to=1-1]
    \end{tikzcd}\]
    and derive the indices as $i = ek, j = kf$. However, this is more restrictive, unless $f$ is a split epi, as only then pairs $i,j$ and arrows $k$ are in bijection. In addition, this makes ornaments harder to work with, because we have to hit the indices definitionally, whereas asking for the square to commute gives us some leeway (i.e., the lift would require the user to transport the ornament). 
\end{remark}


%\investigate{Can these be simpler? Right now, these just construct the ornament and description on the fly, rather than actually asking for less.}



\part{Numerical representations}\label{part:numrep}
While the practical applications of the last example do not stretch very far\footnote{Considering that \AgdaDatatype{ℕ} is a candidate to be replaced by a more suitable unsigned integer type when compiling to Haskell anyway.}, the approach generalizes to the more relevant containers and their associated laws.

In the same vein as the last section, we could define a simple but inefficient array type, and a more efficient implementation using trees. Then we can show that these are equivalent, such that when the simple type satisfies a set of laws, trees will satisfy them as well. We could then start developing all sorts of complex implementations fine-tuned to each operation and figure out how these are equivalent to some simpler type, but let us first take a step back, and investigate how we can make this approach run smoothly in a simpler example.

Rather than inductively defining a container and then showing that it is represented by a lookup function, we can go the other way around and define a type by insisting that it is equivalent to such a function. This approach, in particular the case in which one calculates a container with the same shape as a numeral system, was dubbed numerical representations in \cite{purelyfunctional}, and has some formalized examples in, e.g., \cite{calcdata} and \cite{progorn}. Numerical representations form the starting point for defining more complex datastructures based on simpler ones, so let us demonstrate such a calculation. 

\subsection{Numerical representations: from numbers to containers}\label{ssec:numrep}
We can compute the type of vectors starting from \bN{}.\footnote{This is adapted (and fairly abridged) from \cite{calcdata}} For simplicity, we define them as a type computing function via the ``use-as-definition`` notation from before. We expect vectors to be represented by 
\ExecuteMetaData[Tex/NumRep]{Lookup}
where we use the finite type \AgdaDatatype{Fin} as an index into vector. Using this representation as a specification, we can compute both \AgdaDatatype{Fin} and a type of vectors. The finite type can be computed from the evident definition
\ExecuteMetaData[Tex/NumRep]{Fin-def}
using
\ExecuteMetaData[Tex/NumRep]{leq-split}
Likewise, vectors can be computed by applying a sequence of type isomorphisms
\ExecuteMetaData[Tex/NumRep]{Vec}
\investigate{SIP doesn't mesh very well with indexed stuff, does HSIP help?}
Of course, a container would not be of much use without lookup functions, so we define an interface
\ExecuteMetaData[Tex/NumRep]{Array}
which at the very least has to satisfy laws like
\ExecuteMetaData[Tex/NumRep]{Laws}
We could directly show that \AgdaDatatype{Vec} satisfies this, but now that we defined \AgdaDatatype{Vec} from \AgdaDatatype{Lookup} we might as well use this fact.

The implementation of arrays as functions is very straightforward
\ExecuteMetaData[Tex/NumRep]{FunArray}
and clearly satisfies our interface
\ExecuteMetaData[Tex/NumRep]{FunLaw}
We can implement arrays based on \AgdaDatatype{Vec} as well
\ExecuteMetaData[Tex/NumRep]{VecArray}
and again, we can transport the proofs from \AgdaDatatype{Lookup} to \AgdaDatatype{Vec}.\footnote{Except in this oversimplified case the laws are trivial for \AgdaDatatype{Vec} as well.}\todo{If one was determined to cobble together the path over path over path we need now.}
\investigate{As you can see, taking ``use-as-definition'' too literally prevents Agda from solving a lot of metavariables.}

\investigate{This computation can of course be generalized to any arity zeroless numeral system; unfortunately beyond this set of base types, this ``straightforward'' computation from numeral system to container loses its efficacy. In a sense, the n-ary natural numbers are exactly the base types for which the required steps are convenient type equivalences like $(A + B) \to C = (A \to C) \times (B \to C)$?}

%\subsection{Relating types by structure: Ornamentation (unfinished)}\label{sec:ornament}
\subsection{Numerical representations as ornaments}\label{ssec:ornaments}
We could peform the same computation for \bL{}, which would yield the type of binary trees, but we note that these computations proceed with roughly the same pattern: each constructor of the numeral system gets assigned a value, and is amended with a field holding a number of elements and subnodes using this value as a ``weight''. But wait! Such modifications of constructors are already made formal by the concept of ornamentation!\todo{It seems like Agda forgets that we defined Leibniz if we move between different .tex files, so I'll have to float all AgdaTargets to the top level at some point...}

Ornamentation, as exposed in \cite{algorn} and \cite{progorn}, lets us formulate what it means for two types to have a ``similar'' recursive structure. This is achieved by interpreting (indexed inductive) datatypes from descriptions, between which an ornament is seen as a certificate of similarity, describing which fields or indices need to be introduced or dropped. Furthermore, a one-sided ornament: an ornamental description, lets us describe new datatypes by recording the modifications to an existing description.
\todo{Again not sure how much space I should use to reiterate Desc, Orn, and OrnDesc.}

This links back to the construction in the previous section, since \bN{} and \AgdaDatatype{Vec} share the same recursive structure, so \AgdaDatatype{Vec} can be formed by introducing indices and adding a field holding an element at each node.\footnote{These and similar examples are also documented in \cite{progorn}} 

However, instead deriving \AgdaDatatype{List} from \bN{} generalizes to \bL{} with less notational overhead, so lets tackle that case first. For this, we have to give a description of \bN{} to work with\todo{Clearly this can use more explanation (the question is, how much?)}
\ExecuteMetaData[Tex/NumRepOrn]{NatD}
Recall that \AgdaInductiveConstructor{σ} adds a field, upon which the rest of the description may vary, and \AgdaInductiveConstructor{ṿ} lists the recursive fields and their indices (which can only be \AgdaInductiveConstructor{tt}).
We can now write down the ornament which adds fields to the \AgdaFunction{suc} constructor
\ExecuteMetaData[Tex/NumRepOrn]{ListO}
Here, the \AgdaInductiveConstructor{σ} and \AgdaInductiveConstructor{ṿ} are forced to match those of \AgdaDatatype{NatD},
but the \AgdaInductiveConstructor{Δ} adds a new field. With the least fixpoint and description extraction from \cite{progorn}, this is sufficient to define \AgdaDatatype{List}. Note that we cannot hope to give an unindexed ornament from \bL{}
\ExecuteMetaData[Tex/NumRepOrn]{LeibnizD}
into trees, since trees have a very different recursive structure! Instead, we must keep track at what level we are in the tree so that we can ask for adequately many elements:
\ExecuteMetaData[Tex/NumRepOrn]{TreeO}
We use the \AgdaFunction{power} combinator to ensure that the digit at position $n$, which has weight $2^n$ in the interpretation of a binary number, also holds its value times $2^n$ elements. This makes sure that the number of elements in the tree shaped after a given binary number also is the value of that  binary number.

This ``folding in'' technique using the indices to keep track of structure seems to apply more generally. Let us explore this a bit further, and return later to the generalization of the pattern from numeral systems to datastructures.
% i.e. why did this even work?

\subsection{Folding in}\label{ssec:flattening}
Let us describe this procedure of folding a complex recursive structure into a simpler structure more generally. In particular, we will demonstrate that for linear datatypes, such as \bN{} and \bL{}, and for a given unindexed datatype, there is always an indexed datatype isomorphic to it at some index, and an ornament from the linear type to the indexed type. 

Suppose we are given a description, the first thing we can do to simplify it is collect all fields in one place
\ExecuteMetaData[Tex/Flatten]{RField}
Next, we will certainly have to count the number of recursive occurrences we are tracking, so we define
\ExecuteMetaData[Tex/Flatten]{Number}
where \AgdaInductiveConstructor{𝟙} records that we are at the top level, and \AgdaInductiveConstructor{ṿ} denotes that we are below a constructor with some number of recursive fields. This simplifies our task to implementing the types in
\ExecuteMetaData[Tex/Flatten]{nested}
such a way that we get an isomorphism 
\ExecuteMetaData[Tex/Flatten]{wish}
Thus, \AgdaDatatype{Fields} is forced to have a \AgdaInductiveConstructor{leaf} constructor like 
\ExecuteMetaData[Tex/Flatten]{Fields}
if \AgdaFunction{nested} is to work at \AgdaInductiveConstructor{𝟙}. The \AgdaInductiveConstructor{node} constructor makes sure that if we have collection of \AgdaDatatype{Fields}, then we can gather them in a field at a higher level. We can then count the subnodes of a given \AgdaDatatype{Fields} as
\ExecuteMetaData[Tex/Flatten]{subnodes}
where \AgdaFunction{RSize} counts the number of recursive fields of a particular branch
\ExecuteMetaData[Tex/Flatten]{RSize}
Note that \AgdaFunction{subnodes} effectively keeps the shape of the previous field, but unfolds the recursive fields at the bottom of the tree by one level.

\investigate{I then tried and realized how unpleasant even the functions from the original type to the nested type are to write.}

As a trivialty, we get that any type, interpreted as a container, always decomposes as an ornament over a ``numerical'' base type.\todo{Or at least, that was where I was trying to go with this, but I notice that this still is a bit further away.} This links to the construction of binary heaps in \cite{progorn}, as in hindsight, starting from the usual binary heaps would yield binary numbers and their binary heap ornament (in a much less useful package).


\part{Related work}\label{part:related}
\section{Descriptions and ornaments}
We compare our implementation to a selection of previous work, considering the following features


\begin{tabular}{c | c c c c c}
             & Haskell        & \cite{initenough} & \cite{levitation} & \cite{algorn} & \cite{progorn} \\
    \hline                                                                                             
    Fixpoint & yes*           & yes               & no                & yes?          & yes            \\
    Index    & —              & —                 & first**           & equality      & first          \\
    Poly     & yes            & 1                 & external          & external      & external       \\
    Levels   & —              & —                 & no                & no            & no             \\
    Sums     & list           & —                 & large             & large         & large          \\
    IndArg   & any            & any               & $\dots \to X\ i$  & $X\ i$        & $X\ i$         \\
    Compose  & yes            & yes               & no                & no            & no             \\
    Extension& —              & —                 & no                & —             & —              \\
    Ignore   & —              & —                 & —                 & —             & —              \\
    Set      & —              & —                 & —                 & —             & —              \\
\end{tabular}


\begin{tabular}{c | c c c c c}
             & \cite{sijsling} & \cite{effectfully} & \cite{practgen} & Shallow   & Deep (old) \\
    \hline   
    Fixpoint & yes             & yes                & no              & yes       & yes     \\
    Index    & equality        & equality           & equality        & equality  &         \\
    Poly     & telescope       & external           & telescope       & telescope &         \\
    Levels   & no***           & cumulative         & Typeω           & Type-in-Type &         \\
    Sums     & list            & large              & list            & list      &         \\
    IndArg   & $X\ pv\ i$      & $\dots\to X\ v\ i$ & $\dots\to X\ pv\ i$ & $X (f pv) i$ & ?1 \\
    Compose  & no              & yes?2              & no              & yes       &         \\
    Extension& —               & yes                & yes             & no        &         \\
    Ignore   & no              & ?                  & ?               & transform &         \\
    Set      & no              & no                 & no              & no        & yes     \\
\end{tabular}





\begin{itemize}
    \item IndArg: the allowed shapes of inductive arguments. Note that none other than Haskell, higher-order functors, and potentially ?1, allow full nested types!
    \item Compose: can a description refer to another description?
    \item Extension: do inductive arguments and end nodes, and sums and products coincide through a top-level extension?
    \item Ignore: can subsequent constructor descriptions ignore values of previous ones? (Either this, or thinnings, are essential to make composites work)
    \item Set: are sets internalized in this description?
\end{itemize}

\begin{itemize}
    \item[*] These descriptions are ``coinductive'' in that they can contain themselves, so the ``fixpoint'' is more like a deep interpretation.
    \item[**] This has no fixpoint, and the generalization over the index is external.
    \item[***] But you could bump the parameter telescope to Typeω and lose nothing.
    \item[*4] A variant keeps track of the highest level in the index.
    \item[?1] Deeply encoding all involved functors would remove the need for positivity annotations for full nested types like in other implementations.
    \item[?2] The ``simplicity'' of this implementation, where data and constructor descriptions coincide, automatically allows composite descriptions.
\end{itemize}

We take away some interesting points from this:
\begin{itemize}
    \item Levels are important, because index-first descriptions are incompatible with ``data-cumulativity'' when not emulating it using equalities! (This results in datatypes being forced to have fields of a fixed level).
    \item Coinductive descriptions can generate inductive types!
    \item Typeω descriptions can generate types of any level!
    \item Large sums do not reflect Agda (a datatype instantiated from a derived description looks nothing like the original type)! On the other hand, they make lists unnecessary, and simplify the definition of ornaments as well.
    \item We can group/collapse multiple signatures into one using tags, this might be nice for defining generic functions in a more collected way.
    \item Everything becomes completely unreadable without opacity.
\end{itemize}

\subsection{Merge me}


\subsubsection{Ornamentation}
While we can derive datastructures from number systems by going through their index types \cite{calcdata}, we may also interpret numerical representations more literally as instructions to rewrite a number system to a container type. We can record this transformation internally using ornaments, which can then be used to derive an indexed version of the container \cite{algorn}, or can be modified further to naturally integrate other constraints, e.g., ordering, into the resulting structure \cite{progorn}. Furthermore, we can also use the forgetful functions induced by ornaments to generate specifications for functions defined on the ornamented types \cite{orntrans}.

\subsubsection{Generic constructions}
Being able to define a datatype and reflect its structure in the same language opens doors to many more interesting constructions \cite{practgen}; a lot of ``recipes'' we recognize, such as defining the eliminators for a given datatype, can be formalized and automated using reflection and macros. We expect that other type transformations can also be interpreted as ornaments, like the extraction of heterogeneous binary trees from level-polymorphic binary trees \cite{hetbin}. 


\subsection{Takeways}
At the very least, descriptions will need sums, products, and recursive positions as well. While we could use coinductive descriptions, bringing normal and recursive fields to the same level, we avoid this as it also makes ornaments a bit more wild\footnote{For better or worse, an ornament could refer to a different ornament for a recursive field.}. We represent indexed types by parametrizing over a type $I$. Since we are aiming for nested types, external polymorphism\footnote{E.g., for each type $A$ a description of lists of $A$ à la \cite{progorn}} does not suffice: we need to let descriptions control their contexts.

We describe parameters by defining descriptions relative to a context. Here, a context is a telescope of types, where each type can depend on all preceding types:
\[ \dots \]
Much like the work Escot and Cockx \cite{practgen} we shove everything into \AgdaFunction{Typeω}, but we do not (yet) allow parameters to depend on previous values, or indices on parameters\footnote{I do not know yet what that would mean for ornaments.}.

We use equalities to enforce indices, simply because index-first types are not honest about being finite, and consequently mess up our levels. For an index type and a context a description represents a list of constructors:
\[ \dots \]
These represent lists of alternative constructors, which each represent a list of fields:
\[ \dots \]
We separate mere fields from ``known'' fields, which are given by descriptions rather than arbitrary types. Note that we do not split off fields to another description, as subsequent fields should be able to depend on previous fields
\[ \dots. \]


We parametrize over the levels, because unlike practical generic, we stay at one level.

Q: what happens when you precompose a datatype with a function? E.g. (List . f) A = List (f A) 

Q: practgen is cool, compact, and probably necessary to have all datatypes. Note that in comparison, most other implementations (like Sijsling) do not allow functions as inductive arguments. Reasonably so.

Q: I should probably update my Agda and make use of the new opaque features to make things readable when refining


\towrite{Adapt this to the non-proposal form.}

\section{The Structure Identity Principle}
If we write a program, and replace an expression by an equal one, then we can prove that the behaviour of the program can not change. Likewise, if we replace one implementation of an interface with another, in such a way that the correspondence respects all operations in the interface, then the implementations should be equal when viewed through the interface. Observations like these are instances of ``representation independence'', but even in languages with an internal notation of type equality, the applicability is usually exclusive to the metatheory.

In our case, moving from Agda's ``usual type theory'' to Cubical Agda, \textit{univalence} \cite{cuagda} lets us internalize a kind of representation independence known as the Structure Identity Principle \cite{iri}, and even generalize it from equivalences to quasi-equivalence relations. 
%a cubical homotopy type theory,
We will also be able to apply univalence to get a true ``equational reasoning'' for types when we are looking at numerical representations.

Still, representation independence in may be internalized outside the homotopical setting in some cases \cite{tgalois}, and remains of interest in the context of generic constructions that conflict with cubical type theory.

\section{Numerical Representations}
Rather than equating implementations after the fact, we can also ``compute'' datastructures by imposing equations. In the case of container types, one may observe similarities to number systems \cite{purelyfunctional} and call such containers numerical representations. One can then use these representations to prototype new datastructures that automatically inherit properties and equalities from their underlying number systems \cite{calcdata}.

From another perspective, numerical representations run by using representability as a kind of ``strictification'' of types. %This suggests that we may be able to generalize the approach of numerical representations, using that any (non-indexed) infinitary inductive-recursive type supports a lookup operation \cite{glookup}.

\part{Discussion}
\begin{outline}
:warning: the trie ornament is hard to prove about

:warning: nesting rather than branching

:warning: folds don't give folds over parts

:warning: composites complicate the indexed variant

:warning: the index type becomes more awkward than with big sigmas
\end{outline}


\section{Temporary: future work (\autoref{part:ornament})}
\begin{remark}
    Note that this allows us to express datatypes like finger trees, but not rose trees. Such datatypes would need a way to place a functor ``around the \AgdaInductiveConstructor{ρ}'', which then also requires a description of strictly positive functors. In our setup, this could only be encoded by separating general descriptions from strictly positive descriptions. The non-recursive fields of these strictly positive descriptions then need to be restricted to only allow compositions of strictly positive context functions. 
\end{remark} % \investigate{This setup does not allow nesting over recursive fields, which is necessary for structures like rose trees. This is actually kind of essential for enumeration. Nesting over a recursive field is problematic: we can incorporate it by adding ``this'' implicitly to a \AgdaInductiveConstructor{δ}, but then the \AgdaBoundFontStyle{R} needs to be strictly positive in its last argument, meaning we need to split \AgdaDatatype{Desc} into a strictly positive part and normal part. The strictly positive part should then only allow strictly positive parameter transforms in recursive and non-recursive fields, requiring an embedding of transforms.}

\begin{remark}
    Variable transforms are not essential in these descriptions, but there are a couple of reasons for keeping them. In particular, they make it possible to reuse a description in multiple contexts, and save us from writing complex expressions in the indices of our ornaments. On the other hand, the transforms still make defining ornaments harder (the majority of the commuting squares are from variables). Isolating them into a single constructor of \AgdaDatatype{Desc}, call it \AgdaInductiveConstructor{v}, seems like a good middle ground, but raises some odd questions, like ``why is there no ornament between \AgdaBoundFontStyle{v (g ∘ f) C} and \AgdaBoundFontStyle{v g (v f C)}''. (Furthermore, this also does not simplify the indices of ornaments).
\end{remark} %\investigate{Variable transforms are both less essential and less troublesome than I first thought. We can move variable transforms into a new former, and it probably simplifies the definition of ornaments a lot.}

\begin{remark}
    Rather, ornaments themselves could act as information bundles. If there was a description for \AgdaDatatype{Desc}, that is. Such a scheme of levitation would make it easier to switch between being able to actively manipulate information, and not having to interact with it at all. However, the complexity of our descriptions makes this a non-trivial task; since our \AgdaDatatype{Desc} is given by mutual recursion and induction-recursion, the descriptions, and the ornaments, would have to be amended to encode both forms of recursion as well.
\end{remark} % \investigate{If we levitate, then informed descriptions become ornaments over \AgdaDatatype{Desc}. This gives us the best of both worlds (modulo reflecting the description into a datatype): in plain descriptions, information does not even exist, and in informed descriptions, it is explicit. For levitation, we likely need induction-recursion and mutual recursion.}

\begin{remark}\label{rem:orn-lift}
    Rather than having the user provide two indices and show that the square commutes, we can ask for a ``lift'' $k$
    % https://q.uiver.app/#q=WzAsNCxbMCwwLCJcXGJ1bGxldCJdLFsxLDAsIlxcYnVsbGV0Il0sWzAsMSwiXFxidWxsZXQiXSxbMSwxLCJcXGJ1bGxldCJdLFswLDEsImUiXSxbMiwzLCJmIiwyXSxbMiwwLCJqIl0sWzMsMSwiaSIsMl0sWzMsMCwiayIsMV1d
    \[\begin{tikzcd}
        \bullet & \bullet \\
        \bullet & \bullet
        \arrow["e", from=1-1, to=1-2]
        \arrow["f"', from=2-1, to=2-2]
        \arrow["j", from=2-1, to=1-1]
        \arrow["i"', from=2-2, to=1-2]
        \arrow["k"{description}, from=2-2, to=1-1]
    \end{tikzcd}\]
    and derive the indices as $i = ek, j = kf$. However, this is more restrictive, unless $f$ is a split epi, as only then pairs $i,j$ and arrows $k$ are in bijection. In addition, this makes ornaments harder to work with, because we have to hit the indices definitionally, whereas asking for the square to commute gives us some leeway (i.e., the lift would require the user to transport the ornament). 
\end{remark}




\begin{remark}
    Comparing SOP and computational sigmas. In particular, \texttt{s N (\ n -> v (replicate n tt))} is not in SOP without full nesting. SOP is good for generics in both directions (the conversion in both ways keeps the datatype like it is supposed to). On the other hand, computational sigmas make writing and proving about \texttt{Path} a lot easier.
\end{remark}


\section{Temporary: future work (\autoref{part:numrep})}

\investigate{This implementation of TrieO always computes the random-access variant of the datastructure. Can we implement a variant which computes the ``Braun tree'' variant of the datastructure?}

\investigate{Index types are a simple ornament over number types: paths. This is not quite like \cite{glookup}.}

\investigate{Is Ix x -> A initial for the algebra of the algebraic ornament induced by TrieO? (This is \cite{calcdata}).}

\investigate{While evidently Ix x != Fin (toN x) for arbitrary number systems, does the expected iso Ix x -> A = Trie A x yield Traversable, for free?}



\printbibliography

\part{Appendix}
\begin{outline}
    :warning: The other way of parameters-indices
\end{outline}

\appendix

\section{Finger trees}
%\towrite{Can we prove that the time complexity of head is always less than cons, similarly for lookup and insert?}
\towrite{Cite everything}

We know that some datastructures can be presented as non-redundant numerical representations, for example lists by unary numbers, random access lists by binary numbers [calculating], and, skew binary heaps by skew binary numbers [progorn]. So far, some of these examples do support amortized constant time consing, but they have at best logarithmic time snocing. This is reflected by their number systems, for which either the natural successor operation is constant time, but can only act at the front, or is logarithmic time to begin with. We will instead look at more redundant number systems, and refine these step-by-step to produce structures similar to finger trees, giving us datastructures with fast access to both ends, and some of their properties for free.

\subsection{Binary finger trees}
For any numerical representation, we see that the operations on the represented datatype must be coherent with the corresponding operations on the number system. Hence, if we want to have constant time cons and snoc, we must first have constant time suc anc cus. By starting from a more symmetric number system, we can ensure good performance for both. Note that such a system is necessarily redundant, as there must be cases where neither suc nor cus recurses, yet both must clearly yield different values!

The obvious first candidate are symmetric unary numbers
\[ ... \]
but we can also see that subtraction has linear time
\[ ... \]
which gives a linear lower bound on the time complexity of lookup. By using a binary backbone for the numbers, we can also get a good lookup
\[ ... \]
However, this shape is still not ideal. We can see that for values like
\[ ... \]
The pair of suc
\[ ... \]
and pred
\[ ... \]
can compose to always take logarithmic time [fingertrees anew]. To avoid this, we can give the numbers bigger fingers
\[ ... \]
Now applying pred to the pathological case produces a value for which suc and pred both are constant time
\[ ... \]
More formally, we can use a three-colour scheme [purely functional/tarjan] to prove that any sequence of suc/cus pred/derp will amortize to constant time. Again, the interpretation of this number system is given by
\[ ... \]

To extract the datastructure, we must find a suitable index type for these numbers. Since the numbers are redundant, we can also get trees of different shapes with the same size, each having a different and incompatible index type. Still, the trees of a fixed shape are represented by functions, and the isomorphisms will still hold. The computation of the index type from the interpretation of the numbers is straightforward
\[ ... \]
\dots
This lets us define the tree type as
\[ ... \]
and gives definitions of the basic array operations
\[ ... \]

We can again trieify this to get a concrete datastructure
\[ ... \]
Consequently, since the representable arrays obey head (cons x xs) = x, the concrete arrays obey this as well.
On the other hand, the redundancy of the numbers, for which suc cus = cus suc does not hold, also causes cons snoc = snoc cons to not hold either; it seems that binary finger trees are not a very nice array type. We would like to quotient the redundancy of the number type away, which would also alleviate our issues related to indexing.

We can do this by imposing the following relation on the numbers
\[ ... \]
and turning the number system into a setoid. Eliminating from this setoid should then respect this relation, so for example, we have to prove
\[ ... \]

Similarly, we can keep the implementation of the binary finger trees, but put this under an appropriate relation as well
\[ ... \]
which ensures that the construction of the trees respects the relation on the number as well..
















\section{Heterogenization}
%The situation in which one wants to collect a variety of types is not uncommon, and is typically handled by tuples. However, if e.g., you are making a game in Haskell, you might feel the need to maintain a list of ``Drawables'', which may be of different types. Such a list would have to be a kind of ``heterogeneous list''. In Haskell, this can be resolved by using an existentially quantified list, which, informally speaking, can contain any type implementing a given constraint, but can only be inspected as if it contains the intersection of all types implementing this constraint. 

This ports directly to Agda, but becomes cumbersome quickly, and impractical if we want to be able to inspect the elements. The alternative is to split our heterogeneous list into two parts; one tracking the types, and one tracking the values. In practice, this means that we implement a heterogeneous list as a list of values indexed over a list of types. This approach and mainly its specialization to binary trees is investigated by Swierstra \cite{hetbin}.

We will demonstrate that we can express this ``lift a type over itself'' operation as an ornament. For this, we make a small adjustment to \AgdaDatatype{RDesc} to track a type parameter separately from the fields. Using this we define an ornament-computing function, which given a description computes an ornamental description on top of it:
\ExecuteMetaData[Tex/Heterogenize]{HetO}
This ornament relates the original unindexed type to a type indexed over it; we see that this ornament largely keeps all fields and structure identical, only performing the necessary bookkeeping in the index, and adding extra fields before parameters.

As an example, we adapt the list description
\ExecuteMetaData[Tex/Heterogenize]{List}
which is easily heterogenized to an \AgdaDatatype{HList}. In fact, \AgdaFunction{HetO} seems to act functorially; if we lift \AgdaDatatype{Maybe} like
\ExecuteMetaData[Tex/Heterogenize]{HMaybe}
then we can lift functions like \AgdaFunction{head} as
\ExecuteMetaData[Tex/Heterogenize]{hhead}



\section{More equivalences for less effort}\label{sec:userfriendly}
Noting that constructing equivalences directly or from isomorphisms as in \autoref{ssec:leibniz} can quickly become challenging when one of the sides is complicated, we work out a different approach making use of the initial semantics of W-types instead. We claim that the functions in the isomorphism of \autoref{ssec:leibniz} were partially forced, but this fact was unused there.

First, we explain that if we assume that one of the two sides of the equivalence is a fixpoint or initial algebra of a polynomial functor (that is, the \AgdaDatatype{μ} of a \AgdaDatatype{Desc′}), this simplifies giving an equivalence to showing that the other side is also initial.

We describe how we altered the original ornaments \cite{progorn} to ensure that \AgdaDatatype{μ} remains initial for its base functor in Cubical Agda, explaining why this fails otherwise, and how defining base functors as datatypes avoids this issue.

In a subsection focussing on the categorical point of view, we show how we can describe initial algebras (and truncate the appropriate parts) in such a way that the construction both applies to general types (rather than only sets), and still produces an equivalence at the end. We explain how this definition, like the usual definition, makes sure that a pair of initial objects always induces a pair of conversion functions, which automatically become inverses. Finally, we explain that we can escape our earlier truncation by appealing to the fact that ``being an equivalence'' is a proposition.

Next, we describe some theory, using which other types can be shown to be initial for a given algebra. This is compared to the construction in \autoref{ssec:leibniz}, observing that intuitively, initiality follows because the interpretation of the zero constructor is forced by the square defining algebra maps, and the other values are forced by repeatedly applying similar squares. This is clarified as an instance of recursion over a polynomial functor.

To characterize when this recursion is allowed, we define accessibility with respect to polynomial functors as a mutually recursive datatype as follows. This datatype is constructed using the fibers of the algebra map, defining accessibility of elements of these fibers by cases over the description of the algebra. Then we remark that this construction is an atypical instance of well-founded recursion, and define a type as well-founded for an algebra when all its elements are accessible.

We interpret well-foundedness as an upper bound on the size of a type, leading us to claim that injectivity of the algebra map gives a lower bound, which is sufficient to induce the isomorphism. We sketch the proof of the theorem, relating part of this construction to similar concepts in the formalization of well-founded recursion in the Standard Library. In particular, we prove an irrelevance and an unfolding lemma, which lets us show that the map into any other algebra induced by recursion is indeed an algebra map. By showing that it is also unique, we conclude initiality, and get the isomorphism as a corollary. 

The theorem is applied and demonstrated to the example of binary naturals. We remark that the construction of well-foundedness looks similar to view-patterns. After this, we conclude that this example takes more lines that the direct derivation in \autoref{ssec:leibniz}, but we argue that most of this code can likely be automated.

\towrite{Merge}

%% REPLACE X BY A?
The setup some approaches in earlier sections require makes them tedious or impractical to apply. In this section we will look at some ways how part of this problem could be alleviated through generics, or by alternative descriptions of concepts like equivalences through the lens of initial algebras. 

In later sections we will construct many more equivalences between more complicated types than before, so we will dive right into the latter. Reflecting upon \autoref{sec:leibniz}, we see that when one establishes an equivalence, most of the time is spent working out a series of tedious lemmas to show that the conversion functions are mutual inverses, which tend to be relatively easy to define. We take away two things from this; the first is that the conversion functions are perhaps too obvious, and the second is that we should really avoid talking about sections and retractions lest we incur tedium!\footnote{The latter perhaps less so, because it is useful to show a map to be monic.} We will reuse the machinery of Ko and Gibbons \cite{progorn} to illustrate how the definitions in \autoref{sec:leibniz} were actually forced for a large part.

First, we remark that \AgdaDatatype{μ} is internalization of the representation of simple\footnote{Of course, indexed datatypes are indexed W-types, mutually recursive datatypes are represented yet differently\dots} datatypes as W-types. Thus, we will assume that one of the sides of the equivalence is always represented as an initial algebra of a polynomial functor, and hence the \AgdaDatatype{μ} of a \AgdaDatatype{Desc′}.

\subsection{Well-founded monic algebras are initial}\label{ssec:wellfounded}
Unfortunately, the machinery developed by Ko and Gibbons \cite{progorn} relies on axiom K for a small but crucial part. To be precise, in a cubical setting, the type \AgdaDatatype{μ} as given stops being initial for its base functor! In this section, we will be working with a simplified and repaired version. Namely, we simplify \AgdaDatatype{Desc′} to 
\ExecuteMetaData[Extra/ProgOrn/Desc]{DescS}
To complete the definition of \AgdaDatatype{μ}
\ExecuteMetaData[Extra/ProgOrn/Desc]{mu}
we will need to implement \AgdaDatatype{Base}. We remark that in the original setup, the recursion of \AgdaFunction{mapFold} is a structural descent in \AgdaFunction{⟦ D' ⟧ (μ D)}. Because \AgdaFunction{⟦\_⟧} is a type computing function and not a datatype, this descent becomes invalid\footnote{Refer to the  \href{https://agda.readthedocs.io/en/latest/language/without-k.html\#restrictions-on-termination-checking}{without K} page.}, and \AgdaFunction{mapFold} fails the termination check. We resolve this by defining \AgdaDatatype{Base} as a datatype
\ExecuteMetaData[Extra/ProgOrn/Desc]{Base}
such that this descent is allowed by the termination checker without axiom K.\footnote{This has, again by the absence of axiom K, the consequence of pushing the universe levels up by one. However, this is not too troublesome, as equivalences can go between two levels, and indeed types are equivalent to their lifts.}

Recall that the \AgdaDatatype{Base} functors of descriptions are special polynomial functors, and the fixpoint of a base functor is its initial algebra. We are looking for sufficient conditions on $X$ to get the equivalence $e: X \cong \mu F$. Note that when $X \cong \mu F$, then there necessarily is an initial algebra $F X \to X$. Conversely, if the algebra $(X, f)$ is isomorphic to $(\mu F, \mathrm{con})$, then $X \cong \mu F$ would follow immediately, so it is equivalent to ask for the algebras to be isomorphic instead.

\begin{comment}
The situation so far is summarized by the diagram
% https://q.uiver.app/?q=WzAsMyxbMSwwLCJGXFxtdV9GIl0sWzEsMSwiXFxtdSBGIl0sWzAsMSwiWCJdLFswLDEsIlxcbWF0aHJte2Nvbn0iXSxbMiwxLCJlIiwyLHsic3R5bGUiOnsidGFpbCI6eyJuYW1lIjoiYXJyb3doZWFkIn0sImJvZHkiOnsibmFtZSI6ImRhc2hlZCJ9fX1dXQ==
\[\begin{tikzcd}[ampersand replacement=\&]
	\& {F\mu_F} \\
	X \& {\mu F}
	\arrow["{\mathrm{con}}", from=1-2, to=2-2]
	\arrow["e"', dashed, tail reversed, from=2-1, to=2-2]
\end{tikzcd}\]
\end{comment}
\subsubsection{Datatypes as initial algebras}
To characterize when such algebras are isomorphic, we reiterate some basic category theory, simultaneously rephrasing it in Agda terms.\footnote{We are not reusing a pre-existing category theory library for the simple reasons that it is not that much work to write out the machinery explicitly, and that such libraries tend to phrase initial objects in the correct way, which is too restrictive for us.}

Let $C$ be a category, and let $a, b, c$ be objects of $C$, so that in particular we have identity arrows $1_a : a \to a$ and for arrows $g : b \to c, f : a \to b$ composite arrows $gf : a \to c$ subject to associativity. In our case, $C$ is the category of types, with ordinary functions as arrows.

Recall that an endofunctor, which is simply a functor $F$ from $C$ to itself, assigns objects to objects and sends arrows to arrows
\ExecuteMetaData[Extra/Category]{RawFunctor}
These assignments are subject to the identity and composition laws
\ExecuteMetaData[Extra/Category]{Functor}
An $F$-algebra is just a pair of an object $a$ and an arrow $Fa \to a$
\ExecuteMetaData[Extra/Category]{Algebra}
Algebras themselves again form a category $C^F$. The arrows of $C^F$ are the arrows $f$ of $C$ such that the following square commutes% https://q.uiver.app/?q=WzAsNCxbMCwwLCJGYSJdLFsxLDAsIkZiIl0sWzAsMSwiYSJdLFsxLDEsImIiXSxbMiwzLCJmIiwyXSxbMCwyLCJVX2EiLDJdLFsxLDMsIlVfYiJdLFswLDEsIkZmIl1d
\[\begin{tikzcd}[ampersand replacement=\&]
	Fa \& Fb \\
	a \& b
	\arrow["f"', from=2-1, to=2-2]
	\arrow["{U_a}"', from=1-1, to=2-1]
	\arrow["{U_b}", from=1-2, to=2-2]
	\arrow["Ff", from=1-1, to=1-2]
\end{tikzcd}\]
So we define
\ExecuteMetaData[Extra/Category]{AlgSqr}
and
\ExecuteMetaData[Extra/Category]{AlgMap}
Note that we take the propositional truncation of the square, such that algebra maps with the same underlying morphism become propositionally equal
\ExecuteMetaData[Extra/Category]{AlgPath}
The identity and composition in $C^F$ arise directly from those of the underlying arrows in $C$.

Recall that an object $\emptyset$ is initial when for each other object $a$, there is a unique arrow $!: \emptyset \to a$. By reversing the proofs of initiality of \AgdaDatatype{μ} and the main result of this section, we obtain a slight variation upon the usual definition. Namely, unicity is often expressed as contractability of a type
\ExecuteMetaData[Tex/Snippets]{isContr}
Instead, we again use a truncation
\ExecuteMetaData[Extra/Category]{weakContr}
but note that this also, crucially, slightly stronger than connectedness. We define initiality for arbitrary relations
\ExecuteMetaData[Extra/Category]{Initial}
such that it closely resembles the definition of least element. Then, $A$ is an initial algebra when
\ExecuteMetaData[Extra/Category]{InitAlg}

By basic category theory (using the usual definition of initial objects), two initial objects $a$ and $b$ are always isomorphic;
namely, initiality guarantees that there are arrows $f : a \to b$ and $g : b \to a$, which by initiality must compose to the identities again.

Similarly, we get that
\ExecuteMetaData[Extra/Category]{InitAlg-equiv}
However, we only have the equalities from the isomorphism inside a propositional truncation. But fortunately, being an equivalence is a property, so we can eliminate from the truncations to get the wanted result.

%Note that even though we warned ourselves, we are still talking about sections and retractions to establish that $f$ is an equivalence! However, this result also makes sure we will not have to speak of them again.

\subsubsection{Accessibility}
As a consequence, we get that $X$ is isomorphic to $\mu D$ when $X$ is an initial algebra for the base functor of $D$; $\mu D$ is initial by its fold, and by induction on $\mu D$ using the squares of algebra maps. 

\begin{remark}
    We need (in general) not hope $\mu D$ is a strict initial object in the category of algebras. For a strict initial object, having a map $a \to \emptyset$ implies $a \cong \emptyset$. This is not the case here: strict initial objects satisfy $a \times \emptyset \cong \emptyset$, but for the $X \mapsto 1 + X$-algebras $\mathbb{N}$ and $2^\mathbb{N}$ clearly $2^\mathbb{N} \times \mathbb{N} \cong \mathbb{N}$ does not hold. On the other hand, the ``obvious'' sufficient condition to let $C^F$ have strict initial objects is that $F$ is a left adjoint, but then the carrier of the initial algebra is simply $\bot$.
\end{remark}

Looking back at \autoref{sec:leibniz}, we see that \AgdaDatatype{Leibniz} is an initial $F: X \mapsto 1 + X$ algebra because for any other algebra, the image of \AgdaFunction{0b} is fixed, and by \AgdaFunction{bsuc} all other values are determined by chasing around the square. Thus, we are looking for a similar structure on $f : FX \to X$ that supports recursion.

Clearly we will need something stronger than $FX \cong X$, as in general a functor can have many fixpoints. For this, we define what it means for an element $x$ to be accessible by $f$. This definition uses a mutually recursive datatype as follows:
We state that an element $x$ of $X$ is accessible when there is an accessible $y$ in its fiber over $f$
\ExecuteMetaData[Extra/Category/Poly]{Acc}
Accessibility of an element $x$ of \AgdaFunction{Base A E} is defined by cases on $E$; if $E$ is \AgdaFunction{ṿ n} and $x$ is a \AgdaFunction{Vec A n}, then $x$ is accessible if all its elements are; if $x$ is \AgdaFunction{σ S E'}, then $x$ is accessible if \AgdaFunction{snd x} is
\ExecuteMetaData[Extra/Category/Poly]{Acc'}
Consequently, $X$ is well-founded for an algebra when all its elements are accessible
\ExecuteMetaData[Extra/Category/Poly]{Wf}

We can see well-foundedness as an upper bound on the size of $X$, if it were larger than $\mu D$, some of its elements would inevitably get out of reach of an algebra. \textit{Now} having $FX \cong X$ also gives us a lower bound, but remark that having a well-founded injection $f: FX \to X$ is already sufficient, as accessibility gives a section of $f$, making it an iso. This leads us to claim
\begin{claim}\label{claim:wf-inj-init}
    If there is a mono $f : FX \to X$ and $X$ is well-founded for $f$, then $X$ is an initial $F$-algebra.
\end{claim}

\subsubsection{Proof sketch of \autoref{claim:wf-inj-init}}
Let us be on our way. Suppose $X$ is well-founded for the mono $f : FX \to X$. To show that $(X, f)$ is initial, let us take another algebra $(Y, g)$, and show that there is a unique arrow $(X, f) \to (Y, g)$.\todo[inline]{This section is about as digestable as a brick.}

By \AgdaDatatype{Acc}-recursion and because all $x$ are accessible, we can define a plain map into $Y$
\ExecuteMetaData[Extra/Category/WellFounded]{Wf-rec}
This construction is an instance of the concept of ``well-founded recursion''\footnote{This is formalized in the \href{https://agda.github.io/agda-stdlib/Induction.WellFounded.html}{standard-library} with many other examples.}, so we let ourselves be inspired by these methods. In particular, we prove an irrelevance lemma
\ExecuteMetaData[Extra/Category/WellFounded]{Wf-rec-irr}
which implies the unfolding lemma
\ExecuteMetaData[Extra/Category/WellFounded]{Wf-rec-unfold}
The unfolding lemma ensures that the map we defined by \AgdaFunction{Wf-rec} is a map of algebras. The proof that this map is unique proceeds analogously to that in the proof that $\mu D$ is initial, but here we instead use \AgdaDatatype{Acc}-recursion
\ExecuteMetaData[Extra/Category/WellFounded]{Wf+inj=Init}
Thus, we conclude that $X$ is initial. The main result is then a corollary of initiality of $X$ and the isomorphism of initial objects
\ExecuteMetaData[Extra/Category/WellFounded]{Wf+inj=mu}


\subsubsection{Example}
Let us redo the proof in \autoref{sec:leibniz}, now using this result. Recall the description of naturals \AgdaFunction{NatD}. To show that \AgdaFunction{Leibniz} is isomorphic to \AgdaFunction{Nat}, we will need a \AgdaFunction{NatD}-algebra and a proof of its well-foundedness. We define the algebra
\ExecuteMetaData[Tex/Leibniz2]{bsuc'}

For well-foundedness, we use something similar to view-patterns %[mcbride]
(the main difference being that we look through the entire structure, instead of a single layer)
\ExecuteMetaData[Tex/Leibniz2]{Peano-View}
where the mutually recursive proof of \AgdaFunction{view} is ``almost trivial''. Well-foundedness follows fairly immediately
\ExecuteMetaData[Tex/Leibniz2]{Wf-bsuc}

Injectivity of \AgdaFunction{bsuc\_1} happens to be harder to prove from retractions than directly, so we prove it directly, from which the wanted statement follows
\ExecuteMetaData[Tex/Leibniz2]{L-is-mu-N}

Note that in this case it took us more code to prove the same statement! However, we stress that the code that we did write became more forced, and might be more amenable to automation.





\end{document}
