\documentclass{article}

\usepackage{comment}

\setlength{\marginparwidth}{2cm} % remove when done

\usepackage{todonotes}
\usepackage{xcolor}
\usepackage[hidelinks]{hyperref}
\usepackage{catchfilebetweentags}
\usepackage{quiver} 
\usepackage{amsthm}


\theoremstyle{plain}% default
\newtheorem{theorem}{Theorem}[section]
\newtheorem{lemma}[theorem]{Lemma}
\newtheorem{prop}[theorem]{Proposition}
\newtheorem*{cor}{Corollary}

\theoremstyle{definition}
\newtheorem{defn}{Definition}[section]
\newtheorem{remark}{Remark}[section]
\newtheorem{claim}{Claim}[section]

\usepackage[links]{agda}
\AgdaNoSpaceAroundCode{}

% from: https://agda.readthedocs.io/en/v2.6.3/_downloads/59877ce886494c991a213f09e29b712c/article-luaxelatex-different-fonts.lagda.tex
\usepackage{fontspec}

\usepackage{luaotfload}
\directlua{luaotfload.add_fallback
  ("mycustomfallback",
    { "JuliaMono:style=Regular;"
    , "NotoSansMono:style=Regular;"
    , "NotoSansMath:style=Regular;"
    }
  )}
\defaultfontfeatures{RawFeature={fallback=mycustomfallback}}

\newfontfamily{\AgdaSerifFont}{Linux Libertine O}
\newfontfamily{\AgdaSansSerifFont}{Linux Biolinum O}
\newfontfamily{\AgdaTypewriterFont}{inconsolata}
\renewcommand{\AgdaFontStyle}[1]{{\AgdaSansSerifFont{}#1}}
\renewcommand{\AgdaKeywordFontStyle}[1]{{\AgdaSansSerifFont{}#1}}
\renewcommand{\AgdaStringFontStyle}[1]{{\AgdaTypewriterFont{}#1}}
\renewcommand{\AgdaCommentFontStyle}[1]{{\AgdaTypewriterFont{}#1}}
\renewcommand{\AgdaBoundFontStyle}[1]{\textit{\AgdaSerifFont{}#1}}

\usepackage[style=alphabetic]{biblatex}
\addbibresource{refs.bib}



% macros
\newcommand{\investigate}[1]{\par\vspace{1\baselineskip}\textcolor{gray}{\textit{#1}}\vspace{1\baselineskip}\par}

% symbols
\newcommand{\bN}{\AgdaDatatype{ℕ}}
\newcommand{\bL}{\AgdaDatatype{Leibniz}}


\title{Restoring (part of) the friendship between recursion schemes and without-K (provisional)\\ \small I'll have to grab a UU-template at some point}
\author{Samuel Klumpers\\6057314}

% previous (provisional) titles:
% The gentle art of smashing things to bits and pieces
% Running in circles in Agda

\begin{document}
\maketitle
\tableofcontents

This document is generated from a literate agda file!
\newpage

\begin{abstract}
    The preliminary goal of this thesis is to introduce, among others, the concepts of the structure identity principle, numerical representations, and ornamentations, which are then combined to simplify the presentation and verification of finger trees, as a demonstration of the generalizability and improved compactness and security of the resulting code. 
\end{abstract}

\section{Introduction}\label{sec:intro}
Most of the time when we are Agda-ing \cite{agda} we are trying to un-Haskell ourselves, e.g., not take the head of an empty list. In this example, we can make \AgdaFunction{head} total by switching to length-indexed lists: vectors. We have now effectively doubled the size of our code base, since functions like \AgdaFunction{\_++\_} which we had for lists, will also have to be reimplemented for vectors.

To make things worse; often, after coping with the overloaded names resulting from Agda-ing by shoving them into a different namespace, we also find out that lists nor vectors are efficient containers to begin with. Maybe binary trees are better. We now need four times the number of definitions to keep everything working, and, if we start proving things, we will also have to prove everything fourfold. (Not to mention that reasoning about trees is probably going to be harder than reasoning about lists).  This inefficiency has sparked (my) interest in ways to deal with the situation.

Following \cite{orntrans} and \cite{progorn}, we can describe the relation between list and vector using the mechanism of ornamentation. This leads them to define the concept of patches, which can aid us when defining \AgdaFunction{\_++\_} for the second time by forcing the new version to be coherent. In fact, the algebraic nature of ornaments can even get us the definition of the vector type for free, if we started by defining lists relative to natural numbers \cite{algorn}. Such constructions rely heavily on descriptions of datastructures and often come with limitations in their expressiveness. These descriptions in turn impose additional ballast on the programmer, leading us to investigate reflection like in \cite{practgen} as a means to bring datatypes and descriptions closer when possible.

From a different direction, \cite{calcdata} gives methods by which we can show two implementations of some structure to be equivalent. With this, we can simply transport all proofs about \AgdaFunction{\_++\_} we have for lists over to the implementation for trees, provided that we show them to be equivalent as appendable containers. This process can also be automated by some heavy generics, but instead, we resort to cubical; which hosts a range of research like \cite{iri} tailored to the problem describing equivalences of structures.

We can liken the situation to movement on a plane, where ornamentation moves us vertically by modifying constructors or indices, and structured equivalences move us horizontally to and from equivalent but more equivalent implementations. In this paper, we will investigate a variety of means of moving around structures and proofs, and ways to make this more efficient or less intrusive.

Currently, all sections mainly reintroduce or reformulate existing research, with some spots of new ideas and original examples here and there. In \autoref{sec:leibniz}, we will look at how proofs on unary naturals can be moved to binary naturals. Then in \autoref{sec:numrep} we recall how numeral systems in particular induce container types, which we attempt to reformulate in the language of ornaments in \autoref{ssec:ornaments}.
%Before we try to improve or generalize upon these approaches, let us clarify some parts of the enviroment we are working in, partially by going through some examples.


\section{How Cubical Agda helps our binary numbers (ready)}\label{sec:leibniz}
Let us quickly review some features of Cubical Agda \cite{cuagda} that we will use in this section.

%Of course, this downside is more than offset by the benefits of changing our primitive notion of equality, which we will see makes it easier to show that ``equivalent'' structures behave identically. 
In Cubical Agda, the primitive notion of equality arises not (directly) from the indexed inductive definition we are used to, but rather from the presence of the interval type \AgdaPrimitiveType{I}. This type represents a set of two points \AgdaInductiveConstructor{i0} and \AgdaInductiveConstructor{i1}, which are considered ``identified'' in the sense that they are connected by a path. To define a function out of this type, we also have to define the function on all the intermediate points, which is why we call such a function a ``path''. Terms of other types are then considered identified when there is a path between them.

While the benefits are overwhelming for us\todo[inline, color=red]{Which?}, this is not completely without downsides, such as that
%\ExecuteMetaData[Tex/CubicalAndBinary]{cubical}% \todo[inline]{Not sure if it would be helpful to have a more extensive introduction covering all features used.} % at this moment, probably not, as the cubical usage is rather tame, so I'll probably stick to introducing stuff as it becomes necessary. % TODO then write that somewhere
the negation of axiom K complicates both some termination checking and some universe levels.\footnote{In particular, this prompts rather far-reaching (but not fundamental) changes to the code of previous work, such as to the machinery of ornaments \cite{progorn} in \autoref{sec:userfriendly}.} Furthermore, if we use certain homotopical constructions, and we wish to eliminate from our types as if they were sets, then we will also have to prove that they are indeed sets.

On the positive side, this different perspective gives intuitive interpretations to some proofs of equality, like
\ExecuteMetaData[Tex/CubicalAndBinary]{sym}
where \AgdaFunction{∼\_} is the interval reversal, swapping \AgdaInductiveConstructor{i0} and \AgdaInductiveConstructor{i1}, so that \AgdaFunction{sym} simply reverses the given path.

Furthermore, because we can now interpret paths in record and function types in a new way, we get a host of ``extensionality'' for free. For example, a path in $A \to B$ is indeed a function which takes each $i$ in \AgdaPrimitiveType{I} to a function $A \to B$. Using this, function extensionality becomes tautological 
\ExecuteMetaData[Tex/CubicalAndBinary]{funExt}

Finally, %while in ``non-univalent'' Agda bijections or isomorphisms do not play such a central role,
much of our work will rest on equivalences, as the ``HoTT-compatible'' generalization of bijections. This is because in Cubical Agda, we have the univalence theorem 
%the \AgdaPrimitiveType{Glue} type tells us that equivalent types fit together in a new type, in a way that guarantees univalence
\ExecuteMetaData[Tex/CubicalAndBinary]{ua}
stating that ``equivalent types are identified'', such that type isomorphisms like $1 \to A \simeq A$ become paths $1 \to A \equiv A$, making it so that we can transport proofs along them. We will demonstrate this by a more practical example in the next section.


\subsection{Unary numbers are binary numbers}\label{ssec:binary}
Let us demonstrate an application of univalence by exploiting the equivalence of the ``Peano'' naturals and the ``Leibniz'' naturals. Recall that the Peano naturals are defined as 
\ExecuteMetaData[Tex/CubicalAndBinary]{Peano}
This definition enjoys a simple induction principle and is well-covered in most libraries. However, the definition is also impractically slow, since most arithmetic operations defined on \bN{} have time complexity in the order of the value of the result.

As an alternative we can use binary numbers, for which for example addition has logarithmic time complexity. Standard libraries tend to contain few proofs about binary number properties, but this does not have to be a problem: the \bN{} naturals and the binary numbers should be equivalent after all!

Let us make this formal. We define the Leibniz naturals as follows:
\ExecuteMetaData[Leibniz/Base.tex]{Leibniz}
Here, the \AgdaInductiveConstructor{0b} constructor encodes 0, while the \AgdaInductiveConstructor{\_1b} and \AgdaInductiveConstructor{\_2b} constructors respectively add a 1 and a 2 bit, under the usual interpretation of binary numbers:
\ExecuteMetaData[Leibniz/Base.tex]{toN}
\ExecuteMetaData[Leibniz/Base.tex]{toN-2}
This defines one direction of the equivalence from \bN{} to \bL{}, for the other direction, we can interpret a number in \bN{} as a binary number by repeating the successor operation on binary numbers:
\ExecuteMetaData[Leibniz/Base.tex]{bsuc}
\ExecuteMetaData[Leibniz/Base.tex]{fromN}
To show that \AgdaFunction{toℕ} is an isomorphism, we have to show that it is the inverse of \AgdaFunction{fromℕ}. By induction on \bL{} and basic arithmetic on \bN{} we see that
\ExecuteMetaData[Leibniz/Properties.tex]{toN-suc}
so \AgdaFunction{toℕ} respects successors. Similarly, by induction on \bN{} we get
\ExecuteMetaData[Leibniz/Properties.tex]{fromN-1}
and % I can't get the code blocks to stick together lol
\ExecuteMetaData[Leibniz/Properties.tex]{fromN-2}
so that \AgdaFunction{fromℕ} respects even and odd numbers. We can then prove that applying \AgdaFunction{toℕ} and \AgdaFunction{fromℕ} after each other is the identity by repeating these lemmas
\ExecuteMetaData[Leibniz/Properties.tex]{N-iso-L}
This isomorphism can be promoted to an equivalence
\ExecuteMetaData[Leibniz/Properties.tex]{N-equiv-L}
which, finally, lets us identify \bN{} and \bL{} by univalence
\ExecuteMetaData[Leibniz/Properties.tex]{N-is-L}
The path \AgdaFunction{ℕ≡L} then allows us to transport properties from \bN{} directly to \bL{}; as an example, we have not yet shown that \bL{} is discrete, i.e., has decidable equality. Using substitution, we can quickly derive this\footnote{Of course, this gives a rather inefficient equality test, but for the homotopical consequences this is not a problem.}
\ExecuteMetaData[Leibniz/Properties.tex]{DiscreteL}
This can be generalized even further to transport proofs about operations from \bN{} to \bL{}.

\subsection{Functions from specifications}\label{ssec:useas}
As an example, we will define addition of binary numbers. We could transport \AgdaFunction{\_+\_} as a binary operation
\ExecuteMetaData[Extra/Algebra]{BinOp}
from \bN to \bL to get
\ExecuteMetaData[Tex/CubicalAndBinary]{badplus}
But this is inefficient, incurring an $O(n + m)$ overhead when adding $n$ and $m$. It is more efficient to define addition on \bL{} directly, making use of the binary nature of \bL{}, while agreeing with the addition on \bN{}. Such a definition can be derived from the specification ``agrees with \AgdaFunction{\_+\_}'', so we implement a syntax for giving definitions by equational reasoning, inspired by the ``use-as-definition'' notation used by Hinze and Swierstra \cite{calcdata}: Using an implicit pair type
\ExecuteMetaData[Prelude/UseAs.tex]{isigma}
we define
\ExecuteMetaData[Prelude/UseAs.tex]{Def}
which extracts a definition as the right endpoint of a given path.
% \investigate{As of now, I am unsure if this reduces the effort of implementing a coherent function, or whether it is more typically possible to give a smarter or shorter proof by just giving a definition and proving an easier property of it\footnote{I will put the alternative in the appendix for now}}

With this we can define addition on \bL{} and show it agrees with addition on \bN{} in one motion
\ExecuteMetaData[Leibniz/Properties.tex]{plus-def}
Now we can easily extract the definition of \AgdaFunction{plus} and its correctness with respect to \AgdaFunction{\_+\_} 
\ExecuteMetaData[Leibniz/Properties.tex]{plus-good}

We remark that \AgdaFunction{Def} is close in concept to refinement types\footnote{À la \href{https://agda.github.io/agda-stdlib/Data.Refinement.html}{Data.Refinement}.}, but extracts the value from the proof, rather than requiring it before. \footnote{Unfortunately, normalizing an application of a \AgdaFunction{defined-by} function also causes a lot of unnecessary wrapping and unwrapping, so \AgdaFunction{Def} is mostly only useful for presentation.} %for now..


\subsection{The Structure Identity Principle}
We point out that \bN{} with \AgdaFunction{N.+} and \bL{} with \AgdaFunction{plus} form magmas, that is, inhabitants of
\ExecuteMetaData[Extra/Algebra.tex]{Magma'}
Using that a path in a dependent pair corresponds to a dependent pair of paths, we get a path from (\bN{}, \AgdaFunction{N.+}) to (\bL{}, \AgdaFunction{plus}). %More generally, a magma is simply a type $X$ with some structure, which is a function $f: X \to X \to X$ in the case of a magma. We can see that paths between magmas correspond to paths $p$ between the underlying types $X$ and paths over $p$ between their operations $f$.
This observation is further generalized by the Structure Identity Principle (SIP) as a form of representation independence \cite{iri}. Given a structure, which in our case is just a binary operation
\ExecuteMetaData[Extra/Algebra.tex]{MagmaStr}
this principle produces an appropriate definition ``structured equivalence'' $\iota$. The $\iota$ is such that if structures $X, Y$ are $\iota$-equivalent, then they are identified. In the case of \AgdaFunction{MagmaStr}, the $\iota$ asks us to provide something with the same type as \AgdaFunction{plus-coherent}, so we have just shown that the \AgdaFunction{plus} magma on \bL{}
\ExecuteMetaData[Leibniz/Properties.tex]{magmaL}
and the \AgdaFunction{\_+\_} magma on \bN{} and are identical
\ExecuteMetaData[Leibniz/Properties.tex]{magma-equal}
As a consequence, properties of \AgdaFunction{\_+\_} directly yield corresponding properties of \AgdaFunction{plus}. For example,
\ExecuteMetaData[Leibniz/Properties.tex]{assoc-transport}\todo[inline, color=red]{Express what this accomplishes, and why this is impressive compared to without univalence}

\section{Specifying types (ready)}\label{sec:numrep}
While the practical applications of the last example do not stretch very far\footnote{Considering that \AgdaDatatype{ℕ} is a candidate to be replaced by a more suitable unsigned integer type when compiling to Haskell anyway.}, the approach generalizes to the more relevant containers and their associated laws.

In the same vein as the last section, we could define a simple but inefficient array type, and a more efficient implementation using trees. Then we can show that these are equivalent, such that when the simple type satisfies a set of laws, trees will satisfy them as well. We could then start developing all sorts of complex implementations fine-tuned to each operation and figure out how these are equivalent to some simpler type, but let us first take a step back, and investigate how we can make this approach run smoothly in a simpler example.

Rather than inductively defining a container and then showing that it is represented by a lookup function, we can go the other way around and define a type by insisting that it is equivalent to such a function. This approach, in particular the case in which one calculates a container with the same shape as a numeral system, was dubbed numerical representations in \cite{purelyfunctional}, and has some formalized examples in, e.g., \cite{calcdata} and \cite{progorn}. Numerical representations form the starting point for defining more complex datastructures based on simpler ones, so let us demonstrate such a calculation. 

\subsection{Numerical representations: from numbers to containers}\label{ssec:numrep}
We can compute the type of vectors starting from \bN{}.\footnote{This is adapted (and fairly abridged) from \cite{calcdata}} For simplicity, we define them as a type computing function via the ``use-as-definition`` notation from before. We expect vectors to be represented by 
\ExecuteMetaData[Tex/NumRep]{Lookup}
where we use the finite type \AgdaDatatype{Fin} as an index into vector. Using this representation as a specification, we can compute both \AgdaDatatype{Fin} and a type of vectors. The finite type can be computed from the evident definition
\ExecuteMetaData[Tex/NumRep]{Fin-def}
using
\ExecuteMetaData[Tex/NumRep]{leq-split}
Likewise, vectors can be computed by applying a sequence of type isomorphisms
\ExecuteMetaData[Tex/NumRep]{Vec}
\investigate{SIP doesn't mesh very well with indexed stuff, does HSIP help?}
Of course, a container would not be of much use without lookup functions, so we define an interface
\ExecuteMetaData[Tex/NumRep]{Array}
which at the very least has to satisfy laws like
\ExecuteMetaData[Tex/NumRep]{Laws}
We could directly show that \AgdaDatatype{Vec} satisfies this, but now that we defined \AgdaDatatype{Vec} from \AgdaDatatype{Lookup} we might as well use this fact.

The implementation of arrays as functions is very straightforward
\ExecuteMetaData[Tex/NumRep]{FunArray}
and clearly satisfies our interface
\ExecuteMetaData[Tex/NumRep]{FunLaw}
We can implement arrays based on \AgdaDatatype{Vec} as well
\ExecuteMetaData[Tex/NumRep]{VecArray}
and again, we can transport the proofs from \AgdaDatatype{Lookup} to \AgdaDatatype{Vec}.\footnote{Except in this oversimplified case the laws are trivial for \AgdaDatatype{Vec} as well.}\todo{If one was determined to cobble together the path over path over path we need now.}
\investigate{As you can see, taking ``use-as-definition'' too literally prevents Agda from solving a lot of metavariables.}

\investigate{This computation can of course be generalized to any arity zeroless numeral system; unfortunately beyond this set of base types, this ``straightforward'' computation from numeral system to container loses its efficacy. In a sense, the n-ary natural numbers are exactly the base types for which the required steps are convenient type equivalences like $(A + B) \to C = (A \to C) \times (B \to C)$?}

%\subsection{Relating types by structure: Ornamentation (unfinished)}\label{sec:ornament}
\subsection{Numerical representations as ornaments}\label{ssec:ornaments}
We could peform the same computation for \bL{}, which would yield the type of binary trees, but we note that these computations proceed with roughly the same pattern: each constructor of the numeral system gets assigned a value, and is amended with a field holding a number of elements and subnodes using this value as a ``weight''. But wait! Such modifications of constructors are already made formal by the concept of ornamentation!\todo{It seems like Agda forgets that we defined Leibniz if we move between different .tex files, so I'll have to float all AgdaTargets to the top level at some point...}

Ornamentation, as exposed in \cite{algorn} and \cite{progorn}, lets us formulate what it means for two types to have a ``similar'' recursive structure. This is achieved by interpreting (indexed inductive) datatypes from descriptions, between which an ornament is seen as a certificate of similarity, describing which fields or indices need to be introduced or dropped. Furthermore, a one-sided ornament: an ornamental description, lets us describe new datatypes by recording the modifications to an existing description.
\todo{Again not sure how much space I should use to reiterate Desc, Orn, and OrnDesc.}

This links back to the construction in the previous section, since \bN{} and \AgdaDatatype{Vec} share the same recursive structure, so \AgdaDatatype{Vec} can be formed by introducing indices and adding a field holding an element at each node.\footnote{These and similar examples are also documented in \cite{progorn}} 

However, instead deriving \AgdaDatatype{List} from \bN{} generalizes to \bL{} with less notational overhead, so lets tackle that case first. For this, we have to give a description of \bN{} to work with\todo{Clearly this can use more explanation (the question is, how much?)}
\ExecuteMetaData[Tex/NumRepOrn]{NatD}
Recall that \AgdaInductiveConstructor{σ} adds a field, upon which the rest of the description may vary, and \AgdaInductiveConstructor{ṿ} lists the recursive fields and their indices (which can only be \AgdaInductiveConstructor{tt}).
We can now write down the ornament which adds fields to the \AgdaFunction{suc} constructor
\ExecuteMetaData[Tex/NumRepOrn]{ListO}
Here, the \AgdaInductiveConstructor{σ} and \AgdaInductiveConstructor{ṿ} are forced to match those of \AgdaDatatype{NatD},
but the \AgdaInductiveConstructor{Δ} adds a new field. With the least fixpoint and description extraction from \cite{progorn}, this is sufficient to define \AgdaDatatype{List}. Note that we cannot hope to give an unindexed ornament from \bL{}
\ExecuteMetaData[Tex/NumRepOrn]{LeibnizD}
into trees, since trees have a very different recursive structure! Instead, we must keep track at what level we are in the tree so that we can ask for adequately many elements:
\ExecuteMetaData[Tex/NumRepOrn]{TreeO}
We use the \AgdaFunction{power} combinator to ensure that the digit at position $n$, which has weight $2^n$ in the interpretation of a binary number, also holds its value times $2^n$ elements. This makes sure that the number of elements in the tree shaped after a given binary number also is the value of that  binary number.

This ``folding in'' technique using the indices to keep track of structure seems to apply more generally. Let us explore this a bit further, and return later to the generalization of the pattern from numeral systems to datastructures.
% i.e. why did this even work?

\subsection{Folding in}\label{ssec:flattening}
Let us describe this procedure of folding a complex recursive structure into a simpler structure more generally. In particular, we will demonstrate that for linear datatypes, such as \bN{} and \bL{}, and for a given unindexed datatype, there is always an indexed datatype isomorphic to it at some index, and an ornament from the linear type to the indexed type. 

Suppose we are given a description, the first thing we can do to simplify it is collect all fields in one place
\ExecuteMetaData[Tex/Flatten]{RField}
Next, we will certainly have to count the number of recursive occurrences we are tracking, so we define
\ExecuteMetaData[Tex/Flatten]{Number}
where \AgdaInductiveConstructor{𝟙} records that we are at the top level, and \AgdaInductiveConstructor{ṿ} denotes that we are below a constructor with some number of recursive fields. This simplifies our task to implementing the types in
\ExecuteMetaData[Tex/Flatten]{nested}
such a way that we get an isomorphism 
\ExecuteMetaData[Tex/Flatten]{wish}
Thus, \AgdaDatatype{Fields} is forced to have a \AgdaInductiveConstructor{leaf} constructor like 
\ExecuteMetaData[Tex/Flatten]{Fields}
if \AgdaFunction{nested} is to work at \AgdaInductiveConstructor{𝟙}. The \AgdaInductiveConstructor{node} constructor makes sure that if we have collection of \AgdaDatatype{Fields}, then we can gather them in a field at a higher level. We can then count the subnodes of a given \AgdaDatatype{Fields} as
\ExecuteMetaData[Tex/Flatten]{subnodes}
where \AgdaFunction{RSize} counts the number of recursive fields of a particular branch
\ExecuteMetaData[Tex/Flatten]{RSize}
Note that \AgdaFunction{subnodes} effectively keeps the shape of the previous field, but unfolds the recursive fields at the bottom of the tree by one level.

\investigate{I then tried and realized how unpleasant even the functions from the original type to the nested type are to write.}

As a trivialty, we get that any type, interpreted as a container, always decomposes as an ornament over a ``numerical'' base type.\todo{Or at least, that was where I was trying to go with this, but I notice that this still is a bit further away.} This links to the construction of binary heaps in \cite{progorn}, as in hindsight, starting from the usual binary heaps would yield binary numbers and their binary heap ornament (in a much less useful package).

\section{Reducing friction (work in progress)}\label{sec:userfriendly}
% REPLACE X BY A?
The setup some approaches in earlier sections require makes them tedious or impractical to apply. In this section we will look at some ways how part of this problem could be alleviated through generics, or by alternative descriptions of concepts like equivalences through the lens of initial algebras. 

In later sections we will construct many more equivalences between more complicated types than before, so we will dive right into the latter. Reflecting upon \autoref{sec:leibniz}, we see that when one establishes an equivalence, most of the time is spent working out a series of tedious lemmas to show that the conversion functions are mutual inverses, which tend to be relatively easy to define. We take away two things from this; the first is that the conversion functions are perhaps too obvious, and the second is that we should really avoid talking about sections and retractions lest we incur tedium!\footnote{The latter perhaps less so, because it is useful to show a map to be monic.} We will reuse the machinery of Ko and Gibbons \cite{progorn} to illustrate how the definitions in \autoref{sec:leibniz} were actually forced for a large part.

First, we remark that \AgdaDatatype{μ} is internalization of the representation of simple\footnote{Of course, indexed datatypes are indexed W-types, mutually recursive datatypes are represented yet differently\dots} datatypes as W-types. Thus, we will assume that one of the sides of the equivalence is always represented as an initial algebra of a polynomial functor, and hence the \AgdaDatatype{μ} of a \AgdaDatatype{Desc′}.

\subsection{Well-founded monic algebras are initial}\label{ssec:wellfounded}
Unfortunately, the machinery developed by Ko and Gibbons \cite{progorn} relies on axiom K for a small but crucial part. To be precise, in a cubical setting, the type \AgdaDatatype{μ} as given stops being initial for its base functor! In this section, we will be working with a simplified and repaired version. Namely, we simplify \AgdaDatatype{Desc′} to 
\ExecuteMetaData[Extra/ProgOrn/Desc]{DescS}
To complete the definition of \AgdaDatatype{μ}
\ExecuteMetaData[Extra/ProgOrn/Desc]{mu}
we will need to implement \AgdaDatatype{Base}. We remark that in the original setup, the recursion of \AgdaFunction{mapFold} is a structural descent in \AgdaFunction{⟦ D' ⟧ (μ D)}. Because \AgdaFunction{⟦\_⟧} is a type computing function and not a datatype, this descent becomes invalid\footnote{Refer to the  \href{https://agda.readthedocs.io/en/latest/language/without-k.html\#restrictions-on-termination-checking}{without K} page.}, and \AgdaFunction{mapFold} fails the termination check. We resolve this by defining \AgdaDatatype{Base} as a datatype
\ExecuteMetaData[Extra/ProgOrn/Desc]{Base}
such that this descent is allowed by the termination checker without axiom K.\footnote{This has, again by the absence of axiom K, the consequence of pushing the universe levels up by one. However, this is not too troublesome, as equivalences can go between two levels, and indeed types are equivalent to their lifts.}

Recall that the \AgdaDatatype{Base} functors of descriptions are special polynomial functors, and the fixpoint of a base functor is its initial algebra. We are looking for sufficient conditions on $X$ to get the equivalence $e: X \cong \mu F$. Note that when $X \cong \mu F$, then there necessarily is an initial algebra $F X \to X$. Conversely, if the algebra $(X, f)$ is isomorphic to $(\mu F, \mathrm{con})$, then $X \cong \mu F$ would follow immediately, so it is equivalent to ask for the algebras to be isomorphic instead.

\begin{comment}
The situation so far is summarized by the diagram
% https://q.uiver.app/?q=WzAsMyxbMSwwLCJGXFxtdV9GIl0sWzEsMSwiXFxtdSBGIl0sWzAsMSwiWCJdLFswLDEsIlxcbWF0aHJte2Nvbn0iXSxbMiwxLCJlIiwyLHsic3R5bGUiOnsidGFpbCI6eyJuYW1lIjoiYXJyb3doZWFkIn0sImJvZHkiOnsibmFtZSI6ImRhc2hlZCJ9fX1dXQ==
\[\begin{tikzcd}[ampersand replacement=\&]
	\& {F\mu_F} \\
	X \& {\mu F}
	\arrow["{\mathrm{con}}", from=1-2, to=2-2]
	\arrow["e"', dashed, tail reversed, from=2-1, to=2-2]
\end{tikzcd}\]
\end{comment}
\subsubsection{Datatypes as initial algebras}
To characterize when such algebras are isomorphic, we reiterate some basic category theory, simultaneously rephrasing it in Agda terms.\footnote{We are not reusing a pre-existing category theory library for the simple reasons that it is not that much work to write out the machinery explicitly, and that such libraries tend to phrase initial objects in the correct way, which is too restrictive for us.}

Let $C$ be a category, and let $a, b, c$ be objects of $C$, so that in particular we have identity arrows $1_a : a \to a$ and for arrows $g : b \to c, f : a \to b$ composite arrows $gf : a \to c$ subject to associativity. In our case, $C$ is the category of types, with ordinary functions as arrows.

Recall that an endofunctor, which is simply a functor $F$ from $C$ to itself, assigns objects to objects and sends arrows to arrows
\ExecuteMetaData[Extra/Category]{RawFunctor}
These assignments are subject to the identity and composition laws
\ExecuteMetaData[Extra/Category]{Functor}
An $F$-algebra is just a pair of an object $a$ and an arrow $Fa \to a$
\ExecuteMetaData[Extra/Category]{Algebra}
Algebras themselves again form a category $C^F$. The arrows of $C^F$ are the arrows $f$ of $C$ such that the following square commutes% https://q.uiver.app/?q=WzAsNCxbMCwwLCJGYSJdLFsxLDAsIkZiIl0sWzAsMSwiYSJdLFsxLDEsImIiXSxbMiwzLCJmIiwyXSxbMCwyLCJVX2EiLDJdLFsxLDMsIlVfYiJdLFswLDEsIkZmIl1d
\[\begin{tikzcd}[ampersand replacement=\&]
	Fa \& Fb \\
	a \& b
	\arrow["f"', from=2-1, to=2-2]
	\arrow["{U_a}"', from=1-1, to=2-1]
	\arrow["{U_b}", from=1-2, to=2-2]
	\arrow["Ff", from=1-1, to=1-2]
\end{tikzcd}\]
So we define
\ExecuteMetaData[Extra/Category]{AlgSqr}
and
\ExecuteMetaData[Extra/Category]{AlgMap}
Note that we take the propositional truncation of the square, such that algebra maps with the same underlying morphism become propositionally equal
\ExecuteMetaData[Extra/Category]{AlgPath}
The identity and composition in $C^F$ arise directly from those of the underlying arrows in $C$.

Recall that an object $\emptyset$ is initial when for each other object $a$, there is a unique arrow $!: \emptyset \to a$. By reversing the proofs of initiality of \AgdaDatatype{μ} and the main result of this section, we obtain a slight variation upon the usual definition. Namely, unicity is often expressed as contractability of a type
\ExecuteMetaData[Tex/Snippets]{isContr}
Instead, we again use a truncation
\ExecuteMetaData[Extra/Category]{weakContr}
but note that this also, crucially, slightly stronger than connectedness. We define initiality for arbitrary relations
\ExecuteMetaData[Extra/Category]{Initial}
such that it closely resembles the definition of least element. Then, $A$ is an initial algebra when
\ExecuteMetaData[Extra/Category]{InitAlg}

By basic category theory (using the usual definition of initial objects), two initial objects $a$ and $b$ are always isomorphic;
namely, initiality guarantees that there are arrows $f : a \to b$ and $g : b \to a$, which by initiality must compose to the identities again.

Similarly, we get that
\ExecuteMetaData[Extra/Category]{InitAlg-equiv}
However, we only have the equalities from the isomorphism inside a propositional truncation. But fortunately, being an equivalence is a property, so we can eliminate from the truncations to get the wanted result.

%Note that even though we warned ourselves, we are still talking about sections and retractions to establish that $f$ is an equivalence! However, this result also makes sure we will not have to speak of them again.

\subsubsection{Accessibility}
As a consequence, we get that $X$ is isomorphic to $\mu D$ when $X$ is an initial algebra for the base functor of $D$; $\mu D$ is initial by its fold, and by induction on $\mu D$ using the squares of algebra maps. 

\begin{remark}
    We need (in general) not hope $\mu D$ is a strict initial object in the category of algebras. For a strict initial object, having a map $a \to \emptyset$ implies $a \cong \emptyset$. This is not the case here: strict initial objects satisfy $a \times \emptyset \cong \emptyset$, but for the $X \mapsto 1 + X$-algebras $\mathbb{N}$ and $2^\mathbb{N}$ clearly $2^\mathbb{N} \times \mathbb{N} \cong \mathbb{N}$ does not hold. On the other hand, the ``obvious'' sufficient condition to let $C^F$ have strict initial objects is that $F$ is a left adjoint, but then the carrier of the initial algebra is simply $\bot$.
\end{remark}

Looking back at \autoref{sec:leibniz}, we see that \AgdaDatatype{Leibniz} is an initial $F: X \mapsto 1 + X$ algebra because for any other algebra, the image of \AgdaFunction{0b} is fixed, and by \AgdaFunction{bsuc} all other values are determined by chasing around the square. Thus, we are looking for a similar structure on $f : FX \to X$ that supports recursion.

Clearly we will need something stronger than $FX \cong X$, as in general a functor can have many fixpoints. For this, we define what it means for an element $x$ to be accessible by $f$. This definition uses a mutually recursive datatype as follows:
We state that an element $x$ of $X$ is accessible when there is an accessible $y$ in its fiber over $f$
\ExecuteMetaData[Extra/Category/Poly]{Acc}
Accessibility of an element $x$ of \AgdaFunction{Base A E} is defined by cases on $E$; if $E$ is \AgdaFunction{ṿ n} and $x$ is a \AgdaFunction{Vec A n}, then $x$ is accessible if all its elements are; if $x$ is \AgdaFunction{σ S E'}, then $x$ is accessible if \AgdaFunction{snd x} is
\ExecuteMetaData[Extra/Category/Poly]{Acc'}
Consequently, $X$ is well-founded for an algebra when all its elements are accessible
\ExecuteMetaData[Extra/Category/Poly]{Wf}

We can see well-foundedness as an upper bound on the size of $X$, if it were larger than $\mu D$, some of its elements would inevitably get out of reach of an algebra. \textit{Now} having $FX \cong X$ also gives us a lower bound, but remark that having a well-founded injection $f: FX \to X$ is already sufficient, as accessibility gives a section of $f$, making it an iso. This leads us to claim
\begin{claim}\label{claim:wf-inj-init}
    If there is a mono $f : FX \to X$ and $X$ is well-founded for $f$, then $X$ is an initial $F$-algebra.
\end{claim}

\subsubsection{Proof sketch of \autoref{claim:wf-inj-init}}
Let us be on our way. Suppose $X$ is well-founded for the mono $f : FX \to X$. To show that $(X, f)$ is initial, let us take another algebra $(Y, g)$, and show that there is a unique arrow $(X, f) \to (Y, g)$.\todo[inline]{This section is about as digestable as a brick.}

By \AgdaDatatype{Acc}-recursion and because all $x$ are accessible, we can define a plain map into $Y$
\ExecuteMetaData[Extra/Category/WellFounded]{Wf-rec}
This construction is an instance of the concept of ``well-founded recursion''\footnote{This is formalized in the \href{https://agda.github.io/agda-stdlib/Induction.WellFounded.html}{standard-library} with many other examples.}, so we let ourselves be inspired by these methods. In particular, we prove an irrelevance lemma
\ExecuteMetaData[Extra/Category/WellFounded]{Wf-rec-irr}
which implies the unfolding lemma
\ExecuteMetaData[Extra/Category/WellFounded]{Wf-rec-unfold}
The unfolding lemma ensures that the map we defined by \AgdaFunction{Wf-rec} is a map of algebras. The proof that this map is unique proceeds analogously to that in the proof that $\mu D$ is initial, but here we instead use \AgdaDatatype{Acc}-recursion
\ExecuteMetaData[Extra/Category/WellFounded]{Wf+inj=Init}
Thus, we conclude that $X$ is initial. The main result is then a corollary of initiality of $X$ and the isomorphism of initial objects
\ExecuteMetaData[Extra/Category/WellFounded]{Wf+inj=mu}


\subsubsection{Example}
Let us redo the proof in \autoref{sec:leibniz}, now using this result. Recall the description of naturals \AgdaFunction{NatD}. To show that \AgdaFunction{Leibniz} is isomorphic to \AgdaFunction{Nat}, we will need a \AgdaFunction{NatD}-algebra and a proof of its well-foundedness. We define the algebra
\ExecuteMetaData[Tex/Leibniz2]{bsuc'}

For well-foundedness, we use something similar to view-patterns %[mcbride]
(the main difference being that we look through the entire structure, instead of a single layer)
\ExecuteMetaData[Tex/Leibniz2]{Peano-View}
where the mutually recursive proof of \AgdaFunction{view} is ``almost trivial''. Well-foundedness follows fairly immediately
\ExecuteMetaData[Tex/Leibniz2]{Wf-bsuc}

Injectivity of \AgdaFunction{bsuc\_1} happens to be harder to prove from retractions than directly, so we prove it directly, from which the wanted statement follows
\ExecuteMetaData[Tex/Leibniz2]{L-is-mu-N}

Note that in this case it took us more code to prove the same statement! However, we stress that the code that we did write became more forced, and might be more amenable to automation.



\section{Is equivalence too strong (finger trees)}\label{sec:weakening}



\section{Discussion and future work (aka the union of my to-do list and the actual future work section)}\label{sec:discussion}



\section{Temporary}\label{sec:temp}
\listoftodos
%\subfile{Scratch.tex}


\printbibliography
\end{document}
