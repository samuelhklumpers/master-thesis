% is fold and map really valid and terminating? -> the whole --cubical --safe not being safe really undermines this
% Q: are index-first descriptions even complete? -> everything has to be suffixed with (excluding cramming everything into the index)


\section{δ is conservative over Desc and Orn}
The numerical representation presented in \autoref{sec:trieo} relies on \AIC{δ} to provide the interpretation of the components making up a composite number system such as \AF{PhalanxD}. However, this is not necessary in the presence of \AD{Info}, since we can also request information to turn a \AIC{σ} into a \AIC{δ} when applicable:
\ExecuteMetaData[Tex/Discussion]{Delta-Info}
If we include the ornament \AIC{∇σ} dropping a field by giving a default value \AV{V}\ \AF{⊧}\ \AV{S} in place of a \AIC{σ}:
\ExecuteMetaData[Tex/Discussion]{nabla-sigma}
then we can also represent \AIC{∙δ} without further modifying \AD{ConOrnDesc}. Namely
\ExecuteMetaData[Tex/Discussion]{comp-delta-nabla-sigma}
This emulates the \AIC{∙δ} over an ornament \AV{RR′}, by first adding a field of \AD{μ}\ (\AF{toDesc}\ \AV{RR′}) and then fixing a default value for \AD{μ}\ \AV{R} by using \AF{ornForget}.

This makes the presentation of the descriptions and ornaments, and the interpretations of both simpler. However, this has the downside of needing a transport (or, with-abstraction) for each pattern match on a value which would otherwise be a \AIC{δ}.


\section{Σ-descriptions are more natural for expressing finite types}
One reason we did not present indexed numerical representations is that representing finite types of arbitrary number systems in \AD{DescI} quickly becomes hard. Consider the binary numbers from before
\ExecuteMetaData[Tex/Discussion/Sigma-Desc]{Leibniz}
for which the finite type
\ExecuteMetaData[Tex/Discussion/Sigma-Desc]{FinB}
has more constructors than the numbers themselves, obstructing an ornament from numbers to their finite types. Furthermore, the number of constructors of the finite type depends both on the multipliers and constants in all fields and leaves of the number system, which prevents us from constructing the finite type on-the-fly like for \AF{trieifyOD} (that is, without passing around lists of constructors instead).

This mismatch of the relation between a number and its finite type, and our definition of descriptions and ornaments, stems from our treatment of the field-former \AIC{σ}. Some treatments of descriptions \cite{effectfully,progorn,algorn} encode a dependent field \AV{(s : S)} by asking for a function \AV{C} assigning values \AV{s} to descriptions, while we merely ask for a description in a context extended by \AV{S}. This keeps out some more exotic descriptions \cite{?}, but also prevents us from introducing branches inside a constructor.

\todo{Comparing SOP and computational sigmas. In particular, sigma bN (n -> v (replicate n tt)) is not in SOP without full nesting. SOP is good for generics in both directions (the conversion in both ways keeps the datatype like it is supposed to).}

If we instead started from Σ-descriptions, taking functions into \AD{DescI} to encode dependent fields, we could compute a ``type of paths'' in a number system by adding and deleting the appropriate fields. Consider the universe
\ExecuteMetaData[Tex/Discussion/Sigma-Desc]{Sigma-Desc}
In this universe we can present the binary numbers as
\ExecuteMetaData[Tex/Discussion/Sigma-Desc]{LeibnizD}
The finite type for these numbers can be described by
\ExecuteMetaData[Tex/Discussion/Sigma-Desc]{FinBD}
Since this description of \AF{FinB} largely has the same structure as \AF{Leibniz}, and as a consequence also the numerical representation associated to \AF{Leibniz}, this would simplify proving that the indexed numerical representation is indeed equivalent to the representable representation (the maps out of \AF{FinB}). For more flexible ornaments, we can even describe the finite type as an ornament on the number system.

\todo{Related work glookup. We do path and type simultaneously, for a small class, and with specific behaviour. They do type first, and then compute the paths.}

\section{No Algebraic Ornaments for DescI}
Another reason for not giving the indexed numerical representations is that, provided the descriptions and ornaments are \todo{nice}, these can be computed from their unindexed variants.

Unfortunately, our descriptions are not nice. Morally speaking, the indexed numerical representations are always ornaments over the unindexed numerical representations. We can observe this directly for \AD{List} and \AD{Vec}, likewise, we can imagine how we can ornament \AF{RandomOD} to get its indexed variant. More generally, this ornament is always the algebraic ornament, which is formed by inserting fields which ensure that \AF{ornForget} and the index of the type always line up \cite{algorn}.

However, this is not the case for \AD{DescI}, since the algebras \AF{⟦}\ \AV{D}\ \AF{⟧D}\ \AV{X}\ \AF{⇶}\ \AV{X} for a description \AV{D} do not yield algebras its subdescriptions in a \AIC{δ}. We can see this if we try to derive the wanted algebraic ornament:

Consequently, an algebraic ornament does not have sufficient information to re-index a description \AV{R} in a \AIC{δ}, making it impossible to present the intended indexed numerical representation as an algebraic ornament.

We can remedy this by simply directly asking for the necessary functions, instead of hoping they bubble out of the functions out of interpretations.
\[ ... \]


% \section{Why not indexed numerical representations} % is this even necessary?



%\section{better folds and algebraic ornaments}
%\investigate{Having a function of the same type as \AgdaFunction{ornForget} is not sufficient to deduce an ornament. An obstacle is that the usual empty type (no constructors) and the non-wellfounded empty type (only a recursive field) don't have an ornament. Also, while the leaf-preservation case spells itself out, the substitutions obviously don't give us a way to recover the equalities.} -> idk


% \begin{remark}
%     Not all functions $\mu D \to X$ come from algebras $[D] X \to X$. Consider the function $\mathrm{pred} : N \to N$, sending $\mathrm{suc} n$ to $n$ and 0 to 0.
%     % https://q.uiver.app/#q=WzAsNyxbMCwwLCIxK04iXSxbMCwxLCJOIl0sWzEsMSwiTiJdLFsxLDAsIjErTiJdLFsyLDAsIk4iXSxbMywxLCJOIl0sWzMsMCwiTiJdLFswLDEsIlxcbWF0aHJte3N1Y30iLDJdLFszLDIsIj8iLDAseyJzdHlsZSI6eyJib2R5Ijp7Im5hbWUiOiJkYXNoZWQifX19XSxbMCwzLCIxK1xcbWF0aHJte3ByZWR9Il0sWzEsMiwiXFxtYXRocm17cHJlZH0iLDJdLFswLDIsIlxcbGFuZ2xlMCwgXFxtYXRocm17aWR9XFxyYW5nbGUiLDFdLFs0LDUsIlxcbWF0aHJte2lkfSIsMV0sWzQsNiwiXFxtYXRocm17cHJlZH0iXSxbNiw1LCIhISEiLDAseyJzdHlsZSI6eyJib2R5Ijp7Im5hbWUiOiJkb3R0ZWQifX19XV0=
%     \[\begin{tikzcd}
%         {1+N} & {1+N} & N & N \\
%         N & N && N
%         \arrow["{\mathrm{suc}}"', from=1-1, to=2-1]
%         \arrow["{?}", dashed, from=1-2, to=2-2]
%         \arrow["{1+\mathrm{pred}}", from=1-1, to=1-2]
%         \arrow["{\mathrm{pred}}"', from=2-1, to=2-2]
%         \arrow["{\langle0, \mathrm{id}\rangle}"{description}, from=1-1, to=2-2]
%         \arrow["{\mathrm{id}}"{description}, from=1-3, to=2-4]
%         \arrow["{\mathrm{pred}}", from=1-3, to=1-4]
%         \arrow["{!!!}", dotted, from=1-4, to=2-4]
%     \end{tikzcd}\]
%     We see that $\mathrm{pred}$ cannot come from a map $1 + N \to N$, as that map would be a retraction, while $\mathrm{pred}$ is not mono.
% \end{remark}


\section{Branching numerical representations}
\begin{outline}
We gave numerical representations by using nesting. You can also give the branching ones, if you had sigma-descriptions. This implementation of TrieO always computes the random-access variant of the datastructure. Can we implement a variant which computes the ``Braun tree'' variant of the datastructure?
\end{outline}

% Braun trie? -> NO! not without indices, and then it is no longer an algebraic ornament(?)



\section{Variables slightly later}
%package them with the constructor descriptions rather than only after a sigma
% separate substitutions -> probably 


\section{Less commutative squares}
%\autoref{rem:orn-lift} -> Discussion
% \begin{remark}\label{rem:orn-lift}
%     Rather than having the user provide two indices and show that the square commutes, we can ask for a ``lift'' $k$
%     % https://q.uiver.app/#q=WzAsNCxbMCwwLCJcXGJ1bGxldCJdLFsxLDAsIlxcYnVsbGV0Il0sWzAsMSwiXFxidWxsZXQiXSxbMSwxLCJcXGJ1bGxldCJdLFswLDEsImUiXSxbMiwzLCJmIiwyXSxbMiwwLCJqIl0sWzMsMSwiaSIsMl0sWzMsMCwiayIsMV1d
%     \[\begin{tikzcd}
%         \bullet & \bullet \\
%         \bullet & \bullet
%         \arrow["e", from=1-1, to=1-2]
%         \arrow["f"', from=2-1, to=2-2]
%         \arrow["j", from=2-1, to=1-1]
%         \arrow["i"', from=2-2, to=1-2]
%         \arrow["k"{description}, from=2-2, to=1-1]
%     \end{tikzcd}\]
%     and derive the indices as $i = ek, j = kf$. However, this is more restrictive, unless $f$ is a split epi, as only then pairs $i,j$ and arrows $k$ are in bijection. In addition, this makes ornaments harder to work with, because we have to hit the indices definitionally, whereas asking for the square to commute gives us some leeway (i.e., the lift would require the user to transport the ornament). 
% \end{remark}


\section{No RoseTrees}
\begin{outline}    
    Note that this allows us to express datatypes like finger trees, but not rose trees. Such datatypes would need a way to place a functor ``around the \AgdaInductiveConstructor{ρ}'', which then also requires a description of strictly positive functors. In our setup, this could only be encoded by separating general descriptions from strictly positive descriptions. The non-recursive fields of these strictly positive descriptions then need to be restricted to only allow compositions of strictly positive context functions. 
    
    This setup does not allow nesting over recursive fields, which is necessary for structures like rose trees. This is actually kind of essential for enumeration. Nesting over a recursive field is problematic: we can incorporate it by adding ``this'' implicitly to a \AgdaInductiveConstructor{δ}, but then the \AgdaBoundFontStyle{R} needs to be strictly positive in its last argument, meaning we need to split \AgdaDatatype{Desc} into a strictly positive part and normal part. The strictly positive part should then only allow strictly positive parameter transforms in recursive and non-recursive fields, requiring an embedding of transforms.
\end{outline}


\section{No levitation}
\begin{outline}    
    Rather, ornaments themselves could act as information bundles. If there was a description for \AgdaDatatype{Desc}, that is. Such a scheme of levitation would make it easier to switch between being able to actively manipulate information, and not having to interact with it at all. However, the complexity of our descriptions makes this a non-trivial task; since our \AgdaDatatype{Desc} is given by mutual recursion and induction-recursion, the descriptions, and the ornaments, would have to be amended to encode both forms of recursion as well. 
    
    If we levitate, then informed descriptions become ornaments over \AgdaDatatype{Desc}. This gives us the best of both worlds (modulo reflecting the description into a datatype): in plain descriptions, information does not even exist, and in informed descriptions, it is explicit. For levitation, we likely need induction-recursion and mutual recursion.

    Also trieifyOD would be a fold over DescI lol. 
\end{outline}

\section{Odd numerical representations}
the numerical representation of 2-3 fingertrees is a bit odd, or trivial.
I do not know whether there is a datastructure (let alone numerical representation) which has amortized constant append on both sides, and has logarithmic lookup, but uses only simple nesting (i.e., nesting over a functor with only products and no sums).



\section{Reconcile calculating and trieifyOD}
\begin{outline}
In the computation of generic numerical representations, we gave \AF{trieifyOD} directly, rather than as the consequence of a calculation. %\todo{This is simply because a) the wheels would come off very soon b) trieifyOD is not a definition but rather describes one.}

By abstracting \AF{Def} over a function, we can elegantly describe the kind of object we are looking for
\[ ... \]
but because we factor through an interpretation into \AD{Type}, we still have to give the definition before we can construct the isomorphism.

Maybe this works better for trieifyOD itself, where the isomorphism is really a composition of smaller isomorphisms by analyzing the descriptions, rather than one global isomorphism as is the case when comparing Lookup and VecD.

See Tex.Disussion.Def-cong
\end{outline}



\section{?}
While evidently Ix x != Fin (toN x) for arbitrary number systems, does the expected iso Ix x -> A = Trie A x yield Traversable, for free?





% \section{Useless}
%\todo{We could also use a telescope for indices, but we do not.} -> ok.