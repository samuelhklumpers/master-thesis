%\section{Appendices}

\newpage
\begin{appendix}
\section*{Appendices}
\addcontentsline{toc}{section}{Appendices}
\renewcommand{\thesubsection}{\Alph{subsection}}

\subsection{Folding}\label{app:gfold}
In \autoref{ssec:generic-programming} and \autoref{sec:background-ornaments} we used \AF{fold} as a concept to explain a bit of generic programming. We give its definition here, but for \AD{DescI} instead, since the fold of \AD{U-ix} can be seen as a simplification of it.
\ExecuteMetaData[Ornament/Desc]{fold-type}
As \AF{fold}\ \AV{f} is the algebra map \AIC{con}\ \AF{⇒}\ \AV{f}, the following commutes:
% https://q.uiver.app/#q=WzAsNCxbMSwwLCJGQSJdLFsxLDEsIkEiXSxbMCwxLCJcXG11IEYiXSxbMCwwLCJGXFxtdSBGIl0sWzAsMSwiZiJdLFsyLDEsIlxcbWF0aHJte2ZvbGR9XFwgZiIsMl0sWzMsMCwiRihcXG1hdGhybXtmb2xkfVxcIGYpIl0sWzMsMiwiXFxtYXRocm17Y29ufSIsMl1d
\[\begin{tikzcd}
	{F\mu F} & FA \\
	{\mu F} & A
	\arrow["f", from=1-2, to=2-2]
	\arrow["{\mathrm{fold}\ f}"', from=2-1, to=2-2]
	\arrow["{F(\mathrm{fold}\ f)}", from=1-1, to=1-2]
	\arrow["{\mathrm{con}}"', from=1-1, to=2-1]
\end{tikzcd}\]
However, by defining \AF{fold}\ \AV{f}\ (\AIC{con}\ \AV{x}) as \AV{f}\ (\AF{map}\ (\AF{fold}\ \AV{f}) x), we prevent the termination checker from seeing that \AF{fold} is only applied to terms strictly smaller than \AV{x} (much like our fellow universe constructions find out somewhere along the line). To help out the termination checker, we inline \AF{fold} into \AF{map}, which gives us an equivalent definition:
\ExecuteMetaData[Ornament/Desc]{mapFold}
Here \AF{mapDesc} (and \AF{mapCon}) simply peel off and reassemble all non-recursive structure, applying \AF{fold} to the recursive fields; \AF{fold} is then defined in the usual way by applying its algebra \AV{f} to itself mapped over \AV{x}.

\subsection{Folding without Axiom K}\label{app:withoutk}
The axiom of univalence (or cubical type theory) gives us another interesting context to study ornaments in. In the way we presented it, the theory of ornaments produces a lot of isomorphisms from relations between types, which are not yet as powerful as they could be when comparing properties between related types. Univalence gives us the means to turn equivalences\footnote{Equivalences can be considered as a correction to isomorphisms for types which are not sets (in the sense of being discrete); since all types we describe here are sets, equivalences and isomorphisms coincide.} into equalities, allowing us to put an isomorphism between types to work by transporting properties over it.

Unfortunately, a direct port of ornaments into \AV{--cubical} is quickly thwarted by the absence of Axiom K, as one would discover that the definitions of \AF{mapDesc} and \AF{mapCon} illegally pattern match on the types calculated by interpretations\footnote{The \href{https://agda.readthedocs.io/en/v2.6.4.1-rc1/language/without-k.html}{Without K documentation} explains why pattern matching on non-datatypes is not safe in general.}.

This can be remedied by presenting interpretations as datatypes\footnote{Albeit a bit dubiously; at the time of writing, this is also how you can circumvent a restriction on pattern matching emplaced by \AV{--cubical-compatible}, see \href{https://github.com/agda/agda/issues/5910\#issuecomment-1601301237}{the relevant GitHub issue}.}. Effectively, we are applying the duality between type computing functions and indexed types. Since \AD{Desc} and \AD{Con} are unindexed types, they cannot accidentally carry equational content, and pattern matching on them does not generate transports in \AF{⟦\_⟧D} and \AF{⟦\_⟧C}. Hence, the definition of \AF{fold} is (morally speaking) safe.

With that out of the way, we can define the interpretations as indexed types:
\ExecuteMetaData[Appendix/Intp]{Intp}
Since the interpretations are datatypes now, we can pattern match on them to define \AF{mapDesc} and \AF{mapCon} in a way that is accepted:
\ExecuteMetaData[Appendix/Intp]{mapFold}


\subsection{Nested types as uniformly recursive indexed types}\label{app:unnested}
Although \AD{U-ix} has no direct support for expressing nested types, we can actually give equivalent encodings for some of them.

Indeed, indices are readily repurposed as parameters. If we apply this to random-access lists, we can write:
\ExecuteMetaData[Appendix/NoNesting]{RandomD-1}
%Suppose that we defined \AD{U-nest} as the extension of \AD{U-ix} for nested types, then we can automate this transformation:
%\ExecuteMetaData[Appendix/NoNesting]{uniform}
More interestingly, perhaps, the depth of a random-access list determines the types of its fields. Namely, \AIC{One} will ask for 1 element at the highest level, one level down it asks for 2, and one more it asks for 4, and so on. Hence, in a way that vaguely resembles defunctionalization, we can define
\ExecuteMetaData[Appendix/NoNesting]{Pair}
and describe a field at depth \AV{n} by \AF{power}\ \AV{n}\ \AD{Pair}\ \AV{A}. This can be applied to describe random-access lists which track their depth in their index instead:
\ExecuteMetaData[Appendix/NoNesting]{RandomD-2}
Since we cannot (yet) construct path types generically (\autoref{sec:sigma-desc}), we cannot make this construction generic. If we did have such constructions, the argument for random-access lists generalizes to an operation that splits a nested datatype \AV{D} into three parts:
\begin{enumerate}
    \item a type of paths in \AV{D} (not necessarily pointing to a field)
    \item a lookup function that sends a path to the accumulated parameter transformation
    \item the (uniform) datatype, indexed over the paths, using the lookup function to calculate the types of its fields.
\end{enumerate}
\end{appendix}