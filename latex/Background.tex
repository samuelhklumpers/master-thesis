\section{Agda}
We formalize our work in Agda \cite{agda}, a functional programming language with dependent types. Using dependent types we can use Agda as a proof assistant, allowing us to state and prove theorems about our datastructures and programs. These proofs can then be run as algorithms, or in some cases be extracted to other languages like Haskell\footnote{Or JavaScript, if you want.}.

Syntactically Agda is reminiscent of Haskell. One difference is that Agda allows most characters and words in identifiers with only a small set of exceptions. For example, we can write
\ExecuteMetaData[Tex/Background]{ternary}
Another is that datatypes are always either given as generalized algebraic datatypes (GADTs) or record types. For example, the definition of booleans
\begin{verbatim}
    data Bool = True | False
\end{verbatim}
can be written in Agda as
\ExecuteMetaData[Tex/Background]{bool}
The unit type
\begin{verbatim}
    data Unit = Unit
\end{verbatim}
becomes\footnote{One can also write the unit type as a datatype with one constructor. However, in Agda, records (can) benefit from eta-expansion. In our case, all terms of \AgdaDatatype{⊤} are definitionally equal.}
\ExecuteMetaData[Tex/Background]{true}

The type system of Agda is an extension of (intensional) Martin-Löf type theory (MLTT), a constructive type theory in which we can interpret intuitionistic logic: the Curry-Howard isomorphism states that certain formulas correspond to certain types, and proofs of a formula correspond to terms of the corresponding type. The atomic formula true can be represented by \AgdaDatatype{⊤}, so that \AgdaFunction{tt} always proves true. False can be represented by a datatype with no constructors
\ExecuteMetaData[Tex/Background]{false}
since there is (hopefully) no way to make get a term of \AgdaDatatype{⊥} without inconsistent assumptions. The logical implication $A \implies B$ corresponds to the type of functions $A \to B$: a proof of $A$ can be converted to a proof of $B$. Using implication, we can define the negation $\lnot A$ of a formula $A$ as the type $A \to \bot$. Disjunction (logical or) is described by a sum type $A + B$:
\ExecuteMetaData[Tex/Background]{either}
if we have either $A$ or $B$, we can prove $A + B$. Conjunction (logical and) is given as a product type:
\ExecuteMetaData[Tex/Background]{pair}
we need both $A$ and $B$ to prove $A \times B$. Using the correspondence, we can reason in propositional logic by writing functional programs. As an example, consider the proof of the tautology 
\ExecuteMetaData[Tex/Background]{distr}

Compared to Haskell, Agda allows the type of the codomain of a function to vary with the applied value:
given a function $P$ from $A$ into \AgdaPrimitiveType{Type}, a type family over $A$, we can form the dependent function type $(a : A) \to P\ a$. Applying a function $f: (a : A) \to P\ a$ to a value $a : A$ then will have type $f\ a : P\ a$. Similarly, the type of a field in a record type can depend on values of earlier fields, e.g.,
\ExecuteMetaData[Tex/Background]{exists}
The presence of these dependent types enriches the interpretation of logic into programs. To interpret first-order logic we need to describe formulas containing variables, which are called predicates. Predicates correspond to functions%into \AgdaDatatype{Type}
\ExecuteMetaData[Tex/Background]{predicate}
Using predicates, we can interpret quantifiers as the dependent types above. Universal quantification (for all) is a dependent function type 
\ExecuteMetaData[Tex/Background]{forall}
since for each $a : A$, we have a proof of $P\ a$. Likewise, existential quantification (exists) is the dependent pair type $\exists$, since this gives an $a : A$ and a proof $P\ a$. 

Whereas the parameters of a type are fixed beforehand, an index can be fixed during construction. We can define the finite types as
\ExecuteMetaData[Tex/Background]{fin}
In each constructor, we pick an $n$ and set the index to $n+1$, ensuring that \texttt{Fin (suc n)} always has one element more than \texttt{Fin n}, and \texttt{Fin 0} has zero.

Predicates can also be expressed using indexed datatypes. E.g., equality of elements of a type $A$ can then be interpreted as the type
\ExecuteMetaData[Tex/Background]{eq}
Closed terms of this type can only be constructed for definitionally equal elements, but crucially, variables of this type can contain equalities between different elements. As the second argument is an index, pattern matching on \AgdaFunction{refl} unifies the elements, such that properties like substitution follow
\ExecuteMetaData[Tex/Background]{subst}

Unlike most languages, Agda rules out non-terminating functions by restricting their definitions to structural recursion. The termination checker (together with other restrictions which we will encounter in due time) prevents trivial proofs which would be tolerated in Haskell, like
\ExecuteMetaData[Tex/Background]{loop}
This ensures that all our interpretations mentioned above remain consistent.

\begin{comment}
With this, we can do maths. For example, we could define natural numbers as an inductive type
\[ \dots \]
and prove some properties of prime numbers. But to get the same results to binary numbers (without duplicating the proofs), we need a bit more. The usual notion of equalities of types are isomorphisms: two types $A, B$ are isomorphic if there are functions $A \to B$ and $B \to A$, which are mutually inverse 
\[ \dots \]
In ordinary Agda, we cannot directly apply these to transport along like we can for equalities, however.
\end{comment}


\section{Generic programming}\label{ssec:bg-desc}
%outline:
%we expound on generic programming
%we use this later to automate the construction of datastructures and proofs

%The deriving-mechanism in Haskell can take writing functions which consist primarily of boilerplate out of the hands of the programmer by deriving default implementations. Using reflection we can write similar macros and generic programs inside the type-checking monad; with it one can quote types or values, inspect their definitions, and unquote terms them to inject them into the code as if they were written manually.
%However, programming in this monad is generally not pleasant, as terms enjoy none of the safety we are used to from Agda, and type errors are only detected when applying macros as opposed to when writing them. That is not to say that effective generic programming is impossible in Agda, and quite the opposite is true \cite{practgen}\todo{And more}. We will take a closer look at constructions which we can use for datatype generic programming. 
\subsection{Descriptions}
To be able to reason about datatypes themselves, we first have to represent datatypes by another datatype.
This can be done by defining a datatype of codes instructing how datatypes can be formed, together with a function assigning the meaning to this encoding, henceforth description and interpretation respectively. We will start from an encoding which captures only a small set of types, and work towards an encoding of parametrized indexed types.


We can describe the universe of finite types with the following description:
\ExecuteMetaData{Tex/Desc.tex}[]
Each of the constructors of this description represents a type-former. In this case, the universe only contains sums and products of the 0 and 1; the meaning of the type-formers comes from the interpretation:
\ExecuteMetaData{Tex/Desc.tex}[]
Booleans live in this universe as
\ExecuteMetaData{Tex/Desc.tex}[]
but to encode a type like \bN{} we need a different setup. Consider the definition
\ExecuteMetaData{Tex/Desc.tex}[]
we can interpret this as the declaration \texttt{ℕ ≃ ⊤ ⊎ ℕ}, and formally \AgdaDatatype{ℕ} is indeed the least fixpoint of this equation. In category theoretic terms we would say that \AgdaDatatype{ℕ} is the initial algebra of its base functor \texttt{(⊤ ⊎\_)}.
 
Letting
\ExecuteMetaData{Tex/Desc.tex}[]
assign base functors to descriptions, we can take the fixpoint as
\ExecuteMetaData{Tex/Desc.tex}[]
We see that if \texttt{⟦ ND ⟧} is \texttt{(⊤ ⊎\_)}, then \texttt{μ ND} satisfies the equation for \AgdaDatatype{ℕ}.

We change the codes to
\ExecuteMetaData{Tex/Desc.tex}[]
and describe the base functors:
\ExecuteMetaData{Tex/Desc.tex}[]
Now, \AgdaInductiveConstructor{𝟙} encodes the leaves of a datatype, and \AgdaInductiveConstructor{ρ} encodes a recursive node. The operators \AgdaInductiveConstructor{⊕} and \AgdaInductiveConstructor{⊗} are changed to act pointwise. 
In this universe, we can define \AgdaDatatype{ℕ} by
\ExecuteMetaData{Tex/Desc.tex}[]
To describe complex types more practically, we can merge \AgdaInductiveConstructor{ρ} and \AgdaInductiveConstructor{⊗}, and add a variant \AgdaInductiveConstructor{σ} of \AgdaInductiveConstructor{⊗}, which then represent adding a recursive and a non-recursive field respectively

\ExecuteMetaData{Tex/Desc.tex}[]
In \AgdaInductiveConstructor{σ}, we ask for a function \texttt{S → Desc} rather than just a \AgdaDatatype{Desc}, modelling a \AgdaDatatype{Desc} with a bound variable of type \texttt{S}. The interpretation is similar, interpreting \AgdaInductiveConstructor{ρ} and \AgdaInductiveConstructor{σ} as a product and dependent product respectively.

In this universe we can describe types in which the fields be either \texttt{X}, the type itself, or another type \texttt{S}. For example, we can describe \AgdaDatatype{List} as
\ExecuteMetaData{Tex/Desc.tex}[]
using a type parameter from outside the description. We will soon see how we can internalize parameters, but since internalizing indices is easier, we will tackle indices first.

We should note that there are two strategies we can use to describe an indexed type. First, we can define a description of a type indexed by \texttt{I} to simply be a function \texttt{I → Desc}, yielding a universe of index-first types. Second, we can pull the index completely into \AgdaDatatype{Desc}, and let \AgdaInductiveConstructor{𝟙} declare the index at the leaf of a constructor, more closely resembling Agda's datatypes. Both have their advantages and disadvantages, mainly, index-first datatypes are more space efficient. We opt however for the second option, because as we will see later, this allows us to keep descriptions ``relatively small'' (i.e., something like foldable) and more flexible in their levels.

\ExecuteMetaData{Tex/Desc.tex}[]
Now \texttt{𝟙 j} says that this branch constructs a term of \texttt{X j}, while \texttt{ρ i} asks for a recursive field \texttt{X i}. As \texttt{Desc I} describes a type indexed by \texttt{I}, which is a function \texttt{I → Type}, we also have to interpret \texttt{Desc I} as an indexed functor 
\ExecuteMetaData{Tex/Desc.tex}[]
Applying an interpretation to an index asks for the constructors at that index. We see that by interpreting \texttt{𝟙 j} as an equality, we ensure that asking for an index indeed gives that index.

In this universe we can describe vectors
\ExecuteMetaData{Tex/Desc.tex}[]
making use of the variable binding in \AgdaInductiveConstructor{σ} to state that if we get a vector of length \texttt{n}, then we can construct a vector of length \texttt{suc n}.

The observant reader might have noticed that we claim \texttt{I → Desc} does not give small descriptions, but still allow for \texttt{S → Desc}. We can fix this issue at the same time we implement parameters, keeping a form of variable binding. Implementing types with a single parameter can be done by interpreting to ``endofunctors'' \texttt{Type → I → Type}, adding another type-former accessing the parameter. To handle types with more parameters, which may depend on each other, we abstract descriptions over lists of parameters.

We will first need some structure expressing the kinds of parameters that we can have. We could try using List Type, but this rules out types like \texttt{Σ (A : Type) (B : A → Type)}. Instead, we use a telescope, a list of types which explicitly captures the dependencies. We define telescopes and their meaning by induction-recursion:

\ExecuteMetaData{Tex/Desc.tex}[]
A telescope can either be empty, or be formed from a telescope and a type in the context of that telescope.

Contexts are interpreted as

To deal with variables, we will also need to be able to describe variable telescopes. This means that while the parameter telescope in a description stays constant, the variable telescope grows independently when we add more \AgdaInductiveConstructor{σ}'s. We can represent this by parametrizing telescopes over a type

\ExecuteMetaData{Tex/Desc.tex}[]
We define a shorthand \texttt{Γ ⊢ A} for the type of \texttt{S}, representing a value of \texttt{A} in context \texttt{Γ}. By changing \texttt{⟦\_⟧tel} to depend on a value of \texttt{P} as
\ExecuteMetaData{Tex/Desc.tex}[]
a telescope of the form
\ExecuteMetaData{Tex/Desc.tex}[]

can access all values of \texttt{Γ}, and can be treated as an extension of \texttt{Γ}. To interpret them, we define
\ExecuteMetaData{Tex/Desc.tex}[]
To make use of this we also split \AgdaInductiveConstructor{⊕} and \AgdaDatatype{Desc}, making \AgdaDatatype{Desc} a list of constructors, in line with actual Agda datatypes
\ExecuteMetaData{Tex/Desc.tex}[]
A constructor then starts off with the empty variable context, which grows as fields are added
\ExecuteMetaData{Tex/Desc.tex}[]
replacing \texttt{I} by \texttt{V ⊢ I} in \AgdaInductiveConstructor{𝟙} and \AgdaInductiveConstructor{ρ} allows the index of a constructor or argument to depend on the preceding fields, of which the values are made accessible by appending them to the context as \texttt{V ▷ S} in \AgdaInductiveConstructor{σ}. Finally, we interpret this as
\ExecuteMetaData{Tex/Desc.tex}[]
with the fixpoint
\ExecuteMetaData{Tex/Desc.tex}[]


\subsection{Ornaments}
In the framework of \AD{DescI} in the last section, we can write down a number system and its meaning as the starting point of the construction of a numerical representation. To write down the generic construction of those numerical representations, we will need a language in which we can describe modifications on the number systems.

\changed{Somewhat final version above, draft/notes/rough comments/outline below.}
In this section, we will describe the ornamental descriptions for the \AD{DescI} universe, and explain their working by means of (plenty of examples). We omit the definition of the ornaments, since we will only construct new datatypes, rather than relate pre-existing types\footnote{Maybe, I will throw the ornaments into the appendix along with the conversion from ornamental description to ornament}.

% (Be alarmed, the implicits get out of hand pretty quickly.)
\todo{do we need to remark more?}


\section{Ornamental descriptions}
These ornamental descriptions take the same shape as those in \autoref{sec:background-ornamental-descriptions}, generalized to handle nested types, variable transformations, and composite types. Like the interpretation of a \AD{DescI}, ornaments also completely ignore the \AD{Info} of a \AD{DescI}.

Recall that a \AD{OrnDesc}\ \AV{If′ Δ c J i D} represents the ornament building on top of \AV{D}, which yields a description with information \AV{If′}, parameters \AV{Δ}, and indices \AV{J}. We use \AF{∼} to write down pointwise equality of functions, which in this case are all commutativity squares. Since \AD{ConI} allows the transformation of variable telescopes, we have to dedicate a lot of lines to writing down commutativity squares for variables, which along with the generally high number of arguments and implicits\footnote{Of which even more are hidden!} makes the definition rather dry and long.

One tip is to  ignore all squares involving a \AD{Vxf}, these are trivial when using the \AV{+-} variants of the \AIC{σ} and \AIC{δ} formers anyway! Due to the last constructor \AIC{δ•}, \AD{OrnDesc}, \AD{ConOrnDesc}, and \AF{toDesc}\footnote{We left out the variable square for \AIC{δ•}, because it is honestly just too long. If this was included, then we also would involve \AF{ornForget}.} become tightly connected, so the definition is given in one large mutual block:
\ExecuteMetaData[Ornament/OrnDesc]{OrnDesc}
Here the implicit \AV{If′} contains the information necessary to recover the \AD{DescI} from an \AD{OrnDesc}:\todo{line length}
\ExecuteMetaData[Ornament/OrnDesc]{toDesc}
The commutativity squares again ensure the existence of functions like \AF{ornForget}, and that these ornamental descriptions indeed induce ornaments.

Compared to the previous ornaments, we have the new constructors \AIC{δ}, \AIC{Δδ} and \AIC{δ•}, where the first two are analogues of \AIC{σ} and \AIC{Δσ}. The \AIC{δ•} constructor states that an ornamental description from a description \AV{R} and a (constructor) ornamental description from \AV{CD} can be composed to form an ornamental description from the composition (in the sense of the \AV{δ} type-former) of \AV{CD} with \AV{R}.

Let us make the uses of \AD{OrnDesc} more clear by means of examples, where we make use of the simpler variants: \todo{Oδ•+- needs ornForget to run}
\ExecuteMetaData[Ornament/OrnDesc]{O-sigma-pm}
With these we can give the now familiar ornamental description of \AD{Vec} from \AD{List}:
\ExecuteMetaData[Ornament/OrnDesc]{VecOD}
Using the new flexibility in \AIC{ρ}, we can now start from a description of binary numbers:
\ExecuteMetaData[Ornament/OrnDesc]{LeibnizD}
and give the random access lists from before as an ornamental description as well.
\ExecuteMetaData[Ornament/OrnDesc]{RandomOD}
Likewise, we can use \AIC{δ•} to start from the ``fingertree numbers'':
\todo{finger tree skeleton}
and compose this with the ornamental description of \AD{Digit}
\todo{DigitOD}
to obtain the ornamental description of finger trees:
\todo{FingerTreeOD}

%Again, ornForget, fold blabla.

\todo{Now we can compute everything generically.}


\begin{outline}
\ExecuteMetaData[Ornament/OrnDesc]{ConOrnDesc-type}
The definition of ornamental descriptions can be derived in a straightforward manner from ornaments, removing all mentions of the LHS and making all fields which then no longer appear in the indices explicit\footnote{One might expect to need less equalities, alas, this is difficult because of \autoref{rem:orn-lift}.}. We will show the leaf-preserving rule as an example, the others are derived analogously:
\ExecuteMetaData[Ornament/OrnDesc]{OrnDesc-1}
As we can see, the only change we need to make is that \AgdaBoundFontStyle{k} becomes explicit and fully annotated.

Almost by construction, we have that an ornamental description can be decomposed into a description of the new datatype
\ExecuteMetaData[Ornament/OrnDesc]{toDesc}
and an ornament between the starting description and this new description
\ExecuteMetaData[Ornament/OrnDesc]{toOrn}
\end{outline}


\begin{outline}    
\section{The ornaments}
we could ditch removal of fields: we don't use it. downside: ornament over ornament is the same as field removal for deltas

:warning: match everything, add/remove field, add/remove recursive field, add/remove description field, ornament over ornament

\todo{Nuke ornaments, keep ornamental descriptions}

\towrite{Put something that isn't yet in \autoref{ssec:bg-orn} here.}

\ExecuteMetaData[Ornament/Orn]{Orn-type}
\ExecuteMetaData[Ornament/Orn]{ornForget-type}

We will walk through the constructor ornaments
\ExecuteMetaData[Ornament/Orn]{ConOrn-type}
again, an ornament between datatypes is just a list of ornaments between their constructors
\ExecuteMetaData[Ornament/Orn]{Orn}
Note that all ornaments completely ignore information bundles! They cannot affect the existence of \AgdaFunction{ornForget} after all.

Copying parts from one description to another, up to parameter and index refinement, corresponds to reflexivity. Preservation of leaves follows the rule
\ExecuteMetaData[Ornament/Orn]{Orn-1}
We can see that this commuting square (\texttt{e (k p) ≡ j (over f p)}) is necessary: take a value of \texttt{E} at \texttt{p, i}, where \texttt{i} is given as \texttt{k p}. Then \AgdaFunction{ornForget} has to convert this to a value of \texttt{D} at \texttt{f p , e i}, but since \texttt{e i} must match \texttt{j (f p)}, this is only possible if \texttt{e (k p) = j (f p)}.

Preserving a recursive field similarly requires a square of indices and conversions to commute
\ExecuteMetaData[Ornament/Orn]{Orn-rho}
additionally requiring the recursive parameters to commute with the conversion. \todo{Does adding the derivations for the squares everywhere make this section clearler?}

Preservation of non-recursive fields and description fields is analogous
\ExecuteMetaData[Ornament/Orn]{Orn-sigma-delta}
differing only in that non-recursive fields appears transformed on the right hand, while description fields have their conversions modified instead. For this rule, we need that the variable transformations fit into a commuting square with the parameter conversions. The condition on indices for descriptions, which is a commuting triangle, is encoded in the return type\footnote{Should this become a problem like with \AgdaInductiveConstructor{ρ}, modifying the rule to require a triangle is trivial.}.

Ornaments would not be very interesting if they only related identical structures. The left-hand side can also have more fields than the right-hand side, in which case \AgdaFunction{ornForget} will simply drop the fields which have no counterpart on the right-hand side. As a consequence, the description extending rules have fewer conditions than the description preserving rules: 
\ExecuteMetaData[Ornament/Orn]{Orn-+-rho}
Note that this extension\footnote{Kind of breaking the ``ornaments relate types with similar recursive structure'' interpretation.} with a recursive field has no conditions.

Extending by a non-recursive field or a description field again only requires the variable transform to interact well with the parameter conversion
\ExecuteMetaData[Ornament/Orn]{Orn-+-sigma-delta}

In the other direction, the left-hand side can also omit a field which appears on the right-hand side, provided we can produce a default value
\ExecuteMetaData[Ornament/Orn]{Orn---sigma-delta}
These rules let us describe the basic set of ornaments between datatypes.

Intuitively we also expect a conversion to exist when two constructors have description fields which are not equal, but are only related by an ornament. Such a composition of ornaments takes two ornaments, one between the field, and one between the outer descriptions. This composition rule reads:\todo{The implicits kind of get out of control here, but the rule is also unreadable without them. I might hide the rule altogether and only run an example with it.}
\ExecuteMetaData[Ornament/Orn]{Orn-comp}
We first require two commuting squares, one relating the parameters of the fields to the inner and outer parameter conversions, and one relating the indices of the fields to the inner index conversion and the outer parameter conversion. Then, the last square has a rather complicated equation, which merely states that the variable transforms for the remainder respect the outer parameter conversion.

We will construct \AgdaFunction{ornForget} as a \AgdaFunction{fold}. Using
\ExecuteMetaData[Ornament/Orn]{erase-type}
we can define the algebra which forgets the added structure of the outer layer
\ExecuteMetaData[Ornament/Orn]{ornAlg}
Folding over this algebra gives the wanted function
\ExecuteMetaData[Ornament/Orn]{ornForget}

\todo{NatD was removed here}

We can also relate lists and vectors
\ExecuteMetaData[Ornament/Orn]{ListD-VecD}
Now the parameter conversion is the identity, since both have a single type parameter. The index conversion is \AgdaFunction{!}, since lists have no indices. Again, most structure is preserved, we only note that vectors have an added field carrying the length.

Instantiating \AgdaFunction{ornForget} to these ornaments, we now get the functions \AgdaFunction{length} and \AgdaFunction{toList} for free!

%\investigate{Having a function of the same type as \AgdaFunction{ornForget} is not sufficient to deduce an ornament. An obstacle is that the usual empty type (no constructors) and the non-wellfounded empty type (only a recursive field) don't have an ornament. Also, while the leaf-preservation case spells itself out, the substitutions obviously don't give us a way to recover the equalities.}
\end{outline}
    


\section{Cubical Agda}
%outline:
%all our isomorphisms are nice, but not very powerful without a simple way to apply them
%our ideas generally work without axiom K, provided we take some precautions

Intuitively, one expects that like how isomorphic groups share the same group-theoretical properties, isomorphic types also share the same type-theoretical properties. Meta-theoretically, this is known as \emph{representation independence}, and is evident. Inside (ordinary) Agda this is not so practical, as this independence only holds when applied to concrete types, and is then only realized by manually substituting along the isomorphism. On the other hand, in Cubical Agda, the Structure Identity Principle internalizes a kind of representation independence \cite{iri}.

Cubical Agda modifies the type theory of Agda to a kind of homotopy type theory, looking at equalities as paths between terms rather than the equivalence relation generated by reflexivity. In cubical type theories, the role played by pattern matching on \AgdaFunction{refl} or by axiom J, in MLTT and ``Book HoTT'' respectively, is instead acted out by directly manipulating cubes\footnote{Under the analogy where a term is a point, an equality between points is a line, a line between lines is a square.}. %In Cubical Agda, univalence
%\[ ... \]
%is not an axiom but a theorem.

To give an understanding of the basics of Cubical Agda \cite{cuagda} and the Structure Identity Principle (SIP), we walk through the steps to transport proofs about addition on Peano naturals to Leibniz naturals. %We give an overview of some features of Cubical Agda, such as that paths give the primitive notion of equality, until the simplified statement of univalence.
%We do note that Cubical Agda has two downsides relating to termination checking and universe levels, which we encounter in later sections.

\subsection{Paths}
In Cubical Agda, the primitive notion of equality arises not (directly) from the indexed inductive definition we are used to, but rather from the presence of the interval type \AgdaPrimitiveType{I}. This type represents a set of two points \AgdaInductiveConstructor{i0} and \AgdaInductiveConstructor{i1}, which are considered ``identified'' in the sense that they are connected by a path. To define a function out of this type, we also have to define the function on all the intermediate points, which is why we call such a function a ``path''. Terms of other types are then considered identified when there is a path between them.

Paths between types are incredibly useful, as they effectively let us directly transport properties between isomorphic structures. However, they do not come without downsides, such as that the negation of axiom K complicates both some termination checking and some universe levels.\footnote{In particular, this prompts rather far-reaching (but not fundamental) changes to the code of previous work, such as to the machinery of ornaments \cite{progorn} in \autoref{sec:userfriendly}.} %Furthermore, if we use certain homotopical constructions, and we wish to eliminate from our types as if they were sets, then we will also have to prove that they are indeed sets.

We will discuss how to deal with these issues in later sections, so let us not be distracted from what we \emph{can} do with paths. For example, the different perspective gives intuitive interpretations to some proofs of equality, like
\ExecuteMetaData[Tex/CubicalAndBinary]{sym}
where \AgdaFunction{∼\_} is the interval reversal, swapping \AgdaInductiveConstructor{i0} and \AgdaInductiveConstructor{i1}, so that \AgdaFunction{sym} simply reverses the given path.

Also, because we can now interpret paths in record and function types in a new way, we get a host of ``extensionality'' for free. For example, a path in $A \to B$ is indeed a function which takes each $i$ in \AgdaPrimitiveType{I} to a function $A \to B$. Using this, function extensionality becomes tautological 
\ExecuteMetaData[Tex/CubicalAndBinary]{funExt}


\subsection{Univalence}
Finally, equivalences, the HoTT-compatible variant of bijections, have the univalence theorem 
\ExecuteMetaData[Tex/CubicalAndBinary]{ua}
stating that ``equivalent types are identified'', such that equivalences like $1 \to A \simeq A$ become paths $1 \to A \equiv A$, making it so that we can transport proofs along them. We will demonstrate this by a more practical example in the next section.

%\towrite{Why circles are points with K. Why circles are not points with univalence}

\subsection{The Structure Identity Principle}\label{ssec:leibniz}
%Starting by defining the unary Peano naturals and the binary Leibniz naturals, we prove that they are isomorphic by interpreting them into each other. We observe how the interpretations are mutual inverses by proving lemmas stating that both interpretations ``respect the constructors'' of the types. Next, we demonstrate how this isomorphism can be promoted into an equivalence or an equality, and remark that this is sufficient to transport intrinsic properties, such as having decidable equality, from one natural to the other.

%Noting that transporting unary addition to binary addition is possible but not efficient, we define binary addition while ensuring that it corresponds to unary addition. We present a variant on refinement types as a syntax to recover definition from chains of equality reasoning, allowing one to rewrite definitions while preserving equalities.

%We clarify that to transport proofs referring to addition from unary to binary naturals, we indeed require that these are meaningfully related. Then, we observe that in this instance, the pairs of ``type and operation'' are actually equated as magmas, and explain that this is an instance of the SIP.

%Finally, we describe the use case of the SIP, how it generalizes our observation about magmas, and how it can calculate the minimal requirements to equate to implementations of an interface. This is demonstrated by transporting associativity from unary addition to binary addition, noting that this would save many lines of code provided there is much to be transported.

Let us quickly review the small set of features in Cubical Agda that we will be using extensively throughout this article.\footnote{\cite{cuagda} gives a comprehensive introduction to cubical agda.}

%Of course, this downside is more than offset by the benefits of changing our primitive notion of equality, which we will see makes it easier to show that ``equivalent'' structures behave identically. 
In Cubical Agda, the primitive notion of equality arises not (directly) from the indexed inductive definition we are used to, but rather from the presence of the interval type \AgdaPrimitiveType{I}. This type represents a set of two points \AgdaInductiveConstructor{i0} and \AgdaInductiveConstructor{i1}, which are considered ``identified'' in the sense that they are connected by a path. To define a function out of this type, we also have to define the function on all the intermediate points, which is why we call such a function a ``path''. Terms of other types are then considered identified when there is a path between them.

While the benefits are overwhelming for us, this is not completely without downsides, such as that
%\ExecuteMetaData[Tex/CubicalAndBinary]{cubical}% \todo{Not sure if it would be helpful to have a more extensive introduction covering all features used.} % at this moment, probably not, as the cubical usage is rather tame, so I'll probably stick to introducing stuff as it becomes necessary. % TODO then write that somewhere
the negation of axiom K complicates both some termination checking and some universe levels.\footnote{In particular, this prompts rather far-reaching (but not fundamental) changes to the code of previous work, such as to that of \cite{progorn} in \autoref{sec:userfriendly}.} Furthermore, if we use certain homotopical constructions, like set quotients, we will also have to prove that our types are sets, before we can use them.

On the positive side, this different perspective gives intuitive interpretations to some proofs of equality, like
\ExecuteMetaData[Tex/CubicalAndBinary]{sym}
where \AgdaFunction{∼\_} is the interval reversal, swapping \AgdaInductiveConstructor{i0} and \AgdaInductiveConstructor{i1}, so that \AgdaFunction{sym} simply reverses the given path.

Furthermore, because we can now interpret paths in records and function differently, we get a host of ``extensionality'' for free. For example, a path in $A \to B$ is indeed a function which takes each $i$ in \AgdaPrimitiveType{I} to a function $A \to B$. Using this, function extensionality becomes tautological 
\ExecuteMetaData[Tex/CubicalAndBinary]{funExt}

Finally, while in ``non-univalent'' Agda bijections or isomorphisms do not play such a central role, much of our work will rest on equivalences, as the ``HoTT-compatible'' generalization of bijections. This is because the \AgdaPrimitiveType{Glue} type tells us that equivalent types fit together in a new type, in a way that guarantees univalence
\ExecuteMetaData[Tex/CubicalAndBinary]{ua}
This essentially states that ``equivalent types are identified'', such that type isomorphisms like $1 \to A \simeq A$ actually become paths $1 \to A \equiv A$, making it so that we can transport proofs along them. We will demonstrate this by a slightly more practical example in the next section.


\subsection{Binary numbers}\label{ssec:binary}
Let us demonstrate an application of univalence by exploiting the equivalence of the ``Peano'' naturals and the ``Leibniz'' naturals. Recall that the Peano naturals are defined as 
\ExecuteMetaData[Tex/CubicalAndBinary]{Peano}
This definition enjoys a simple induction principle and has many proofs of its properties in standard libraries. However, it is too slow to be of practical use: most arithmetic operations defined on \bN{} have time complexity in the order of the value of the result.

Of course, the alternative are the more performant binary numbers: the time complexities for binary numbers are usually logarithmic in the resultant values. However, the number of cases for a proof about binary numbers also grows quicker than it would for unary numbers. This does not have to be a problem, because the \bN{} naturals and the binary numbers should be equivalent after all!

Let us make this formal. We define the Leibniz naturals as follows:
\ExecuteMetaData[Leibniz/Base.tex]{Leibniz}
Here, the \AgdaInductiveConstructor{0b} constructor encodes 0, while the \AgdaInductiveConstructor{\_1b} and \AgdaInductiveConstructor{\_2b} constructors respectively add a 1 and a 2 bit, under the usual interpretation of binary numbers:
\ExecuteMetaData[Leibniz/Base.tex]{toN}
This defines one direction of the equivalence from \bN{} to \bL{}, for the other direction, we can interpret a number in \bN{} as a binary number by repeating the successor operation on binary numbers:
\ExecuteMetaData[Leibniz/Base.tex]{bsuc}
\ExecuteMetaData[Leibniz/Base.tex]{fromN}
To show that \AgdaFunction{toℕ} is an isomorphism, we have to show that it is the inverse of \AgdaFunction{fromℕ}. By induction on \bL{} and basic arithmetic on \bN{} we see that
\ExecuteMetaData[Leibniz/Properties.tex]{toN-suc}
so \AgdaFunction{toℕ} respects successors. Similarly, by induction on \bN{} we get
\ExecuteMetaData[Leibniz/Properties.tex]{fromN-1}
and % I can't get the code blocks to stick together lol
\ExecuteMetaData[Leibniz/Properties.tex]{fromN-2}
so that \AgdaFunction{fromℕ} respects even and odd numbers. We can then prove that applying \AgdaFunction{toℕ} and \AgdaFunction{fromℕ} after each other is the identity by repeating these lemmas
\ExecuteMetaData[Leibniz/Properties.tex]{N-iso-L}
This isomorphism can be promoted to an equivalence
\ExecuteMetaData[Leibniz/Properties.tex]{N-equiv-L}
which, finally, lets us identify \bN{} and \bL{} by univalence
\ExecuteMetaData[Leibniz/Properties.tex]{N-is-L}
The path \AgdaFunction{ℕ≡L} then allows us to transport properties from \bN{} directly to \bL{}, e.g.,
\ExecuteMetaData[Leibniz/Properties.tex]{isSetL}
This can be generalized even further to transport proofs about operations from \bN{} to \bL{}. 

\subsection{Use as definition: functions from specifications}\label{ssec:useas}
As an example, we will define addition of binary numbers. We could transport binary operations
\ExecuteMetaData[Extra/Algebra]{BinOp}
to get
\ExecuteMetaData[Tex/CubicalAndBinary]{badplus}
but this would be rather inefficient, incurring an $O(n + m)$ overhead when adding $n + m$. It is more efficient to define addition on \bL{} directly, making use of the binary nature of \bL{}, while agreeing with the addition on \bN{}. Such a definition can be derived from the specification ``agrees with \AgdaFunction{\_+\_}'', so we implement the following syntax for giving definitions by equational reasoning, inspired by the ``use-as-definition'' notation from \cite{calcdata}:
\ExecuteMetaData[Prelude/UseAs.tex]{Def}
which infers the definition from the right endpoint of a path using an implicit pair type
\ExecuteMetaData[Prelude/UseAs.tex]{isigma}
% \investigate{As of now, I am unsure if this reduces the effort of implementing a coherent function, or whether it is more typically possible to give a smarter or shorter proof by just giving a definition and proving an easier property of it\footnote{I will put the alternative in the appendix for now}}

With this we can define addition on \bL{} and show it agrees with addition on \bN{} in one motion
\ExecuteMetaData[Leibniz/Properties.tex]{plus-def}
Now we can easily extract the definition of \AgdaFunction{plus} and its correctness with respect to \AgdaFunction{\_+\_} 
\ExecuteMetaData[Leibniz/Properties.tex]{plus-good}

We remark \AgdaFunction{Def} is close in concept to refinement types\footnote{À la \href{https://agda.github.io/agda-stdlib/Data.Refinement.html}{Data.Refinement}.}, but importantly, the equality proof is relevant for us, and the value is inferred rather than given. \footnote{Unfortunately, normalizing an application of a \AgdaFunction{defined-by} function also causes a lot of unnecessary wrapping and unwrapping, so \AgdaFunction{Def} is mostly only useful for presentation.} %for now..


\subsection{Structure Identity Principle}
Now \bN{} with \AgdaFunction{N.+} form, in particular, a magma. The same goes for \bL{} and \AgdaFunction{plus}, but notice that a path in a \AgdaDatatype{Σ} type is just a \AgdaDatatype{Σ} of paths! This means that we get a path from (\bN{}, \AgdaFunction{N.+}) to (\bL{}, \AgdaFunction{plus}). More generally, a magma is simply a type $X$ with some structure, which is a function $f: X \to X \to X$ in the case of a magma. We can see that paths between magmas correspond to paths $p$ between the underlying types $X$ and paths over $p$ between their operations $f$. This observation is further generalized by the Structure Identity Principle (SIP), formalized in \cite{iri}. Given a structure, which in our case is just a binary operation
\ExecuteMetaData[Extra/Algebra.tex]{MagmaStr}
this principle produces an appropriate definition ``structured equivalence'' $\iota$. The $\iota$ is such that if structures $X, Y$ are $\iota$-equivalent, then they are identified. In the case of \AgdaFunction{MagmaStr}, the $\iota$ asks us to provide something with the same type as \AgdaFunction{plus-coherent}, so we have just shown that the \AgdaFunction{plus} magma on \bL{}
\ExecuteMetaData[Leibniz/Properties.tex]{magmaL}
and the \AgdaFunction{\_+\_} magma on \bN{} and are identical
\ExecuteMetaData[Leibniz/Properties.tex]{magma-equal}
As a consequence, properties of \AgdaFunction{\_+\_} directly yield corresponding properties of \AgdaFunction{plus}. For example,
\ExecuteMetaData[Leibniz/Properties.tex]{assoc-transport}

