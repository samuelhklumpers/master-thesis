\subsection{Numerical Representations}\label{sec:desc-numrep}
Before we start rebuilding our universe, let us look at the construction of the simplest numerical representation \AD{Vec} from \bN{}. At first, we defined \AD{Vec} as the length-indexed variant of \AD{List}, so that \AF{lookup} becomes total and satisfies nice properties like \AF{lookup-insert}. Later, we gave another description of \AD{Vec} as an ornament on top of \AD{List}. More abstractly, \AD{Vec} is an implementation of finite maps with domain \AD{Fin}, where finite maps are simply those types with operations like \AF{insert}, \AF{remove}, \AF{lookup}, and \AF{tabulate}\footnote{The function \AF{tabulate}\ \AV{:}\ (\AD{Fin}\ \AV{n}\ \AV{→}\ \AV{A})\ \AV{→}\ \AV{Vec}\ \AV{A}\ \AV{n} collects an assignment of elements \AV{f} into a vector \AF{tabulate}\ \AV{f}.}, satisfying relations or laws like \AF{lookup-insert} and \AF{lookup}\ \AF{∘}\ \AF{tabulate}\ \AD{≡}\ \AF{id}.  \todo{chopping}

For comparison, we can define a trivial implementation of finite maps, by reading \AF{lookup} as a prescript
\ExecuteMetaData[Tex/Descriptions/Numrep]{Lookup2}
Since \AF{lookup} is simply the identity function on \AF{Lookup}, this immediately satisfies the laws of finite maps, provided we define \AF{insert} and \AF{remove} correctly.

Unsurprisingly, \AD{Vec} is \emph{representable}, that is, we have that \AF{Lookup} and \AD{Vec} are equivalent, in the sense that there is an \emph{isomorphism} between \AF{Lookup} and \AD{Vec}\footnote{Since \AF{lookup} is an isomorphism with \AF{tabulate} as inverse, as we see from the relations \AF{lookup}\ \AF{∘}\ \AF{tabulate}\ \AD{≡}\ \AF{id} and  \AF{tabulate}\ \AF{∘}\ \AF{lookup}\ \AD{≡}\ \AF{id}. Without further assumptions, we cannot use the equality type \AD{≡} for this notion of equivalence of types: a type with a different name but exactly the same constructors as \AD{Vec} would not be equal to \AD{Vec}.}
\ExecuteMetaData[Tex/Descriptions/Numrep]{Iso}
An \AD{Iso} from \AV{A} to \AV{B} is a map from \AV{A} to \AV{B} with a (two-sided) inverse\footnote{In our situation (\AV{--with-K}), this is the same as the other notion of equivalence: there is a map $f : A \to B$, and for each \AV{b} in \AV{B} there is exactly one \AV{a} in \AV{A} for which $f(a) = b$.}. In terms of elements, this means that elements of \AV{A} and \AV{B} are in one-to-one correspondence.

Rather than deriving them ourselves, we can also establish properties like \AF{lookup-insert} from this equivalence. Instead of finding the properties of \AD{Vec} that were already there, let us view \AD{Vec} as a consequence of the definition of \bN{} and \AF{lookup}. By turning the \AD{Iso} on its head, and starting from the equation that \AD{Vec} is equivalent to \AD{Lookup}, we derive a definition of \AD{Vec} as if were solving an equation \cite{calcdata}.

As a warm-up, we can also derive \AD{Fin} from the fact that \AD{Fin}\ \AV{n} should contain \AV{n} elements, and thus be isomorphic to \AV{Σ[ m ∈ ℕ ] m < n}. To express such a definition by isomorphism, we define:
\ExecuteMetaData[Tex/Descriptions/Numrep]{Def}
using
\ExecuteMetaData[Tex/Descriptions/Numrep]{isigma}
The type \AD{Def}\ \AV{A} is deceptively simple, after all, there is (up to isomorphism) only one unique term in it! However, when using \AD{Def}initions, the implicit \AD{Σ'} extracts the right-hand side of a proof of an isomorphism, allowing us to reinterpret a proof as a definition.

To keep the isomorphisms we are trying to construct readable, we construct them as chains of simpler isomorphisms, using a variant of \emph{equational reasoning} \cite{agdastdlib, plfa}, letting us compose isomorphisms while displaying the intermediate steps. In the calculation of \AD{Fin}, we will use the following lemmas
\ExecuteMetaData[Tex/Descriptions/Numrep]{Fin-lemmas}
If we allow reading isomorphisms as ``\emph{is}'', then the terminology of \autoref{sec:background-proving}, \AF{⊥-strict} states that ``if A is false, then A \emph{is} empty'', while \AF{<-split} states that the set of numbers below $n+1$ has one more element than the set of numbers below $n$. Using these, we can calculate\footnote{Making non-essential use of \AF{cong} for type families. In the derivation of \AD{Vec} we use function extensionality, which has to be postulated, or can be obtained by using the cubical path types.}
\ExecuteMetaData[Tex/Descriptions/Numrep]{Fin-def}
This gives a different (but equivalent) definition of \AD{Fin} compared to \AF{FinD}; the description \AF{FinD} describes \AD{Fin} as an inductive family, whereas \AF{Fin-def} describes \AD{Fin} equivalently as a type-computing function \cite{progorn}. From this \AD{Def} we can extract a definition of \AD{Fin}
\ExecuteMetaData[Tex/Descriptions/Numrep]{Fin}
To derive \AD{Vec}, we use the isomorphisms
\ExecuteMetaData[Tex/Descriptions/Numrep]{Vec-lemmas}
which one can compare to the familiar exponential laws. With these laws, we calculate the type of vectors
\ExecuteMetaData[Tex/Descriptions/Numrep]{Vec-def}
yielding a definition of vectors and the \AD{Iso} to \AF{Lookup} in one go
\ExecuteMetaData[Tex/Descriptions/Numrep]{Vec}
In conclusion, we computed a type of finite maps (the numerical representation \AD{Vec}) from a number system (\bN{}), by cases on the number system and making use of the values represented by the number system.


\subsection{Room for Improvement}
We could now carry on and attempt to generalize this calculation to other number systems, but we would quickly run into dead ends for certain numerical representations. Let us give an overview of what bits of \AD{U-ix} are still missing if we are going to generically construct all numerical representations we promised.

%The universe \AD{U-ix} is missing some bits we will need to describe some numerical representations. \todo{Write}
%In this section, we will point out which bits are missing, and the modifications to \AD{U-ix} we make in order to express all number systems and numerical representations we want.

\subsubsection{Number systems}\label{ssec:numbers}
In the calculation \AD{Vec} from \bN{}, we analyze and replicate the structure of \bN{}, for which we use the meaning of these constructors as numbers assigned to them by our explanation of \bN{} in words\footnote{More accurately, the meaning of \bN{} comes from \AF{Fin}, which gets its meaning from our definition of \AF{\_<\_}.}. Based on that interpretation of constructors as numbers, we deliberately choose to add zero fields in the case corresponding to \AIC{zero} and choose to add one field in the case of \AIC{one}.

However, if we want to compute numerical representations generically, we also have to convince the computer that our datatypes indeed represent number systems. As a first step, let us fix \bN{} as the primordial number system, so that we can compare other number systems by how they are mapped into \bN{}. Trivially, \bN{} can be interpreted as a number system via \AF{id}\ \AV{:}\ \bN{}\ \AV{→}\ \bN{}.

The binary numbers, as described in the introduction, can be mapped to \bN{} by
\ExecuteMetaData[Tex/Descriptions/Number]{Bin}
As a more exotic example, we can describe a number system
\ExecuteMetaData[Tex/Descriptions/Number]{Carpal}
which consists of smaller ``number systems''
\ExecuteMetaData[Tex/Descriptions/Number]{Phalanx}
We could now define a general number system as a type \AV{N} equipped with a map \AV{N →}\ \bN{}, but this would both be too general for our purpose and opaque to generic programs. On the other hand, allowing only traditional positional number systems excludes number systems like \AD{Carpal}, which would otherwise still have valid numerical representations, as we will see later.

Instead, we observe that across the above examples, the interpretation of a number is computed by a simple fold. In particular, leaves have associated constants, recursive fields correspond to multiplication and addition, while fields can defer to another function. If we encode number systems in \AD{U-sop}, we can thus encode each of these systems by associating a single number to each \AIC{𝟙} and \AIC{ρ}, and a function to each \AIC{σ}. This essentially encodes number systems as trees that evaluate nodes by linearly combining values of subnodes, generalizing \emph{dense} representations of positional number systems\footnote{As a consequence, this excludes the \emph{sparse} number systems, as we discuss in \autoref{sec:discussion-no-sparse}.}.

Using a modified version of \AD{U-sop}, we can encode the examples we gave as follows. Note that in \bN{}, we insert redundant fields of \AD{⊤} in order to express that the second constructor acts as $x \mapsto x + 1$
\ExecuteMetaData[Tex/Descriptions/Number]{Nat-num}
marking all leaves as zero. The binary numbers admit a similar encoding, but multiply their recursive fields by two instead
\ExecuteMetaData[Tex/Descriptions/Number]{Bin-num}
Finally, the \AD{Carpal} system can be encoded by using the interpretation of \AD{Phalanx}
\ExecuteMetaData[Tex/Descriptions/Number]{Carpal-num}


\subsubsection{Nested types}
If our construction is going to cast \AD{Random}, as defined in \autoref{sec:introduction}, as the numerical representation associated to \AD{Bin}, then \AD{Random} needs to have a description to begin with. However, the recursive fields of \AD{Random} are not given the parameter \AV{A}, but \AV{A}\ \AD{×}\ \AV{A}. This makes \AD{Random} a nested type, as opposed to a uniformly recursive type, in which the parameters of the recursive fields would be identical to the top-level parameters. Consequently, \AD{Random} has no adequate description in \AD{U-ix}\footnote{Here, the ``inadequate'' descriptions either hardly resemble the user defined \AD{Random}, use indices to store the depth of a node (see \autoref{app:unnested}), or only have a complicated isomorphism to \AD{Random}.}. 

Due to the work of Johann and Ghani \cite{initenough}, we can model general nested types as fixpoints of \emph{higher-order functors} (i.e., endofunctors on the category of endofunctors)
\ExecuteMetaData[Tex/Descriptions/Nested]{HMu}
By placing the recursive field \AD{Mu}\ \AV{F} under \AV{F}, the functor \AV{F} can modify \AD{Mu}\ \AV{F} and \AV{A} to determine the type of the recursive field. We can encode \AD{Random} by a \AF{HFun} as
\ExecuteMetaData[Tex/Descriptions/Nested]{HRandom}
However, this definition is unsafe\footnote{As you might have deduced from the pragma disabling the positivity checker. Consider \AF{HBad}\ \AV{F}\ \AV{A}\ \AV{=}\ \AV{F}\ \AV{A}\ \AV{→}\ \AD{⊥}.}, so we settle for the weaker but safe inner nesting instead. This kind of nesting is described by a simple modification to the recursive field \AIC{ρ} in \AD{U-ix}
\ExecuteMetaData[Tex/Descriptions/Nested]{rho-nest}
allowing a recursive field specify a transformation \AF{Cxf} that is applied to the parameters before they are passed to the recursive field. Correspondingly, the interpretation of \AIC{ρ} applies \AV{f} before passing \AV{p} to the recursive field \AV{X}
\ExecuteMetaData[Tex/Descriptions/Nested]{rho-nest-int}
With this modification, \AD{Random} can be directly transcribed 
\ExecuteMetaData[Tex/Descriptions/Nested]{Random}
using the map $A \mapsto A \times A$ to describe its nesting as usual.

Of course, the nesting which is useful here is only an inconvenience for uniformly recursive types, so we define a shorthand
\ExecuteMetaData[Tex/Descriptions/Nested]{rho0}
emulating the behaviour of a uniformly recursive field.

\subsubsection{Composite types}
In \autoref{ssec:numbers}, we defined the number system \AF{Carpal-num} as a \emph{composite type}, in the sense that its description references another concrete type \AD{Phalanx}. By the same argument as there, the description \AD{Carpal-num} which relies on \AF{toℕ-Phalanx} to describe the value of \AD{Phalanx}, turns out to be too imprecise to recover the complete numerical representation generically. 

In comparison, generic programming facilities like the deriving-mechanism in Haskell allow for code like
\begin{verbatim}
{-# LANGUAGE DeriveFunctor #-}

data Two a = Two a a deriving Functor
data Even a = Zero | More (Two a) (Even a) deriving Functor    
\end{verbatim}
In this example, we can define lists of even numbers of elements as lists of pairs of elements, and the functor instance for \AD{Even} can be derived generically, using that \AD{Two} has a (derived) functor instance. This would not work for \AD{U-ix} or \AD{U-num}, as a generic function would not be able to decide whether a field is of the form \AD{μ}\ \AV{D} to begin with. 

Inlining the constructors of \AD{Phalanx} into \AD{Carpal} does allow generic constructions to see the structure of \AD{Phalanx}, but is undesirable in this case and in general, as this yields a type with two of the original constructors of \AD{Carpal}, and 9 more constructors for each combination of constructors of \AD{Phalanx}\footnote{If working with 11 constructors sounds too feasible, consider that defining addition on types like \AD{Carpal} (or concatenation on its numerical representation) is not (yet) generic and, if fully written out, will instead demand 121 manually written cases.}.

In order to make the descriptions of fields that have them visible to generics, we simply add a more specific former of fields to the universe, specialized to adding \emph{composite fields} from provided descriptions
\ExecuteMetaData[Tex/Descriptions/Composite]{delta}
taking the functions \AV{d} and \AV{j} to determine the parameters and indices passed to \AV{R}. A composite field encoded by \AIC{δ} is then interpreted identically to how it would be if we used \AIC{σ} and \AD{μ} instead\footnote{The omission of \AD{μ}\ \AV{R} from the variable telescope is intentional. While adding it is workable, it also significantly complicates the treatment of ornaments.}:
\ExecuteMetaData[Tex/Descriptions/Composite]{delta-int}
Using \AIC{δ} rather than \AIC{σ} allows us to reveal the description of a field to a generic program. Rather than adding \AD{Phalanx} via a \AIC{σ}, we would use \AIC{δ} to directly add \AF{Phalanx-num} instead.

\subsubsection{Hiding variables}
With the modifications described above, we can describe all the structures we want. However, there is one peculiarity in the way \AD{U-ix} handles variables. Namely, each field added by a \AIC{σ} is treated as bound, and even if the value is then unused, all fields afterwards need to work around it. This only poses a minor inconvenience, but this does mean that two subsequent fields which refer to the same variable will have to be encoded differently. Furthermore, adding fields of complicated types can quickly clutter the context when writing or inspecting a generic program.

%\todo{With rather significant consequences on the definition of ornaments down the line\dots}
With a simple modification to the handling of telescopes in \AD{U-ix}, we can emulate both bound and unbound fields, without adding more formers to \AD{U-ix}. By accepting a transformation of variables \AF{Vxf}\ \AV{Γ}\ \AV{(V ▷ S)}\ \AV{W} after a \AIC{σ}\ \AV{S} in the context of \AV{V}, the remainder of the fields can be described in the context \AV{W}:
\ExecuteMetaData[Tex/Descriptions/Variable]{sigma-var}
Of course, it would be no use to redefine \AIC{σ} in an attempt to save the user some effort, while leaving them with the burden of manually adding these transformations. So, we define shorthands emulating precisely the bound field
\ExecuteMetaData[Tex/Descriptions/Variable]{sigma-plus}
and the unbound field
\ExecuteMetaData[Tex/Descriptions/Variable]{sigma-minus}


\subsection{A new Universe}\label{ssec:desc}
\todo{Chopping from here}
We can now define a new universe based on \AD{U-ix} by incorporating all modifications we described above. This universe is again the type of lists of constructors
\ExecuteMetaData[Ornament/Desc]{Desc}
Compared to \AD{U-ix}, \AD{DescI} is also parametrized over the metadata \AD{Meta}, which we will use later to encode number systems in \AD{DescI}.

The constructors for this universe are defined as follows
\ExecuteMetaData[Ornament/Desc]{Con}
Remark that \AIC{𝟙} remains the same, but \AIC{ρ} can now accept the transformation \AF{Cxf}\ \AV{Γ}\ \AV{Γ} to encode non-uniform parameters. Likewise, \AIC{σ} now also takes the transformation \AV{w} from \AV{V}\ \AIC{▷}\ \AV{S} to \AV{W} allowing us to replace the context after a field with \AV{W} rather than \AV{V}\ \AIC{▷}\ \AV{S}. Finally, \AIC{δ} is added to directly describe composite datatypes by giving a description \AV{R} to represent a field \AD{μ}\ \AV{R}.

Let us take a fresh look at some datatypes from before, now through the lens of \AD{DescI}. We will leave the metadata aside for now by using
\ExecuteMetaData[Ornament/Desc]{Plain-synonyms}
Like before, we use the shorthands \AF{σ+}, \AF{σ-}, and \AF{ρ0} to avoid cluttering descriptions which do not make use of the corresponding features. 

We can describe \bN{} and \AD{List} as before
\ExecuteMetaData[Ornament/Desc]{NatD-and-ListD}
replacing \AIC{σ} with \AF{σ-} and \AIC{ρ} with \AIC{ρ0}.

On the other hand, if we define \AD{Vec}, we bind the length as a (implicit) field, so we use \AF{σ+} instead
\ExecuteMetaData[Ornament/Desc]{VecD}
and extract the length \AV{n} like we would in \AD{U-ix}.

%Recall that for random-access arrays, we have that an array with parameter \AV{A} contains zero, one, or two values of \AV{A}, but the recursive field must contain an array of twice the weight. Hence, the parameter passed to the recursive field is \AV{A times A}, for which we define
%\ExecuteMetaData[Ornament/Desc]{Pair}
%Passing \AF{Pair} to \AIC{rho} we can define random access lists:  
With the nested recursive field \AIC{ρ}, we can almost repeat the definition of \AD{Random} from \AD{U-nest}:
\ExecuteMetaData[Ornament/Desc]{RandomD}

%To represent finger trees, we first represent the type of digits \AD{Digit}: \todo{reminder to cite this here if I end up not referencing finger trees earlier.}
Binary finger trees (as a simplification of 2-3 finger trees \cite{fingertrees}), a nested datatype like \AD{Random}, instead storing elements in variably sized digits on both sides, can be composed from digits
\ExecuteMetaData[Ornament/Desc]{DigitD}
using \AIC{δ} to add fields represented by \AF{DigitD}
\ExecuteMetaData[Ornament/Desc]{FingerD}
%Here, the fact that the first \AIC{δ-} drops its field from the telescope makes it possible to reuse of \AD{Digit} in the second \AIC{δ-}.
These descriptions can be instantiated as before by taking the fixpoint%\footnote{Note that these (obviously?) ignore the \AD{Meta} of a description.}
\ExecuteMetaData[Ornament/Desc]{fpoint}
of their interpretations as functors
\ExecuteMetaData[Ornament/Desc]{interpretation}
inserting the transformations of parameters \AV{g} in \AIC{ρ} and the transformations of variables \AV{w} in \AIC{σ}.

Like \AD{U-ix}, \AD{DescI} comes with a generic \AF{fold}
\ExecuteMetaData[Ornament/Desc]{fold-type}
which is defined analogously.


\subsubsection{Annotating Descriptions with Metadata}
We promised encodings of number systems in \AD{DescI}, so let us explain how number systems can be described as \emph{metadata} and how this lets use \AD{DescI} in the same way we used \AD{U-num} to describe type and numerical value in the same description.

By generalizing \AD{DescI} over \AD{Meta}, rather than coding the specification of number systems into the universe directly, we give ourselves the flexibility to both represent plain datatypes and number systems in the same universe. \AD{DescI} then uses the specific type of \AD{Meta} to query bits of information in the implicit fields in each of the type-formers. A term of \AD{Meta} simply lists the type of information to be queried at each type former\footnote{One can compare this to how generic representations of datatypes in Haskell can be (optionally) annotated with metadata making the names of datatypes, constructors and fields available on the type level.}:
\ExecuteMetaData[Ornament/Desc]{Meta}
In composite fields \AIC{δ}, the metadata on the other description is not necessarily related to the top-level metadata. When this happens, we ask that both sides is related by a transformation
\ExecuteMetaData[Ornament/Desc]{MetaF}
making it possible to downcast (or upcast) between the different types of metadata. This allows us to include an annotated type \AD{DescI}\ \AV{Me} into an ordinary datatype \AD{Desc} without duplicating the former definition in \AD{Desc} first.

The encoding of number systems by associating numbers to \AIC{𝟙} and \AIC{ρ}, and functions to \AIC{σ}, can be summarized as
\ExecuteMetaData[Ornament/Desc]{Number} 
The \AIC{δ}-former, which was not described when we discussed encoding number systems in \AD{U-num}, is assigned a single number, representing multiplication analogous to \AIC{ρ}. The equalities in the metadata of a \AIC{δ} ensure that number systems have no parameters or indices. 

Using \AF{Number}, we can for example reproduce the binary numbers \AF{Bin-num} in \AD{DescI} as
\ExecuteMetaData[Ornament/Desc]{BinND} 
Functions between metadata come in when we represent \AF{Carpal-num} in its more accurate form by first defining 
\ExecuteMetaData[Ornament/Desc]{PhalanxND} 
and directly including it into \AD{Carpal}
\ExecuteMetaData[Ornament/Desc]{CarpalND} 
where we can use the identity function as both sides exactly use \AF{Number}.

The metadata on a \AD{DescI}\ \AF{Number} can then be used to define a generic function sending terms of number systems to their \AF{value} in \bN{}
\ExecuteMetaData[Ornament/Desc]{toN-type}
which is defined by generalizing over the inner metadata and \AF{fold}ing using
\ExecuteMetaData[Ornament/Desc]{toN-con}

On the other hand, we can also declare that a description has no metadata at all by querying \AD{⊤} for all type-formers:
\ExecuteMetaData[Ornament/Desc]{Plain}
By making the fields querying information implicit in the type of descriptions, we can ensure that descriptions from \AD{U-ix} can be imported into \AD{Desc} without having to insert metadata anywhere.

But it is also possible to use \AD{Meta} to encode conventionally useful metadata such as field names
\ExecuteMetaData[Ornament/Desc]{Names}