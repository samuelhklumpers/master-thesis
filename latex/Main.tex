\documentclass[10pt]{article}

\usepackage[style=alphabetic]{biblatex}
\addbibresource{refs.bib}

\usepackage{comment}

\setlength{\marginparwidth}{2cm} % remove when done

\usepackage{todonotes}
\usepackage{xcolor}
\usepackage[hidelinks]{hyperref}

%\hypersetup{
%    colorlinks=true,
%    linkcolor=cyan
%    }

\usepackage{catchfilebetweentags}
\usepackage{quiver} 
\usepackage{tabularx}
\usepackage{adjustbox}
\usepackage{longtable}
\usepackage{amsthm}
\usepackage{amsmath}


\theoremstyle{plain}% default
\newtheorem{theorem}{Theorem}[section]
\newtheorem{lemma}[theorem]{Lemma}
\newtheorem{prop}[theorem]{Proposition}
\newtheorem{conjecture}{Conjecture}[section]
\newtheorem*{cor}{Corollary}

\theoremstyle{definition}
\newtheorem{defn}{Definition}[section]
\newtheorem{remark}{Remark}[section]
\newtheorem{claim}{Claim}[section]
\newtheorem{example}{Example}[section]

\renewcommand{\partautorefname}{Part}%
\renewcommand{\sectionautorefname}{Section}%
\renewcommand{\subsectionautorefname}{Subsection}%

\providecommand{\theoremautorefname}{Theorem}%
\providecommand{\lemmaautorefname}{Lemma}%
\providecommand{\propautorefname}{Proposition}%
\providecommand{\conjectureautorefname}{Conjecture}%
\providecommand{\corautorefname}{Corollary}%

\providecommand{\defnautorefname}{Definition}%
\providecommand{\remarkautorefname}{Remark}%
\providecommand{\exampleautorefname}{Example}%
\providecommand{\claimautorefname}{Claim}%


\usepackage[links]{agda}
%\AtBeginEnvironment{code}{\fontsize{8}{10}}
\AgdaNoSpaceAroundCode{}

% from: https://agda.readthedocs.io/en/v2.6.3/_downloads/59877ce886494c991a213f09e29b712c/article-luaxelatex-different-fonts.lagda.tex
\usepackage{fontspec}

\usepackage{luaotfload}

\directlua{luaotfload.add_fallback
  ("myfallback",
    { "JuliaMono:style=Regular;"
    , "NotoSansMono:style=Regular;"
    , "NotoSansMath:style=Regular;"
    , "Segoe UI Emoji:mode=harf;"
    }
  )}
\defaultfontfeatures{RawFeature={fallback=myfallback}}

\setmainfont{Latin Modern Roman}

\newfontfamily{\AgdaSerifFont}{JuliaMono Regular}[Scale=0.8]
\newfontfamily{\AgdaSansSerifFont}{JuliaMono Regular}[Scale=0.8]
\newfontfamily{\AgdaTypewriterFont}{JuliaMono Regular}[Scale=0.8]
\setmonofont{JuliaMono Regular}[Scale=0.8]
\renewcommand{\AgdaFontStyle}[1]{{\AgdaSansSerifFont{}#1}}
\renewcommand{\AgdaKeywordFontStyle}[1]{{\AgdaSansSerifFont{}#1}}
\renewcommand{\AgdaStringFontStyle}[1]{{\AgdaTypewriterFont{}#1}}
\renewcommand{\AgdaCommentFontStyle}[1]{{\AgdaTypewriterFont{}#1}}
\renewcommand{\AgdaBoundFontStyle}[1]{\textit{\AgdaSerifFont{}#1}}

\definecolor{git-green}{HTML}{13A10E}
\definecolor{git-orange}{HTML}{C69026}

\newcommand{\added}[1]{\textcolor{git-green}{+#1}}
\newcommand{\changed}[1]{\textcolor{git-orange}{$\bullet$#1}}
\newcommand{\towrite}[1]{\todo[color=cyan]{#1}}
\newcommand{\toremove}[1]{\textcolor{red}{This is going to be (re)moved: ``#1''}}


% macros
\newcommand{\investigate}[1]{\par\vspace{1\baselineskip}\textcolor{gray}{#1}\vspace{1\baselineskip}\par}

% symbols
\newcommand{\bN}{\AgdaDatatype{ℕ}}
\newcommand{\bL}{\AgdaDatatype{Leibniz}}


\title{Ornaments and Proof Transport applied to Numerical Representations}
\author{Samuel Klumpers\\6057314}


\begin{document}
\maketitle

%\begin{abstract}
%This thesis explains the concepts of the structure identity principle, numerical representations, and ornaments, and aims to combine these to simplify the presentation and verification of finger trees, demonstrating the generalizability and improved compactness and security of the resulting code. Consequently, we also investigate to which extent ornaments, and other generic programs relying on axiom K, remain applicable in the cubical setting required for the structure identity principle.
%\end{abstract}

\tableofcontents

\listoftodos

\section{Introduction}\label{sec:intro}
\todo[inline, color=red]{Spend some more time on the context}
Agda \cite{agda} is a functional programming language and a proof assistant, taking inspiration from languages like Haskell and other proof assistants such as Coq. We can write programs as we would in Haskell, and then express and prove their properties all inside Agda. This allows us to demonstrate the correctness of programs by formal proof rather than by testing. 

However, this level of formality also trades-off the uncertainty of testing for a time-investment to produce these proofs. In this thesis, we will explore a variety of methods of proving properties of our programs, focussing on the problems that one may encounter, presenting solutions as they arise. Let us sketch some of these problems.

First, merely adapting a program to Agda may already require changes to the datatypes used in it; for example, if a program manipulating a \AgdaDatatype{List} uses the unsafe \AgdaFunction{head} function, then one is forced to replace the \AgdaDatatype{List} by a datatype that ensures non-emptiness, such as a \AgdaDatatype{NonEmpty} list or a length-aware vector \AgdaDatatype{Vec}. On the other hand, there might be sections of a program where the concrete length is not relevant for correctness and only gets in the way. As a result, one might find themselves duplicating common functions like concatenation \AgdaFunction{\_++\_} to only alter their signatures.

However, the ``new'' datatype (\AgdaDatatype{Vec}) is typically a simple variation on the old datatype (\AgdaDatatype{List}) making small adjustments to the existing constructors; in this case, we decorate the nil and cons constructors with natural numbers representing the length. This kind of modification of types falls in the framework of ornamentation as described by Ko and Gibbons \cite{progorn}; if two types are reified to their \textit{descriptions}, then \textit{ornaments} express whether the types are ``similar'' by acting as a recipe to produce one type from the other. By restricting the operations to the copying of corresponding parts, and the introduction of fields or dropping of indices, the existence of such an ornament ensures that the types have the same recursive structure.

\towrite{Something about patches.}

\towrite{For each invariant a new datatype? Still ornaments}

Now that we know we can organize similar datatypes using ornaments, it is time to look at dissimilar datatypes. It is conventional to prototype a program using simpler types or implementations, and only replace these with more performant alternatives in critical places; knowing that this is eventually going to happen, one might as well prepare for it. While this may quickly turn into a refactoring nightmare in the general case, we can hope for a more satisfying transition if we restrict our attention to a narrower scope. As an example, we might start programming using \AgdaDatatype{List}s, but replace this with a \AgdaDatatype{Tree} if we notice that the program spends most of its time in \AgdaFunction{lookup} operations. To gain a speedup, we will have to reimplement the operations on \AgdaDatatype{Tree}. This would also double the number of necessary proofs; however, we have two ways to avoid this problem. 

We will look at the more specific solution first. This solution is guided by the realization that \AgdaDatatype{List} and \AgdaDatatype{Tree}, like most other containers, still have similarities if their recursive structure is very different. That is, both resemble a number system, and, Okasaki \cite{purelyfunctional} notes that this resemblance to number systems is ``surprisingly common''. In the case of lists and Braun trees\footnote{Braun trees are a kind of binary tree, of which its shape is determined by its size.}, one can present both by deriving them from unary and binary numbers respectively, as is made formal by Hinze and Swierstra \cite{calcdata}. One can then apply this \textit{numerical representation} to simplify or make trivial the proofs of the properties we hesitated to duplicate before.

\towrite{If we instead hide our datatypes behind interfaces, we can use proof transport as an alternative.}

\towrite{Something about fingertrees, leading into the research question and proposed work}


\section{\toremove{Introduction}}
%The dependently typed functional programming language Agda \cite{agda} can, when restricted to its reasonable parts, be translated into readable and safe Haskell \cite{agda2hs}. However, the intrinsic safety of languages like Agda can also lead to code duplication by encouraging the use of multiple variants of the same datatype. As an example, the coverage check forces the \AgdaFunction{head} function on \AgdaDatatype{List} to return a \AgdaDatatype{Maybe}. This \AgdaDatatype{Maybe} can be avoided by moving to the length-indexed list type \AgdaDatatype{Vec}, at the cost of duplicating functions like \AgdaFunction{\_++\_}, which we need at both types.

Something similar happens when replacing an implementation with a more efficient one. For example, when implementing binary trees as a more efficient alternative to lists, the proofs of the same properties will differ between list and tree, and tend to be more difficult for the latter. Switching between implementations of an interface not only duplicates code, but also (and sometimes more than) doubles the effort required to verify both.\todo[inline, color=red]{concrete example?}

%There is plenty of prior work dealing with problems like these. The work in \cite{orntrans} and \cite{progorn} provides the means to relate similar datatypes, such as lists and vectors, using the mechanism of ornamentation, letting us organize variants of the same datatype in a rigid framework.  %This leads them to define the concept of patches, which can aid us when defining \AgdaFunction{\_++\_} for the second time by forcing the new version to be coherent.
%In fact, the algebraic nature of ornaments yields the definition of the vector type for free, provided we relate lists to natural numbers \cite{algorn}. %Such constructions rely heavily on descriptions of datastructures and often come with limitations in their expressiveness. These descriptions in turn impose additional ballast on the programmer, leading us to investigate reflection like in \cite{practgen} as a means to bring datatypes and descriptions closer when possible.

Other work like \cite{calcdata}\todo[inline, color=red]{don't use \textbackslash cite as noun} simplifies the proofs relating to certain containers directly, formally executing the way of though of numerical representations as noted in \cite{purelyfunctional}.
%From another point of view, lists and trees are not so different at all, provided we look at them through the interface of one-sided flexible arrays; this idea noted in \cite{purelyfunctional} and formalized in \cite{calcdata} where both are shown to be instances of numerical representations by calculating them from a numeral system. 

When two types are isomorphic and equivalent under an interface, proofs of properties of these implementations should be interconvertible. By using structured equivalences and univalence, \cite{iri} characterizes equivalences under interfaces.
%While this is achievable through meta-programming, substituting conversions to and from into the proof terms, this is internally expressible in Cubical Agda.

%We can liken the situation to movement on a plane, where ornamentation moves us vertically by modifying constructors or indices, and structured equivalences move us horizontally to and from equivalent but more equivalent implementations. In this paper, we will investigate a variety of means of moving around structures and proofs, and ways to make this more efficient or less intrusive.

In \autoref{sec:leibniz}, we will follow \cite{iri}, and look at how proofs on unary naturals can be transported to the binary naturals. Then in \autoref{sec:numrep} we recall how numeral systems in particular induce container types in \cite{calcdata}, which we attempt to reformulate in the language of ornaments in \autoref{ssec:ornaments}, using the framework of \cite{progorn}. In \autoref{sec:userfriendly} we investigate how we can make the earlier methods more easily accessible to the user, and, ourselves, when we give a description of finger trees in \autoref{sec:fingertrees}.\todo[inline, color=red]{Ok, but make the research question more concrete}



\subsection{The Problem}
The main question of this project is: \textit{can we describe finger trees \cite{fingertrees} in the frameworks of numerical representations and ornamentation \cite{progorn}, simplifying the verification of their properties as flexible two-sided arrays?}\todo{Revisit this when further} This question generates a number of interesting subproblems, such as that the number system corresponding to finger trees has many representations for the same number, which we expect to describe using quotients \cite{cuagda} and reason about using representation independence \cite{iri}. %If this is accomplished or deemed infeasible at an early stage, we can generalize the results we have to other related problems; for example, we may view the problem of generating arbitrary values for testing as an instance of an enumeration problem.


\subsection{Contributions}
In this paper, we:\todo{Revisit this when further}
\begin{itemize}
    \item[x] Adapt ornaments to nested types.
    \item[x] Allow ornaments to refer to sub-ornaments.
    \item[x] Define a small universe of typical number systems.
    \item[x] Give a generic derivation of numerical representations as ornaments from these number systems.
%    \item Instantiate a Structure Identity Principle for these representations.
\end{itemize}
We follow this up by enumerating these, and more structures. We:
\begin{itemize}
    \item[x] Define hierarchies to enumerate terms by levels.
    \item[x] Track the cardinalities of each level.
    \item Include parametrized datatypes into this setup.
    \item Modify this to include nested types.
    \item[?] Adapt this approach to index-first datatypes.
    \item Iterate the accessible indices per level.
\end{itemize}
Along the way, we also:
\begin{itemize}
    \item[x] Characterize identities of W-types.
    \item[x] Express heterogeneous variants of datastructures as ornaments.
\end{itemize}


\part*{Background}\label{part:background}
\addcontentsline{toc}{part}{\nameref{part:background}}
\subsection{Agda}
We formalize our work in Agda \cite{agda}, a functional programming language with dependent types. Using dependent types we can use Agda as a proof assistant, allowing us to state and prove theorems about our datastructures and programs. These proofs can then be run as algorithms, or in some cases be extracted to a Haskell program\footnote{Or JavaScript, if you want.}.

The type system of Agda is an extension of (intensional) Martin-Löf type theory (MLTT), a constructive type theory in which we can interpret intuitionistic logic. Compared to Haskell, which extends a polymorphic lambda calculus with inductive types, MLTT allows the type of the codomain of a function to vary with the values in the domain and the type of the second field of a pair type to vary with the value of the first. The interpretation of logic into programs is known as the Curry-Howard isomorphism: propositions or logical formulas are related to types, such that a term of a type constitutes a proof of the related proposition.

Syntactically Agda is similar to Haskell, with a few notable differences. One is that Agda allows most characters and words in identifiers with only a small set of exceptions. For example, we can write
\ExecuteMetaData[Tex/Snippets]{ternary}
The other is that datatypes are given either as generalized algebraic datatypes (GADTs) or record types in Haskell.

An essential semantical difference is that Agda rules out non-termination by restricting function definitions to structural recursion. The termination checker (together with other restrictions which we will encounter in due time) ensures that the logic interpreted in Agda remains consistent, and does not allow trivial proofs which would be tolerated in Haskell, like
\ExecuteMetaData[Tex/Snippets]{loop}
The propositional part of the Curry-Howard correspondence can then be formulated by the usual type formers. The atomic formulas, true and false, can be represented respectively as the empty record: there always is a proof \AgdaFunction{tt} of true
\ExecuteMetaData[Tex/Snippets]{true}
and the type with no constructors: there is no way to make a proof of false
\ExecuteMetaData[Tex/Snippets]{false}
Implication $A \implies B$ corresponds to function types $A \to B$: a proof of $A$ can be converted to a proof of $B$. Implication also gives an interpretation of negation as functions into false $A \to \bot$. Disjunction (logical or) is described by a sum type $A + B$: either of $A$ or $B$ can prove $A + B$
\ExecuteMetaData[Tex/Snippets]{either}
Conjunction (logical and) is given as a product type: having both $A$ and $B$ proves $A \times B$
\ExecuteMetaData[Tex/Snippets]{pair}
Predicates, formulas containing variables, correspond to functions into the type of formulas
\ExecuteMetaData[Tex/Snippets]{predicate}
allowing interpretations of higher-order logic. Quantifiers are interpreted via dependent types, universal quantification (for all) is a dependent function type: for each $a : A$, give a proof of $P\ a$
\ExecuteMetaData[Tex/Snippets]{forall}
Likewise, existential quantification (exists) is a dependent pair type: there is an $a : A$ and a proof $P\ a$
\ExecuteMetaData[Tex/Snippets]{exists}
Predicates can also be expressed using indexed datatypes, in which the choice of constructor can influence the index, whereas parameters must be constant over all constructors. Equality of elements of a type $A$ can then be interpreted as the type
\ExecuteMetaData[Tex/Snippets]{eq}
Closed terms of this type can only be constructed for definitionally equal elements, but crucially, variables can contain equalities between different elements. As the second argument is an index, pattern matching on \AgdaFunction{refl} unifies the elements, such that properties like substitution follow
\ExecuteMetaData[Tex/Snippets]{subst}
\towrite{Isomorphisms}

\subsection{Cubical Agda}
\towrite{Gluing everything together, making representation independence run.}

The methods described in later sections yield type isomorphisms. One might expect that like how isomorphic groups share the same group-theoretical properties, isomorphic types also share the same type-theoretical properties. Meta-theoretically, this is known as \emph{representation independence}, and is evident: if $e: A \simeq B$, simply modify the type by substituting all variables $x : B$ with $e x'$ for $x' : A$, and replacing the resulting terms $t : B$ by $e^{-1} t$. Then the proof term can be recovered by substituting along the equalities $e^{-1}(e x) \equiv x$ as needed.

Inside (ordinary) Agda this is not so practical, as this independence only holds when applied to concrete types, and is then only realized by manually performing these substitutions. On the other hand, in Cubical Agda, the Structure Identity Principle internalizes a kind of representation independence \cite{iri}.

Cubical Agda modifies the type theory of Agda to a kind of homotopy type theory, looking at equalities as paths between terms rather than the equivalence relation generated by reflexivity. In cubical type theories, the role played by pattern matching on \AgdaFunction{refl} or by axiom J, in MLTT and ``Book HoTT'' respectively, is instead acted out by directly manipulating cubes\footnote{Under the analogy where a term is a point, an equality between points is a line, a line between lines is a square.}. In Cubical Agda, univalence
% ua
is not an axiom but a theorem.

% Why circles are points with K

% Why circles are not points with univalence

\subsection{Numerical representations}
\towrite{Generalizing the observation that lists look like unary naturals and Braun trees look like binary naturals.}

\subsection{Generic programming and ornaments}
\towrite{Taking the writing out of our hands, formalizing the ``looks like relation''.}




\part{Descriptions and ornaments}\label{part:ornament}
In the framework of \AD{DescI} in the last section, we can write down a number system and its meaning as the starting point of the construction of a numerical representation. To write down the generic construction of those numerical representations, we will need a language in which we can describe modifications on the number systems.

\changed{Somewhat final version above, draft/notes/rough comments/outline below.}
In this section, we will describe the ornamental descriptions for the \AD{DescI} universe, and explain their working by means of (plenty of examples). We omit the definition of the ornaments, since we will only construct new datatypes, rather than relate pre-existing types\footnote{Maybe, I will throw the ornaments into the appendix along with the conversion from ornamental description to ornament}.

% (Be alarmed, the implicits get out of hand pretty quickly.)
\todo{do we need to remark more?}


\section{Ornamental descriptions}
These ornamental descriptions take the same shape as those in \autoref{sec:background-ornamental-descriptions}, generalized to handle nested types, variable transformations, and composite types. Like the interpretation of a \AD{DescI}, ornaments also completely ignore the \AD{Info} of a \AD{DescI}.

Recall that a \AD{OrnDesc}\ \AV{If′ Δ c J i D} represents the ornament building on top of \AV{D}, which yields a description with information \AV{If′}, parameters \AV{Δ}, and indices \AV{J}. We use \AF{∼} to write down pointwise equality of functions, which in this case are all commutativity squares. Since \AD{ConI} allows the transformation of variable telescopes, we have to dedicate a lot of lines to writing down commutativity squares for variables, which along with the generally high number of arguments and implicits\footnote{Of which even more are hidden!} makes the definition rather dry and long.

One tip is to  ignore all squares involving a \AD{Vxf}, these are trivial when using the \AV{+-} variants of the \AIC{σ} and \AIC{δ} formers anyway! Due to the last constructor \AIC{δ•}, \AD{OrnDesc}, \AD{ConOrnDesc}, and \AF{toDesc}\footnote{We left out the variable square for \AIC{δ•}, because it is honestly just too long. If this was included, then we also would involve \AF{ornForget}.} become tightly connected, so the definition is given in one large mutual block:
\ExecuteMetaData[Ornament/OrnDesc]{OrnDesc}
Here the implicit \AV{If′} contains the information necessary to recover the \AD{DescI} from an \AD{OrnDesc}:\todo{line length}
\ExecuteMetaData[Ornament/OrnDesc]{toDesc}
The commutativity squares again ensure the existence of functions like \AF{ornForget}, and that these ornamental descriptions indeed induce ornaments.

Compared to the previous ornaments, we have the new constructors \AIC{δ}, \AIC{Δδ} and \AIC{δ•}, where the first two are analogues of \AIC{σ} and \AIC{Δσ}. The \AIC{δ•} constructor states that an ornamental description from a description \AV{R} and a (constructor) ornamental description from \AV{CD} can be composed to form an ornamental description from the composition (in the sense of the \AV{δ} type-former) of \AV{CD} with \AV{R}.

Let us make the uses of \AD{OrnDesc} more clear by means of examples, where we make use of the simpler variants: \todo{Oδ•+- needs ornForget to run}
\ExecuteMetaData[Ornament/OrnDesc]{O-sigma-pm}
With these we can give the now familiar ornamental description of \AD{Vec} from \AD{List}:
\ExecuteMetaData[Ornament/OrnDesc]{VecOD}
Using the new flexibility in \AIC{ρ}, we can now start from a description of binary numbers:
\ExecuteMetaData[Ornament/OrnDesc]{LeibnizD}
and give the random access lists from before as an ornamental description as well.
\ExecuteMetaData[Ornament/OrnDesc]{RandomOD}
Likewise, we can use \AIC{δ•} to start from the ``fingertree numbers'':
\todo{finger tree skeleton}
and compose this with the ornamental description of \AD{Digit}
\todo{DigitOD}
to obtain the ornamental description of finger trees:
\todo{FingerTreeOD}

%Again, ornForget, fold blabla.

\todo{Now we can compute everything generically.}


\begin{outline}
\ExecuteMetaData[Ornament/OrnDesc]{ConOrnDesc-type}
The definition of ornamental descriptions can be derived in a straightforward manner from ornaments, removing all mentions of the LHS and making all fields which then no longer appear in the indices explicit\footnote{One might expect to need less equalities, alas, this is difficult because of \autoref{rem:orn-lift}.}. We will show the leaf-preserving rule as an example, the others are derived analogously:
\ExecuteMetaData[Ornament/OrnDesc]{OrnDesc-1}
As we can see, the only change we need to make is that \AgdaBoundFontStyle{k} becomes explicit and fully annotated.

Almost by construction, we have that an ornamental description can be decomposed into a description of the new datatype
\ExecuteMetaData[Ornament/OrnDesc]{toDesc}
and an ornament between the starting description and this new description
\ExecuteMetaData[Ornament/OrnDesc]{toOrn}
\end{outline}


\begin{outline}    
\section{The ornaments}
we could ditch removal of fields: we don't use it. downside: ornament over ornament is the same as field removal for deltas

:warning: match everything, add/remove field, add/remove recursive field, add/remove description field, ornament over ornament

\todo{Nuke ornaments, keep ornamental descriptions}

\towrite{Put something that isn't yet in \autoref{ssec:bg-orn} here.}

\ExecuteMetaData[Ornament/Orn]{Orn-type}
\ExecuteMetaData[Ornament/Orn]{ornForget-type}

We will walk through the constructor ornaments
\ExecuteMetaData[Ornament/Orn]{ConOrn-type}
again, an ornament between datatypes is just a list of ornaments between their constructors
\ExecuteMetaData[Ornament/Orn]{Orn}
Note that all ornaments completely ignore information bundles! They cannot affect the existence of \AgdaFunction{ornForget} after all.

Copying parts from one description to another, up to parameter and index refinement, corresponds to reflexivity. Preservation of leaves follows the rule
\ExecuteMetaData[Ornament/Orn]{Orn-1}
We can see that this commuting square (\texttt{e (k p) ≡ j (over f p)}) is necessary: take a value of \texttt{E} at \texttt{p, i}, where \texttt{i} is given as \texttt{k p}. Then \AgdaFunction{ornForget} has to convert this to a value of \texttt{D} at \texttt{f p , e i}, but since \texttt{e i} must match \texttt{j (f p)}, this is only possible if \texttt{e (k p) = j (f p)}.

Preserving a recursive field similarly requires a square of indices and conversions to commute
\ExecuteMetaData[Ornament/Orn]{Orn-rho}
additionally requiring the recursive parameters to commute with the conversion. \todo{Does adding the derivations for the squares everywhere make this section clearler?}

Preservation of non-recursive fields and description fields is analogous
\ExecuteMetaData[Ornament/Orn]{Orn-sigma-delta}
differing only in that non-recursive fields appears transformed on the right hand, while description fields have their conversions modified instead. For this rule, we need that the variable transformations fit into a commuting square with the parameter conversions. The condition on indices for descriptions, which is a commuting triangle, is encoded in the return type\footnote{Should this become a problem like with \AgdaInductiveConstructor{ρ}, modifying the rule to require a triangle is trivial.}.

Ornaments would not be very interesting if they only related identical structures. The left-hand side can also have more fields than the right-hand side, in which case \AgdaFunction{ornForget} will simply drop the fields which have no counterpart on the right-hand side. As a consequence, the description extending rules have fewer conditions than the description preserving rules: 
\ExecuteMetaData[Ornament/Orn]{Orn-+-rho}
Note that this extension\footnote{Kind of breaking the ``ornaments relate types with similar recursive structure'' interpretation.} with a recursive field has no conditions.

Extending by a non-recursive field or a description field again only requires the variable transform to interact well with the parameter conversion
\ExecuteMetaData[Ornament/Orn]{Orn-+-sigma-delta}

In the other direction, the left-hand side can also omit a field which appears on the right-hand side, provided we can produce a default value
\ExecuteMetaData[Ornament/Orn]{Orn---sigma-delta}
These rules let us describe the basic set of ornaments between datatypes.

Intuitively we also expect a conversion to exist when two constructors have description fields which are not equal, but are only related by an ornament. Such a composition of ornaments takes two ornaments, one between the field, and one between the outer descriptions. This composition rule reads:\todo{The implicits kind of get out of control here, but the rule is also unreadable without them. I might hide the rule altogether and only run an example with it.}
\ExecuteMetaData[Ornament/Orn]{Orn-comp}
We first require two commuting squares, one relating the parameters of the fields to the inner and outer parameter conversions, and one relating the indices of the fields to the inner index conversion and the outer parameter conversion. Then, the last square has a rather complicated equation, which merely states that the variable transforms for the remainder respect the outer parameter conversion.

We will construct \AgdaFunction{ornForget} as a \AgdaFunction{fold}. Using
\ExecuteMetaData[Ornament/Orn]{erase-type}
we can define the algebra which forgets the added structure of the outer layer
\ExecuteMetaData[Ornament/Orn]{ornAlg}
Folding over this algebra gives the wanted function
\ExecuteMetaData[Ornament/Orn]{ornForget}

\todo{NatD was removed here}

We can also relate lists and vectors
\ExecuteMetaData[Ornament/Orn]{ListD-VecD}
Now the parameter conversion is the identity, since both have a single type parameter. The index conversion is \AgdaFunction{!}, since lists have no indices. Again, most structure is preserved, we only note that vectors have an added field carrying the length.

Instantiating \AgdaFunction{ornForget} to these ornaments, we now get the functions \AgdaFunction{length} and \AgdaFunction{toList} for free!

%\investigate{Having a function of the same type as \AgdaFunction{ornForget} is not sufficient to deduce an ornament. An obstacle is that the usual empty type (no constructors) and the non-wellfounded empty type (only a recursive field) don't have an ornament. Also, while the leaf-preservation case spells itself out, the substitutions obviously don't give us a way to recover the equalities.}
\end{outline}
    


\part{Numerical representations}\label{part:numrep}
%\section{Types from Specifications: Ornamentation and Calculation}\label{sec:numrep}
\documentclass[Main.tex]{subfiles}

\begin{document}
While the practical applications of the last example do not stretch very far\footnote{Considering that \AgdaDatatype{ℕ} is a candidate to be replaced by a more suitable unsigned integer type when compiling to Haskell anyway.}, the approach generalizes to the more relevant containers and their associated laws.

In the same vein as the last section, we could define a simple but inefficient array type, and a more efficient implementation using trees. Then we can show that these are equivalent, such that when the simple type satisfies a set of laws, trees will satisfy them as well. We could then start developing all sorts of complex implementations fine-tuned to each operation and figure out how these are equivalent to some simpler type, but let us first take a step back, and investigate how we can make this approach run smoothly in an even simpler example.

Rather than defining inductively defining a container and then showing that it is represented by a lookup function, we can go the other way and define a type by insisting that it is equivalent to such a function. This approach, in particular the case in which one calculates a container with the same shape as a numeral system, was dubbed numerical representations in \cite{purelyfunctional}, and has some formalized examples in, e.g., \cite{calcdata} and \cite{progorn}. Numerical representations form the starting point for defining more complex datastructures based off of simpler basic structures, so let us run through an example.

\subsection{Numerical representations: from numbers to containers}
We can compute the type of vectors starting from \bN{}.\footnote{This is adapted (and fairly abridged) from \cite{calcdata}} For simplicity, we define them as a type computing function via the ``use-as-definition`` notation from before. We expect vectors to be represented by 
\[ lookup \]
where we use the finite type \AgdaDatatype{Fin} as an index into vector. We define this as
\[ finfromsigma \]
The computation of vectors proceeds as follows
\[ vectors \]

\investigate{SIP doesn't mesh very well with indexed stuff, does HSIP help?}

Arrays are made to be indexed, but let us list some expectations
\todo{do this}

The implementation of vectors as functions is very straightforward
\[ \]
and clearly satisfies our interface
\[ \]
Again these proofs transport to vectors.\todo{If one was determined to cobble together the path over path over path we need now.}

(This computation can of course be generalized to any arity zeroless numeral system; unfortunately beyond this set of base types, this ``straightforward'' computation from numeral system to container loses its efficacy. In a sense, the n-ary natural numbers are exactly the base types for which the required steps are convenient type equivalences like $(A + B) \to C = (A \to C) \times (B \to C)$?)

%\subsection{Relating types by structure: Ornamentation (unfinished)}\label{sec:ornament}
\subsection{Numerical representations as ornaments}
We could peform the same computation for \bL{}, which would yield the type of binary trees, but we note that these computations proceed with roughly the same pattern: each constructor of the numeral system gets assigned a value, and is amended with a field holding a number of elements and subnodes using this value as a ``weight''. But wait! Such modifications of constructors are already made formal by the concept of ornamentation!

Ornamentation, as exposed in \cite{algorn} and \cite{progorn}, lets us formulate what it means for two types to have a ``similar'' recursive structure. This is achieved by interpreting (indexed inductive) datatypes from descriptions, between which an ornament is seen as a certificate of similarity, describing which fields or indices need to be introduced or dropped. Furthermore, a one-sided ornament: an ornamental description, lets us describe new datatypes by recording the modifications to an existing description.
\todo{Again not sure if it helps to reiterate Desc, Orn, and OrnDesc.}

This links back to the construction in the previous section, since \bN{} and \AgdaDatatype{Vec} share the same recursive structure, so \AgdaDatatype{Vec} can be formed by introducing indices and adding a field holding an element at each node.\footnote{These and similar examples are also documented in \cite{progorn}} For this, we first have to give a description of \bN{} to work with\todo{Clearly this can use more explanation (the question is, how much?)}
\todo{include this}
Now we can write down the ornament which adds fields to the \AgdaFunction{suc} constructor
\[ include me \]
With the least fixpoint and description extraction from \cite{progorn}, this is sufficient to define \AgdaDatatype{Vec}.

Note that we cannot hope to give an unindexed ornament from \bL{} into trees, since trees have a very different recursive structure! Instead, we must keep track at what level we are in the tree so that we can ask for adequately many elements. 
\todo{include this}

In fact, this ``folding in'' technique seems to apply rather generally, let us digress.

\subsection{Folding in}
Let us describe this procedure of folding a complex recursive structure into a simpler structure more generally, and relate this to the construction of binary heaps in \cite{progorn}.
\todo{go}


\end{document}


\part{Enumeration}\label{part:enum}
Property based testing frameworks often rely on random generation of values, consider for example the Arbitrary class of Quickcheck \cite{quickcheck}. How these values are best generated depends on the property being tested; if we are testing an implementation of \AgdaFunction{insertSorted}, we should probably generate sorted lists \cite{rest}! Some frameworks like Quickcheck do provide deriving mechanisms for Arbitrary instances, but this relinquishes most control over the distribution. This leaves manually re-implementing Arbitrary as necessary as the only option for a user who wants to test properties with more sophisticated preconditions.

A more controllable alternative to random generation is the complete enumeration of all values. Provided that such an enumeration supports efficient (and fair) indexing, one can adjust a random distribution of values by controlling the sampling from enumerations. There is rich theory of enumeration, and this problem has also been tackled numerous times in the context of functional programming. Some approaches focus on the efficient indexing of enumerations \cite{feat}, others focus on generating indexed types as a means of enumerating values with invariants \cite{uqenum}.

We will describe a framework generalizing these approaches, which will support:
\begin{enumerate}
    \item unique and complete enumeration
    \item indexing by (exact) recursive depth
    \item fast skipping through the enumeration
    \item indexed, nested, and mutually recursive types
\end{enumerate}

We will follow an approach similar to the list-to-list approach \cite{uqenum}, but rather than expressing enumerations as a step-function, computing the next generation of values from a list of predecessors, we will keep track of the entire depth indexed hierarchies.

\section{Basic strategy}
We define a hierarchy of elements as
\ExecuteMetaData[Enumeration/Approach1]{hierarchy}
When applied to a number $n$, a hierarchy should then return the list of elements of exactly depth $n$. To iteratively approximate hierarchies, we define a hierarchy-builder type
\ExecuteMetaData[Enumeration/Approach1]{buildertype}
Hierarchy-builders should be able to take a partially defined hierarchy, and return a hierarchy which is defined at one higher level.

We implement some basic hierarchy building blocks, such as the one-element builder
\ExecuteMetaData[Enumeration/Approach1]{pure}
which represents nullary constructors, and the shift builder
\ExecuteMetaData[Enumeration/Approach1]{rec}
which represents recursive fields.

To interpret sum types, we use an interleaving operation. Consider that for the disjoint sum, the elements at level $n$ must be formed from elements which are also at level $n$, regardless whether they are from the left summand or the right.
\ExecuteMetaData[Enumeration/Approach1]{alternative}
For product types, the elements at level $n$ are those which contain at least one component at level $n$, so we have to sum all possible combinations of products
\ExecuteMetaData[Enumeration/Approach1]{pair}
We claim that this is sufficient to enumerate the following simple universe of types
\ExecuteMetaData[Enumeration/Approach1]{Desc}
In the same vein as other generic constructions, we can define a generic builder by cases over the interpretetation
\ExecuteMetaData[Enumeration/Approach1]{builder}
By applying constructors, we can wrap this up into an endomorphism at a fixpoint
\ExecuteMetaData[Enumeration/Approach1]{gbuilder}
Finally, we observe that applying this builder $n+1$ times to the empty hierarchy is sufficient to approximate the hierarchy up to level $n$
\ExecuteMetaData[Enumeration/Approach1]{build}
which gives us the generic \AgdaFunction{hierarchy}

We can for example apply this to generate binary trees of given depths
\ExecuteMetaData[Enumeration/Approach1]{TreeD}
which returns the following trees of level 2 
\ExecuteMetaData[Enumeration/Approach1]{trees-2}


However, it would be even cooler if
\begin{enumerate}
    \item An enumeration could tell us how many elements there are of some depth
    \item An enumeration was a map from constructor to subsequent enumerations
    \item The possible indices get computed as we go down.
\end{enumerate}
The first is essential for sampling. The second would give the user total control over the shapes of their generated values. And the third is particularly crucial when the set of possible indices is small.

\section{Cardinalities}
Simplifying our earlier approach a bit, we can tinker 
\ExecuteMetaData[Enumeration/Counting]{hierarchy}
to track the sizes. For example, our interleaving operation becomes
\ExecuteMetaData[Enumeration/Counting]{alternative}
We can write down a generic hierarchy
\ExecuteMetaData[Enumeration/Counting]{ghierarchy}
Then we can count 
\ExecuteMetaData[Enumeration/Counting]{numTrees}
and see that there are 210065930571 trees of level 6, wow! It still takes a bit of time to walk across all branches and products in the description, because there is no memoization at all, but it's a lot better than counting the trees after generating them. Also indexing will be slow, even knowing this information, because we're working with plain lists. Things would probably already get a lot better if we worked with trees that know the sizes of their children.

\section{Indexed types}
Ideally, we get a meaningful list or enumeration of indices at the end: the non-empty ones. However, we do not (yet) require the index type to be enumerable.

The index-first presentation of indexed datatypes, while efficient and succinct, does not seem suitable for this. After all, the descriptions for such a presentation live in the function space from the index to the base descriptions. We would rather want to start ``recklessly applying'' constructors and seeing what kinds of indices that leaves us with.

This example explains why it's also pretty hopeless for Sijsling's descriptions:
\ExecuteMetaData[Enumeration/Indexed]{Counter}
We would need a notion of ``forward indexed type'' in which the indices in the arguments must be strictly less crazy than those in the resulting type.

Anyway, we restrict our attention to indexed types that work, that is, we can decide whether an index fits. In the previous example, the constructor would instead compute whether $n$ is $n' + 2$, and return $n'$ if it is. This completely breaks any attempt at counting the enumeration.

In comparison, the index-first presentation tells us nothing about which indices are reachable, but probably does better with counting. I suppose you could combine them at the cost of a lot, and first run the forward idea on only the indices, and then see how much each index has, or something.


%\part{Temporary}

\part{Related work}\label{part:related}
\section{Descriptions and ornaments}
We compare our implementation to a selection of previous work, considering the following features


\begin{tabular}{c | c c c c c}
             & Haskell        & \cite{initenough} & \cite{levitation} & \cite{algorn} & \cite{progorn} \\
    \hline                                                                                             
    Fixpoint & yes*           & yes               & no                & yes?          & yes            \\
    Index    & —              & —                 & first**           & equality      & first          \\
    Poly     & yes            & 1                 & external          & external      & external       \\
    Levels   & —              & —                 & no                & no            & no             \\
    Sums     & list           & —                 & large             & large         & large          \\
    IndArg   & any            & any               & $\dots \to X\ i$  & $X\ i$        & $X\ i$         \\
    Compose  & yes            & yes               & no                & no            & no             \\
    Extension& —              & —                 & no                & —             & —              \\
    Ignore   & —              & —                 & —                 & —             & —              \\
    Set      & —              & —                 & —                 & —             & —              \\
\end{tabular}


\begin{tabular}{c | c c c c c}
             & \cite{sijsling} & \cite{effectfully} & \cite{practgen} & Shallow   & Deep (old) \\
    \hline   
    Fixpoint & yes             & yes                & no              & yes       & yes     \\
    Index    & equality        & equality           & equality        & equality  &         \\
    Poly     & telescope       & external           & telescope       & telescope &         \\
    Levels   & no***           & cumulative         & Typeω           & Type-in-Type &         \\
    Sums     & list            & large              & list            & list      &         \\
    IndArg   & $X\ pv\ i$      & $\dots\to X\ v\ i$ & $\dots\to X\ pv\ i$ & $X (f pv) i$ & ?1 \\
    Compose  & no              & yes?2              & no              & yes       &         \\
    Extension& —               & yes                & yes             & no        &         \\
    Ignore   & no              & ?                  & ?               & transform &         \\
    Set      & no              & no                 & no              & no        & yes     \\
\end{tabular}



\begin{itemize}
    \item IndArg: the allowed shapes of inductive arguments. Note that none other than Haskell, higher-order functors, and potentially ?1, allow full nested types!
    \item Compose: can a description refer to another description?
    \item Extension: do inductive arguments and end nodes, and sums and products coincide through a top-level extension?
    \item Ignore: can subsequent constructor descriptions ignore values of previous ones? (Either this, or thinnings, are essential to make composites work)
    \item Set: are sets internalized in this description?
\end{itemize}

\begin{itemize}
    \item[*] These descriptions are ``coinductive'' in that they can contain themselves, so the ``fixpoint'' is more like a deep interpretation.
    \item[**] This has no fixpoint, and the generalization over the index is external.
    \item[***] But you could bump the parameter telescope to Typeω and lose nothing.
    \item[*4] A variant keeps track of the highest level in the index.
    \item[?1] Deeply encoding all involved functors would remove the need for positivity annotations for full nested types like in other implementations.
    \item[?2] The ``simplicity'' of this implementation, where data and constructor descriptions coincide, automatically allows composite descriptions.
\end{itemize}

We take away some interesting points from this:
\begin{itemize}
    \item Levels are important, because index-first descriptions are incompatible with ``data-cumulativity'' when not emulating it using equalities! (This results in datatypes being forced to have fields of a fixed level).
    \item Coinductive descriptions can generate inductive types!
    \item Typeω descriptions can generate types of any level!
    \item Large sums do not reflect Agda (a datatype instantiated from a derived description looks nothing like the original type)! On the other hand, they make lists unnecessary, and simplify the definition of ornaments as well.
    \item We can group/collapse multiple signatures into one using tags, this might be nice for defining generic functions in a more collected way.
    \item Everything becomes completely unreadable without opacity.
\end{itemize}

\subsection{Merge me}


\subsubsection{Ornamentation}
While we can derive datastructures from number systems by going through their index types \cite{calcdata}, we may also interpret numerical representations more literally as instructions to rewrite a number system to a container type. We can record this transformation internally using ornaments, which can then be used to derive an indexed version of the container \cite{algorn}, or can be modified further to naturally integrate other constraints, e.g., ordering, into the resulting structure \cite{progorn}. Furthermore, we can also use the forgetful functions induced by ornaments to generate specifications for functions defined on the ornamented types \cite{orntrans}.

\subsubsection{Generic constructions}
Being able to define a datatype and reflect its structure in the same language opens doors to many more interesting constructions \cite{practgen}; a lot of ``recipes'' we recognize, such as defining the eliminators for a given datatype, can be formalized and automated using reflection and macros. We expect that other type transformations can also be interpreted as ornaments, like the extraction of heterogeneous binary trees from level-polymorphic binary trees \cite{hetbin}. 


\subsection{Takeways}
At the very least, descriptions will need sums, products, and recursive positions as well. While we could use coinductive descriptions, bringing normal and recursive fields to the same level, we avoid this as it also makes ornaments a bit more wild\footnote{For better or worse, an ornament could refer to a different ornament for a recursive field.}. We represent indexed types by parametrizing over a type $I$. Since we are aiming for nested types, external polymorphism\footnote{E.g., for each type $A$ a description of lists of $A$ à la \cite{progorn}} does not suffice: we need to let descriptions control their contexts.

We describe parameters by defining descriptions relative to a context. Here, a context is a telescope of types, where each type can depend on all preceding types:
\[ \dots \]
Much like the work Escot and Cockx \cite{practgen} we shove everything into \AgdaFunction{Typeω}, but we do not (yet) allow parameters to depend on previous values, or indices on parameters\footnote{I do not know yet what that would mean for ornaments.}.

We use equalities to enforce indices, simply because index-first types are not honest about being finite, and consequently mess up our levels. For an index type and a context a description represents a list of constructors:
\[ \dots \]
These represent lists of alternative constructors, which each represent a list of fields:
\[ \dots \]
We separate mere fields from ``known'' fields, which are given by descriptions rather than arbitrary types. Note that we do not split off fields to another description, as subsequent fields should be able to depend on previous fields
\[ \dots. \]


We parametrize over the levels, because unlike practical generic, we stay at one level.

Q: what happens when you precompose a datatype with a function? E.g. (List . f) A = List (f A) 

Q: practgen is cool, compact, and probably necessary to have all datatypes. Note that in comparison, most other implementations (like Sijsling) do not allow functions as inductive arguments. Reasonably so.

Q: I should probably update my Agda and make use of the new opaque features to make things readable when refining


\towrite{Adapt this to the non-proposal form.}

\section{The Structure Identity Principle}
If we write a program, and replace an expression by an equal one, then we can prove that the behaviour of the program can not change. Likewise, if we replace one implementation of an interface with another, in such a way that the correspondence respects all operations in the interface, then the implementations should be equal when viewed through the interface. Observations like these are instances of ``representation independence'', but even in languages with an internal notation of type equality, the applicability is usually exclusive to the metatheory.

In our case, moving from Agda's ``usual type theory'' to Cubical Agda, \textit{univalence} \cite{cuagda} lets us internalize a kind of representation independence known as the Structure Identity Principle \cite{iri}, and even generalize it from equivalences to quasi-equivalence relations. 
%a cubical homotopy type theory,
We will also be able to apply univalence to get a true ``equational reasoning'' for types when we are looking at numerical representations.

Still, representation independence in may be internalized outside the homotopical setting in some cases \cite{tgalois}, and remains of interest in the context of generic constructions that conflict with cubical type theory.

\section{Numerical Representations}
Rather than equating implementations after the fact, we can also ``compute'' datastructures by imposing equations. In the case of container types, one may observe similarities to number systems \cite{purelyfunctional} and call such containers numerical representations. One can then use these representations to prototype new datastructures that automatically inherit properties and equalities from their underlying number systems \cite{calcdata}.

From another perspective, numerical representations run by using representability as a kind of ``strictification'' of types. %This suggests that we may be able to generalize the approach of numerical representations, using that any (non-indexed) infinitary inductive-recursive type supports a lookup operation \cite{glookup}.

\part{Discussion}

\section{Temporary: future work (\autoref{part:ornament})}
\begin{remark}
    Note that this allows us to express datatypes like finger trees, but not rose trees. Such datatypes would need a way to place a functor ``around the \AgdaInductiveConstructor{ρ}'', which then also requires a description of strictly positive functors. In our setup, this could only be encoded by separating general descriptions from strictly positive descriptions. The non-recursive fields of these strictly positive descriptions then need to be restricted to only allow compositions of strictly positive context functions. 
\end{remark} % \investigate{This setup does not allow nesting over recursive fields, which is necessary for structures like rose trees. This is actually kind of essential for enumeration. Nesting over a recursive field is problematic: we can incorporate it by adding ``this'' implicitly to a \AgdaInductiveConstructor{δ}, but then the \AgdaBoundFontStyle{R} needs to be strictly positive in its last argument, meaning we need to split \AgdaDatatype{Desc} into a strictly positive part and normal part. The strictly positive part should then only allow strictly positive parameter transforms in recursive and non-recursive fields, requiring an embedding of transforms.}

\begin{remark}
    Variable transforms are not essential in these descriptions, but there are a couple of reasons for keeping them. In particular, they make it possible to reuse a description in multiple contexts, and save us from writing complex expressions in the indices of our ornaments. On the other hand, the transforms still make defining ornaments harder (the majority of the commuting squares are from variables). Isolating them into a single constructor of \AgdaDatatype{Desc}, call it \AgdaInductiveConstructor{v}, seems like a good middle ground, but raises some odd questions, like ``why is there no ornament between \AgdaBoundFontStyle{v (g ∘ f) C} and \AgdaBoundFontStyle{v g (v f C)}''. (Furthermore, this also does not simplify the indices of ornaments).
\end{remark} %\investigate{Variable transforms are both less essential and less troublesome than I first thought. We can move variable transforms into a new former, and it probably simplifies the definition of ornaments a lot.}

\begin{remark}
    Rather, ornaments themselves could act as information bundles. If there was a description for \AgdaDatatype{Desc}, that is. Such a scheme of levitation would make it easier to switch between being able to actively manipulate information, and not having to interact with it at all. However, the complexity of our descriptions makes this a non-trivial task; since our \AgdaDatatype{Desc} is given by mutual recursion and induction-recursion, the descriptions, and the ornaments, would have to be amended to encode both forms of recursion as well.
\end{remark} % \investigate{If we levitate, then informed descriptions become ornaments over \AgdaDatatype{Desc}. This gives us the best of both worlds (modulo reflecting the description into a datatype): in plain descriptions, information does not even exist, and in informed descriptions, it is explicit. For levitation, we likely need induction-recursion and mutual recursion.}

\begin{remark}\label{rem:orn-lift}
    Rather than having the user provide two indices and show that the square commutes, we can ask for a ``lift'' $k$
    % https://q.uiver.app/#q=WzAsNCxbMCwwLCJcXGJ1bGxldCJdLFsxLDAsIlxcYnVsbGV0Il0sWzAsMSwiXFxidWxsZXQiXSxbMSwxLCJcXGJ1bGxldCJdLFswLDEsImUiXSxbMiwzLCJmIiwyXSxbMiwwLCJqIl0sWzMsMSwiaSIsMl0sWzMsMCwiayIsMV1d
    \[\begin{tikzcd}
        \bullet & \bullet \\
        \bullet & \bullet
        \arrow["e", from=1-1, to=1-2]
        \arrow["f"', from=2-1, to=2-2]
        \arrow["j", from=2-1, to=1-1]
        \arrow["i"', from=2-2, to=1-2]
        \arrow["k"{description}, from=2-2, to=1-1]
    \end{tikzcd}\]
    and derive the indices as $i = ek, j = kf$. However, this is more restrictive, unless $f$ is a split epi, as only then pairs $i,j$ and arrows $k$ are in bijection. In addition, this makes ornaments harder to work with, because we have to hit the indices definitionally, whereas asking for the square to commute gives us some leeway (i.e., the lift would require the user to transport the ornament). 
\end{remark}

\begin{remark}
    Comparing SOP and computational sigmas. In particular, \texttt{s N (\ n -> v (replicate n tt))} is not in SOP without full nesting. SOP is good for generics in both directions (the conversion in both ways keeps the datatype like it is supposed to). On the other hand, computational sigmas make writing and proving about \texttt{Path} a lot easier.
\end{remark}


\section{Temporary: future work (\autoref{part:numrep})}

\investigate{This implementation of TrieO always computes the random-access variant of the datastructure. Can we implement a variant which computes the ``Braun tree'' variant of the datastructure?}

\investigate{Index types are a simple ornament over number types: paths. This is not quite like \cite{glookup}.}

\investigate{Is Ix x -> A initial for the algebra of the algebraic ornament induced by TrieO? (This is \cite{calcdata}).}

\investigate{While evidently Ix x != Fin (toN x) for arbitrary number systems, does the expected iso Ix x -> A = Trie A x yield Traversable, for free?}



\printbibliography

\part{Appendix}
\appendix

\section{Finger trees}
%\towrite{Can we prove that the time complexity of head is always less than cons, similarly for lookup and insert?}
We know that some datastructures can be presented as non-redundant numerical representations, for example lists by unary numbers, random access lists by binary numbers \cite{calcdata}, and, skew binary heaps by skew binary numbers \cite{progorn}. So far, some of these examples do support amortized constant time \AgdaFunction{cons}, but they have at best logarithmic time \AgdaFunction{snoc}. This is reflected by their number systems, for which either the natural successor operation is logarithmic time, or is constant time, but can only act at the front. Instead, we will look at more redundant number systems, and refine these step-by-step to produce structures similar to finger trees. This gives us datastructures with fast access to both ends, and some of their properties for free.

\subsection{Binary finger trees}
If a datastructure has a numerical representation, we see that the operations on the datastructure must be coherent with the number system. Hence, if we want to have constant time \AgdaFunction{cons} and \AgdaFunction{snoc}, we must first have constant time \AgdaFunction{suc} anc \AgdaFunction{cus}. By starting from a symmetric number system, we can ensure good performance for both.

Note that such a system is necessarily redundant: if \AgdaFunction{suc} and \AgdaFunction{cus} both are amortized constant time, there must be cases where neither recurses (otherwise, there is a value and a sequence of \AgdaFunction{suc}s and \AgdaFunction{cus}s which cannot be amortized constant). On the other hand, both must clearly yield different values!

Symmetric unary numbers could be represented by a pair of Peano naturals, but would lead to a linear time \AgdaFunction{lookup}. 
By using a binary backbone for the numbers, we can get good \AgdaFunction{suc} and \AgdaFunction{lookup}
\ExecuteMetaData[FingerTrees/Simple]{bin-bad}
However, this shape is still not ideal. We can see that for values like
\ExecuteMetaData[FingerTrees/Simple]{bad-1}
applying \AgdaFunction{suc} would give
\ExecuteMetaData[FingerTrees/Simple]{bad-2}
Applying \AgdaFunction{pred} would take us back, so composing the two always takes logarithmic time \cite{ftanew}. To avoid this, we can give the numbers bigger digits (the system merely goes from redundant to slightly more redundant)
\ExecuteMetaData[FingerTrees/Simple]{bin-good}
Now applying \AgdaFunction{suc} to the pathological case
\ExecuteMetaData[FingerTrees/Simple]{good-1}
produces
\ExecuteMetaData[FingerTrees/Simple]{good-2}
instead, for which both \AgdaFunction{suc} and \AgdaFunction{pred} are constant time \footnote{More formally, we can use recursive slowdown \cite{purelyfunctional,recursiveslowdown} to show that any sequence of operations amortizes to constant time.}. We interpret this number system as
\ExecuteMetaData[FingerTrees/Simple]{interpret}
To extract the datastructure, we must find a suitable index type for these numbers. Since the numbers are redundant, we can also get trees of different shapes with the same size, each having a different and incompatible index type. However, the trees of a fixed shape are represented by functions, and the isomorphisms will still hold.

The computation of the index type from the interpretation of the numbers is straightforward. We first compute the indices for digits, which yields the indices for the numbers
\ExecuteMetaData[FingerTrees/Simple]{ix}
This represents simple finger trees as
\ExecuteMetaData[FingerTrees/Simple]{rep}
To define the basic array operations like \AgdaFunction{cons} on these functions as datastructures, we again construct a \AgdaFunction{Fin}-like view for the indices. For this we produce values corresponding to zero
\ExecuteMetaData[FingerTrees/Simple]{izero}
and induce the successor on the indices using
\ExecuteMetaData[FingerTrees/Simple]{isucc}
The view is similarly defined by
\ExecuteMetaData[FingerTrees/Simple]{iview}
letting us define
\ExecuteMetaData[FingerTrees/Simple]{ops}
We can again trieify this to get a concrete datastructure\footnote{I'll probably not do this manually, because it is theoretically analogous to the other trees, but hellish in practice}
\ExecuteMetaData[FingerTrees/Simple]{trieified}
Consequently, the concrete version will now obey all the relations the representable arrays obey as well. For example, for representable arrays we can easily see 
\ExecuteMetaData[FingerTrees/Simple]{head-cons}
hence, the concrete arrays obey this as well.

On the other hand, as
\ExecuteMetaData[FingerTrees/Simple]{succ-noncomm}
does not generally hold for symmetric binary, \AgdaFunction{cons} will not interchange with \AgdaFunction{snoc} for finger trees either\footnote{For starters, the types are different}; it seems that binary finger trees are not a very nice array type. Likewise, indexing into the finger trees is impractical, as changing shapes would require inefficient re-indexing. 

\subsection{Restoring efficient lookup}
Can we restore lookup? We can probably do something similar to the original finger trees, and maintain the sizes internally (hopelessly breaking the isomorphism\footnote{Or would it stay intact, since the shape determines the size anyway?}). Then we could state that a fingertree of a given size is just a finger tree of a shape paired with a proof that this shape has the right size.

\begin{comment}
We would like to quotient the redundancy of the number type away, which would also alleviate our issues related to indexing.

We can do this by imposing the following relation on the numbers
\[ \dots \]
and turning the number system into a setoid. This ensures that numbers with equal interpretetations are related, e.g.,
\[ \dots \]
Eliminating from this setoid should then respect this relation, so for example, we have to prove
\[ \dots \]

Similarly, we can keep the implementation of the binary finger trees, but also put this under an appropriate relation
\[ \dots \]
which ensures that the construction of the trees respects the relation on the number as well.
\end{comment}

\section{Heterogenization}
The situation in which one wants to collect a variety of types is not uncommon, and is typically handled by tuples. However, if e.g., you are making a game in Haskell, you might feel the need to maintain a list of ``Drawables'', which may be of different types. Such a list would have to be a kind of ``heterogeneous list''. In Haskell, this can be resolved by using an existentially quantified list, which, informally speaking, can contain any type implementing a given constraint, but can only be inspected as if it contains the intersection of all types implementing this constraint. 

This ports directly to Agda, but becomes cumbersome quickly, and impractical if we want to be able to inspect the elements. The alternative is to split our heterogeneous list into two parts; one tracking the types, and one tracking the values. In practice, this means that we implement a heterogeneous list as a list of values indexed over a list of types. This approach and mainly its specialization to binary trees is investigated by Swierstra \cite{hetbin}.

We will demonstrate that we can express this ``lift a type over itself'' operation as an ornament. For this, we make a small adjustment to \AgdaDatatype{RDesc} to track a type parameter separately from the fields. Using this we define an ornament-computing function, which given a description computes an ornamental description on top of it:
\ExecuteMetaData[Tex/Heterogenize]{HetO}
This ornament relates the original unindexed type to a type indexed over it; we see that this ornament largely keeps all fields and structure identical, only performing the necessary bookkeeping in the index, and adding extra fields before parameters.

As an example, we adapt the list description
\ExecuteMetaData[Tex/Heterogenize]{List}
which is easily heterogenized to an \AgdaDatatype{HList}. In fact, \AgdaFunction{HetO} seems to act functorially; if we lift \AgdaDatatype{Maybe} like
\ExecuteMetaData[Tex/Heterogenize]{HMaybe}
then we can lift functions like \AgdaFunction{head} as
\ExecuteMetaData[Tex/Heterogenize]{hhead}



\section{More equivalences for less effort}\label{sec:userfriendly}
Noting that constructing equivalences directly or from isomorphisms as in \autoref{ssec:leibniz} can quickly become challenging when one of the sides is complicated, we work out a different approach making use of the initial semantics of W-types instead. We claim that the functions in the isomorphism of \autoref{ssec:leibniz} were partially forced, but this fact was unused there.

First, we explain that if we assume that one of the two sides of the equivalence is a fixpoint or initial algebra of a polynomial functor (that is, the \AgdaDatatype{μ} of a \AgdaDatatype{Desc′}), this simplifies giving an equivalence to showing that the other side is also initial.

We describe how we altered the original ornaments \cite{progorn} to ensure that \AgdaDatatype{μ} remains initial for its base functor in Cubical Agda, explaining why this fails otherwise, and how defining base functors as datatypes avoids this issue.

In a subsection focussing on the categorical point of view, we show how we can describe initial algebras (and truncate the appropriate parts) in such a way that the construction both applies to general types (rather than only sets), and still produces an equivalence at the end. We explain how this definition, like the usual definition, makes sure that a pair of initial objects always induces a pair of conversion functions, which automatically become inverses. Finally, we explain that we can escape our earlier truncation by appealing to the fact that ``being an equivalence'' is a proposition.

Next, we describe some theory, using which other types can be shown to be initial for a given algebra. This is compared to the construction in \autoref{ssec:leibniz}, observing that intuitively, initiality follows because the interpretation of the zero constructor is forced by the square defining algebra maps, and the other values are forced by repeatedly applying similar squares. This is clarified as an instance of recursion over a polynomial functor.

To characterize when this recursion is allowed, we define accessibility with respect to polynomial functors as a mutually recursive datatype as follows. This datatype is constructed using the fibers of the algebra map, defining accessibility of elements of these fibers by cases over the description of the algebra. Then we remark that this construction is an atypical instance of well-founded recursion, and define a type as well-founded for an algebra when all its elements are accessible.

We interpret well-foundedness as an upper bound on the size of a type, leading us to claim that injectivity of the algebra map gives a lower bound, which is sufficient to induce the isomorphism. We sketch the proof of the theorem, relating part of this construction to similar concepts in the formalization of well-founded recursion in the Standard Library. In particular, we prove an irrelevance and an unfolding lemma, which lets us show that the map into any other algebra induced by recursion is indeed an algebra map. By showing that it is also unique, we conclude initiality, and get the isomorphism as a corollary. 

The theorem is applied and demonstrated to the example of binary naturals. We remark that the construction of well-foundedness looks similar to view-patterns. After this, we conclude that this example takes more lines that the direct derivation in \autoref{ssec:leibniz}, but we argue that most of this code can likely be automated.

\towrite{Merge}

% REPLACE X BY A?
The setup some approaches in earlier sections require makes them tedious or impractical to apply. In this section we will look at some ways how part of this problem could be alleviated through generics, or by alternative descriptions of concepts like equivalences through the lens of initial algebras. 

In later sections we will construct many more equivalences between more complicated types than before, so we will dive right into the latter. Reflecting upon \autoref{sec:leibniz}, we see that when one establishes an equivalence, most of the time is spent working out a series of tedious lemmas to show that the conversion functions are mutual inverses, which tend to be relatively easy to define. We take away two things from this; the first is that the conversion functions are perhaps too obvious, and the second is that we should really avoid talking about sections and retractions lest we incur tedium!\footnote{The latter perhaps less so, because it is useful to show a map to be monic.} We will reuse the machinery from \cite{progorn} to illustrate how the definitions in \autoref{sec:leibniz} were actually forced for a large part.

First, we remark that \AgdaDatatype{μ} is internalization of the representation of simple\footnote{Of course, indexed datatypes are indexed W-types, mutually recursive datatypes are represented yet differently\dots} datatypes as W-types. Thus, we will assume that one of the sides of the equivalence is always represented as an initial algebra of a polynomial functor, and hence the \AgdaDatatype{μ} of a \AgdaDatatype{Desc′}.

\subsection{Well-founded monic algebras are initial}\label{ssec:wellfounded}
Unfortunately, the machinery from \cite{progorn} relies on axiom K for a small but crucial part. To be precise, in a cubical setting, the type \AgdaDatatype{μ} as given stops being initial for its base functor! In this section, we will be working with a simplified and repaired version. Namely, we simplify \AgdaDatatype{Desc′} to 
\ExecuteMetaData[Extra/ProgOrn/Desc]{DescS}
To complete the definition of \AgdaDatatype{μ}
\ExecuteMetaData[Extra/ProgOrn/Desc]{mu}
we will need to implement \AgdaDatatype{Base}. We remark that in \cite{progorn}, the recursion of \AgdaFunction{mapFold} is a structural descent in \AgdaFunction{⟦ D' ⟧ (μ D)}. Because \AgdaFunction{⟦\_⟧} is a type computing function and not a datatype, this descent becomes invalid\footnote{Refer to the  \href{https://agda.readthedocs.io/en/latest/language/without-k.html\#restrictions-on-termination-checking}{without K} page.}, and \AgdaFunction{mapFold} fails the termination check. We resolve this by defining \AgdaDatatype{Base} as a datatype
\ExecuteMetaData[Extra/ProgOrn/Desc]{Base}
such that this descent is allowed by the termination checker without axiom K.\footnote{This has, again by the absence of axiom K, the consequence of pushing the universe levels up by one. However, this is not too troublesome, as equivalences can go between two levels, and indeed types are equivalent to their lifts.}

Recall that the \AgdaDatatype{Base} functors of descriptions are special polynomial functors, and the fixpoint of a base functor is its initial algebra. The situation so far is summarized by the diagram
% https://q.uiver.app/?q=WzAsMyxbMSwwLCJGXFxtdV9GIl0sWzEsMSwiXFxtdSBGIl0sWzAsMSwiWCJdLFswLDEsIlxcbWF0aHJte2Nvbn0iXSxbMiwxLCJlIiwyLHsic3R5bGUiOnsidGFpbCI6eyJuYW1lIjoiYXJyb3doZWFkIn0sImJvZHkiOnsibmFtZSI6ImRhc2hlZCJ9fX1dXQ==
\[\begin{tikzcd}[ampersand replacement=\&]
	\& {F\mu_F} \\
	X \& {\mu F}
	\arrow["{\mathrm{con}}", from=1-2, to=2-2]
	\arrow["e"', dashed, tail reversed, from=2-1, to=2-2]
\end{tikzcd}\]
so, we are looking for sufficient conditions on $X$ to get the equivalence $e: X \cong \mu F$. Note that when $X \cong \mu F$, then there necessarily is an initial algebra $F X \to X$. Conversely, if the algebra $(X, f)$ is isomorphic to $(\mu F, \mathrm{con})$, then $X \cong \mu F$ would follow immediately, so it is equivalent to ask for the algebras to be isomorphic instead.

\subsubsection{Datatypes as initial algebras}
To characterize when such algebras are isomorphic, we reiterate some basic category theory, simultaneously rephrasing it in Agda terms.\footnote{We are not reusing a pre-existing category theory library for the simple reasons that it is not that much work to write out the machinery explicitly, and that such libraries tend to phrase initial objects in the correct way, which is too restrictive for us.}\todo{Maybe category theory reference}

Let $C$ be a category, and let $a, b, c$ be objects of $C$, so that in particular we have identity arrows $1_a : a \to a$ and for arrows $g : b \to c, f : a \to b$ composite arrows $gf : a \to c$ subject to associativity. In our case, $C$ is the category of types, with ordinary functions as arrows.

Recall that an endofunctor, which is simply a functor $F$ from $C$ to itself, assigns objects to objects and sends arrows to arrows
\ExecuteMetaData[Extra/Category]{RawFunctor}
These assignments are subject to the identity and composition laws
\ExecuteMetaData[Extra/Category]{Functor}
An $F$-algebra is just a pair of an object $a$ and an arrow $Fa \to a$
\ExecuteMetaData[Extra/Category]{Algebra}
Algebras themselves again form a category $C^F$. The arrows of $C^F$ are the arrows $f$ of $C$ such that the following square commutes% https://q.uiver.app/?q=WzAsNCxbMCwwLCJGYSJdLFsxLDAsIkZiIl0sWzAsMSwiYSJdLFsxLDEsImIiXSxbMiwzLCJmIiwyXSxbMCwyLCJVX2EiLDJdLFsxLDMsIlVfYiJdLFswLDEsIkZmIl1d
\[\begin{tikzcd}[ampersand replacement=\&]
	Fa \& Fb \\
	a \& b
	\arrow["f"', from=2-1, to=2-2]
	\arrow["{U_a}"', from=1-1, to=2-1]
	\arrow["{U_b}", from=1-2, to=2-2]
	\arrow["Ff", from=1-1, to=1-2]
\end{tikzcd}\]
So we define
\ExecuteMetaData[Extra/Category]{AlgSqr}
and
\ExecuteMetaData[Extra/Category]{AlgMap}
Note that we take the propositional truncation of the square, such that algebra maps with the same underlying morphism become propositionally equal
\ExecuteMetaData[Extra/Category]{AlgPath}
The identity and composition in $C^F$ arise directly from those of the underlying arrows in $C$.

Recall that an object $\emptyset$ is initial when for each other object $a$, there is an unique arrow $!: \emptyset \to a$. By reversing the proofs of initiality of \AgdaDatatype{μ} and the main result of this section, we obtain a slight variation upon the usual definition. Namely, unicity is often expressed as contractability of a type
\ExecuteMetaData[Tex/Snippets]{isContr}
Instead, we again use a truncation
\ExecuteMetaData[Extra/Category]{weakContr}
but note that this also, crucially, slightly stronger than connectedness. We define initiality for arbitrary relations
\ExecuteMetaData[Extra/Category]{Initial}
such that it closely resembles the definition of least element. Then, $A$ is an initial algebra when
\ExecuteMetaData[Extra/Category]{InitAlg}

By basic category theory (using the usual definition of initial objects), two initial objects $a$ and $b$ are always isomorphic;
namely, initiality guarantees that there are arrows $f : a \to b$ and $g : b \to a$, which by initiality must compose to the identities again.

Similarly, we get that
\ExecuteMetaData[Extra/Category]{InitAlg-equiv}
However, we only have the equalities from the isomorphism inside a propositional truncation. But fortunately, being an equivalence is a property, so we can eliminate from the truncations to get the wanted result.

Note that even though we warned ourselves, we are still talking about sections and retractions to establish that $f$ is an equivalence! However, this result also makes sure we will not have to speak of them again.\footnote{For now...}

\subsubsection{Accessibility}
As a consequence, we get that $X$ is isomorphic to $\mu D$ when $X$ is an initial algebra for the base functor of $D$; $\mu D$ is initial by its fold, and by induction on $\mu D$ using the squares of algebra maps. 

\begin{remark}
    We need (in general) not hope $\mu D$ is a strict initial object in the category of algebras. For a strict initial object, having a map $a \to \emptyset$ implies $a \cong \emptyset$. This is not the case here: strict initial objects satisfy $a \times \emptyset \cong \emptyset$, but for the $X \mapsto 1 + X$-algebras $\mathbb{N}$ and $2^\mathbb{N}$ clearly $2^\mathbb{N} \times \mathbb{N} \cong \mathbb{N}$ does not hold. On the other hand, the ``obvious'' sufficient condition to let $C^F$ have strict initial objects is that $F$ is a left adjoint, but then the carrier of the initial algebra is simply $\bot$.
\end{remark}

Looking back at \autoref{sec:leibniz}, we see that \AgdaDatatype{Leibniz} is an initial $F: X \mapsto 1 + X$ algebra because for any other algebra, the image of \AgdaFunction{0b} is fixed, and by \AgdaFunction{bsuc} all other values are determined by chasing around the square. Thus, we are looking for a similar structure on $f : FX \to X$ that supports recursion.

Clearly we will need something stronger than $FX \cong X$, as in general a functor can have many fixpoints. For this, we define what it means for an element $x$ to be accessible by $f$. This definition uses a mutually recursive datatype as follows:
We state that an element $x$ of $X$ is accessible when there is an accessible $y$ in its fiber over $f$
\ExecuteMetaData[Extra/Category/Poly]{Acc}
Accessibility of an element $x$ of \AgdaFunction{Base A E} is defined by cases on $E$; if $E$ is \AgdaFunction{ṿ n} and $x$ is a \AgdaFunction{Vec A n}, then $x$ is accessible if all its elements are; if $x$ is \AgdaFunction{σ S E'}, then $x$ is accessible if \AgdaFunction{snd x} is
\ExecuteMetaData[Extra/Category/Poly]{Acc'}
Consequently, $X$ is well-founded for an algebra when all its elements are accessible
\ExecuteMetaData[Extra/Category/Poly]{Wf}

We can see well-foundedness as an upper bound on the size of $X$, if it were larger than $\mu D$, some of its elements would inevitably get out of reach of an algebra. \textit{Now} having $FX \cong X$ also gives us a lower bound, but remark that having a well-founded injection $f: FX \to X$ is already sufficient, as accessibility gives a section of $f$, making it an iso. This leads us to claim
\begin{claim}\label{claim:wf-inj-init}
    If there is a mono $f : FX \to X$ and $X$ is well-founded for $f$, then $X$ is an initial $F$-algebra.
\end{claim}

\subsubsection{Proof sketch of \autoref{claim:wf-inj-init}}
Let us be on our way. Suppose $X$ is well-founded for the mono $f : FX \to X$. To show that $(X, f)$ is initial, let us take another algebra $(Y, g)$, and show that there is a unique arrow $(X, f) \to (Y, g)$.\todo{This section is about as digestable as a brick.}

By \AgdaDatatype{Acc}-recursion and because all $x$ are accessible, we can define a plain map into $Y$
\ExecuteMetaData[Extra/Category/WellFounded]{Wf-rec}
This construction is an instance of the concept of ``well-founded recursion''\footnote{This is formalized in the \href{https://agda.github.io/agda-stdlib/Induction.WellFounded.html}{standard-library} with many other examples.}, so we let ourselves be inspired by these methods. In particular, we prove an irrelevance lemma
\ExecuteMetaData[Extra/Category/WellFounded]{Wf-rec-irr}
which implies the unfolding lemma
\ExecuteMetaData[Extra/Category/WellFounded]{Wf-rec-unfold}
The unfolding lemma ensures that the map we defined by \AgdaFunction{Wf-rec} is a map of algebras. The proof that this map is unique proceeds analogously to that in the proof that $\mu D$ is initial, but here we instead use \AgdaDatatype{Acc}-recursion
\ExecuteMetaData[Extra/Category/WellFounded]{Wf+inj=Init}
Thus, we conclude that $X$ is initial. The main result is then a corollary of initiality of $X$ and the isomorphism of initial objects
\ExecuteMetaData[Extra/Category/WellFounded]{Wf+inj=mu}


\subsubsection{Example}
Let us redo the proof in \autoref{sec:leibniz}, now using this result. 

\end{document}
