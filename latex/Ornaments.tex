\section{Ornamental descriptions}
The ornamental descriptions for \AD{DescI} take the same shape as those in \autoref{sec:background-ornamental-descriptions}, generalized to handle nested types, variable transformations, and composite types. These ornamental descriptions are defined such that a \AD{OrnDesc}\ \AV{Me′ Δ re-par J re-index D} represents a patch from a base description \AV{D} to a description with metadata \AV{Me′}, parameters \AV{Δ} and indices \AV{J}. Note that metadata, as a non-structural property, no direct influence on ornaments, so we simply generalize over the information on \AV{D}, and query the information for the new description without imposing constraints.

%We use \AF{∼} to write down pointwise equality of functions, which in this case all are commutativity squares. 
Ornamental descriptions themselves are again lists of constructor ornaments 
\ExecuteMetaData[Ornament/OrnDesc]{OrnDesc}
The constructor ornaments are also where we pay the price for the flexibility we built into \AD{ConI}. For example, as \AD{ConI} allows us to transform variables, \AD{ConOrnDesc} has to relate the transformations on both sides to guarantee the existence of \AF{ornForget}. A lot of lines are dedicated to the commutativity squares for variables, but these squares involving \AD{Vxf} can generally ignored, as witnessed by the \AF{Oσ+} and \AF{Oσ-} variants of the \AIC{σ} ornament, automatically filling those squares in the usual cases of binding or ignoring fields.

The structure-preserving ornaments are defined as usual
\ExecuteMetaData[Ornament/OrnDesc]{ConOrn-preserve}
where \AIC{ρ} has a new field relating the old and new nesting transforms \AV{g} and \AV{d}. Likewise, \AIC{σ} now has a field relating the old and new variable transforms, which for example prevents us from unbinding a field in the new description which was used in the old description. The ornament \AIC{δ} now represents the direct copying of a \AIC{δ} in descriptions up to \AV{re-par} and \AV{re-var}.

Where only \AIC{Δσ} could add fields before, we can now also add fields described by \AIC{δ} using \AIC{Δδ}
\ExecuteMetaData[Ornament/OrnDesc]{ConOrn-extend}
Again, \AIC{Δσ} now requires the relation of old and new variables.

The last ornament represents an ornament \emph{inside} a \AIC{δ}: If we have a description \AV{D'} = \AIC{δ}\ \AV{R}\ \AV{d}\ \AV{j}\ \AV{R}\ \AV{D} referencing a description \AV{R}, then we may expect that an ornamental description on top of \AV{R} also induces an ornamental description on top of \AV{D'}. We generalize this by defining a kind of orthogonal composition of ornaments\footnote{As opposed to Ko's parallel composition \cite{ko}.}, taking ornamental descriptions \AV{RR′} on \AV{R} and \AV{DD′} on \AV{D}, and producing an ornamental description on \AV{D'}:
\ExecuteMetaData[Ornament/OrnDesc]{ConOrn-compose} 
Roughly speaking, the equality \AV{p₁}, respectively \AV{p₂}, demands that the parameter, respectively index, passed to \AV{R} as computed before and after transforming the outer parameters and variables agree.

%Compared to the previous ornaments, we have the new constructors \AIC{δ}, \AIC{Δδ} and \AIC{δ•}, where the first two are analogues of \AIC{σ} and \AIC{Δσ}. The \AIC{δ•} constructor states that an ornamental description from a description \AV{R} and a (constructor) ornamental description from \AV{CD} can be composed to form an ornamental description from the composition (in the sense of the \AV{δ} type-former) of \AV{CD} with \AV{R}. 
As before we can define \AF{ornForget} by erasing ornaments, now using the new commutativity squares. The precise meaning of ornamental descriptions as descriptions is given by the conversion:
\ExecuteMetaData[Ornament/OrnDesc]{toDesc}
which makes use of the implicit metadata fields in the constructor ornaments to reconstruct the metadata on the target description.

With \AD{OrnDesc} we can reproduce the examples of the ornamental descriptions for \AD{U-ix}, but also present some previously inexpressible types as ornamental descriptions. Using the variants of some ornaments specialized to binding or ignoring fields:
\ExecuteMetaData[Ornament/OrnDesc]{O-sigma-pm}
we can give the familiar ornamental description from \AD{List} to \AD{Vec}:
\ExecuteMetaData[Ornament/OrnDesc]{VecOD}
Rather than defining \AD{Random} in a vacuum, we can use the new flexibility in \AIC{ρ} and describe random access lists as an ornament from binary numbers:
\ExecuteMetaData[Ornament/OrnDesc]{RandomOD}
Likewise, we can give an ornament turning phalanges into digits
\ExecuteMetaData[Ornament/OrnDesc]{DigitOD}
and assemble these into fingertrees with \AIC{δ•}
\ExecuteMetaData[Ornament/OrnDesc]{FingerOD}
