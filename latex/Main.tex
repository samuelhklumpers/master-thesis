\documentclass[10pt, final]{article}

\usepackage[style=alphabetic]{biblatex}
\addbibresource{refs.bib}

\usepackage{comment}

% \setlength{\marginparwidth}{2cm} % this makes \todo{}s explode less, remove when done

\usepackage{todonotes}
\usepackage{xcolor}
\usepackage[toc]{appendix}
\usepackage[hidelinks]{hyperref}


\usepackage{catchfilebetweentags}
\usepackage{quiver} 
\usepackage{tabularx}
\usepackage{amsthm}
\usepackage{amsmath}
\usepackage{listings}


\theoremstyle{plain}
\newtheorem{theorem}{Theorem}[section]
\newtheorem{lemma}[theorem]{Lemma}
\newtheorem{prop}[theorem]{Proposition}
\newtheorem{conjecture}{Conjecture}[section]
\newtheorem*{cor}{Corollary}

\theoremstyle{definition}
\newtheorem{defn}{Definition}[section]
\newtheorem{remark}{Remark}[section]
\newtheorem{claim}{Claim}[section]
\newtheorem{example}{Example}[section]

\renewcommand{\sectionautorefname}{Section}
\renewcommand{\subsectionautorefname}{Section}
\renewcommand{\subsubsectionautorefname}{Section}
\providecommand{\appendixautorefname}{Appendix}

\providecommand{\theoremautorefname}{Theorem}
\providecommand{\lemmaautorefname}{Lemma}
\providecommand{\propautorefname}{Proposition}
\providecommand{\conjectureautorefname}{Conjecture}
\providecommand{\corautorefname}{Corollary}
\providecommand{\defnautorefname}{Definition}
\providecommand{\remarkautorefname}{Remark}
\providecommand{\exampleautorefname}{Example}
\providecommand{\claimautorefname}{Claim}


\usepackage[links]{agda}
\AgdaNoSpaceAroundCode{}

\usepackage{fontspec}
\usepackage{luaotfload}

\directlua{luaotfload.add_fallback
  ("myfallback",
    { "JuliaMono:style=Regular;"
    , "NotoSansMono:style=Regular;"
    , "NotoSansMath:style=Regular;"
    , "Segoe UI Emoji:mode=harf;"
    }
  )}
\defaultfontfeatures{RawFeature={fallback=myfallback}}

\setmainfont{Latin Modern Roman}

\newfontfamily{\AgdaSerifFont}{JuliaMono Regular}[Scale=0.8]
\newfontfamily{\AgdaSansSerifFont}{JuliaMono Regular}[Scale=0.8]
\newfontfamily{\AgdaTypewriterFont}{JuliaMono Regular}[Scale=0.8]
\setmonofont{JuliaMono Regular}[Scale=0.8]
\renewcommand{\AgdaFontStyle}[1]{{\AgdaSansSerifFont{}#1}}
\renewcommand{\AgdaKeywordFontStyle}[1]{{\AgdaSansSerifFont{}#1}}
\renewcommand{\AgdaStringFontStyle}[1]{{\AgdaTypewriterFont{}#1}}
\renewcommand{\AgdaCommentFontStyle}[1]{{\AgdaTypewriterFont{}#1}}
\renewcommand{\AgdaBoundFontStyle}[1]{\textit{\AgdaSerifFont{}#1}}

\newcommand{\AF}[1]{\AgdaFunction{#1}}
\newcommand{\AD}[1]{\AgdaDatatype{#1}}
\newcommand{\AR}[1]{\AgdaRecord{#1}}
\newcommand{\AV}[1]{\AgdaBoundFontStyle{#1}}
\newcommand{\AIC}[1]{\AgdaInductiveConstructor{#1}}
\newcommand{\ARF}[1]{\AgdaField{#1}}

\definecolor{nondescriptyellow}{HTML}{D6B656}

\newcommand{\towrite}[1]{\todo[color=cyan]{#1}}
\newcommand{\marker}[1]{\todo[color=green]{#1}}
\newcommand{\lowprio}[1]{\todo[color=nondescriptyellow]{#1}}

  
\lstnewenvironment{semicomment}{\color{gray}\lstset{breaklines=true}}{}
\lstnewenvironment{outline}{\color{nondescriptyellow}\lstset{breaklines=true}}{}

\newcommand{\investigate}[1]{\par\vspace{1\baselineskip}\textcolor{gray}{#1}\vspace{1\baselineskip}\par}

\newcommand{\bN}{\AgdaDatatype{ℕ}}
\newcommand{\bL}{\AgdaDatatype{Leibniz}}

\title{\large Utrecht University \\ Master Thesis \\ Computing Science \& Mathematical Sciences \\ \Large Generic Numerical Representations as Ornaments}

\author{Samuel Klumpers\\6057314\\ \\ \textbf{Supervisors} \\ dr. Wouter Swierstra \\ dr. Paige North}

\begin{document}
\maketitle

\newpage
\begin{abstract}
The concept of numerical representations as defined by Okasaki \cite{purelyfunctional} explains how certain datastructures resemble number systems, and motivates how number systems can be used as a basis to design datastructures from, which can be made precise by using McBride's ornaments \cite{algorn}. In order to study a broad spectrum of indexed and unindexed numerical representations, we encode a universe allowing the expression of nested datatypes and internalization descriptions of composite types. By equipping the universe with metadata, number systems and numerical representations can be described in the same setup. Adapting ornaments to this universe allows us to define two generic ornament computing functions, generalizing well-known sequences of ornaments, such as naturals-lists-vectors, by computing the indexed and unindexed numerical representations as a sequence of ornaments on a number system.
\end{abstract}


\newpage
\tableofcontents
\listoftodos
\newpage

\section{Introduction}\label{sec:introduction}
\section{Introduction}\label{sec:intro}
Agda \cite{agda} is a functional programming language and a proof assistant, taking inspiration from languages like Haskell and other proof assistants like Coq. We can write programs like we would in Haskell, and then express and prove their properties all inside Agda. This allows us to demonstrate the correctness of programs by formal proof%, which is sound,
rather than by testing%, which is complete
; of course, producing an often tedious proof typically demands more effort than covering the relevant code with testcases.

In this thesis, we will explore some methods of proving properties of our programs, focussing on the problems or inconveniences that may arise, and how to deal with them. Let us sketch some problems and their remedies to get an idea of what awaits us, before we dive into the nitty-gritty details.

First, merely adapting a program to Agda may already require changes to the datatypes used in it; for example, if a program manipulating a \AgdaDatatype{List} uses the unsafe \AgdaFunction{head} function, then one is forced to replace the \AgdaDatatype{List} by a datatype that ensures non-emptyness, such as a \AgdaDatatype{NonEmpty} list or a length-aware vector \AgdaDatatype{Vec}. On the other hand, there might be sections of a program where the concrete length is not relevant for correctness and only gets in the way. As a result, one might find themselves duplicating common functions like concatenation \AgdaFunction{\_++\_} and only altering their signatures.

Clearly we would not be writing this if there was no way out, which there is! Often, the ``new'' datatype (\AgdaDatatype{Vec}) is simply a variation on the old datatype (\AgdaDatatype{List}) making small adjustments to the existing constructors; in this case, we decorate the nil and cons constructors with natural numbers representing the length. This kind of modification of types falls in the framework of ornamentation as described by Ko and Gibbons \cite{progorn}; if two types are reified to their \textit{descriptions}, then \textit{ornaments} express whether the types are ``similar'' by acting as a recipe to produce one type from the other. By restricting the operations to the copying of corresponding parts, and the introduction of fields or dropping of indices, the existence of such an ornament ensures that the types have the same recursive structure.

\towrite{Something about patches.}

\towrite{For each invariant a new datatype? Still ornaments}

Now that we know we can decently satisfyingly organize similar datatypes, it is time to look at dissimilar datatypes. It is certainly not foolish to prototype a program using simpler types or implementations, and only replace these with more performant alternatives in critical places; knowing that this is eventually going to happen, one might as well prepare for it. While this may quickly turn into a refactoring nightmare in the general case, we can hope for a more satisfying transition if we restrict our attention to a more narrow scope. As an example, we might start programming something with a \AgdaDatatype{List}, but replace this with a \AgdaDatatype{Tree} if we notice that the program spends most of its time in \AgdaFunction{lookup} operations. We are of course reimplementing the operations on \AgdaDatatype{Tree} utilizing their binary nature to gain a speedup, but it also seems as though we are about to double the number of necessary proofs; however, we have two ways to avoid this problem. 

We will look at the more specific solution first. This solution is guided by the realization that \AgdaDatatype{List} and \AgdaDatatype{Tree}, like most other containers, still have something in common even if their recursive structure is very different. That is, both resemble a number system, and, Okasaki \cite{purelyfunctional} notes that this resemblance to number systems is ``surprisingly common''. In the case of lists and Braun trees\footnote{Braun trees are a kind of binary tree, of which its shape is determined by its size.}, one can present both by deriving them from unary and binary numbers respectively, as is made formal by Hinze and Swierstra in \cite{calcdata}. One can then apply this \textit{numerical representation} to simplify or make trivial the proofs of the properties we hesitated to duplicate before.

\towrite{If we instead hide our datatypes behind interfaces, we can use proof transport as an alternative.}


\section{\toremove{Introduction (old)}}
%The dependently typed functional programming language Agda \cite{agda} can, when restricted to its reasonable parts, be translated into readable and safe Haskell \cite{agda2hs}. However, the intrinsic safety of languages like Agda can also lead to code duplication by encouraging the use of multiple variants of the same datatype. As an example, the coverage check forces the \AgdaFunction{head} function on \AgdaDatatype{List} to return a \AgdaDatatype{Maybe}. This \AgdaDatatype{Maybe} can be avoided by moving to the length-indexed list type \AgdaDatatype{Vec}, at the cost of duplicating functions like \AgdaFunction{\_++\_}, which we need at both types.

Something similar happens when replacing an implementation with a more efficient one. For example, when implementing binary trees as a more efficient alternative to lists, the proofs of the same properties will differ between list and tree, and tend to be more difficult for the latter. Switching between implementations of an interface not only duplicates code, but also (and sometimes more than) doubles the effort required to verify both.

%There is plenty of prior work dealing with problems like these. The work in \cite{orntrans} and \cite{progorn} provides the means to relate similar datatypes, such as lists and vectors, using the mechanism of ornamentation, letting us organize variants of the same datatype in a rigid framework.  %This leads them to define the concept of patches, which can aid us when defining \AgdaFunction{\_++\_} for the second time by forcing the new version to be coherent.
%In fact, the algebraic nature of ornaments yields the definition of the vector type for free, provided we relate lists to natural numbers \cite{algorn}. %Such constructions rely heavily on descriptions of datastructures and often come with limitations in their expressiveness. These descriptions in turn impose additional ballast on the programmer, leading us to investigate reflection like in \cite{practgen} as a means to bring datatypes and descriptions closer when possible.

Other work like \cite{calcdata} simplifies the proofs relating to certain containers directly, formally executing the way of though of numerical representations as noted in \cite{purelyfunctional}.
%From another point of view, lists and trees are not so different at all, provided we look at them through the interface of one-sided flexible arrays; this idea noted in \cite{purelyfunctional} and formalized in \cite{calcdata} where both are shown to be instances of numerical representations by calculating them from a numeral system. 

When two types are isomorphic and equivalent under an interface, proofs of properties of these implementations should be interconvertible. By using structured equivalences and univalence, \cite{iri} characterizes equivalences under interfaces.
%While this is achievable through meta-programming, substituting conversions to and from into the proof terms, this is internally expressible in Cubical Agda.

%We can liken the situation to movement on a plane, where ornamentation moves us vertically by modifying constructors or indices, and structured equivalences move us horizontally to and from equivalent but more equivalent implementations. In this paper, we will investigate a variety of means of moving around structures and proofs, and ways to make this more efficient or less intrusive.

In \autoref{sec:leibniz}, we will follow \cite{iri}, and look at how proofs on unary naturals can be transported to the binary naturals. Then in \autoref{sec:numrep} we recall how numeral systems in particular induce container types in \cite{calcdata}, which we attempt to reformulate in the language of ornaments in \autoref{ssec:ornaments}, using the framework of \cite{progorn}. In \autoref{sec:userfriendly} we investigate how we can make the earlier methods more easily accessible to the user, and, ourselves, when we give a description of finger trees in \autoref{sec:fingertrees}.




\section{Background}\label{part:background}
Many of our constructions extend upon or are inspired by existing work in the domain of generic programming and ornaments, so let us take a closer look at the nuts and bolts to see what all the concepts are about.

This section describes some common datatypes and their usages, exploring how dependent types let us prove properties of programs, or write programs that are correct-by-construction. We then discuss certain proofs or programs can be generalized to classes of types by encoding datatypes using descriptions. Finally, we take a look at ornaments as a means to relate datatypes by their structure, or construct more datatypes of a given structure, but also as a way to identify comparable programs on structurally similar datatypes.

\section{Agda}\label{sec:background-agda}
We formalize our work in the programming language Agda \cite{agda}. While we will only occasionally reference Haskell, those more familiar with Haskell might understand (the reasonable part of) Agda as the subset of total Haskell programs \cite{agda2hs}.

Agda is a total functional programming language with dependent types. Here, totality means that functions of a given type always terminate in a value of that type, ruling out non-terminating (and not obviously terminating) programs. Using dependent types we can use Agda as a proof assistant, allowing us to state and prove theorems about our datastructures and programs. 

In this section, we will explain and highlight some parts of Agda which we use in the later sections. 
Many of the types we use in this section are also described and explained in most Agda tutorials (\cite{ulftutorial}, \cite{plfa}, etc.), and can be imported from the standard library \cite{agdastdlib}.

%Note that we use \texttt{--type-in-type} to keep the explanations more readable. 
%and \texttt{--with-K}
% Even though the former makes Agda inconsistent, and the latter is not strictly necessary, we know that our work can be ported to a setting with neither option \autoref{app:withoutK}.


\section{Data in Agda}\label{sec:background-data}
At the level of generalized algebraic datatypes Agda is close to Haskell. In both languages, one can define objects using data declarations, and interact with them using function declarations. For example, we can define the type of \emph{booleans}:
\ExecuteMetaData[Tex/Background]{Bool}
The constructors of this type state that we can make values of \AD{Bool} in exactly two ways: \AIC{false} and \AIC{true}. We can then define functions on \AD{Bool} by pattern matching. As an example, we can define the conditional operator as
\ExecuteMetaData[Tex/Background]{conditional}
When \emph{pattern matching}, the coverage checker ensures we define the function on all cases of the type matched on, and thus the function is completely defined. % mark: shuffle

We can also define a type representing the natural numbers
\ExecuteMetaData[Tex/Background]{Nat}
Here, \bN{} always has a \AF{zero} element, and for each element $n$ the constructor \AIC{suc} expresses that there is also an element representing $n + 1$. Hence, \bN{} represents the \textit{naturals} by encoding the existential axioms of the Peano axioms. By pattern matching and recursion on \bN{}, we define the less-than operator:
\ExecuteMetaData[Tex/Background]{lt}
One of the cases contains a recursive instance of \bN{}, so termination checker also verifies that this recursion indeed terminates, ensuring that we still define \AV{n}\ \AF{<?} \AV{m} for all possible combinations of \AV{n} and \AV{m}. %Essentially, the coverage and termination checker make sure that any valid definition by pattern matching corresponds to a valid proof by cases and induction.
In this case the recursion is valid, since both arguments decrease before the recursive call, meaning that at some point \AV{n} or \AV{m} hits \AIC{zero} and the recursion terminates.

Like in Haskell, we can \emph{parametrize} a datatype over other types to make \emph{polymorphic} type, which we can use to define lists of values for all types:
\ExecuteMetaData[Tex/Background]{List} 
A list of \AV{A} can either be empty \AIC{[]}, or contain an element of \AV{A} and another list via \AIC{\_∷\_}. In other words, \AD{List} is a type of \emph{finite sequences} in \AV{A} (in the sense of sequences as an abstract type \cite{purelyfunctional}).

Using polymorphic functions, we can manipulate and inspect lists by inserting or extracting elements. For example, we can define a function to look up the value at some position \AV{n} in a list
\ExecuteMetaData[Tex/Background]{lookup-list}
However, this function \emph{partial}, as we are relying on the type
\ExecuteMetaData[Tex/Background]{Maybe}
to handle the case where the position falls outside the list and we cannot return an element. 
If we know the length of the list \AV{xs}, then we also know for which positions \AF{lookup} will succeed, and for which it will not. We define 
\ExecuteMetaData[Tex/Background]{length}
so that we can test whether the position \AV{n} lies inside the list by checking \AV{n}\ \AF{<?}\ \AF{length}\ \AV{xs}. If we declare \AF{lookup} as a dependent function consuming a proof of \AV{n}\ \AF{<?}\ \AF{length}\ \AV{xs}, then \AF{lookup} always succeeds. However, this actually only moves the burden of checking whether the output was \AIC{nothing} afterwards to proving that \AV{n}\ \AF{<?}\ \AF{length}\ \AV{xs} beforehand.

We can avoid both by defining an \emph{indexed type} representing numbers below an upper bound
\ExecuteMetaData[Tex/Background]{Fin}
Like parameters, indices add a variable to the context of a datatype, but unlike parameters, indices can influence the availability of constructors. The type \AD{Fin} is defined such that a variable of type \AD{Fin}\ \AV{n} represents a number less than \AV{n}. Since both constructors \AIC{zero} and \AIC{suc} dictate that the index is the \AIC{suc} of some natural \AV{n}, we see that \AD{Fin}\ \AIC{zero} has no values. On the other hand, \AIC{suc} gives a value of \AD{Fin}\ (\AIC{suc}\ \AV{n}) for each value of \AD{Fin}\ \AV{n}, and \AIC{zero} gives exactly one additional value of \AD{Fin}\ (\AIC{suc}\ \AV{n}) for each \AV{n}. By induction (externally), we find that \AD{Fin}\ \AV{n} has exactly \AV{n} closed terms, each representing a number less than \AV{n}.

To complement \AD{Fin}, we define another indexed type representing lists of a known length, also known as vectors:
\ExecuteMetaData[Tex/Background]{Vec}
The \AIC{[]} constructor of this type produces the only term of type \AD{Vec}\ \AV{A}\ \AIC{zero}. The \AIC{\_∷\_} constructor ensures that a \AD{Vec}\ \AV{A}\ (\AIC{suc}\ \AV{n}) always consists of an element of \AV{A} and a \AD{Vec}\ \AV{A}\ \AV{n}. By induction, we find that a \AD{Vec}\ \AV{A}\ \AV{n} contains exactly \AV{n} elements of \AV{A}. Thus, we conclude that \AD{Fin}\ \AV{n} is exactly the type of positions in a \AD{Vec}\ \AV{A}\ \AV{n}. In comparison to \AD{List}, we can say that \AD{Vec} is a type of arrays (in the sense of arrays as the abstract type of sequences of a fixed length). Furthermore, knowing the index of a term \AV{xs} of type \AV{Vec}\ \AV{A}\ \AV{n} uniquely determines the the constructor it was formed by. Namely, if \AV{n} is \AIC{zero}, then \AV{xs} is \AIC{[]}, and if \AV{n} is \AIC{suc} of \AV{m}, then \AV{xs} is formed by \AIC{\_∷\_}. 

Using this, we define a variant of \AF{lookup} for \AD{Fin} and \AD{Vec}, taking a vector of length \AV{n} and a position below \AV{n}:
\ExecuteMetaData[Tex/Background]{lookup}
The case in which we would return \AIC{nothing} for lists, which is when \AV{xs} is \AIC{[]}, is omitted. This happens because \AV{x} of type \AD{Fin}\ \AV{n} is either \AIC{zero} or \AIC{suc}\ \AV{i}, and both cases imply that \AV{n} is \AIC{suc}\ \AV{m} for some \AV{m}. As we saw above, a \AD{Vec}\ \AV{A}\ (\AIC{suc} \AV{m}) is always formed by \AIC{\_∷\_}, making the case in which \AV{xs} is \AIC{[]} impossible. Consequently, lookup always succeeds for vectors,
% demonstrating that vectors are correct-by-construction. 
however, this does not yet prove that \AF{lookup} necessarily returns the right element, we will need some more logic to verify this.

\section{Proving in Agda}\label{sec:background-proving}
To describe equality of terms we define a new type
\ExecuteMetaData[Tex/Background]{equiv}
If we have a value \AV{x} of \AV{a}\ \AD{≡}\ \AV{b}, then, as the only constructor of \AD{\_≡\_} is \AIC{refl}, we must have that \AV{a} is equal to \AV{b}. We can use this type to describe the behaviour of functions like \AF{lookup}: If we insert elements into a vector with
\ExecuteMetaData[Tex/Background]{insert}
we can express the correctness of \AF{lookup} as
\ExecuteMetaData[Tex/Background]{lookup-insert-type}
stating that we expect to find an element where we insert it.

% When we use pattern matching in a function, the coverage and termination checker ensure that the resulting function is total and defined by well-founded recursion\cite{?}. If we are proving some statement by constructing a function as a proof, this means that we can interpret a function definition by (dependent) pattern matching and well-founded recursion as a proof by well-founded induction\cite{?}.

%So, to 
To prove the statement, we proceed as when defining any other function. 
By simultaneous induction on the position and vector, we prove
\ExecuteMetaData[Tex/Background]{lookup-insert}
In the first two cases, where we \AF{lookup} the first position, \AF{insert}\ \AV{xs}\ \AIC{zero}\ \AV{y} simplifies to \AV{y}\ \AF{∷}\ \AF{xs}, so the lookup immediately returns \AV{y} as wanted. In the last case, we have to prove that \AF{lookup} is correct for \AV{x}\ \AF{∷}\ \AF{xs}, so we use that the \AF{lookup} ignores the term \AV{x} and we appeal to the correctness of \AF{lookup} on the smaller list \AV{xs} to complete the proof.

Like \AD{\_≡\_}, we can encode many other logical operations into datatypes, which establishes a correspondence between types and formulas, known as the Curry-Howard isomorphism. For example, we can encode disjunctions (the logical `or' operation) as
\ExecuteMetaData[Tex/Background]{uplus}

The other components of the isomorphism are as follows. Conjunction (logical `and') can be represented by\footnote{We use a record here, rather than a datatype with a constructor \AV{A → B →}\ \AV{A}\ \AD{×}\ \AV{B}. The advantage of using a record is that this directly gives us projections like \ARF{fst}\ \AV{:}\ \AV{A}\ \AD{×}\ \AV{B}\ \AV{→ A}, and lets us use eta equality, making $(a, b) = (c , d) \iff a = c \land b = d$ holds automatically.}
\ExecuteMetaData[Tex/Background]{product}
True and false are respectively represented by
\ExecuteMetaData[Tex/Background]{true}
so that always \AIC{tt}\ \AV{:}\ \AD{⊤}, and 
\ExecuteMetaData[Tex/Background]{false}
The body of \AD{⊥} is not accidentally left out: because \AD{⊥} has no constructors, there is no proof of false\footnote{If we did not use \AV{--type-in-type}, and even in that case I can only hope.}.

Because we identify function types with logical implications, we can also define the negation of a formula \AV{A} as ``\AV{A} implies false'':
\ExecuteMetaData[Tex/Background]{not}
The logical quantifiers $\forall$ and $\exists$ act on formulas with a free variable in a specific domain of discourse. We represent closed formulas by types, so we can represent a formula with a free variable of type \AV{A} by a function values of \AV{A} to types \AV{A}\ \AV{→}\ \AD{Type}, also known as a predicate. The universal quantifier $\forall a P(a)$ is true when for all $a$ the formula $P(a)$ is true, so we represent the universal quantification of a predicate \AV{P} as a dependent function type \AV{(a : A) → P a}, producing for each \AV{a} of type \AV{A} a proof of \AV{P}\ \AV{a}. The existential quantifier $\exists a P(a)$ is true when there is some $a$ such that $P(a)$ is true, so we represent the existential quantification as
\ExecuteMetaData[Tex/Background]{exists}
so that we have \AD{Σ}\ \AV{A}\ \AV{P} iff we have an element \AV{fst} of \AV{A} and a proof \AV{snd} of \AV{P}\ \AV{a}. To avoid the need for lambda abstractions in existentials, we define the syntax
\ExecuteMetaData[Tex/Background]{sigma-syntax}
letting us write \AD{Σ[}\ \AV{a}\ \AD{∈}\ \AV{A}\ \AD{]}\ \AV{P a} for $\exists a P(a)$.

\section{Descriptions}\label{sec:background-descriptions}
In the previous sections we completed a quadruple of types (\bN{}, \AD{List}, \AD{Vec}, \AD{Fin}), 
%, even computing the latter two from \bN{}.
which have nice interactions (\AF{length}, \AF{lookup}). Similar to the type of \AF{length}\ \AV{:}\ \AD{List}\ \AV{A}\ \AV{→}\ \bN{}, we can define
\ExecuteMetaData[Tex/Background]{toList}
converting vectors back to lists. In the other direction, we can also promote a list to a vector by recomputing its index:
\ExecuteMetaData[Tex/Background]{toVec}
We claim that is not a coincidence, but rather happens because \bN{}, \AD{List}, and \AD{Vec} have the same ``shape''.

But what is the shape of a datatype? In this section, we will explain a framework of datatype descriptions and ornaments, allowing us to describe the shapes of datatypes and use these for generic programming \cite{ulftutorial, genericsamm, effectfully, practgen}. Recall that while polymorphism allows us to write one program for many types at once, those programs act parametrically \cite{reynolds1983types, wadlerfree}: polymorphic functions must work for all types, thus they cannot inspect values of their type argument. Generic programs, by design, do use the structure of a datatype, allowing for more complex functions that do inspect values\footnote{Think of JSON encoding types with encodable fields \cite{truesop}, or deriving functor instances for a broad class of types \cite{haskellderiving}.}.

Using datatype descriptions we can then relate \bN{}, \AD{List} and \AD{Vec}, explaining how \AF{length} and \AF{toList} are instances of a generic construction. Let us walk through some ways of defining descriptions. We will start from simpler descriptions, building our way up to more general types, until we reach a framework in which we can describe \bN{}, \AD{List}, \AD{Vec} and \AD{Fin}. 
%, which, as a bonus, gives some insight into the meaning of datatypes.


\subsection{Finite types}\label{ssec:background-fin}
A datatype description, which are datatypes of which each value again represents a datatype, consist of two components. Namely, a type of descriptions \AV{U}, also referred to as codes, and an interpretation \AV{U}\ \AV{→}\ \AD{Type}, decoding descriptions to the represented types. In the terminology of Martin-L{\"{o}}f type theory (MLTT)\cite{levitation}, %\todo{No citation for MLTT? Agda is a rather loose extension, none of the original papers really match.}
where types of types like \AD{Type} are called universes, we can think of a type of descriptions as an internal universe.

As a start, we define a basic universe with two codes \AIC{𝟘} and \AIC{𝟙}, respectively representing the types \AD{⊥} and \AD{⊤}, and the requirement that the universe is closed under sums and products:
\ExecuteMetaData[Tex/Background]{U-fin}
The meaning of the codes in this universe is then assigned by the interpretation
\ExecuteMetaData[Tex/Background]{int-fin}
which indeed sends \AIC{𝟘} to \AD{⊥}, \AIC{𝟙} to \AD{⊤}, sums to sums and products to products\footnote{One might recognize that \AF{⟦\_⟧fin} is a morphism between the rings (\AD{U-fin}, \AIC{⊕}, \AIC{⊗}) and (\AD{Type}, \AD{⊎}, \AD{×}). Similarly, \AD{Fin} also gives a ring morphism from \bN{} with \AF{+} and \AF{×} to \AD{Type}, and in fact \AF{⟦\_⟧fin} factors through \AD{Fin} via the map sending the expressions in \AD{U-fin} to their value in \bN{}.}.

In this universe, we can encode the type of booleans simply as 
\ExecuteMetaData[Tex/Background]{BoolD}
The types \AIC{𝟘} and \AIC{𝟙} are finite, and sums and products of finite types are also finite, which is why we call \AD{U-fin} the universe of finite types. Consequently, the type of naturals \bN{} cannot fit in \AD{U-fin}.

\subsection{Recursive types}\label{ssec:background-rec}
To accommodate \bN{}, we need to be able to express recursive types. By adding a code \AIC{ρ} to \AD{U-fin} representing recursive type occurrences, we can express those types: 
\ExecuteMetaData[Tex/Background]{U-rec}
However, the interpretation cannot be defined like in the previous example: when interpreting \AIC{𝟙}\ \AIC{⊕}\ \AIC{ρ}, we need to know that the whole type was \AIC{𝟙}\ \AIC{⊕}\ \AIC{ρ} while processing \AIC{ρ}. As a consequence, we have to split the interpretation in two phases. First, we interpret the descriptions into polynomial functors
\ExecuteMetaData[Tex/Background]{int-rec}
Then, by viewing such a functor as a type with a free type variable, the functor can model a recursive type by setting the variable to the type itself:
\ExecuteMetaData[Tex/Background]{mu-rec}
Recall the definition of \bN{}, which can be read as the declaration that \AD{ℕ} is a fixpoint: \AD{ℕ}\ \AD{≡}\ \AV{F}\ \AD{ℕ} for \AV{F X = ⊤ ⊎ X}. This makes representing \bN{} as simple as:
\ExecuteMetaData[Tex/Background]{NatD}

\subsection{Sums of products}\label{ssec:background-sop}
A downside of \AD{U-rho} is that the definitions of types do not mirror their equivalent definitions in user-written Agda. We can define a similar universe using that polynomials can always be canonically written as sums of products. For this, we split the descriptions into a stage in which we can form sums, on top of a stage where we can form products.
\ExecuteMetaData[Tex/Background]{U-sop}
When doing this, we can also let the left-hand side of a product be any type, allowing us to represent ordinary fields:
\ExecuteMetaData[Tex/Background]{Con-sop}
The interpretation of this universe, while analogous to the one in the previous section, is also split into two parts:
\ExecuteMetaData[Tex/Background]{int-sop}
In this universe, we can define the type of lists as a description quantified over a type:
\ExecuteMetaData[Tex/Background]{ListD-bad}
Using this universe requires us to split functions on descriptions into multiple parts, but makes interconversion between representations and concrete types straightforward.

\subsection{Parametrized types}\label{ssec:background-par}
The encoding of fields in \AD{U-sop} makes the descriptions large in the following sense: by letting \AV{S} in \AIC{σ} be an infinite type, we can get a description referencing infinitely many other descriptions. As a consequence, we cannot inspect an arbitrary description in its entirety. We will introduce parameters in such a way that we recover the finiteness of descriptions as a bonus.

In the last section, we saw that we could define the parametrized type \AD{List} by quantifying over a type. However, in some cases, we will want to be able to inspect or modify the parameters belonging to a type. % mark: why
%footnote{For example, deriving Traversable for parametrized types as functions would not be possible (without macros), as one could not decide whether the signature of a type in a field is compatible.}
To represent the parameters of a type, we will need a new gadget.

In a naive attempt, we can represent the parameters of a type as \AD{List}\ \AD{Type}. However, this cannot represent many useful types, of which the parameters depend on each other. For example, in the existential quantifier \AD{Σ\_}, the type \AV{A}\ \AV{→}\ \AD{Type} of second parameter \AV{B} references back to the first parameter \AV{A}.

In a general parametrized type, parameters can refer to the values of all preceding parameters. The parameters of a type are thus a sequence of types depending on each other, which we call telescopes \cite{practgen, sijsling, telescopes} (also known as contexts in MLTT). We define telescopes using induction-recursion:
\ExecuteMetaData[Tex/Background]{Tel-simple}
A telescope can either be empty, or be formed from a telescope and a type in the context of that telescope. Here, we used the meaning of a telescope \AF{⟦\_⟧tel} to define types in the context of a telescope. This meaning represents the valid assignment of values to parameters:
\ExecuteMetaData[Tex/Background]{int-simple}
interpreting a telescope into the dependent product of all the parameter types.

This definition of telescopes would let us write down the type of \AD{Σ}:
\ExecuteMetaData[Tex/Background]{sigma-tel}
but is not sufficient to define \AD{Σ}, as we need to be able to bind a value \AV{a} of \AV{A} and reference it in the field \AV{P}\ \AV{a}. By quantifying telescopes over a type \cite{practgen}, we can represent bound arguments using almost the same setup:
\ExecuteMetaData[Tex/Background]{Tel-type}
A \AD{Tel}\ \AV{P} then represents a telescope for each value of \AV{P}, which we can view as a telescope in the context of \AV{P}. For readability, we redefine values in the context of a telescope as:
\ExecuteMetaData[Tex/Background]{entails}
so we can define telescopes and their interpretations as:
\ExecuteMetaData[Tex/Background]{Tel-def}
By setting \AV{P}\ \AV{=}\ \AD{⊤}, we recover the previous definition of parameter-telescopes. We can then define an extension of a telescope as a telescope in the context of a parameter telescope:
\ExecuteMetaData[Tex/Background]{ExTel}
representing a telescope of variables over the fixed parameter-telescope \AV{Γ}, which can be extended independently of \AV{Γ}. Extensions can be interpreted by interpreting the variable part given the interpretation of the parameter part:
\ExecuteMetaData[Tex/Background]{int-ExTel}
We will name maps \AV{Δ → Γ} of telescopes \AF{Cxf}\ \AV{Δ}\ \AV{Γ}. Given such a map \AV{g}, name maps \AV{W → V} between extensions \AF{Vxf}\ \AV{g}\ \AV{W}\ \AV{V}:
\ExecuteMetaData[Tex/Background]{tele-helpers} %mark: map-var
We also defined two functions we will use extensively later: \AF{var→par} states that a map of extensions extend to a map of the whole telescope, and \AF{Vxf-▷} lets us extend a map of extensions by acting as the identity on a new variable. 

In the descriptions directly relay the parameter telescope to the constructors, resetting the variable telescope to \AIC{∅} for each constructor:
\ExecuteMetaData[Tex/Background]{U-par}
Of the constructors we only modify the \AIC{σ} to request a type \AV{S} in the context of \AV{V}, and to extend the context for the subsequent fields by \AV{S}:
\ExecuteMetaData[Tex/Background]{Con-par}
Replacing the function \AV{S →}\ \AD{U-sop} by \AD{Con-par}\ (\AV{V}\ \AIC{▷}\ \AV{S}) allows us to bind the value of \AV{S} while avoiding the higher order argument. The interpretation of the universe is then:
\ExecuteMetaData[Tex/Background]{int-par}
In particular, we provide \AV{X} the parameters and variables in the \AIC{σ} case, and extend context by \AV{s} before passing to the rest of the interpretation.

In this universe, we can describe lists using a one-type telescope:
\ExecuteMetaData[Tex/Background]{ListD}
This description declares that \AD{List} has two constructors, one with no fields, corresponding to \AIC{[]}, and the second with one field and a recursive field, representing \AIC{\_∷\_}. In the second constructor, we used pattern lambdas to deconstruct the telescope\footnote{Due to a quirk in the interpretation of telescopes, the \AIC{∅} part always contributes a value \ARF{tt} we explicitly ignore, which also explicitly needs to be provided when passing parameters and variables.} and extract the type \AV{A}.
Using the variable bound in \AIC{σ}, we can also define the existential quantifier:
\ExecuteMetaData[Tex/Background]{SigmaD}
having one constructor with two fields. Here, the first field of type \AV{A} adds a value \AV{a} to the variable telescope, which we recover in the second field by pattern matching, before passing it to \AV{B}.


\subsection{Indexed types}\label{ssec:background-ix}
Lastly, we can integrate indexed types \cite{iir} into the universe by abstracting over indices
\ExecuteMetaData[Tex/Background]{U-ix}
Recall that in native Agda datatypes, a choice of constructor can fix the indices of the recursive fields and the resultant type, so we encode:
\ExecuteMetaData[Tex/Background]{Con-ix}
%In most cases, the index is simply threaded through the interpretation, allowing for a choice in the relevant codes.
If we are constructing a term of some indexed type, then the previous choices of constructors and arguments build up the actual index of this term. This actual index must then match the index we expected in the declaration of this term. This means that in the case of a leaf, we have to replace the unit type with the necessary equality between the expected and actual indices \cite{algorn}:
\ExecuteMetaData[Tex/Background]{int-ix}
In a recursive field, the expected index can be chosen based on parameters and variables. % mark: wording

In this universe, we can define finite types and vectors as:
\ExecuteMetaData[Tex/Background]{FinD}
and
\ExecuteMetaData[Tex/Background]{VecD}
These are equivalent, but since we do not model implicit fields, they are slightly different in use compared to \AD{Fin} and \AD{Vec}. In the first constructor of \AF{VecD} we report an actual index of \AIC{zero}. In the second, we have a field \bN{} to bring the index \AV{n} into scope, which is used to request a recursive field with index \AV{n}, and report the actual index of \AIC{suc}\ \AV{n}. 

Let us also show how the definitions of naturals and lists from earlier sections can be replicated in \AD{U-ix}
\ExecuteMetaData[Tex/Background]{new-Nat-List}
Writing the descriptions \AF{NatD}, \AF{ListD} and \AF{VecD} next to each other makes it easy to see the similarities: \AF{ListD} is the same as \AF{NatD} with a type parameter and one more \AIC{σ}. Likewise, \AF{VecD} is the same as \AF{ListD}, but now indexing over \bN{} and with yet one more \AIC{σ} of \bN{}. This kind of analysis is the focus of \autoref{sec:background-ornaments}.

\subsubsection{Generic Programming}
As a bonus, we can also use \AD{U-ix} for generic programming. For example, by a long construction which can be found in \autoref{app:gfold}, we can define the generic \AF{fold} operation:
\ExecuteMetaData[Tex/Background]{fold-type}
Let us describe how \AF{fold} works intuitively. We can interpret a term of \AF{⟦}\ \AV{D}\ \AF{⟧D}\ \AV{X} as a term of \AF{μ-ix}\ \AV{D}, where the recursive positions hold values of \AV{X} rather than values of \AF{μ-ix}\ \AV{D}. Then \AF{fold} states that a function collapsing such terms into values of \AV{X} extends to a function collapsing \AF{μ-ix}\ \AV{D} into \AV{X}, recursively collapsing applications of \AIC{con} from the bottom up.

As a more concrete example, when instantiating \AF{fold} to \AF{ListD}, the type \AF{⟦}\ \AV{ListD}\ \AF{⟧D}\ \AV{X} reduces (up to equivalence) to \AD{⊤}\ \AD{⊎}\ (\AV{A}\ \AD{×}\ \AV{X}\ \AV{A})\ \AF{→}\ \AV{X}\ \AV{A}, and \AF{fold} becomes
\ExecuteMetaData[Tex/Background]{foldr-type}
which, much like the familiar \AF{foldr} operation lets us consume a \AD{List}\ \AV{A} to produce a value \AV{X A}, provided a value \AV{X A} in the empty case, and a means to convert a pair (\AV{A}, \AV{X A}) to \AV{X A}.

Do note that this version takes a polymorphic function as an argument, as opposed to the usual fold which has the quantifiers on the outside:
\ExecuteMetaData[Tex/Background]{usual-fold}
Like a couple of constructions we will encounter in later sections, we can recover the usual fold into a type \AV{C} by generalizing \AV{C} to some kind of maps into \AV{C}. For example, by letting \AV{X} be continuation-passing computations into \bN{}, we can recover
\ExecuteMetaData[Tex/Background]{foldr-sum}


\section{Ornaments}\label{sec:background-ornaments}
In this section we will introduce a simplified definition of ornaments, which we will use to compare descriptions. Purely looking at their descriptions, \bN{} and \AD{List} are rather similar, except that \AD{List} has a parameter and an extra field \bN{} does not have. We could say that we can form the type of lists by starting from \bN{} and adding this parameter and field, while keeping everything else the same. In the other direction, we see that each list corresponds to a natural by stripping this information. Likewise, the type of vectors is almost identical to \AD{List}, can be formed from it by adding indices, and each vector corresponds to a list by dropping the indices.

Observations like these can be generalized using ornaments \cite{algorn, progorn, sijsling}, which define a binary relation describing which datatypes can be formed by ``decorating'' others. Conceptually, a type can be decorated by adding or modifying fields, extending its parameters, or refining its indices.

Essential to the concept of ornaments is the ability to convert back, forgetting the extra structure. After all, if there is an ornament from \AV{A} to \AV{B}, then \AV{B} is \AV{A} with extra fields and parameters, and more specific indices. In that case, we should also be able to discard those extra fields, parameters, and more specific indices, obtaining a conversion from \AV{B} to \AV{A}. If \AV{A} is a \AD{U-ix}\ \AV{Γ}\ \AV{I} and \AV{B} is a \AD{U-ix}\ \AV{Δ}\ \AV{J}, then a conversion from \AV{B} to \AV{A} presupposes a function \AV{re-par :}\ \AF{Cxf}\ \AV{Δ}\ \AV{Γ} for re-parametrization, and a function \AV{re-index :}\ \AV{J}\ \AV{→}\ \AV{I} for re-indexing.

In the same way that descriptions in \AD{U-ix} are lists of constructor descriptions, ornaments are lists of constructor ornaments. We define the type of ornaments reparametrizing with \AV{re-par} and reindexing with \AV{re-index} as a type indexed over \AD{U-ix}:
\ExecuteMetaData[Tex/Background]{Orn}
The conversion between types induced by an ornament is then embodied by the forgetful map
\ExecuteMetaData[Tex/Background]{bimap}
\ExecuteMetaData[Tex/Background]{ornForget-type}
which will revert the modifications made by the constructor ornaments, and restores the original indices and parameters.

The allowed modifications are controlled by the definition of constructor ornaments \AD{ConOrn}. We must keep in mind that each constructor of \AD{ConOrn} also has to be reverted by \AF{ornForget}, accordingly, some modifications have preconditions, which are in this case always pointwise equalities:
\ExecuteMetaData[Tex/Background]{htpy}
Since constructors exist in the context of variables, we let constructor ornaments transform variables with \AV{re-var}, in addition to parameters and indices.

The first three constructors of \AD{ConOrn} represent the operations which copy the corresponding constructors of \AD{Con-ix}\footnote{Viewing \AD{ConOrn} as a binary relation on \AD{Con-ix}, these represent the preservation of \AD{ConOrn} by \AIC{𝟙}, \AIC{ρ}, and \AIC{σ}, up to parameters, variables, and indices.}. The \AIC{Δσ} constructors allows one to add fields which are not present on the original datatype.
\ExecuteMetaData[Tex/Background]{ConOrn}
% yes re-par can be implicit most of the time
% when you actually start using ornaments generically, it will come back to bite you though
The commuting square \AF{re-index}\ \AF{∘}\ \AV{j}\ \AF{∼}\ \AV{i}\ \AF{∘}\ \AF{var→par}\ \AV{re-var} in the first two constructors ensures that the indices on both sides are indeed related, up to \AV{re-index} and \AV{re-var}.

Now, we can show that lists are indeed naturals decorated with fields:
\ExecuteMetaData[Tex/Background]{NatD-ListD}
This ornament preserves most structure of \bN{}, only adding a field using \AIC{∆σ}\footnote{Note that \AV{S}, and some later arguments we provide to ornaments, are implicit argument: Agda would happily infer them from \AF{ListD} and later \AF{VecD} had we omitted them.}. As \bN{} has no parameters or indices, \AD{List} has more specific parameters, namely a single type parameter. Consequently, all commuting squares factor through the unit type and can be satisfied with \AV{λ}\ \AV{\_}\ \AV{→}\ \AIC{refl}. 

We can also ornament lists to get vectors by reindexing them over \bN{}
\ExecuteMetaData[Tex/Background]{ListD-VecD}
We bind a new field of \bN{} with \AIC{∆σ}, extracting it in \AIC{𝟙} and \AIC{ρ} to declare that the constructor corresponding to \AIC{\_∷\_} takes a vector of length \AV{n} and returns a vector of length \AIC{suc}\ \AV{n}. 

The conversions from lists to naturals, and from vectors to lists are given by \AF{ornForget}. We define \AF{ornForget} as a \AF{fold} over an algebra that erases a single layer of decorations
\ExecuteMetaData[Tex/Background]{ornForget}
Recursively applying this algebra, which reinterprets values of \AV{E} as values of \AV{D}, lets us take apart a value in the fixpoint \AD{μ-ix}\ \AV{E} and rebuild it to a value of \AF{μ-ix}\ \AV{D}. This algebra
\ExecuteMetaData[Tex/Background]{ornAlg}
is a special case of the erasing function, which undecorates interpretations of arbitrary types \AV{X}:
\ExecuteMetaData[Tex/Background]{ornErase}
Reading off the ornament, we see which bits of \AV{CE} are new and which are copied from \AV{CD}, and consequently which parts of a term \AV{x} under an interpretation of \AV{CE} need to be forgotten, and which needs to be copied or translated. Specifically, the first three cases of \AF{conOrnErase} correspond to the structure-preserving ornaments, and merely translate equivalent structures from \AV{CE} to \AV{CD}.

For example, in the first case the ornament \AIC{𝟙}\ \AV{sq} copies leaves, telling us that \AV{CD} is \AIC{𝟙 i'} and \AV{CE} is \AIC{𝟙 j'}. The interpretation \AV{⟦ 𝟙 j' ⟧C \_ p j} of a leaf \AV{𝟙 j'} at parameters \AV{p} and index \AV{j} is simply the equality of expected and actual indices \AV{j ≡ (j' p)}. The term \AV{x} of \AV{j ≡ (j' p)}, then only has to be converted to the corresponding proof of equality on the \AV{CD} side: \AV{re-index j ≡ (i' (var→par re-var p))}. This is precisely accomplished by applying \AF{re-index} to both sides and composing with the square \AV{sq} at \AV{p}.

Likewise, in the case of \AIC{ρ} we only have to show that \AV{x} can be converted from one \AIC{ρ} to the other \AIC{ρ} by translating its parameters, and in the \AIC{σ} case the field is directly copied. The only other ornament \AIC{Δσ} adding fields, is easily undone by removing those fields. 

Thus, \AF{ornForget} establishes that \AV{E} in an ornament \AD{Orn}\ \AV{g}\ \AV{i}\ \AV{D}\ \AV{E} is an adorned version of \AV{D} by associating to each value of \AV{E} its an underlying value in \AV{D}. Additionally, \AF{ornForget} makes it simple to relate functions between related types. For example, instantiating \AF{ornForget} for \AF{NatD-ListD} yields \AF{length}. Hence, the statement that \AF{length} sends concatenation \AF{\_++\_} to addition \AF{\_+\_}, i.e. \AV{length (xs ++ ys) ≡ length xs + length ys}, is equivalent to the statement that \AF{\_++\_} and \AF{\_+\_} are related, or that \AF{\_++\_} is a lifting of \AF{\_+\_} \cite{orntrans}. %\marker{Ik hoop dat dit minder wazig is en de mental typechecking load wat reduceert.}

% remark, ornForget is not epi in general because of ∆σ ⊥

\section{Ornamental Descriptions}\label{sec:background-ornamental-descriptions}
By defining the ornaments \AF{NatD-ListD} and \AF{ListD-VecD} we could show that lists are numbers with fields and vectors are lists with fixed lengths. Even though we had to give \AF{ListD} before we could define \AF{NatD-ListD}, the value of \AF{NatD-ListD} actually forces the right-hand side to be \AF{ListD}.

This means we can also use an ornament to represent a description as a patch on top of another description, if we leave out the right-hand side of the ornament. Ornamental descriptions are precisely defined as ornaments without the right-hand side, and effectively bundle a description and an ornament to it\footnote{Consequently, \AD{OrnDesc}\ \AV{Δ}\ \AV{J}\ \AV{g}\ \AV{i}\ \AV{D} must simply be a convenient representation of \AD{Σ}\ (\AD{U-ix}\ \AV{Δ}\ \AV{J})\ (\AD{Orn}\ \AV{g}\ \AV{i}\ \AV{D}).}. Their definition is analogous to that of ornaments, making the arguments which would only appear in the new description explicit:
\ExecuteMetaData[Tex/Background]{OrnDesc}
\ExecuteMetaData[Tex/Background]{ConOrnDesc}
Using \AD{OrnDesc} we can describe lists as the patch on \AF{NatD} which inserts a \AIC{σ} in the constructor corresponding to \AIC{suc}:
\ExecuteMetaData[Tex/Background]{NatOD}
To extract \AF{ListD} from \AF{NatOD}, we can use the projection applying the patch in an ornamental description:
\ExecuteMetaData[Tex/Background]{toDesc}
The other projection reconstructs the ornament \AF{NatD-ListD} from \AF{NatOD}:
\ExecuteMetaData[Tex/Background]{toOrn}
As a consequence, \AD{OrnDesc} enjoys the features of both \AD{Desc} and \AD{Orn}, such as interpretation into a datatype by \AF{μ} and the conversion to the underlying type by \AF{ornForget}, by factoring through these projections.

In later sections, %mark: precisely?
we will routinely use \AD{OrnDesc} to view triples like (\AF{NatD}, \AF{ListD}, \AF{VecD}) as a base type equipped with two patches in sequence.


% exercise to reader: show OrnDesc AD ~ Exist[ BD in Desc ] Orn AD BD  


\section{Descriptions}\label{part:descriptions}
Before we can analyze number systems and their numerical representations using generic programs, we first have to ensure that these types fit into the descriptions. Some numerical representations are hard to describe using only the descriptions of parametric indexed inductive types \AD{U-ix}; in order to keep things running smoothly for the generic programmer, we present an extension of \AD{U-ix} incorporating metadata, parameter transformation, description composition, and variable transformation.

\section{Numerical Representations}\label{sec:desc-numrep}
Before we dive into descriptions, let us revisit the situation of \bN{}, \AD{List} and \AD{Vec}. If it was not coincidence that gave us ornaments from \bN{} to \AD{List} and from \AD{List} to \AD{Vec}, then we can expect to find ornaments beforehand, instead of as a consequence of the definitions of \AD{List} and \AD{Vec}.

Rather than finding the properties of \AD{Vec} that were already there, let us view \AD{Vec} as a consequence of the definition of \bN{} and \AF{lookup}. From \bN{}, we obtain a trivial type of arrays by reading \AF{lookup} as a prescript:
\ExecuteMetaData[Tex/Descriptions/Numrep]{Lookup2}
For this definition, the lookup function is simply the identity function on \AD{Lookup}. As this is the prototypical array corresponding to natural numbers, any other array type we define should satisfy all the same properties and laws \AD{Lookup} does, and should in fact be equivalent.


We remark that without further assumptions, we cannot use the equality type \AD{≡} for this notion of sameness of types: repeating the definition of a type gives two distinct types with no equality between them. Instead, we import another notion of sameness, known as isomorphisms:
\ExecuteMetaData[Tex/Descriptions/Numrep]{Iso}
An \AD{Iso} from \AV{A} to \AV{B} is a map from \AV{A} to \AV{B} with a (two-sided) inverse\footnote{This is equivalent to the other notion of equivalence: there is a map $f : A \to B$, and for each \AV{b} in \AV{B} there is exactly one \AV{a} in \AV{A} for which $f(a) = b$.}. In terms of elements, this means that elements of \AV{A} and \AV{B} are in one-to-one correspondence.

Now, rather than defining \AD{Vec} ``out of the blue'' and proving that it is correct or isomorphic to \AD{Lookup}, we can also turn the \AD{Iso} on its head: Starting from the equation that \AD{Vec} is equivalent to \AD{Lookup}, we derive a definition of \AD{Vec} as if solving that equation \cite{calcdata}. As a warm-up, we can also derive \AD{Fin} from the fact that \AD{Fin}\ \AV{n} should contain \AV{n} elements, and thus be isomorphic to \AV{Σ[ m ∈ ℕ ] m < n}.

To express such a definition by isomorphism, we define:
\ExecuteMetaData[Tex/Descriptions/Numrep]{Def}
using
\ExecuteMetaData[Tex/Descriptions/Numrep]{isigma}
The type \AD{Def}\ \AV{A} is deceptively simple, after all, there is (up to isomorphism) only one unique term in it! However, when using \AD{Def}initions, the implicit \AD{Σ'} extracts the right-hand side of a proof of an isomorphism, allowing us to reinterpret a proof as a definition.

To keep the resulting \AD{Iso}s readable, we construct them as chains of smaller \AD{Iso}s using a variant of ``equational reasoning'' \cite{agdastdlib, plfa}, which lets us compose \AD{Iso}s while displaying the intermediate steps. In the calculation of \AD{Fin}, we will use the following lemmas
\ExecuteMetaData[Tex/Descriptions/Numrep]{Fin-lemmas}
In the terminology of \autoref{sec:background-proving}, \AF{⊥-strict} states that ``if A is false, then A \emph{is} false'', if we allow reading isomorphisms as ``\emph{is}'', while \AF{<-split} states that the set of numbers below $n+1$ is 1 greater than the set of numbers below $n$.

Using these, we can calculate\footnote{Here we make non-essential use of \AF{cong}, later we do need function extensionality, which has to be postulated or brought in via Cubical Agda.}
\ExecuteMetaData[Tex/Descriptions/Numrep]{Fin-def}
This gives a different (but equivalent) definition of \AD{Fin} compared to \AF{FinD}: the description \AF{FinD} describes \AD{Fin} as an inductive family, whereas \AF{Fin-def} gives the same definition as a type-computing function \cite{progorn}.

This \AD{Def} then extracts to a definition of \AD{Fin}
\ExecuteMetaData[Tex/Descriptions/Numrep]{Fin}
To derive \AD{Vec}, we will use the isomorphisms
\ExecuteMetaData[Tex/Descriptions/Numrep]{Vec-lemmas}
which one can compare to the familiar exponential laws. These compose to calculate
\ExecuteMetaData[Tex/Descriptions/Numrep]{Vec-def}
which yields us a definition of vectors
\ExecuteMetaData[Tex/Descriptions/Numrep]{Vec}
and the \AD{Iso} to \AF{Lookup} in one go.

This explains how we can compute a type of lists or arrays (a numerical representation, here, \AD{Vec}) from a number system (\bN{}).


\section{Augmented Descriptions}
To describe more general numerical representations, we must first describe more general number systems. We do so very loosely, however, allowing for tree-like number systems so long as the values of nodes are linear combinations of the values of subnodes. This generalizes positional number systems such as \bN{} and binary numbers, and allows for more exotic number systems, but for example does not include \bN{}\ \AD{×}\ \bN{} with the Cantor pairing function as a number system.

By requiring that nodes are interpreted as linear combinations of subnodes, we can implement a universe of number systems as a special case of earlier universes by baking the relevant multipliers into the type-formers. Descriptions in the universe of number systems can then both be interpreted to datatypes, and can evaluate their values to \bN{} using the multipliers in their structure.

For there to be an ornament between a number system and its numerical representation, the descriptions of both need to live in the same universe. Hence, we will generalize the type of descriptions over information such as multipliers later, rather than defining a new universe of number systems here. The information needed to describe a number system can be separated between the type-formers. Namely, a leaf \AIC{𝟙} requires a constant in \bN{}, a recursive field \AIC{ρ} requires a multiplier in \bN{}, while a field \AIC{σ} will need a function to convert values to \bN{}.

To facilitate marking type-formers with specific bits of information, we define \todo{Compare this with the usual metadata in generics like in Haskell, but then a bit more wild. Also think of annotations on fingertrees.}
\ExecuteMetaData[Ornament/Desc]{Info}
to record the type of information corresponding to each type-former. We can summarize the information which makes a description into a number system as the following \AD{Info}:
\ExecuteMetaData[Ornament/Numerical]{Number} 
which will then ensure that each \AIC{𝟙} and \AIC{ρ} both are assigned a number \bN{}, and each \AIC{σ} is assigned a function that converts values of the type of its field to \bN{}.

On the other hand, we can also declare that a description needs no further information by:
\ExecuteMetaData[Ornament/Desc]{Plain}
By making the fields of information implicit in the type of descriptions, we can ensure that descriptions from \AD{U-ix} can be imported into the generalized universe without change.

%In a more interesting example, we can define
%\ExecuteMetaData[Ornament/Desc]{Countable}
%so that to define a \AD{DescI}\ \AF{Countable}, one has to provide a proof that each field is countable, which could be used to prove that each type represented by a \AD{DescI}\ \AF{Countable} is in turn also countable.

In the descriptions, the \AIC{δ} type-former, which we will discuss in closer detail in the next section, represents the inclusion of one description in a larger description. When we include another description, this description will also be equipped with extra information, which we allow to be different from the kind of information in the description it is included in. When this happens, we ask that the information on both sides is related by a transformation:
\ExecuteMetaData[Ornament/Desc]{InfoF}
which makes it possible to downcast (or upcast) between different types of information. This, for example, allows the inclusion of a number system \AD{DescI}\ \AF{Number} into an ordinary datatype \AD{Desc} without rewriting the former as a \AD{Desc} first.


\section{The Universe}\label{ssec:desc}
We also need to take care that the numerical representations we will construct indeed fit in our universe. The final universe \AD{U-ix} of \autoref{ssec:background-ix}, while already quite general, still excludes many interesting datastructures. In particular, the encoding of parameters forces recursive type occurrences to have the same applied parameters, ruling out nested datatypes such as (binary) random-access lists \cite{calcdata,purelyfunctional}:
\ExecuteMetaData[Tex/DescOrn]{random-access}
and finger trees \cite{fingertrees}:
\ExecuteMetaData[Tex/DescOrn]{finger-tree}
Even if we could represent nested types in \AD{U-ix} we would find it still struggles with finger trees: Because adding non-recursive fields modifies the variable telescope, it becomes hard to reuse parts of a description in different places. Apart from that, the number of constructors needed to describe finger trees and similar types also grows quickly when adding fields like \AD{Digit}.

We will resolve these issues as follows. We can describe nested types by allowing parameters to be transformed before they are passed to recursive fields \cite{initenough}. By transforming variables before they are passed to subsequent fields, it becomes possible to hide fields that are not referenced later and to share or reuse constructor descriptions. Finally, by adding a variant of \AIC{σ} specialized to descriptions, we can describe composite datatypes more succinctly\todo{Compare this to Haskell, in which representations are type classes, which directly refer to other types (even to the type itself in a recursive instance). (But that's also just there because in Haskell the type always already exists and they do not care about positivity and termination).}.

Combining these changes, we define the following universe:
\ExecuteMetaData[Ornament/Desc]{Desc}
where the constructors are defined as:
\ExecuteMetaData[Ornament/Desc]{Con}
From this definition, we can recover the ordinary descriptions as
\ExecuteMetaData[Ornament/Desc]{Plain-synonyms}
%and first explain the differences in type-formers compared to \AD{U-ix}, and how these 
Let us explain this universe by discussing some of the old and new datatypes we can describe using it. Some of these datatypes do not make use of the full generality of this universe, so we define some shorthands to emulate the simpler descriptions. Using 
\ExecuteMetaData[Ornament/Desc]{sigma-pm}
(and the analogues for \AIC{δ}) we emulate unbound and bound fields respectively, and with 
\ExecuteMetaData[Ornament/Desc]{rho-zero}
we emulate an ordinary (as opposed to nested) recursive field. We can then describe \bN{} and \AD{List} as before
\ExecuteMetaData[Ornament/Desc]{NatD}
\ExecuteMetaData[Ornament/Desc]{ListD}
by replacing \AIC{σ} with \AF{σ-} and \AIC{ρ} with \AIC{ρ0}.

On the other hand, we bind the length of a vector as a field when defining vectors, so there we use \AF{σ+} instead:
\ExecuteMetaData[Ornament/Desc]{VecD}
With the nested recursive field \AIC{ρ}, we can define the type of binary random-access arrays. Recall that for random-access arrays, we have that an array with parameter \AV{A} contains zero, one, or two values of \AV{A}, but the recursive field must contain an array of twice the weight. Hence, the parameter passed to the recursive field is \AV{A times A}, for which we define
\ExecuteMetaData[Ornament/Desc]{Pair}
Passing \AF{Pair} to \AIC{rho} we can define random access lists:  
\ExecuteMetaData[Ornament/Desc]{RandomD}
To represent finger trees, we first represent the type of digits \AD{Digit}: \todo{reminder to cite this here if I end up not referencing finger trees earlier.}
\ExecuteMetaData[Ornament/Desc]{DigitD}
We can then define finger trees as a composite type from \AD{Digit}:
\ExecuteMetaData[Ornament/Desc]{FingerD}
Here, the fact that the first \AIC{δ-} drops its field from the telescope makes it possible to reuse of \AD{Digit} in the second \AIC{δ-}.

These descriptions can be instantiated as before by taking the fixpoint\footnote{Note that these (obviously?) ignore the \AD{Info} of a description.}
\ExecuteMetaData[Ornament/Desc]{fpoint}
of their interpretations as functors
\ExecuteMetaData[Ornament/Desc]{interpretation}
In this universe, we also need to insert the transformations of parameters \AV{f} in \AIC{ρ} and the transformations of variables \AV{h} in \AIC{σ} and \AIC{δ}.

%Like for \AD{U-ix}, we can give the generic \AF{fold} for \AD{DescI}\todo{But why}
%\ExecuteMetaData[Ornament/Desc]{fold-type}




\section{Ornaments}\label{part:ornaments}
In the framework of \AD{DescI} of the last section, we can write down a number system and its meaning in one description, and we can use this as the starting point for constructing numerical representations. To write down a generic construction of numerical representations from number systems, we will need a language in which we can describe modifications on the number systems.

In this section, we will describe the ornamental descriptions for the \AD{DescI} universe, and explain their working by means of examples. As we will be constructing new datatypes, rather than relating pre-existing ones, we omit the definition of the ornaments.

%outline:
%we explained why descriptions and ornaments are crucial to achieve our goals
%however, the descriptions we explained earlier are not powerful enough to house finger trees


%To capture finger trees as an ornament over a number system, we will need to describe ornaments over nested datatypes. In this section we will work out descriptions and ornaments suitable for nested datatypes.
If we are going to simplify working with complex containers %, such as finger trees,
by instantiating generic programs to them, we should first make sure that these types fit into the descriptions.

We construct descriptions for nested datatypes by extending the encoding of parametric and indexed datatypes from \autoref{ssec:bg-desc} with three features: information bundles, parameter transformation, and description composition. Also, to make sharing constructors easier, we introduce variable transformations. Transforming variables before they are passed to child descriptions allows both aggressively hiding variables and introducing values as if by let-constructs.

We base the encoding of off existing encodings \cite{sijsling,practgen}. The descriptions take shape as sums of products, enforce indices at leaf nodes, and have explicit parameter and variable telescopes. Unlike some encodings \cite{effectfully, practgen}, we do not allow higher-order inductive arguments. 

We use type-in-type and with-K to simplify the presentation, noting that these can be eliminated respectively by moving to Typeω and by implementing interpretations as datatypes.

\section{The descriptions}\label{ssec:desc}
We use telescopes identical to those in \autoref{ssec:bg-desc}:
\ExecuteMetaData[Ornament/Desc]{telescopes}
Recall that a \AgdaDatatype{Tel} represents a sequence of types, which can depend on the external type $P$. This lets us represent a telescope succeeding another using \AgdaDatatype{ExTel}. A term of the interpretation \AgdaFunction{⟦\_⟧tel} is then a sequence of terms of all the types in the telescope.

We use some shorthands
\ExecuteMetaData[Ornament/Desc]{tele-shorthands}
\ExecuteMetaData[Ornament/Desc]{shorthands}

As we will see in \autoref{sec:trieo}, some generics require descriptions augmented with more information. For example, a number system needs to describe both a datatype and its interpretation into naturals. This can be incorporated into a description by allowing description formers to query specific pieces of information. We will control where and when which pieces get queried by parametrizing descriptions over information bundles  
\ExecuteMetaData[Ornament/Desc]{Info}
Here a bundle declares for example that \AgdaField{𝟙i} is the type of information has to be provided at a \AgdaInductiveConstructor{𝟙} former. Remark that in \AgdaField{σi}, the bundle can ask for something depending on the type of the field. In \AgdaField{δi}, the bundle can ask something regarding the parameters and indices (e.g., it can force only unindexed subdescriptions.).

\begin{example}
    For example, we can encode a class of number systems using the information 
    \ExecuteMetaData[Ornament/Numerical]{Number}
    (refer to \autoref{sec:trieo}). If we then define the unit type, when viewed as a \AgdaFunction{Number}
    \ExecuteMetaData{Ornament/Numerical}{Unit}
    we have to provide the information that the only value of the unit type evaluates to 1.
\end{example}

We can recover the conventional descriptions by providing the plain bundle:
\ExecuteMetaData[Ornament/Desc]{Plain}
We define the ``down-casting'' of information as
\ExecuteMetaData[Ornament/Desc]{InfoF}
allowing us to reuse more specific descriptions in less specific ones, so that e.g., a number system can be used in a plain datatype.

We can now define the descriptions, which should represent a mapping between parametrized indexed functors
\ExecuteMetaData[Ornament/Desc]{PIType}
Recall that a description 
\ExecuteMetaData[Ornament/Desc]{DescI}
is simply a list of constructor descriptions
\ExecuteMetaData[Ornament/Desc]{Con}
The interpretations \hyperlink{desc-desc-interpretation}{\AgdaFunction{⟦\_⟧}} of the formers can be found below.

Leaves are formed by
\ExecuteMetaData[Ornament/Desc]{Con-1}
Here \AgdaBoundFontStyle{if} queries information according to \texttt{If}, and \AgdaBoundFontStyle{j} computes the index of the leaf from the parameters and variables.

A recursive field is formed by
\ExecuteMetaData[Ornament/Desc]{Con-rho}
where \AgdaBoundFontStyle{j} now determines the index of the recursive field. The function \AgdaBoundFontStyle{g} represents a parameter transform: the parameters of the recursive field can now changed at each recursive level, allowing us to describe nested datatypes. The remainder of the fields are described by \AgdaBoundFontStyle{C}. Note that a recursive field is intentionally not brought into scope: making use of it requires induction-recursion anyway!

A non-recursive field is formed similarly to a recursive field
\ExecuteMetaData[Ornament/Desc]{Con-sigma}
The type of the field is given by \AgdaBoundFontStyle{S}, which may depend on the values of the preceding fields. We bring the field into scope, so we continue the description in an extended context. However, we allow the remainder of the description to provide a conversion from \texttt{V ▷ S} into \texttt{W} to select a new context. This makes it possible to hide fields which are unused in the remainder.

Almost analogously, we make composition of descriptions internal by a variant of \AgdaInductiveConstructor{σ}
\ExecuteMetaData[Ornament/Desc]{Con-delta}
This takes a description \texttt{R}, and acts like the \AgdaInductiveConstructor{σ} of \texttt{μ R}, only with more ceremony. This will allow us to form descriptions by composing other descriptions, avoiding multiplying the number of constructors of composite datatypes.

Similar to \AgdaInductiveConstructor{ρ}, the functions \AgdaBoundFontStyle{j} and \AgdaBoundFontStyle{g} control indices and parameters, only now of the applied description. As we allow the description \AgdaBoundFontStyle{R} of the field to have a different kind of information bundle \AgdaBoundFontStyle{If′}, we must ask that we can down-cast it into \AgdaBoundFontStyle{If} via \AgdaBoundFontStyle{iff}. 

Descriptions and constructor descriptions can then be interpreted to appropriate kind of functor, constructor descriptions also taking variables
\hypertarget{desc-desc-interpretation}{}
\ExecuteMetaData[Ornament/Desc]{interpretation}
We see that a leaf becomes a constraint between expected index and the actual index. A recursive field passes down a transformation of the current parameters and the expected index computed from the variables, before interpreting the remainder of the description. Likewise, a non-recursive field adds a field with type depending on variables, but also adds this field to the variables, which are then transformed and passed on to the remainder. The composite field is analogous, only adding a field from a description rather than a type. Finally, the list of constructor descriptions are interpreted as alternatives.

The fixpoint can then be taken over the interpretation of a description
\ExecuteMetaData[Ornament/Desc]{fpoint}
giving the datatype represented by the description.

We can then give a generic fold for the represented datatypes
\ExecuteMetaData[Ornament/Desc]{fold}
which descends the description, mapping itself over all recursive fields before applying the folding function.
\begin{remark}
    The situation of \AgdaFunction{fold} is very common when dealing with different kinds of recursive interpretations: functions from the fixpoint are generally defined from functions out of the interpretation, generalizing over the inner description while pattern matching on the outer description. 
\end{remark}
Note that the fold requires a rather general function, limiting its usefulness: because of the parameter transformations, we cannot instantiate the fold to a single parameter. Defining, e.g., the vector sum, would require us to inspect the description, and ask that a vector of naturals can be converted into a vector of naturals, which is trivial in this case.

\todo{Sigma plus/minus}

Let's look at some examples. We can encode the naturals as a type parametrized by \AgdaInductiveConstructor{∅} and indexed by \AgdaDatatype{⊤}
\ExecuteMetaData[Ornament/Desc]{NatD}
Lists can be encoded similarly, but this time using the telescope
\ExecuteMetaData[Ornament/Desc]{ListTel}
declaring that lists have a single type parameter. Compared to the naturals, the description now also asks for a field in the second case
\ExecuteMetaData[Ornament/Desc]{ListD}
Since the type parameter is at the top of the parameter telescope, the type of the field is given as \AgdaBoundFontStyle{par top}.

Vectors are described using the same structure, but have indices in \bN{}.
\ExecuteMetaData[Ornament/Desc]{VecD}
In the first case, the index is fixed at 0. The second case declares that to construct a vector of length \AgdaBoundFontStyle{suc ∘ top}, the recursive field must have length \AgdaFunction{top}. Note that unlike index-first types, we cannot know the expected index from inside the description, so much like native indexed types, we must add a field choosing an index.

Recall the type of finger trees. Using parameter transformations and composition, we can give a description of full-fledges finger trees! First, we describe the digits
\ExecuteMetaData[Ornament/Desc]{DigitD}
and define the nodes\footnote{We could give the nodes as a description, but in this case we only use them in the recursive fields, so we would take the fixpoint without looking at their description anyway.}
\ExecuteMetaData[Ornament/Desc]{Node}
We encode finger trees as
\ExecuteMetaData[Ornament/Desc]{FingerD}
In the third case, we have digits which are passed the parameters on both sides in composite fields, and a recursive field in the middle. The recursive field has a parameter transformation, turning the type parameter \AgdaBoundFontStyle{A} into a \AgdaBoundFontStyle{Node A} in the recursive child.

%\investigate{Making \AgdaDatatype{Desc} coinductive would do a couple of things. First, recursion and composition become identical. Second, nesting of both types becomes easier to describe, but potentially impossible to prove strictly positive.}

%\investigate{We intentionally dodge having index telescopes (or having the index type depend on the parameters and values). Does this really change anything?}



\section{The ornaments}
\towrite{Put something that isn't yet in \autoref{ssec:bg-orn} here.}

\ExecuteMetaData[Ornament/Orn]{Orn-type}
\ExecuteMetaData[Ornament/Orn]{ornForget-type}
%Thus, the relation should be precise enough pairs of \AgdaBoundFontStyle{E} and \AgdaBoundFontStyle{D} for which we could not define \AgdaFunction{ornForget}.

We will walk through the constructor ornaments
\ExecuteMetaData[Ornament/Orn]{ConOrn-type}
again, an ornament between datatypes is just a list of ornaments between their constructors
\ExecuteMetaData[Ornament/Orn]{Orn}
Note that all ornaments completely ignore information bundles! They cannot affect the existence of \AgdaFunction{ornForget} after all.

Copying parts from one description to another, up to parameter and index refinement, corresponds to reflexivity. Preservation of leaves follows the rule
\ExecuteMetaData[Ornament/Orn]{Orn-1}
We can see that this commuting square (\texttt{e (k p) ≡ j (over f p)}) is necessary: take a value of \texttt{E} at \texttt{p, i}, where \texttt{i} is given as \texttt{k p}. Then \AgdaFunction{ornForget} has to convert this to a value of \texttt{D} at \texttt{f p , e i}, but since \texttt{e i} must match \texttt{j (f p)}, this is only possible if \texttt{e (k p) = j (f p)}.

Preserving a recursive field similarly requires a square of indices and conversions to commute
\ExecuteMetaData[Ornament/Orn]{Orn-rho}
additionally requiring the recursive parameters to commute with the conversion. \todo{Does adding the derivations for the squares everywhere make this section clearler?}

Preservation of non-recursive fields and description fields is analogous
\ExecuteMetaData[Ornament/Orn]{Orn-sigma-delta}
differing only in that non-recursive fields appears transformed on the right hand, while description fields have their conversions modified instead. For this rule, we need that the variable transformations fit into a commuting square with the parameter conversions. The condition on indices for descriptions, which is a commuting triangle, is encoded in the return type\footnote{Should this become a problem like with \AgdaInductiveConstructor{ρ}, modifying the rule to require a triangle is trivial.}.

Ornaments would not be very interesting if they only related identical structures. The left-hand side can also have more fields than the right-hand side, in which case \AgdaFunction{ornForget} will simply drop the fields which have no counterpart on the right-hand side. As a consequence, the description extending rules have fewer conditions than the description preserving rules: 
\ExecuteMetaData[Ornament/Orn]{Orn-+-rho}
Note that this extension\footnote{Kind of breaking the ``ornaments relate types with similar recursive structure'' interpretation.} with a recursive field has no conditions.

Extending by a non-recursive field or a description field again only requires the variable transform to interact well with the parameter conversion
\ExecuteMetaData[Ornament/Orn]{Orn-+-sigma-delta}

In the other direction, the left-hand side can also omit a field which appears on the right-hand side, provided we can produce a default value
\ExecuteMetaData[Ornament/Orn]{Orn---sigma-delta}
These rules let us describe the basic set of ornaments between datatypes.

Intuitively we also expect a conversion to exist when two constructors have description fields which are not equal, but are only related by an ornament. Such a composition of ornaments takes two ornaments, one between the field, and one between the outer descriptions. This composition rule reads:\todo{The implicits kind of get out of control here, but the rule is also unreadable without them. I might hide the rule altogether and only run an example with it.}
\ExecuteMetaData[Ornament/Orn]{Orn-comp}
We first require two commuting squares, one relating the parameters of the fields to the inner and outer parameter conversions, and one relating the indices of the fields to the inner index conversion and the outer parameter conversion. Then, the last square has a rather complicated equation, which merely states that the variable transforms for the remainder respect the outer parameter conversion.

We will construct \AgdaFunction{ornForget} as a \AgdaFunction{fold}. Using
\ExecuteMetaData[Ornament/Orn]{erase-type}
we can define the algebra which forgets the added structure of the outer layer
\ExecuteMetaData[Ornament/Orn]{ornAlg}
Folding over this algebra gives the wanted function
\ExecuteMetaData[Ornament/Orn]{ornForget}

\todo{NatD was removed here}

We can also relate lists and vectors
\ExecuteMetaData[Ornament/Orn]{ListD-VecD}
Now the parameter conversion is the identity, since both have a single type parameter. The index conversion is \AgdaFunction{!}, since lists have no indices. Again, most structure is preserved, we only note that vectors have an added field carrying the length.

Instantiating \AgdaFunction{ornForget} to these ornaments, we now get the functions \AgdaFunction{length} and \AgdaFunction{toList} for free!

%\investigate{Having a function of the same type as \AgdaFunction{ornForget} is not sufficient to deduce an ornament. An obstacle is that the usual empty type (no constructors) and the non-wellfounded empty type (only a recursive field) don't have an ornament. Also, while the leaf-preservation case spells itself out, the substitutions obviously don't give us a way to recover the equalities.}


\section{Ornamental descriptions}
A description can say ``this is how you make this datatype'', an ornament can say ``this is how you go between these types''. However, an ornament needs its left-hand side to be predefined before it can express the relation, while we might also interpret an ornament as a set of instructions to translate one description into another. A slight variation on ornaments can make this kind of usage possible: ornamental descriptions.

An ornamental description drops the left-hand side when compared to an ornament, and interprets the remaining right-hand side as the starting point of the new datatype:
\ExecuteMetaData[Ornament/OrnDesc]{ConOrnDesc-type}
The definition of ornamental descriptions can be derived in a straightforward manner from ornaments, removing all mentions of the LHS and making all fields which then no longer appear in the indices explicit\footnote{One might expect to need less equalities, alas, this is difficult because of \autoref{rem:orn-lift}.}. We will show the leaf-preserving rule as an example, the others are derived analogously:
\ExecuteMetaData[Ornament/OrnDesc]{OrnDesc-1}
As we can see, the only change we need to make is that \AgdaBoundFontStyle{k} becomes explicit and fully annotated.

Almost by construction, we have that an ornamental description can be decomposed into a description of the new datatype
\ExecuteMetaData[Ornament/OrnDesc]{toDesc}
and an ornament between the starting description and this new description
\ExecuteMetaData[Ornament/OrnDesc]{toOrn}


\section{Temporary: future work}
\begin{remark}
    Note that this allows us to express datatypes like finger trees, but not rose trees. Such datatypes would need a way to place a functor ``around the \AgdaInductiveConstructor{ρ}'', which then also requires a description of strictly positive functors. In our setup, this could only be encoded by separating general descriptions from strictly positive descriptions. The non-recursive fields of these strictly positive descriptions then need to be restricted to only allow compositions of strictly positive context functions. 
\end{remark} % \investigate{This setup does not allow nesting over recursive fields, which is necessary for structures like rose trees. This is actually kind of essential for enumeration. Nesting over a recursive field is problematic: we can incorporate it by adding ``this'' implicitly to a \AgdaInductiveConstructor{δ}, but then the \AgdaBoundFontStyle{R} needs to be strictly positive in its last argument, meaning we need to split \AgdaDatatype{Desc} into a strictly positive part and normal part. The strictly positive part should then only allow strictly positive parameter transforms in recursive and non-recursive fields, requiring an embedding of transforms.}

\begin{remark}
    Variable transforms are not essential in these descriptions, but there are a couple of reasons for keeping them. In particular, they make it possible to reuse a description in multiple contexts, and save us from writing complex expressions in the indices of our ornaments. On the other hand, the transforms still make defining ornaments harder (the majority of the commuting squares are from variables). Isolating them into a single constructor of \AgdaDatatype{Desc}, call it \AgdaInductiveConstructor{v}, seems like a good middle ground, but raises some odd questions, like ``why is there no ornament between \AgdaBoundFontStyle{v (g ∘ f) C} and \AgdaBoundFontStyle{v g (v f C)}''. (Furthermore, this also does not simplify the indices of ornaments).
\end{remark} %\investigate{Variable transforms are both less essential and less troublesome than I first thought. We can move variable transforms into a new former, and it probably simplifies the definition of ornaments a lot.}

\begin{remark}
    Rather, ornaments themselves could act as information bundles. If there was a description for \AgdaDatatype{Desc}, that is. Such a scheme of levitation would make it easier to switch between being able to actively manipulate information, and not having to interact with it at all. However, the complexity of our descriptions makes this a non-trivial task; since our \AgdaDatatype{Desc} is given by mutual recursion and induction-recursion, the descriptions, and the ornaments, would have to be amended to encode both forms of recursion as well.
\end{remark} % \investigate{If we levitate, then informed descriptions become ornaments over \AgdaDatatype{Desc}. This gives us the best of both worlds (modulo reflecting the description into a datatype): in plain descriptions, information does not even exist, and in informed descriptions, it is explicit. For levitation, we likely need induction-recursion and mutual recursion.}

\begin{remark}\label{rem:orn-lift}
    Rather than having the user provide two indices and show that the square commutes, we can ask for a ``lift'' $k$
    % https://q.uiver.app/#q=WzAsNCxbMCwwLCJcXGJ1bGxldCJdLFsxLDAsIlxcYnVsbGV0Il0sWzAsMSwiXFxidWxsZXQiXSxbMSwxLCJcXGJ1bGxldCJdLFswLDEsImUiXSxbMiwzLCJmIiwyXSxbMiwwLCJqIl0sWzMsMSwiaSIsMl0sWzMsMCwiayIsMV1d
    \[\begin{tikzcd}
        \bullet & \bullet \\
        \bullet & \bullet
        \arrow["e", from=1-1, to=1-2]
        \arrow["f"', from=2-1, to=2-2]
        \arrow["j", from=2-1, to=1-1]
        \arrow["i"', from=2-2, to=1-2]
        \arrow["k"{description}, from=2-2, to=1-1]
    \end{tikzcd}\]
    and derive the indices as $i = ek, j = kf$. However, this is more restrictive, unless $f$ is a split epi, as only then pairs $i,j$ and arrows $k$ are in bijection. In addition, this makes ornaments harder to work with, because we have to hit the indices definitionally, whereas asking for the square to commute gives us some leeway (i.e., the lift would require the user to transport the ornament). 
\end{remark}


%\investigate{Can these be simpler? Right now, these just construct the ornament and description on the fly, rather than actually asking for less.}



\section{Generic Numerical Representations}\label{part:numrep}
The ornamental descriptions together with the descriptions and number systems from before complete the toolset we will use to construct numerical representations as ornaments.

In summary, using \AD{DescI}\ \AF{Number} to represent number systems, we paraphrase calculations like in \autoref{sec:desc-numrep} as ornaments, rather than direct definitions. In fact, we have already seen ornaments to numerical representations before, such as \AF{ListOD} and \AF{RandomOD}. Generalizing those ornaments, we construct numerical representations by means of an ornament-computing function, sending number systems to the ornamental descriptions that describe their numerical representations. 

\begin{comment}
3 Calculating datastructures using Ornaments

In this part we return to the matter numerical representations. With 2.3 in mind, we can rephrase part our original question to ask

> Can numerical representations be described as ornaments on their number systems?

Let us look at a numerical representation presented as ornament in action.

\section{Numerical representations as ornaments}\label{sec:ornaments}
Reflecting on this derivation for \bN{}, we could perform the same computation for \bL{} to get Braun trees. However, we note that these computations proceed with roughly the same pattern: each constructor of the numeral system gets assigned a value, and is amended with a field holding a number of elements and subnodes using this value as a ``weight''. This kind of ``modifying constructors'' is formalized by ornamentation \cite{progorn}, which lets us formulate what it means for two types to have a ``similar'' recursive structure. This is achieved by interpreting (indexed inductive) datatypes from descriptions, between which an ornament is seen as a certificate of similarity, describing which fields or indices need to be introduced or dropped to go from one description to the other. \textit{Ornamental descriptions}, which act as one-sided ornaments, let us describe new datatypes by recording the modifications to an existing description.
\todo[inline]{Put some minimal definitions here.}

Looking back at \AgdaDatatype{Vec}, ornaments let us show that express that \AgdaDatatype{Vec} can be formed by introducing indices and adding a fields holding an elements to \bN{}.
However, deriving \AgdaDatatype{List} from \bN{} generalizes to \bL{} with less notational overhead, so we tackle that case first. We use the following description of \bN{}
\ExecuteMetaData[Tex/NumRepOrn]{NatD}
Here, \AgdaInductiveConstructor{σ} adds a field to the description, upon which the rest of the description can vary, and \AgdaInductiveConstructor{ṿ} lists the recursive fields and their indices (which can only be \AgdaInductiveConstructor{tt}).
We can now write down the ornament which adds fields to the \AgdaFunction{suc} constructor
\ExecuteMetaData[Tex/NumRepOrn]{ListO}
Here, the \AgdaInductiveConstructor{σ} and \AgdaInductiveConstructor{ṿ} are forced to match those of \AgdaDatatype{NatD},
but the \AgdaInductiveConstructor{Δ} adds a new field. Using the least fixpoint and description extraction, we can then define \AgdaDatatype{List} from this ornamental description. Note that we cannot hope to give an unindexed ornament from \bL{}
\ExecuteMetaData[Tex/NumRepOrn]{LeibnizD}
into trees, since trees have a very different recursive structure! Thus, we must keep track at what level we are in the tree so that we can ask for adequately many elements:
\ExecuteMetaData[Tex/NumRepOrn]{TreeO}
We use the \AgdaFunction{power} combinator to ensure that the digit at position $n$, which has weight $2^n$ in the interpretation of a binary number, also holds its value times $2^n$ elements. This makes sure that the number of elements in the tree shaped after a given binary number also is the value of that  binary number.
\end{comment}


\section{Generic numerical representations}\label{sec:trieo}
In this section, we will demonstrate how we can use ornamental descriptions to generically compute numerical representations. 

The reasoning here proceeds differently from that in the calculation of \AD{Vec} from \bN{}. Indeed, we directly construct datatypes, 
%and only prove it is the correct type after,
rather than deriving them step-by-step using isomorphism reasoning. Nevertheless, the choices of fields depending on the analysis of a number system follow the same strategy. We will first present the unindexed numerical representations, explaining which fields it adds and why, by cases on the number system. Then, we will show the indexed numerical representations as an ornament on top of the unindexed variant, and how the indices built up incrementally as we descend over the structure of the number system.

Recall the ``natural numbers''-information \AF{Number}, which gets its semantics from the conversion to \bN{}:
\ExecuteMetaData[Ornament/Numerical]{toN-type}
which is defined by generalizing over the inner information bundle and folding using
\ExecuteMetaData[Ornament/Numerical]{toN-con}
The choice of interpretation restricts the numbers to the class of numbers which are evaluated as linear combinations of ``digits''\footnote{An arbitrary \AF{Number} system is not necessarily isomorphic to \bN{}, as the system can still be incomplete (i.e., it cannot express some numbers) or redundant (it has multiple representations of some numbers).}. This class certainly does not include all interesting number systems, but does include many systems that have associated arrays\footnote{Notably, arbitrary polynomials also have numerical representations, interpreting multiplication as precomposition.}. 

We let this interpretation into \bN{} guide the construction of the associated numerical representation. In each case, we follow the computation in \AF{value} by inserting vectors of sizes corresponding to the weights of the number system:
\ExecuteMetaData[Ornament/Numerical]{trieifyOD}
In\todo{Explain better} the case of a leaf \AIC{𝟙} of weight \AV{k}, we insert a vector of size \AV{k}. Similarly, in a field \AIC{σ}, where the weight is determined by a value \AV{s} of \AV{S}, we insert a vector of the weight corresponding to the value of \AV{s}. Note that the actual value/number of elements a leaf or field contributes depends on the preceding multipliers of recursive fields: a recursive field of a number can have a weight \AV{k}, so we multiply the number of elements in a recursive sequence by wrapping the parameter in a vector of size \AV{k}. By roughly the same reasoning we pass the trieification of a subdescription \AV{R} the parameter wrapped in a vector, which we compose into the current numerical description by using the ornament \AIC{∙δ}. Since \AV{R} can have a different \AD{Info}, we generalized the whole construction over \AV{ϕ}\ \AV{:} \AD{InfoF}\ \AV{If}\ \AF{Number}.

As an example, let us define \AF{PhalanxD} as a number system and walk through the computation of its \AF{trieifyOD}. We define
\ExecuteMetaData[Ornament/Numerical]{PhalanxND}
Now, we see that applying \AF{trieifyOD} sends leaves with a value of \AV{k} to \AD{Vec}\ \AV{A}\ \AV{k}, so applying it to \AF{DigitND} yields
\ExecuteMetaData[Ornament/Numerical]{DigitOD-2}
which is equivalent to the \AF{DigitOD} from before, up to expanding a vector of \AV{k} elements into \AV{k} fields. The same happens for the first two constructors of \AF{PhalanxND}, replacing them with an empty vector and a one-element vector respectively. The \AF{ThreeND} in the last constructor gets trieified to \AF{DigitOD′} and composed by \AF{O∙δ+}, and the recursive field gets replaced by a recursive field nesting over vectors of length. Again, this is equivalent to \AF{FingerOD}, up to wrapping values in length one vectors, replacing \AD{Pair} with a two-element vector, and inserting empty vectors.

The indexed numerical representations can be constructed from the unindexed numerical representations\todo{Explain better} by adding fields to track the indices where necessary, incrementally combining these into the indices at the leaves. In contrast to the computation of \AD{Vec} in \autoref{sec:desc-numrep}, where we gave the definition of the datatype by cases on the index, we will have to track the indices as we descend over structure of the number system by generalizing over an index-computing algebra: 
\ExecuteMetaData[Ornament/Numerical]{itrieifyOD}
Explain\todo{This concludes a bunch of things, including this thesis.}




\section{Conclusion and Discussion}
\section{δ is conservative}\label{sec:redundant-delta}
We define our universe \AD{DescI} with \AIC{δ} as a former of fields with known descriptions, because this makes it easier to write down \AF{trieifyOD}, even though \AIC{δ} is redundant. If more concise universes and ornaments are preferable, we can actually get all the features of \AIC{δ} and ornaments like \AIC{∙δ} by describing them using \AIC{σ}, annotations, and other ornaments.

Indeed, rather than using \AIC{δ} to add a field from a description \AV{R}, we can simply use \AIC{σ} to add \AV{S}\ \AV{=}\ \AD{μ}\ \AV{R}, and remember that \AV{S} came from \AV{R} in the information
\ExecuteMetaData[Tex/Discussion]{Delta-Info}
We can then define \AIC{δ} as a pattern synonym matching on the \AIC{just} case, and \AIC{σ} matching on the \AIC{nothing} case.

Recall that the ornament \AIC{∙δ} lets us compose an ornament from \AV{D} to \AV{D'} with an ornament from \AV{R} to \AV{R'}, yielding an ornament from \AIC{δ}\ \AV{D}\ \AV{R} to \AIC{δ}\ \AV{D'}\ \AV{R'}. This ornament can be modelled by first adding a new field \AD{μ}\ \AV{R'}, and then deleting the original \AD{μ}\ \AV{R} field. The ornament \AIC{∇} \cite{kophd} allows one to provide a default value for a field, deleting it from the description. Hence, we can model \AIC{∙δ} by binding a value \AV{r'} of \AD{μ}\ \AV{R'} with \AF{OΔσ+} and deleting the field \AD{μ}\ \AV{R} using a default value computed by \AF{ornForget}.


\section{Indices do not depend on parameters}\label{sec:no-dep-ix}
In \AD{DescI}, we represent the indices of a description as a single constant type, as opposed to an extension of the parameter telescope \cite{practgen}. This simplification keeps the treatment of ornaments and numerical representations more to the point, but rules out types like the identity type \AD{≡}. Another consequence of not allowing indices to depend on parameters is that algebraic ornaments \cite{algorn} can not be formulated in \AD{OrnDesc} in their fully general form.

By replacing index computing functions \AV{Γ}\ \AF{\&}\ \AV{V}\ \AF{⊢}\ \AV{I} with dependent functions
\ExecuteMetaData[Tex/Discussion]{index-interpretation}
we can allow indices to depend on parameters in our framework. As a consequence, we have to modify nested recursive fields to ask for the index type \AF{⟦}\ \AV{I}\ \AF{⟧tel} precomposed with \AV{g :}\ \AF{Cxf}\ \AV{Γ Γ}, and we have to replace the square like \AV{i}\ \AF{∘}\ \AV{j′}\ \AF{∼}\ \AV{i′}\ \AF{∘}\ \AF{over}\ \AV{v} in the definition of ornaments with heterogeneous squares.


\section{Σ-descriptions are more natural for expressing finite types}\label{sec:closed-universe}
Due to our representation of types as sums of products, representing the finite types of arbitrary number systems quickly becomes hard. Consider the binary numbers from before
\ExecuteMetaData[Tex/Discussion/Sigma-Desc]{Leibniz}
for which the finite type
\ExecuteMetaData[Tex/Discussion/Sigma-Desc]{FinB}
has more constructors than the numbers themselves. In general, the number of constructors of a finite type depends both on the multipliers and constants in all fields and leaves of the number system, which prevents us from constructing the finite type by simple recursion on \AD{DescI} (that is, without passing around lists of constructors instead). Furthermore, since our definition of ornaments insists a type and an ornament on it have the same number of constructors, there can also not be a generic ornament from numbers to their finite types. 

The apparent mismatch of number systems and their finite types stems from the treatment of the field-former \AIC{σ} in \AD{DescI}. In such a sums-of-products universe \cite{practgen,sijsling}, a \AIC{σ}\ \AV{S}\ \AV{C} represents a field of type \AV{S}, where the subsequent fields described by \AV{C} have their context extended by \AV{S}. In contrast, a Σ-descriptions universe \cite{effectfully,progorn,algorn} (in the terminology of \cite{sijsling}) encodes a dependent field \AV{(s : S)} by asking for a function \AV{C} assigning values \AV{s} to descriptions.

In comparison, a sums-of-products universe keeps out some more exotic descriptions\footnote{Consider the constructor \AIC{σ}\ \bN{}\ \AV{λ}\ \AV{n}\ \AV{→}\ \AF{power}\ \AIC{ρ}\ \AV{n}\ \AIC{𝟙} which takes a number \AV{n} and asks for \AV{n} recursive fields (where \AF{power}\ \AV{f}\ \AV{n}\ \AV{x} applies \AV{f} \AV{n} times to \AV{x}). This description, resembling a rose tree, does not (trivially) lie in a sums-of-products universe.} which do not have an obvious associated Agda datatype. As a consequence, this also prevents us from introducing new branches inside a constructor.

If we instead started from Σ-descriptions, taking functions into \AD{DescI} to encode dependent fields, we could compute a ``type of paths'' in a number system by adding and deleting the appropriate fields. Consider the universe
\ExecuteMetaData[Tex/Discussion/Sigma-Desc]{Sigma-Desc}
In this universe we can present the binary numbers as
\ExecuteMetaData[Tex/Discussion/Sigma-Desc]{LeibnizD}
The finite type for these numbers can be described by
\ExecuteMetaData[Tex/Discussion/Sigma-Desc]{FinBD}
Since this description of \AF{FinB} largely has the same structure as \AF{Leibniz}, and as a consequence also the numerical representation associated to \AF{Leibniz}, this would simplify proving that the indexed numerical representation is indeed equivalent to the representable representation (the maps out of \AF{FinB}). In a more flexible framework ornaments, we can even describe the finite type as an ornament on the number system.


\section{Indexed numerical representations are not algebraic ornaments}\label{sec:ix-not-alg}
From the theory of algebraic ornaments \cite{algorn}, we find that the type \AD{Vec} as an indexed variant of \AD{List}, can also be seen as an algebraic ornament. This construction takes an ornament between types \AV{A} and \AV{B}, and returns an ornament from \AV{B} to a type indexed over \AV{A}, representing ``\AV{B}s of a given underlying \AV{A}''. Instantiating this for naturals, lists and vectors, the algebraic ornament takes the ornament from naturals to lists, and returns an ornament from lists to vectors, by which vectors are lists of a fixed length.

While we gave an explicit ornament from the unindexed to the indexed numerical representations, we might expect that ornament to be the algebraic ornament of \AF{trieifyOD}. However, this fails if we want to describe composite types like \AD{FingerTree} (unless we first flatten \AD{Digit} into the description of \AD{FingerTree}): Since the algebraic ornament (obviously) preserves a \AIC{σ}, it can not convert the unindexed numerical representation under a \AIC{δ} to the indexed variant. This means that the algebraic ornament on \AD{FingerTree} (given by \AF{toDesc}\ \AF{trieifyOD}) would only index the outer structure, leaving the \AD{Digit} fields unindexed.

Nevertheless, we expect that a modifying the inlining ornamental algebras into algebraic ornaments, in the same way that \AF{itrieifyOD} diverges from the algebraic ornament, yields a variant of \AF{aOoA} which does coincide with \AF{itrieifyOD}.


\section{Branching numerical representations}
The numerical representations we construct via \AF{itrieifyOD} come in the form of heaps. Like random-access lists and finger trees, these numerical representations typically look like spines, storing elements in ever-growing trees hanging off this spine.

This is much in contrast with the Braun trees Hinze and Swierstra \cite{calcdata} compute. We can modify our construction of \AF{itrieifyOD} using 
\ExecuteMetaData[Tex/Discussion]{power}
to apply \AIC{ρ} \AV{k}-fold in the case of \AIC{ρ}\ \AV{\{if = k\}}, rather than applying \AIC{ρ} by nesting with a \AV{k}-element vector. \todo{Is that all there is to say? No algorn, no unindexed variant?}


\section{No RoseTrees}
In \AD{DescI}, we encode nested types by allowing nesting over a function of parameters \AF{Cxf}\ \AV{Γ}\ \AV{Γ}. This is less expressive than full nested types, which may also nest a recursive field under a strictly positive functor. For example, rose trees
\ExecuteMetaData[Tex/Discussion]{RoseTree}
cannot be directly expressed as a \AD{DescI}\footnote{And, since we do not have higher-order inductive arguments like Escot and Cockx \cite{practgen}, we can also not give an essentially equivalent definition.}. We can still describe a similar datatype of ``fixed-height'' rose trees
\ExecuteMetaData[Tex/Discussion]{almost-RoseTree}
which nests on the inside instead. Since lists can be empty, the rose trees are not necessarily of an actual fixed-height, and still coincide with rose trees.

If we were to describe full nested types, by allowing application functors over recursive arguments, we would need to convince Agda that these functors are indeed positive through, for example polarity annotations. Alternatively, we could encode strictly positive functors in a separate universe, which only allows to applications of parameters strictly positive contexts \cite{sijsling}. Finally, we could modify \AD{DescI} in such a way that we can decide if a description uses a parameter strictly positively, which would have to involve modifying \AIC{ρ} and \AIC{σ}, or adding variants of those formers only allowing strictly positive usage of parameters.


\section{No levitation}
Since our encoding does not support higher-order inductive arguments, let alone definitions by induction-recursion, \AD{DescI} certainly does not have a \AD{DescI}. Such self-describing universes have been described by Chapman et al. \cite{levitation}, and we expect that the other features of \AD{DescI}, parameters, nesting, and composition, would not obstruct a similar levitating variant of \AD{DescI}. Due to the work of Dagand and McBride \cite{orntrans}, ornaments might even be generalized to inductive-recursive descriptions.

If that is the case, then a part of our definitions and constructions could be expressed inside such a framework, rather than by modifying the universe to suit our needs. In particular, rather than describing \AD{DescI} to describe datatypes with even more information, \AD{DescI} should be expressible as an ornament on plain descriptions, in contrast to how \AD{Desc} is an instance of \AD{DescI} in our framework. This would allow treating information explicitly in \AD{DescI}, and not at all in \AD{Desc}.

Furthermore, constructions like \AF{trieifyOD}, which have the recursive structure of a fold over \AD{DescI}, could indeed be expressed by instantiating \AF{fold} to \AD{DescI}.\todo{Maybe a bit too dreamy.}


%\section{Variables slightly later}
%package them with the constructor descriptions rather than only after a sigma
% separate substitutions -> probably 


%\section{Less commutative squares}
%\autoref{rem:orn-lift} -> Discussion
% \begin{remark}\label{rem:orn-lift}
%     Rather than having the user provide two indices and show that the square commutes, we can ask for a ``lift'' $k$
%     % https://q.uiver.app/#q=WzAsNCxbMCwwLCJcXGJ1bGxldCJdLFsxLDAsIlxcYnVsbGV0Il0sWzAsMSwiXFxidWxsZXQiXSxbMSwxLCJcXGJ1bGxldCJdLFswLDEsImUiXSxbMiwzLCJmIiwyXSxbMiwwLCJqIl0sWzMsMSwiaSIsMl0sWzMsMCwiayIsMV1d
%     \[\begin{tikzcd}
%         \bullet & \bullet \\
%         \bullet & \bullet
%         \arrow["e", from=1-1, to=1-2]
%         \arrow["f"', from=2-1, to=2-2]
%         \arrow["j", from=2-1, to=1-1]
%         \arrow["i"', from=2-2, to=1-2]
%         \arrow["k"{description}, from=2-2, to=1-1]
%     \end{tikzcd}\]
%     and derive the indices as $i = ek, j = kf$. However, this is more restrictive, unless $f$ is a split epi, as only then pairs $i,j$ and arrows $k$ are in bijection. In addition, this makes ornaments harder to work with, because we have to hit the indices definitionally, whereas asking for the square to commute gives us some leeway (i.e., the lift would require the user to transport the ornament). 
% \end{remark}



%\section{δ is conservative over Desc and Orn}
%If we include the ornament \AIC{∇σ} dropping a field by giving a default value \AV{V}\ \AF{⊧}\ \AV{S} in place of a \AIC{σ}:
%\ExecuteMetaData[Tex/Discussion]{nabla-sigma}
%then we can also represent \AIC{∙δ} without further modifying \AD{ConOrnDesc}. Namely
%\ExecuteMetaData[Tex/Discussion]{comp-delta-nabla-sigma}
%This emulates the \AIC{∙δ} over an ornament \AV{RR′}, by first adding a field of \AD{μ}\ (\AF{toDesc}\ \AV{RR′}) and then fixing a default value for \AD{μ}\ \AV{R} by using \AF%{ornForget}.
%This makes the presentation of the descriptions and ornaments, and the interpretations of both simpler. However, this has the downside of needing a transport (or, %with-abstraction) for each pattern match on a value which would otherwise be a \AIC{δ}.


%\section{Reconcile calculating and trieifyOD}
%\begin{outline}
%    In the computation of generic numerical representations, we gave \AF{trieifyOD} directly, rather than as the consequence of a calculation. %\todo{This is simply because a) %the wheels would come off very soon b) trieifyOD is not a definition but rather describes one.}
%
%    By abstracting \AF{Def} over a function, we can elegantly describe the kind of object we are looking for
%    \[ ... \]
%    but because we factor through an interpretation into \AD{Type}, we still have to give the definition before we can construct the isomorphism.
%
%    Maybe this works better for trieifyOD itself, where the isomorphism is really a composition of smaller isomorphisms by analyzing the descriptions, rather than one global %isomorphism as is the case when comparing Lookup and VecD.
%
%    See Tex.Disussion.Def-cong
%\end{outline}



%\section{?}
%While evidently Ix x != Fin (toN x) for arbitrary number systems, does the expected iso Ix x -> A = Trie A x yield Traversable, for free?

%\section{Odd numerical representations} % mark: but that's not _my_ problem is it?
%the numerical representation of 2-3 fingertrees is a bit odd, or trivial.
%I do not know whether there is a datastructure (let alone numerical representation) which has amortized constant append on both sides, and has logarithmic lookup, but uses only simple nesting (i.e., nesting over a functor with only products and no sums).

% is fold and map really valid and terminating? -> the whole --cubical --safe not being safe really undermines this




%\section{Related work}\label{part:related}
%summarizing why everything that is in my references is there - > merge into discussion and conclusion

\printbibliography

%\section{Appendices}

\newpage
\begin{appendix}
\section*{Appendices}
\addcontentsline{toc}{section}{Appendices}
\renewcommand{\thesubsection}{\Alph{subsection}}

\subsection{Folding}\label{app:gfold}
In \autoref{ssec:generic-programming} and \autoref{sec:background-ornaments} we used \AF{fold} as a concept to explain a bit of generic programming. We give its definition here, but for \AD{DescI} instead, since the fold of \AD{U-ix} can be seen as a simplification of it.
\ExecuteMetaData[Ornament/Desc]{fold-type}
As \AF{fold}\ \AV{f} is the algebra map \AIC{con}\ \AF{⇒}\ \AV{f}, the following commutes:
% https://q.uiver.app/#q=WzAsNCxbMSwwLCJGQSJdLFsxLDEsIkEiXSxbMCwxLCJcXG11IEYiXSxbMCwwLCJGXFxtdSBGIl0sWzAsMSwiZiJdLFsyLDEsIlxcbWF0aHJte2ZvbGR9XFwgZiIsMl0sWzMsMCwiRihcXG1hdGhybXtmb2xkfVxcIGYpIl0sWzMsMiwiXFxtYXRocm17Y29ufSIsMl1d
\[\begin{tikzcd}
	{F\mu F} & FA \\
	{\mu F} & A
	\arrow["f", from=1-2, to=2-2]
	\arrow["{\mathrm{fold}\ f}"', from=2-1, to=2-2]
	\arrow["{F(\mathrm{fold}\ f)}", from=1-1, to=1-2]
	\arrow["{\mathrm{con}}"', from=1-1, to=2-1]
\end{tikzcd}\]
However, by defining \AF{fold}\ \AV{f}\ (\AIC{con}\ \AV{x}) as \AV{f}\ (\AF{map}\ (\AF{fold}\ \AV{f}) x), we prevent the termination checker from seeing that \AF{fold} is only applied to terms strictly smaller than \AV{x} (much like our fellow universe constructions find out somewhere along the line). To help out the termination checker, we inline \AF{fold} into \AF{map}, which gives us an equivalent definition:
\ExecuteMetaData[Ornament/Desc]{mapFold}
Here \AF{mapDesc} (and \AF{mapCon}) simply peel off and reassemble all non-recursive structure, applying \AF{fold} to the recursive fields; \AF{fold} is then defined in the usual way by applying its algebra \AV{f} to itself mapped over \AV{x}.

\subsection{Folding without Axiom K}\label{app:withoutk}
The axiom of univalence (or cubical type theory) gives us another interesting context to study ornaments in. In the way we presented it, the theory of ornaments produces a lot of isomorphisms from relations between types, which are not yet as powerful as they could be when comparing properties between related types. Univalence gives us the means to turn equivalences\footnote{Equivalences can be considered as a correction to isomorphisms for types which are not sets (in the sense of being discrete); since all types we describe here are sets, equivalences and isomorphisms coincide.} into equalities, allowing us to put an isomorphism between types to work by transporting properties over it.

Unfortunately, a direct port of ornaments into \AV{--cubical} is quickly thwarted by the absence of Axiom K, as one would discover that the definitions of \AF{mapDesc} and \AF{mapCon} illegally pattern match on the types calculated by interpretations\footnote{The \href{https://agda.readthedocs.io/en/v2.6.4.1-rc1/language/without-k.html}{Without K documentation} explains why pattern matching on non-datatypes is not safe in general.}.

This can be remedied by presenting interpretations as datatypes\footnote{Albeit a bit dubiously; at the time of writing, this is also how you can circumvent a restriction on pattern matching emplaced by \AV{--cubical-compatible}, see \href{https://github.com/agda/agda/issues/5910\#issuecomment-1601301237}{the relevant GitHub issue}.}. Effectively, we are applying the duality between type computing functions and indexed types. Since \AD{Desc} and \AD{Con} are unindexed types, they cannot accidentally carry equational content, and pattern matching on them does not generate transports in \AF{⟦\_⟧D} and \AF{⟦\_⟧C}. Hence, the definition of \AF{fold} is (morally speaking) safe.

With that out of the way, we can define the interpretations as indexed types:
\ExecuteMetaData[Appendix/Intp]{Intp}
Since the interpretations are datatypes now, we can pattern match on them to define \AF{mapDesc} and \AF{mapCon} in a way that is accepted:
\ExecuteMetaData[Appendix/Intp]{mapFold}


\subsection{Nested types as uniformly recursive indexed types}\label{app:unnested}
Although \AD{U-ix} has no direct support for expressing nested types, we can actually give equivalent encodings for some of them.

Indeed, indices are readily repurposed as parameters. If we apply this to, say, random-access lists, we can write:
\ExecuteMetaData[Appendix/NoNesting]{RandomD-1}
%Suppose that we defined \AD{U-nest} as the extension of \AD{U-ix} for nested types, then we can automate this transformation:
%\ExecuteMetaData[Appendix/NoNesting]{uniform}
More interestingly, perhaps, the depth of a random-access list determines the types of its fields. Namely, at the highest level, \AIC{One} will ask for 1 element, one level down it asks for 2, and one more it asks for 4, and so on. Hence, in a way that vaguely resembles defunctionalization, we can define
\ExecuteMetaData[Appendix/NoNesting]{Pair}
and describe a field at depth \AV{n} by \AF{power}\ \AV{n}\ \AD{Pair}\ \AV{A}. This can be applied to describe random-access lists which track their depth in their index instead:
\ExecuteMetaData[Appendix/NoNesting]{RandomD-2}
Since we cannot (yet) construct path types generically (\autoref{sec:sigma-desc}), we cannot make this construction generic. If we did have such constructions, the argument for random-access lists generalizes to an operation that splits a nested datatype \AV{D} into three parts:
\begin{enumerate}
    \item a type of paths in \AV{D} (not necessarily pointing to a field)
    \item a lookup function that sends a path to the accumulated parameter transformation
    \item the (uniform) datatype, indexed over the paths, using the lookup function to calculate the types of its fields.
\end{enumerate}
\end{appendix}
\end{document}
