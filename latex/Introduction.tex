Agda \cite{agda} is a functional programming language and a proof assistant, taking inspiration from languages like Haskell and other proof assistants such as Coq. We can write programs as we would in Haskell, and then express and prove their properties all inside Agda. This allows us to demonstrate the correctness of programs by formal proof rather than by testing. However, this level of formality also trades-off the uncertainty of testing for a time-investment to produce these proofs. In this thesis, we will explore a variety of methods of proving properties of our programs, focusing on the problems that one may encounter, presenting solutions as they arise. Let us sketch some of these problems.

First, merely adapting a program to Agda may already require changes to the datatypes used in it; for example, if a program manipulating a \AgdaDatatype{List} uses the unsafe \AgdaFunction{head} function, then one is forced to replace the \AgdaDatatype{List} by a datatype that ensures non-emptiness, such as a \AgdaDatatype{NonEmpty} list or a length-aware vector \AgdaDatatype{Vec}. On the other hand, there might be sections of a program where the concrete length is not relevant for correctness and only gets in the way. As a result, one might find themselves duplicating common functions like concatenation \AgdaFunction{\_++\_} to only alter their signatures.

However, the ``new'' datatype (\AgdaDatatype{Vec}) is typically a simple variation on the old datatype (\AgdaDatatype{List}) making small adjustments to the existing constructors; in this case, we decorate the nil and cons constructors with natural numbers representing the length. This kind of modification of types falls in the framework of \textit{ornamentation} \cite{progorn}; if two types are reified to their \textit{descriptions}, then \textit{ornaments} express whether the types are ``similar'' by acting as a recipe to produce one type from the other. By restricting the operations to the copying of corresponding parts, and the introduction of fields or dropping of indices, the existence of such an ornament ensures that the types have the same recursive structure. In general, ornaments allow us to introduce invariants into existing types, so that, as an example, one can derive ordered versions of lists or trees from their ordinary variants. Furthermore, using \textit{patches} \cite{orntrans}, we can in one direction ensure that \AgdaFunction{\_++\_} on \AgdaDatatype{Vec} agrees with its version for \AgdaDatatype{List} under the ornament; in the other direction, a patch can also help us while defining this lifted variant.

Using ornaments, we can organize similar datatypes using ornaments; but we will also make use of relations between dissimilar datatypes. It is conventional to prototype a program using simpler types or implementations, and only replace these with more performant alternatives in critical places. %knowing that this is eventually going to happen, one might as well prepare for it.
While this may quickly turn into a refactoring nightmare in the general case, we can hope for a more satisfying transition if we restrict our attention to a narrower scope. As an example, we might start programming using \AgdaDatatype{List}s, but replace this with a \AgdaDatatype{Tree} if we notice that the program spends most of its time in \AgdaFunction{lookup} operations. To gain a speedup, we will have to reimplement the operations on \AgdaDatatype{Tree}. This would also double the number of necessary proofs; however, we have two ways to avoid this problem. 

We will look at the more specific solution first. This solution is guided by the realization that even though \AgdaDatatype{List} and \AgdaDatatype{Tree} have different recursive structures, they have one commonality; namely, both resemble a number system. Lists and Braun trees\footnote{Braun trees are a kind of binary tree, of which the shape is determined by its size.} can both be presented by deriving them from unary and binary numbers respectively, as is made formal by Hinze and Swierstra \cite{calcdata}. One can then apply this \textit{numerical representation} \cite{purelyfunctional} to simplify or trivialize properties of these datastructures. We will also see that we can interpret numerical interpretations more literally, and construct the representation directly as an ornament.

In the general case, we can apply representation independence. Equality of indiscernables ensures that substituting terms for equal terms cannot change the behaviour of a program, and, as types are terms, the same should hold for types. If we consider two types implementing a given interface, with an operation-preserving isomorphism, then representation independence tells us that the implementations must be functionally equivalent. In the case of trees and lists, this states that since converting a list to a tree preserves \AgdaFunction{lookup}, the outcome of a program that only uses \AgdaFunction{lookup} cannot change when substituting trees for lists. While a proof of this statement usually either exists in the meta-theory, or is produced by manually weaving the conversions through our proofs, Cubical Agda allows us to internalize this independence \cite{iri}.  

We will first take a closer look at SIP \cite{iri} and give concrete examples of proof transport, which we can use to characterize equivalences of flexible two-sided arrays. Then we recall the constructions of numerical representations \cite{calcdata} and ornamentation \cite{progorn}, illustrating how we can define arrays from simpler types by providing interpretations into naturals. We will test these methods by using them to simplify the presentation of finger trees\footnote{A finger tree is a nested type representing a sequence, designed to support amortized constant time en-/dequeueing at both ends, and logarithmic time concatenation and lookup.} \cite{fingertrees}. After that, we will investigate other generic operations, such as the presentation of certain type transformations as ornaments, and the fair enumeration of recursive datatypes.