\subsection{Agda}
We formalize our work in Agda \cite{agda}, a functional programming language with dependent types. Using dependent types we can use Agda as a proof assistant, allowing us to state and prove theorems about our datastructures and programs. \todo{Make sure this does not literally repeat the introduction too often.} These proofs can then be run as algorithms, or in some cases be extracted to a Haskell program\footnote{Or JavaScript, if you want.}.

Syntactically Agda is similar to Haskell, with a few notable differences. One is that Agda allows most characters and words in identifiers with only a small set of exceptions. For example, we can write
\ExecuteMetaData[Tex/Snippets]{ternary}
The other is that datatypes are given either as generalized algebraic datatypes (GADTs) or record types in Haskell.

The type system of Agda is an extension of (intensional) Martin-Löf type theory (MLTT), a constructive type theory in which we can interpret intuitionistic logic. Compared to Haskell, which extends a polymorphic lambda calculus with inductive types, MLTT allows types of codomains of functions to vary with values in the domains: Whereas in Haskell only datatypes can map into types\footnote{Excluding extensions}, in Agda we define functions into \AgdaPrimitiveType{Type}
\[ \dots \]
given a function $f$ from $A$ into \AgdaPrimitiveType{Type}, we can then form the type
\[ \dots \]
Likewise, the type of the second field of a pair type can vary with the value of the first. The presence of these types enriches the interpretation of logic into programs, known as the Curry-Howard isomorphism: propositions or logical formulas are related to types, such that a term of a type constitutes a proof of the related proposition.

To ensure that the logic interpreted by this isomorphism remains consistent, Agda rules out non-terminating functions by restricting their definitions to structural recursion. The termination checker (together with other restrictions which we will encounter in due time) prevents trivial proofs which would be tolerated in Haskell, like
\ExecuteMetaData[Tex/Snippets]{loop}
The propositional part of the Curry-Howard correspondence can then be formulated by the usual type formers. The atomic formulas, true and false, can be represented respectively as the empty record: there always is a proof \AgdaFunction{tt} of true
\ExecuteMetaData[Tex/Snippets]{true}
and the type with no constructors: there is no way to make a proof of false
\ExecuteMetaData[Tex/Snippets]{false}
Implication $A \implies B$ corresponds to function types $A \to B$: a proof of $A$ can be converted to a proof of $B$. Implication also gives an interpretation of negation as functions into false $A \to \bot$. Disjunction (logical or) is described by a sum type $A + B$: either of $A$ or $B$ can prove $A + B$
\ExecuteMetaData[Tex/Snippets]{either}
Conjunction (logical and) is given as a product type: having both $A$ and $B$ proves $A \times B$
\ExecuteMetaData[Tex/Snippets]{pair}
Predicates, formulas containing variables, correspond to functions into the type of formulas
\ExecuteMetaData[Tex/Snippets]{predicate}
allowing interpretations of higher-order logic. Quantifiers are interpreted via dependent types. Universal quantification (for all) is a dependent function type: for each $a : A$, give a proof of $P\ a$
\ExecuteMetaData[Tex/Snippets]{forall}
Likewise, existential quantification (exists) is a dependent pair type: there is an $a : A$ and a proof $P\ a$
\ExecuteMetaData[Tex/Snippets]{exists}
Predicates can also be expressed using indexed datatypes, in which the choice of constructor can influence the index, whereas parameters must be constant over all constructors. Equality of elements of a type $A$ can then be interpreted as the type
\ExecuteMetaData[Tex/Snippets]{eq}
Closed terms of this type can only be constructed for definitionally equal elements, but crucially, variables can contain equalities between different elements. As the second argument is an index, pattern matching on \AgdaFunction{refl} unifies the elements, such that properties like substitution follow
\ExecuteMetaData[Tex/Snippets]{subst}

With this, we can do math. For example, we could define natural numbers as an inductive type
\[ \dots \]
and set out to prove the elementary properties of prime numbers. But to get the same results to binary numbers (without duplicating the proofs), we need a bit more. The usual notion of equalities of types are isomorphisms: two types $A, B$ are isomorphic if there are functions $A \to B$ and $B \to A$, which are mutually inverse 
\[ \dots \]
In ordinary Agda, we cannot directly apply these to transport along like we can for equalities, however.

\subsection{Cubical Agda}
Intuitively, one expects that like how isomorphic groups share the same group-theoretical properties, isomorphic types also share the same type-theoretical properties. Meta-theoretically, this is known as \emph{representation independence}, and is evident. Inside (ordinary) Agda this is not so practical, as this independence only holds when applied to concrete types, and is then only realized by manually substituting along the isomorphism. On the other hand, in Cubical Agda, the Structure Identity Principle internalizes a kind of representation independence \cite{iri}.

Cubical Agda modifies the type theory of Agda to a kind of homotopy type theory, looking at equalities as paths between terms rather than the equivalence relation generated by reflexivity. In cubical type theories, the role played by pattern matching on \AgdaFunction{refl} or by axiom J, in MLTT and ``Book HoTT'' respectively, is instead acted out by directly manipulating cubes\footnote{Under the analogy where a term is a point, an equality between points is a line, a line between lines is a square.}. In Cubical Agda, univalence
% ua
is not an axiom but a theorem.

% Why circles are points with K

% Why circles are not points with univalence


\subsection{The Structure Identity Principle}\label{sec:leibniz}
To give an understanding of the basics of Cubical Agda \cite{cuagda} and the Structure Identity Principle (SIP), we walk through the steps to transport proofs about addition on Peano naturals to Leibniz naturals. We give an overview of some features of Cubical Agda, such as that paths give the primitive notion of equality, until the simplified statement of univalence. We do note that Cubical Agda has two downsides relating to termination checking and universe levels, which we encounter in later sections.

Starting by defining the unary Peano naturals and the binary Leibniz naturals, we prove that they are isomorphic by interpreting them into eachother. We explain that these interpretations are easily seen to be mutual inverses by proving lemmas stating that both interpretations ``respect the constructors'' of the types. Next, we demonstrate how this isomorphism can be promoted into an equivalence or an equality, and remark that this is sufficient to transport intrinsic properties, such as having decidable equality, from one natural to the other.

Noting that transporting unary addition to binary addition is possible but not efficient, we define binary addition while ensuring that it corresponds to unary addition. We present a variant on refinement types as a syntax to recover definition from chains of equality reasoning, allowing one to rewrite definitions while preserving equalities.

We clarify that to transport proofs referring to addition from unary to binary naturals, we indeed require that these are meaningfully related. Then, we observe that in this instance, the pairs of ``type and operation'' are actually equated as magmas, and explain that this is an instance of the SIP.

Finally, we describe the use case of the SIP, how it generalizes our observation about magmas, and how it can calculate the minimal requirements to equate to implementations of an interface. This is demonstrated by transporting associativity from unary addition to binary addition, noting that this would save many lines of code provided there is much to be transported.

\towrite{Merge}

Let us quickly review the small set of features in Cubical Agda that we will be using extensively throughout this article.\footnote{\cite{cuagda} gives a comprehensive introduction to cubical agda.}

%Of course, this downside is more than offset by the benefits of changing our primitive notion of equality, which we will see makes it easier to show that ``equivalent'' structures behave identically. 
In Cubical Agda, the primitive notion of equality arises not (directly) from the indexed inductive definition we are used to, but rather from the presence of the interval type \AgdaPrimitiveType{I}. This type represents a set of two points \AgdaInductiveConstructor{i0} and \AgdaInductiveConstructor{i1}, which are considered ``identified'' in the sense that they are connected by a path. To define a function out of this type, we also have to define the function on all the intermediate points, which is why we call such a function a ``path''. Terms of other types are then considered identified when there is a path between them.

While the benefits are overwhelming for us, this is not completely without downsides, such as that
%\ExecuteMetaData[Tex/CubicalAndBinary]{cubical}% \todo{Not sure if it would be helpful to have a more extensive introduction covering all features used.} % at this moment, probably not, as the cubical usage is rather tame, so I'll probably stick to introducing stuff as it becomes necessary. % TODO then write that somewhere
the negation of axiom K complicates both some termination checking and some universe levels.\footnote{In particular, this prompts rather far-reaching (but not fundamental) changes to the code of previous work, such as to that of \cite{progorn} in \autoref{sec:userfriendly}.} Furthermore, if we use certain homotopical constructions, like set quotients, we will also have to prove that our types are sets, before we can use them.

On the positive side, this different perspective gives intuitive interpretations to some proofs of equality, like
\ExecuteMetaData[Tex/CubicalAndBinary]{sym}
where \AgdaFunction{∼\_} is the interval reversal, swapping \AgdaInductiveConstructor{i0} and \AgdaInductiveConstructor{i1}, so that \AgdaFunction{sym} simply reverses the given path.

Furthermore, because we can now interpret paths in records and function differently, we get a host of ``extensionality'' for free. For example, a path in $A \to B$ is indeed a function which takes each $i$ in \AgdaPrimitiveType{I} to a function $A \to B$. Using this, function extensionality becomes tautological 
\ExecuteMetaData[Tex/CubicalAndBinary]{funExt}

Finally, while in ``non-univalent'' Agda bijections or isomorphisms do not play such a central role, much of our work will rest on equivalences, as the ``HoTT-compatible'' generalization of bijections. This is because the \AgdaPrimitiveType{Glue} type tells us that equivalent types fit together in a new type, in a way that guarantees univalence
\ExecuteMetaData[Tex/CubicalAndBinary]{ua}
This essentially states that ``equivalent types are identified'', such that type isomorphisms like $1 \to A \simeq A$ actually become paths $1 \to A \equiv A$, making it so that we can transport proofs along them. We will demonstrate this by a slightly more practical example in the next section.


\subsection{Binary numbers}\label{ssec:binary}
Let us demonstrate an application of univalence by exploiting the equivalence of the ``Peano'' naturals and the ``Leibniz'' naturals. Recall that the Peano naturals are defined as 
\ExecuteMetaData[Tex/CubicalAndBinary]{Peano}
This definition enjoys a simple induction principle and has many proofs of its properties in standard libraries. However, it is too slow to be of practical use: most arithmetic operations defined on \bN{} have time complexity in the order of the value of the result.

Of course, the alternative are the more performant binary numbers: the time complexities for binary numbers are usually logarithmic in the resultant values. However, the number of cases for a proof about binary numbers also grows quicker than it would for unary numbers. This does not have to be a problem, because the \bN{} naturals and the binary numbers should be equivalent after all!

Let us make this formal. We define the Leibniz naturals as follows:
\ExecuteMetaData[Leibniz/Base.tex]{Leibniz}
Here, the \AgdaInductiveConstructor{0b} constructor encodes 0, while the \AgdaInductiveConstructor{\_1b} and \AgdaInductiveConstructor{\_2b} constructors respectively add a 1 and a 2 bit, under the usual interpretation of binary numbers:
\ExecuteMetaData[Leibniz/Base.tex]{toN}
This defines one direction of the equivalence from \bN{} to \bL{}, for the other direction, we can interpret a number in \bN{} as a binary number by repeating the successor operation on binary numbers:
\ExecuteMetaData[Leibniz/Base.tex]{bsuc}
\ExecuteMetaData[Leibniz/Base.tex]{fromN}
To show that \AgdaFunction{toℕ} is an isomorphism, we have to show that it is the inverse of \AgdaFunction{fromℕ}. By induction on \bL{} and basic arithmetic on \bN{} we see that
\ExecuteMetaData[Leibniz/Properties.tex]{toN-suc}
so \AgdaFunction{toℕ} respects successors. Similarly, by induction on \bN{} we get
\ExecuteMetaData[Leibniz/Properties.tex]{fromN-1}
and % I can't get the code blocks to stick together lol
\ExecuteMetaData[Leibniz/Properties.tex]{fromN-2}
so that \AgdaFunction{fromℕ} respects even and odd numbers. We can then prove that applying \AgdaFunction{toℕ} and \AgdaFunction{fromℕ} after each other is the identity by repeating these lemmas
\ExecuteMetaData[Leibniz/Properties.tex]{N-iso-L}
This isomorphism can be promoted to an equivalence
\ExecuteMetaData[Leibniz/Properties.tex]{N-equiv-L}
which, finally, lets us identify \bN{} and \bL{} by univalence
\ExecuteMetaData[Leibniz/Properties.tex]{N-is-L}
The path \AgdaFunction{ℕ≡L} then allows us to transport properties from \bN{} directly to \bL{}, e.g.,
\ExecuteMetaData[Leibniz/Properties.tex]{isSetL}
This can be generalized even further to transport proofs about operations from \bN{} to \bL{}. 

\subsection{Use as definition: functions from specifications}\label{ssec:useas}
As an example, we will define addition of binary numbers. We could transport binary operations
\ExecuteMetaData[Extra/Algebra]{BinOp}
to get
\ExecuteMetaData[Tex/CubicalAndBinary]{badplus}
but this would be rather inefficient, incurring an $O(n + m)$ overhead when adding $n + m$. It is more efficient to define addition on \bL{} directly, making use of the binary nature of \bL{}, while agreeing with the addition on \bN{}. Such a definition can be derived from the specification ``agrees with \AgdaFunction{\_+\_}'', so we implement the following syntax for giving definitions by equational reasoning, inspired by the ``use-as-definition'' notation from \cite{calcdata}:
\ExecuteMetaData[Prelude/UseAs.tex]{Def}
which infers the definition from the right endpoint of a path using an implicit pair type
\ExecuteMetaData[Prelude/UseAs.tex]{isigma}
% \investigate{As of now, I am unsure if this reduces the effort of implementing a coherent function, or whether it is more typically possible to give a smarter or shorter proof by just giving a definition and proving an easier property of it\footnote{I will put the alternative in the appendix for now}}

With this we can define addition on \bL{} and show it agrees with addition on \bN{} in one motion
\ExecuteMetaData[Leibniz/Properties.tex]{plus-def}
Now we can easily extract the definition of \AgdaFunction{plus} and its correctness with respect to \AgdaFunction{\_+\_} 
\ExecuteMetaData[Leibniz/Properties.tex]{plus-good}

We remark \AgdaFunction{Def} is close in concept to refinement types\footnote{À la \href{https://agda.github.io/agda-stdlib/Data.Refinement.html}{Data.Refinement}.}, but importantly, the equality proof is relevant for us, and the value is inferred rather than given. \footnote{Unfortunately, normalizing an application of a \AgdaFunction{defined-by} function also causes a lot of unnecessary wrapping and unwrapping, so \AgdaFunction{Def} is mostly only useful for presentation.} %for now..


\subsection{Structure Identity Principle}
Now \bN{} with \AgdaFunction{N.+} form, in particular, a magma. The same goes for \bL{} and \AgdaFunction{plus}, but notice that a path in a \AgdaDatatype{Σ} type is just a \AgdaDatatype{Σ} of paths! This means that we get a path from (\bN{}, \AgdaFunction{N.+}) to (\bL{}, \AgdaFunction{plus}). More generally, a magma is simply a type $X$ with some structure, which is a function $f: X \to X \to X$ in the case of a magma. We can see that paths between magmas correspond to paths $p$ between the underlying types $X$ and paths over $p$ between their operations $f$. This observation is further generalized by the Structure Identity Principle (SIP), formalized in \cite{iri}. Given a structure, which in our case is just a binary operation
\ExecuteMetaData[Extra/Algebra.tex]{MagmaStr}
this principle produces an appropriate definition ``structured equivalence'' $\iota$. The $\iota$ is such that if structures $X, Y$ are $\iota$-equivalent, then they are identified. In the case of \AgdaFunction{MagmaStr}, the $\iota$ asks us to provide something with the same type as \AgdaFunction{plus-coherent}, so we have just shown that the \AgdaFunction{plus} magma on \bL{}
\ExecuteMetaData[Leibniz/Properties.tex]{magmaL}
and the \AgdaFunction{\_+\_} magma on \bN{} and are identical
\ExecuteMetaData[Leibniz/Properties.tex]{magma-equal}
As a consequence, properties of \AgdaFunction{\_+\_} directly yield corresponding properties of \AgdaFunction{plus}. For example,
\ExecuteMetaData[Leibniz/Properties.tex]{assoc-transport}



\subsection{Numerical representations}
\towrite{Generalizing the observation that lists look like unary naturals and Braun trees look like binary naturals.}

\subsection{Generic programming and ornaments}
\towrite{Taking the writing out of our hands, formalizing the ``looks like relation''.}

