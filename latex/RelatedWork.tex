


\section{Generic Path types}
We could, given \todo{Sigma-descriptions}, compute finite types, as an instance of the more general Path types, from the number system underlying a numerical representation. We could then define a generic and polymorphic lookup function for numerical representations.

Diehl and Sheard \cite{glookup} derive Path types for infinitary inductive-recursive types, which come with a generic lookup function with a computation return type.




\begin{comment}
\section{Descriptions and ornaments}
We compare our implementation to a selection of previous work, considering the following features


\begin{tabular}{c | c c c c c}
                & Haskell        & \cite{initenough} & \cite{levitation} & \cite{algorn} & \cite{progorn} \\
    \hline                                                                                             
    Fixpoint & yes*           & yes               & no                & yes?          & yes            \\
    Index    & —              & —                 & first**           & equality      & first          \\
    Poly     & yes            & 1                 & external          & external      & external       \\
    Levels   & —              & —                 & no                & no            & no             \\
    Sums     & list           & —                 & large             & large         & large          \\
    IndArg   & any            & any               & $\dots \to X\ i$  & $X\ i$        & $X\ i$         \\
    Compose  & yes            & yes               & no                & no            & no             \\
    Extension& —              & —                 & no                & —             & —              \\
    Ignore   & —              & —                 & —                 & —             & —              \\
    Set      & —              & —                 & —                 & —             & —              \\
\end{tabular}


\begin{tabular}{c | c c c c c}
                & \cite{sijsling} & \cite{effectfully} & \cite{practgen} & Shallow   & Deep (old) \\
    \hline   
    Fixpoint & yes             & yes                & no              & yes       & yes     \\
    Index    & equality        & equality           & equality        & equality  &         \\
    Poly     & telescope       & external           & telescope       & telescope &         \\
    Levels   & no***           & cumulative         & Typeω           & Type-in-Type &         \\
    Sums     & list            & large              & list            & list      &         \\
    IndArg   & $X\ pv\ i$      & $\dots\to X\ v\ i$ & $\dots\to X\ pv\ i$ & $X (f pv) i$ & ?1 \\
    Compose  & no              & yes?2              & no              & yes       &         \\
    Extension& —               & yes                & yes             & no        &         \\
    Ignore   & no              & ?                  & ?               & transform &         \\
    Set      & no              & no                 & no              & no        & yes     \\
\end{tabular}





\begin{itemize}
    \item IndArg: the allowed shapes of inductive arguments. Note that none other than Haskell, higher-order functors, and potentially ?1, allow full nested types!
    \item Compose: can a description refer to another description?
    \item Extension: do inductive arguments and end nodes, and sums and products coincide through a top-level extension?
    \item Ignore: can subsequent constructor descriptions ignore values of previous ones? (Either this, or thinnings, are essential to make composites work)
    \item Set: are sets internalized in this description?
\end{itemize}

\begin{itemize}
    \item[*] These descriptions are ``coinductive'' in that they can contain themselves, so the ``fixpoint'' is more like a deep interpretation.
    \item[**] This has no fixpoint, and the generalization over the index is external.
    \item[***] But you could bump the parameter telescope to Typeω and lose nothing.
    \item[*4] A variant keeps track of the highest level in the index.
    \item[?1] Deeply encoding all involved functors would remove the need for positivity annotations for full nested types like in other implementations.
    \item[?2] The ``simplicity'' of this implementation, where data and constructor descriptions coincide, automatically allows composite descriptions.
\end{itemize}

We take away some interesting points from this:
\begin{itemize}
    \item Levels are important, because index-first descriptions are incompatible with ``data-cumulativity'' when not emulating it using equalities! (This results in datatypes being forced to have fields of a fixed level).
    \item Coinductive descriptions can generate inductive types!
    \item Typeω descriptions can generate types of any level!
    \item Large sums do not reflect Agda (a datatype instantiated from a derived description looks nothing like the original type)! On the other hand, they make lists unnecessary, and simplify the definition of ornaments as well.
    \item We can group/collapse multiple signatures into one using tags, this might be nice for defining generic functions in a more collected way.
    \item Everything becomes completely unreadable without opacity.
\end{itemize}

\subsection{Merge me}


\subsubsection{Ornamentation}
While we can derive datastructures from number systems by going through their index types \cite{calcdata}, we may also interpret numerical representations more literally as instructions to rewrite a number system to a sequence type. We can record this transformation internally using ornaments, which can then be used to derive the associated type of arrays\cite{algorn}, or can be modified further to naturally integrate other constraints, e.g., ordering, into the resulting structure \cite{progorn}. Furthermore, we can also use the forgetful functions induced by ornaments to generate specifications for functions defined on the ornamented types \cite{orntrans}.

\subsubsection{Generic constructions}
Being able to define a datatype and reflect its structure in the same language opens doors to many more interesting constructions \cite{practgen}; a lot of ``recipes'' we recognize, such as defining the eliminators for a given datatype, can be formalized and automated using reflection and macros. We expect that other type transformations can also be interpreted as ornaments, like the extraction of heterogeneous binary trees from level-polymorphic binary trees \cite{hetbin}. 


\subsection{Takeways}
At the very least, descriptions will need sums, products, and recursive positions as well. While we could use coinductive descriptions, bringing normal and recursive fields to the same level, we avoid this as it also makes ornaments a bit more wild\footnote{For better or worse, an ornament could refer to a different ornament for a recursive field.}. We represent indexed types by parametrizing over a type $I$. Since we are aiming for nested types, external polymorphism\footnote{E.g., for each type $A$ a description of lists of $A$ à la \cite{progorn}} does not suffice: we need to let descriptions control their contexts.

We describe parameters by defining descriptions relative to a context. Here, a context is a telescope of types, where each type can depend on all preceding types:
\[ \dots \]
Much like the work Escot and Cockx \cite{practgen} we shove everything into \AgdaFunction{Typeω}, but we do not (yet) allow parameters to depend on previous values, or indices on parameters\footnote{I do not know yet what that would mean for ornaments.}.

We use equalities to enforce indices, simply because index-first types are not honest about being finite, and consequently mess up our levels. For an index type and a context a description represents a list of constructors:
\[ \dots \]
These represent lists of alternative constructors, which each represent a list of fields:
\[ \dots \]
We separate mere fields from ``known'' fields, which are given by descriptions rather than arbitrary types. Note that we do not split off fields to another description, as subsequent fields should be able to depend on previous fields
\[ \dots. \]


We parametrize over the levels, because unlike practical generic, we stay at one level.

Q: what happens when you precompose a datatype with a function? E.g. (List . f) A = List (f A) 

Q: practgen is cool, compact, and probably necessary to have all datatypes. Note that in comparison, most other implementations (like Sijsling) do not allow functions as inductive arguments. Reasonably so.

Q: I should probably update my Agda and make use of the new opaque features to make things readable when refining


\towrite{Adapt this to the non-proposal form.}

\section{The Structure Identity Principle}
If we write a program, and replace an expression by an equal one, then we can prove that the behaviour of the program can not change. Likewise, if we replace one implementation of an interface with another, in such a way that the correspondence respects all operations in the interface, then the implementations should be equal when viewed through the interface. Observations like these are instances of ``representation independence'', but even in languages with an internal notation of type equality, the applicability is usually exclusive to the metatheory.

In our case, moving from Agda's ``usual type theory'' to Cubical Agda, \textit{univalence} \cite{cuagda} lets us internalize a kind of representation independence known as the Structure Identity Principle \cite{iri}, and even generalize it from equivalences to quasi-equivalence relations. 
%a cubical homotopy type theory,
We will also be able to apply univalence to get a true ``equational reasoning'' for types when we are looking at numerical representations.

Still, representation independence in may be internalized outside the homotopical setting in some cases \cite{tgalois}, and remains of interest in the context of generic constructions that conflict with cubical type theory.

\section{Numerical Representations}
Rather than equating implementations after the fact, we can also ``compute'' datastructures by imposing equations. In the case of container types, one may observe similarities to number systems \cite{purelyfunctional} and call such containers numerical representations. One can then use these representations to prototype new datastructures that automatically inherit properties and equalities from their underlying number systems \cite{calcdata}.

From another perspective, numerical representations run by using representability as a kind of ``strictification'' of types. %This suggests that we may be able to generalize the approach of numerical representations, using that any (non-indexed) infinitary inductive-recursive type supports a lookup operation \cite{glookup}.

\end{comment}
