While the practical applications of the last example do not stretch very far\footnote{Considering that \AgdaDatatype{ℕ} is a candidate to be replaced by a more suitable unsigned integer type when compiling to Haskell anyway.}, the approach generalizes to the more relevant containers and their associated laws.

In the same vein as the last section, we could define a simple but inefficient array type, and a more efficient implementation using trees. Then we can show that these are equivalent, such that when the simple type satisfies a set of laws, trees will satisfy them as well. We could then start developing all sorts of complex implementations fine-tuned to each operation and figure out how these are equivalent to some simpler type, but let us first take a step back, and investigate how we can make this approach run smoothly in a simpler example.

Rather than inductively defining a container and then showing that it is represented by a lookup function, we can go the other way around and define a type by insisting that it is equivalent to such a function. This approach, in particular the case in which one calculates a container with the same shape as a numeral system, was dubbed numerical representations in \cite{purelyfunctional}, and has some formalized examples in, e.g., \cite{calcdata} and \cite{progorn}. Numerical representations form the starting point for defining more complex datastructures based on simpler ones, so let us demonstrate such a calculation. 

\subsection{Numerical representations: from numbers to containers}\label{ssec:numrep}
We can compute the type of vectors starting from \bN{}.\footnote{This is adapted (and fairly abridged) from \cite{calcdata}} For simplicity, we define them as a type computing function via the ``use-as-definition`` notation from before. We expect vectors to be represented by 
\ExecuteMetaData[Tex/NumRep]{Lookup}
where we use the finite type \AgdaDatatype{Fin} as an index into vector. Using this representation as a specification, we can compute both \AgdaDatatype{Fin} and a type of vectors. The finite type can be computed from the evident definition
\ExecuteMetaData[Tex/NumRep]{Fin-def}
using
\ExecuteMetaData[Tex/NumRep]{leq-split}
Likewise, vectors can be computed by applying a sequence of type isomorphisms
\ExecuteMetaData[Tex/NumRep]{Vec}
\investigate{SIP doesn't mesh very well with indexed stuff, does HSIP help?}
Of course, a container would not be of much use without lookup functions, so we define an interface
\ExecuteMetaData[Tex/NumRep]{Array}
which at the very least has to satisfy laws like
\ExecuteMetaData[Tex/NumRep]{Laws}
We could directly show that \AgdaDatatype{Vec} satisfies this, but now that we defined \AgdaDatatype{Vec} from \AgdaDatatype{Lookup} we might as well use this fact.

The implementation of arrays as functions is very straightforward
\ExecuteMetaData[Tex/NumRep]{FunArray}
and clearly satisfies our interface
\ExecuteMetaData[Tex/NumRep]{FunLaw}
We can implement arrays based on \AgdaDatatype{Vec} as well
\ExecuteMetaData[Tex/NumRep]{VecArray}
and again, we can transport the proofs from \AgdaDatatype{Lookup} to \AgdaDatatype{Vec}.\footnote{Except in this oversimplified case the laws are trivial for \AgdaDatatype{Vec} as well.}\todo{If one was determined to cobble together the path over path over path we need now.}
\investigate{As you can see, taking ``use-as-definition'' too literally prevents Agda from solving a lot of metavariables.}

\investigate{This computation can of course be generalized to any arity zeroless numeral system; unfortunately beyond this set of base types, this ``straightforward'' computation from numeral system to container loses its efficacy. In a sense, the n-ary natural numbers are exactly the base types for which the required steps are convenient type equivalences like $(A + B) \to C = (A \to C) \times (B \to C)$?}

%\subsection{Relating types by structure: Ornamentation (unfinished)}\label{sec:ornament}
\subsection{Numerical representations as ornaments}\label{ssec:ornaments}
We could peform the same computation for \bL{}, which would yield the type of binary trees, but we note that these computations proceed with roughly the same pattern: each constructor of the numeral system gets assigned a value, and is amended with a field holding a number of elements and subnodes using this value as a ``weight''. But wait! Such modifications of constructors are already made formal by the concept of ornamentation!\todo{It seems like Agda forgets that we defined Leibniz if we move between different .tex files, so I'll have to float all AgdaTargets to the top level at some point...}

Ornamentation, as exposed in \cite{algorn} and \cite{progorn}, lets us formulate what it means for two types to have a ``similar'' recursive structure. This is achieved by interpreting (indexed inductive) datatypes from descriptions, between which an ornament is seen as a certificate of similarity, describing which fields or indices need to be introduced or dropped. Furthermore, a one-sided ornament: an ornamental description, lets us describe new datatypes by recording the modifications to an existing description.
\todo{Again not sure how much space I should use to reiterate Desc, Orn, and OrnDesc.}

This links back to the construction in the previous section, since \bN{} and \AgdaDatatype{Vec} share the same recursive structure, so \AgdaDatatype{Vec} can be formed by introducing indices and adding a field holding an element at each node.\footnote{These and similar examples are also documented in \cite{progorn}} 

However, instead deriving \AgdaDatatype{List} from \bN{} generalizes to \bL{} with less notational overhead, so lets tackle that case first. For this, we have to give a description of \bN{} to work with\todo{Clearly this can use more explanation (the question is, how much?)}
\ExecuteMetaData[Tex/NumRepOrn]{NatD}
Recall that \AgdaInductiveConstructor{σ} adds a field, upon which the rest of the description may vary, and \AgdaInductiveConstructor{ṿ} lists the recursive fields and their indices (which can only be \AgdaInductiveConstructor{tt}).
We can now write down the ornament which adds fields to the \AgdaFunction{suc} constructor
\ExecuteMetaData[Tex/NumRepOrn]{ListO}
Here, the \AgdaInductiveConstructor{σ} and \AgdaInductiveConstructor{ṿ} are forced to match those of \AgdaDatatype{NatD},
but the \AgdaInductiveConstructor{Δ} adds a new field. With the least fixpoint and description extraction from \cite{progorn}, this is sufficient to define \AgdaDatatype{List}. Note that we cannot hope to give an unindexed ornament from \bL{}
\ExecuteMetaData[Tex/NumRepOrn]{LeibnizD}
into trees, since trees have a very different recursive structure! Instead, we must keep track at what level we are in the tree so that we can ask for adequately many elements:
\ExecuteMetaData[Tex/NumRepOrn]{TreeO}
We use the \AgdaFunction{power} combinator to ensure that the digit at position $n$, which has weight $2^n$ in the interpretation of a binary number, also holds its value times $2^n$ elements. This makes sure that the number of elements in the tree shaped after a given binary number also is the value of that  binary number.

This ``folding in'' technique using the indices to keep track of structure seems to apply more generally. Let us explore this a bit further, and return later to the generalization of the pattern from numeral systems to datastructures.
% i.e. why did this even work?

\subsection{Folding in}\label{ssec:flattening}
Let us describe this procedure of folding a complex recursive structure into a simpler structure more generally. In particular, we will demonstrate that for linear datatypes, such as \bN{} and \bL{}, and for a given unindexed datatype, there is always an indexed datatype isomorphic to it at some index, and an ornament from the linear type to the indexed type. 

Suppose we are given a description, the first thing we can do to simplify it is collect all fields in one place
\ExecuteMetaData[Tex/Flatten]{RField}
Next, we will certainly have to count the number of recursive occurrences we are tracking, so we define
\ExecuteMetaData[Tex/Flatten]{Number}
where \AgdaInductiveConstructor{𝟙} records that we are at the top level, and \AgdaInductiveConstructor{ṿ} denotes that we are below a constructor with some number of recursive fields. This simplifies our task to implementing the types in
\ExecuteMetaData[Tex/Flatten]{nested}
such a way that we get an isomorphism 
\ExecuteMetaData[Tex/Flatten]{wish}
Thus, \AgdaDatatype{Fields} is forced to have a \AgdaInductiveConstructor{leaf} constructor like 
\ExecuteMetaData[Tex/Flatten]{Fields}
if \AgdaFunction{nested} is to work at \AgdaInductiveConstructor{𝟙}. The \AgdaInductiveConstructor{node} constructor makes sure that if we have collection of \AgdaDatatype{Fields}, then we can gather them in a field at a higher level. We can then count the subnodes of a given \AgdaDatatype{Fields} as
\ExecuteMetaData[Tex/Flatten]{subnodes}
where \AgdaFunction{RSize} counts the number of recursive fields of a particular branch
\ExecuteMetaData[Tex/Flatten]{RSize}
Note that \AgdaFunction{subnodes} effectively keeps the shape of the previous field, but unfolds the recursive fields at the bottom of the tree by one level.

\investigate{I then tried and realized how unpleasant even the functions from the original type to the nested type are to write.}

As a trivialty, we get that any type, interpreted as a container, always decomposes as an ornament over a ``numerical'' base type.\todo{Or at least, that was where I was trying to go with this, but I notice that this still is a bit further away.} This links to the construction of binary heaps in \cite{progorn}, as in hindsight, starting from the usual binary heaps would yield binary numbers and their binary heap ornament (in a much less useful package).