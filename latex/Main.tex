\documentclass[10pt]{article}

\usepackage[style=alphabetic]{biblatex}
\addbibresource{refs.bib}

\usepackage{comment}

\setlength{\marginparwidth}{2cm} % this makes \todo{}s explode less, remove when done

\usepackage{todonotes}
\usepackage{xcolor}
\usepackage[hidelinks]{hyperref}


\usepackage{catchfilebetweentags}
\usepackage{quiver} 
\usepackage{tabularx}
\usepackage{amsthm}
\usepackage{amsmath}
\usepackage{listings}

\theoremstyle{plain}
\newtheorem{theorem}{Theorem}[section]
\newtheorem{lemma}[theorem]{Lemma}
\newtheorem{prop}[theorem]{Proposition}
\newtheorem{conjecture}{Conjecture}[section]
\newtheorem*{cor}{Corollary}

\theoremstyle{definition}
\newtheorem{defn}{Definition}[section]
\newtheorem{remark}{Remark}[section]
\newtheorem{claim}{Claim}[section]
\newtheorem{example}{Example}[section]

\renewcommand{\partautorefname}{Part}
\renewcommand{\sectionautorefname}{Section}
\renewcommand{\subsectionautorefname}{Subsection}

\providecommand{\theoremautorefname}{Theorem}
\providecommand{\lemmaautorefname}{Lemma}
\providecommand{\propautorefname}{Proposition}
\providecommand{\conjectureautorefname}{Conjecture}
\providecommand{\corautorefname}{Corollary}
\providecommand{\defnautorefname}{Definition}
\providecommand{\remarkautorefname}{Remark}
\providecommand{\exampleautorefname}{Example}
\providecommand{\claimautorefname}{Claim}


\usepackage[links]{agda}
\AgdaNoSpaceAroundCode{}

\usepackage{fontspec}
\usepackage{luaotfload}

\directlua{luaotfload.add_fallback
  ("myfallback",
    { "JuliaMono:style=Regular;"
    , "NotoSansMono:style=Regular;"
    , "NotoSansMath:style=Regular;"
    , "Segoe UI Emoji:mode=harf;"
    }
  )}
\defaultfontfeatures{RawFeature={fallback=myfallback}}

\setmainfont{Latin Modern Roman}

\newfontfamily{\AgdaSerifFont}{JuliaMono Regular}[Scale=0.8]
\newfontfamily{\AgdaSansSerifFont}{JuliaMono Regular}[Scale=0.8]
\newfontfamily{\AgdaTypewriterFont}{JuliaMono Regular}[Scale=0.8]
\setmonofont{JuliaMono Regular}[Scale=0.8]
\renewcommand{\AgdaFontStyle}[1]{{\AgdaSansSerifFont{}#1}}
\renewcommand{\AgdaKeywordFontStyle}[1]{{\AgdaSansSerifFont{}#1}}
\renewcommand{\AgdaStringFontStyle}[1]{{\AgdaTypewriterFont{}#1}}
\renewcommand{\AgdaCommentFontStyle}[1]{{\AgdaTypewriterFont{}#1}}
\renewcommand{\AgdaBoundFontStyle}[1]{\textit{\AgdaSerifFont{}#1}}

\newcommand{\AF}[1]{\AgdaFunction{#1}}
\newcommand{\AD}[1]{\AgdaDatatype{#1}}
\newcommand{\AR}[1]{\AgdaRecord{#1}}
\newcommand{\AV}[1]{\AgdaBoundFontStyle{#1}}
\newcommand{\AIC}[1]{\AgdaInductiveConstructor{#1}}
\newcommand{\ARF}[1]{\AgdaField{#1}}

\definecolor{git-green}{HTML}{13A10E}
\definecolor{git-orange}{HTML}{C69026}
\definecolor{nondescriptyellow}{HTML}{D6B656}

%\newcommand{\added}[1]{\textcolor{git-green}{+#1}}
%\newcommand{\changed}[1]{\textcolor{git-orange}{$\bullet$#1}}
\newcommand{\towrite}[1]{\todo[color=cyan]{#1}}
\newcommand{\marker}[1]{\todo[color=green]{#1}}
\newcommand{\lowprio}[1]{\todo[color=nondescriptyellow]{#1}}
%\newcommand{\toremove}[1]{\textcolor{red}{This is going to be (re)moved: ``#1''}}

  
\lstnewenvironment{semicomment}{\color{gray}\lstset{breaklines=true}}{}
\lstnewenvironment{outline}{\color{nondescriptyellow}\lstset{breaklines=true}}{}

\newcommand{\investigate}[1]{\par\vspace{1\baselineskip}\textcolor{gray}{#1}\vspace{1\baselineskip}\par}

\newcommand{\bN}{\AgdaDatatype{ℕ}}
\newcommand{\bL}{\AgdaDatatype{Leibniz}}

\title{Generic Numerical Representations via Ornaments}
\author{Samuel Klumpers\\6057314}

%\usepackage{subfiles}
%example: latexmk -pdf -use-make -lualatex -halt-on-error -synctex=1 Introduction.tex

\begin{document}
\maketitle


\begin{comment}
    The dependently typed functional programming language Agda encourages defining custom datatypes to write correct-by-construction programs with\todo{long}. In some cases, even those datatypes can be made correct-by-construction, by manually distilling them from a mixture of requirements, as opposed to pulling them out of thin air\todo{distracted}. This is in particular the case for numerical representations, a class of datastructures inspired by number systems, containing structures such as linked lists and binary trees. However, constructing datatypes in this manner, and establishing the necessary relations between them can quickly become tedious and duplicative.

    In the general case, employing datatype-generic programming can curtail code-duplication by allowing the definition of constructions that can be instantiated to a class of types. Furthermore, ornaments make it possible to succinctly describe relations between structurally similar types.
    
    In this thesis, we apply generic programming and ornaments to numerical representations, giving a recipe to compute such a representation from a provided number system.
    For this, we describe a generic universe and a type of ornaments on it, allowing us to formulate the recipe as an ornament from a number system to the computed datatype.
    
    
    Todo legend:
    \todo{To do}
    \towrite{There is something missing here}
    \lowprio{General and vague self-criticism, might ignore. Might do something if I have time on my hands.}
\end{abstract}


\begin{comment}
    This thesis explains the concepts numerical representations and ornaments, and aims to combine these to simplify the presentation and verification of finger trees. We demonstrate the generalizability and easier verification of the resulting code. Further, we also investigate to which extent descriptions and ornaments, and generic programs built on top of these, remain effective in a setting without axiom K.
\end{comment}


\newpage
\tableofcontents

\listoftodos


\section{Introduction}\label{sec:introduction}
\todo[inline, color=red]{Spend some more time on the context}
Agda \cite{agda} is a functional programming language and a proof assistant, taking inspiration from languages like Haskell and other proof assistants such as Coq. We can write programs as we would in Haskell, and then express and prove their properties all inside Agda. This allows us to demonstrate the correctness of programs by formal proof rather than by testing. 

However, this level of formality also trades-off the uncertainty of testing for a time-investment to produce these proofs. In this thesis, we will explore a variety of methods of proving properties of our programs, focussing on the problems that one may encounter, presenting solutions as they arise. Let us sketch some of these problems.

First, merely adapting a program to Agda may already require changes to the datatypes used in it; for example, if a program manipulating a \AgdaDatatype{List} uses the unsafe \AgdaFunction{head} function, then one is forced to replace the \AgdaDatatype{List} by a datatype that ensures non-emptiness, such as a \AgdaDatatype{NonEmpty} list or a length-aware vector \AgdaDatatype{Vec}. On the other hand, there might be sections of a program where the concrete length is not relevant for correctness and only gets in the way. As a result, one might find themselves duplicating common functions like concatenation \AgdaFunction{\_++\_} to only alter their signatures.

However, the ``new'' datatype (\AgdaDatatype{Vec}) is typically a simple variation on the old datatype (\AgdaDatatype{List}) making small adjustments to the existing constructors; in this case, we decorate the nil and cons constructors with natural numbers representing the length. This kind of modification of types falls in the framework of ornamentation as described by Ko and Gibbons \cite{progorn}; if two types are reified to their \textit{descriptions}, then \textit{ornaments} express whether the types are ``similar'' by acting as a recipe to produce one type from the other. By restricting the operations to the copying of corresponding parts, and the introduction of fields or dropping of indices, the existence of such an ornament ensures that the types have the same recursive structure.

\towrite{Something about patches.}

\towrite{For each invariant a new datatype? Still ornaments}

Now that we know we can organize similar datatypes using ornaments, it is time to look at dissimilar datatypes. It is conventional to prototype a program using simpler types or implementations, and only replace these with more performant alternatives in critical places; knowing that this is eventually going to happen, one might as well prepare for it. While this may quickly turn into a refactoring nightmare in the general case, we can hope for a more satisfying transition if we restrict our attention to a narrower scope. As an example, we might start programming using \AgdaDatatype{List}s, but replace this with a \AgdaDatatype{Tree} if we notice that the program spends most of its time in \AgdaFunction{lookup} operations. To gain a speedup, we will have to reimplement the operations on \AgdaDatatype{Tree}. This would also double the number of necessary proofs; however, we have two ways to avoid this problem. 

We will look at the more specific solution first. This solution is guided by the realization that \AgdaDatatype{List} and \AgdaDatatype{Tree}, like most other containers, still have similarities if their recursive structure is very different. That is, both resemble a number system, and, Okasaki \cite{purelyfunctional} notes that this resemblance to number systems is ``surprisingly common''. In the case of lists and Braun trees\footnote{Braun trees are a kind of binary tree, of which its shape is determined by its size.}, one can present both by deriving them from unary and binary numbers respectively, as is made formal by Hinze and Swierstra \cite{calcdata}. One can then apply this \textit{numerical representation} to simplify or make trivial the proofs of the properties we hesitated to duplicate before.

\towrite{If we instead hide our datatypes behind interfaces, we can use proof transport as an alternative.}

\towrite{Something about fingertrees, leading into the research question and proposed work}


\section{\toremove{Introduction}}
%The dependently typed functional programming language Agda \cite{agda} can, when restricted to its reasonable parts, be translated into readable and safe Haskell \cite{agda2hs}. However, the intrinsic safety of languages like Agda can also lead to code duplication by encouraging the use of multiple variants of the same datatype. As an example, the coverage check forces the \AgdaFunction{head} function on \AgdaDatatype{List} to return a \AgdaDatatype{Maybe}. This \AgdaDatatype{Maybe} can be avoided by moving to the length-indexed list type \AgdaDatatype{Vec}, at the cost of duplicating functions like \AgdaFunction{\_++\_}, which we need at both types.

Something similar happens when replacing an implementation with a more efficient one. For example, when implementing binary trees as a more efficient alternative to lists, the proofs of the same properties will differ between list and tree, and tend to be more difficult for the latter. Switching between implementations of an interface not only duplicates code, but also (and sometimes more than) doubles the effort required to verify both.\todo[inline, color=red]{concrete example?}

%There is plenty of prior work dealing with problems like these. The work in \cite{orntrans} and \cite{progorn} provides the means to relate similar datatypes, such as lists and vectors, using the mechanism of ornamentation, letting us organize variants of the same datatype in a rigid framework.  %This leads them to define the concept of patches, which can aid us when defining \AgdaFunction{\_++\_} for the second time by forcing the new version to be coherent.
%In fact, the algebraic nature of ornaments yields the definition of the vector type for free, provided we relate lists to natural numbers \cite{algorn}. %Such constructions rely heavily on descriptions of datastructures and often come with limitations in their expressiveness. These descriptions in turn impose additional ballast on the programmer, leading us to investigate reflection like in \cite{practgen} as a means to bring datatypes and descriptions closer when possible.

Other work like \cite{calcdata}\todo[inline, color=red]{don't use \textbackslash cite as noun} simplifies the proofs relating to certain containers directly, formally executing the way of though of numerical representations as noted in \cite{purelyfunctional}.
%From another point of view, lists and trees are not so different at all, provided we look at them through the interface of one-sided flexible arrays; this idea noted in \cite{purelyfunctional} and formalized in \cite{calcdata} where both are shown to be instances of numerical representations by calculating them from a numeral system. 

When two types are isomorphic and equivalent under an interface, proofs of properties of these implementations should be interconvertible. By using structured equivalences and univalence, \cite{iri} characterizes equivalences under interfaces.
%While this is achievable through meta-programming, substituting conversions to and from into the proof terms, this is internally expressible in Cubical Agda.

%We can liken the situation to movement on a plane, where ornamentation moves us vertically by modifying constructors or indices, and structured equivalences move us horizontally to and from equivalent but more equivalent implementations. In this paper, we will investigate a variety of means of moving around structures and proofs, and ways to make this more efficient or less intrusive.

In \autoref{sec:leibniz}, we will follow \cite{iri}, and look at how proofs on unary naturals can be transported to the binary naturals. Then in \autoref{sec:numrep} we recall how numeral systems in particular induce container types in \cite{calcdata}, which we attempt to reformulate in the language of ornaments in \autoref{ssec:ornaments}, using the framework of \cite{progorn}. In \autoref{sec:userfriendly} we investigate how we can make the earlier methods more easily accessible to the user, and, ourselves, when we give a description of finger trees in \autoref{sec:fingertrees}.\todo[inline, color=red]{Ok, but make the research question more concrete}




\part*{Background}\label{part:background}
\addcontentsline{toc}{part}{\nameref{part:background}}
\marker{Start A}
We extend upon existing work in the domain of generic programming and ornaments, so let us take a closer look at the nuts and bolts to see what all the concepts are about.

We will describe some common datatypes and how they can be used for programming, exploring how dependent types also let us use datatypes to prove properties of programs, or write programs that are correct-by-construction, leading us to discuss descriptions of datatypes and ornaments.

\subsection{Agda}
We formalize our work in Agda \cite{agda}, a functional programming language with dependent types. Using dependent types we can use Agda as a proof assistant, allowing us to state and prove theorems about our datastructures and programs. These proofs can then be run as algorithms, or in some cases be extracted to a Haskell program\footnote{Or JavaScript, if you want.}.

The type system of Agda is an extension of (intensional) Martin-Löf type theory (MLTT), a constructive type theory in which we can interpret intuitionistic logic. Compared to Haskell, which extends a polymorphic lambda calculus with inductive types, MLTT allows the type of the codomain of a function to vary with the values in the domain and the type of the second field of a pair type to vary with the value of the first. The interpretation of logic into programs is known as the Curry-Howard isomorphism: propositions or logical formulas are related to types, such that a term of a type constitutes a proof of the related proposition.

Syntactically Agda is similar to Haskell, with a few notable differences. One is that Agda allows most characters and words in identifiers with only a small set of exceptions. For example, we can write
\ExecuteMetaData[Tex/Snippets]{ternary}
The other is that datatypes are given either as generalized algebraic datatypes (GADTs) or record types in Haskell.

An essential semantical difference is that Agda rules out non-termination by restricting function definitions to structural recursion. The termination checker (together with other restrictions which we will encounter in due time) ensures that the logic interpreted in Agda remains consistent, and does not allow trivial proofs which would be tolerated in Haskell, like
\ExecuteMetaData[Tex/Snippets]{loop}
The propositional part of the Curry-Howard correspondence can then be formulated by the usual type formers. The atomic formulas, true and false, can be represented respectively as the empty record: there always is a proof \AgdaFunction{tt} of true
\ExecuteMetaData[Tex/Snippets]{true}
and the type with no constructors: there is no way to make a proof of false
\ExecuteMetaData[Tex/Snippets]{false}
Implication $A \implies B$ corresponds to function types $A \to B$: a proof of $A$ can be converted to a proof of $B$. Implication also gives an interpretation of negation as functions into false $A \to \bot$. Disjunction (logical or) is described by a sum type $A + B$: either of $A$ or $B$ can prove $A + B$
\ExecuteMetaData[Tex/Snippets]{either}
Conjunction (logical and) is given as a product type: having both $A$ and $B$ proves $A \times B$
\ExecuteMetaData[Tex/Snippets]{pair}
Predicates, formulas containing variables, correspond to functions into the type of formulas
\ExecuteMetaData[Tex/Snippets]{predicate}
allowing interpretations of higher-order logic. Quantifiers are interpreted via dependent types, universal quantification (for all) is a dependent function type: for each $a : A$, give a proof of $P\ a$
\ExecuteMetaData[Tex/Snippets]{forall}
Likewise, existential quantification (exists) is a dependent pair type: there is an $a : A$ and a proof $P\ a$
\ExecuteMetaData[Tex/Snippets]{exists}
Predicates can also be expressed using indexed datatypes, in which the choice of constructor can influence the index, whereas parameters must be constant over all constructors. Equality of elements of a type $A$ can then be interpreted as the type
\ExecuteMetaData[Tex/Snippets]{eq}
Closed terms of this type can only be constructed for definitionally equal elements, but crucially, variables can contain equalities between different elements. As the second argument is an index, pattern matching on \AgdaFunction{refl} unifies the elements, such that properties like substitution follow
\ExecuteMetaData[Tex/Snippets]{subst}
\towrite{Isomorphisms}

\subsection{Cubical Agda}
\towrite{Gluing everything together, making representation independence run.}

The methods described in later sections yield type isomorphisms. One might expect that like how isomorphic groups share the same group-theoretical properties, isomorphic types also share the same type-theoretical properties. Meta-theoretically, this is known as \emph{representation independence}, and is evident: if $e: A \simeq B$, simply modify the type by substituting all variables $x : B$ with $e x'$ for $x' : A$, and replacing the resulting terms $t : B$ by $e^{-1} t$. Then the proof term can be recovered by substituting along the equalities $e^{-1}(e x) \equiv x$ as needed.

Inside (ordinary) Agda this is not so practical, as this independence only holds when applied to concrete types, and is then only realized by manually performing these substitutions. On the other hand, in Cubical Agda, the Structure Identity Principle internalizes a kind of representation independence \cite{iri}.

Cubical Agda modifies the type theory of Agda to a kind of homotopy type theory, looking at equalities as paths between terms rather than the equivalence relation generated by reflexivity. In cubical type theories, the role played by pattern matching on \AgdaFunction{refl} or by axiom J, in MLTT and ``Book HoTT'' respectively, is instead acted out by directly manipulating cubes\footnote{Under the analogy where a term is a point, an equality between points is a line, a line between lines is a square.}. In Cubical Agda, univalence
% ua
is not an axiom but a theorem.

% Why circles are points with K

% Why circles are not points with univalence

\subsection{Numerical representations}
\towrite{Generalizing the observation that lists look like unary naturals and Braun trees look like binary naturals.}

\subsection{Generic programming and ornaments}
\towrite{Taking the writing out of our hands, formalizing the ``looks like relation''.}




\part{Descriptions}\label{part:descriptions}
Before we can analyse and describe number systems and their numerical representations using generic programs, we first have to ensure that these types fit into the descriptions. In this section we discuss how some numerical representations are hard to describe using only the descriptions of parametric indexed inductive types \AD{U-ix}, and based on this discussion we present an extension of \AD{U-ix} incorporating metadata, parameter transformation, description composition, and variable transformation.

%Unlike some other encodings \cite{effectfully, practgen}, we do not allow higher-order inductive arguments. %Like before, we use \texttt{--type-in-type} and \texttt{--with-K} to simplify the presentation, noting that these can be eliminated respectively by moving to \AD{Typeω} and by implementing interpretations as datatypes, as described in \autoref{app:withoutk}.
%We base the encoding of off existing encodings \cite{sijsling,practgen}. The descriptions take shape as sums of products, enforce indices at leaf nodes, and have explicit parameter and variable telescopes. 

\subsection{Numerical Representations}\label{sec:desc-numrep}
Before we start rebuilding our universe, let us look at the construction of the simplest numerical representation \AD{Vec} from \bN{}. At first, we defined \AD{Vec} as the length-indexed variant of \AD{List}, so that \AF{lookup} becomes total and satisfies nice properties like \AF{lookup-insert}. Later, we gave another description of \AD{Vec} as an ornament on top of \AD{List}. More abstractly, \AD{Vec} is an implementation of finite maps with domain \AD{Fin}, where finite maps are simply those types with operations like \AF{insert}, \AF{remove}, \AF{lookup}, and \AF{tabulate}\footnote{The function \AF{tabulate}\ \AV{:}\ (\AD{Fin}\ \AV{n}\ \AV{→}\ \AV{A})\ \AV{→}\ \AV{Vec}\ \AV{A}\ \AV{n} collects an assignment of elements \AV{f} into a vector \AF{tabulate}\ \AV{f}.}, satisfying relations or laws like \AF{lookup-insert} and \AF{lookup}\ \AF{∘}\ \AF{tabulate}\ \AD{≡}\ \AF{id}.

For comparison, we can define a trivial implementation of finite maps, by reading \AF{lookup} as a prescript:
\ExecuteMetaData[Tex/Descriptions/Numrep]{Lookup2}
Since \AF{lookup} is simply the identity function on \AF{Lookup}, this immediately satisfies the laws of finite maps, provided we define \AF{insert} and \AF{remove} correctly.

Unsurprisingly, \AD{Vec} is \emph{representable}, that is, we have that \AF{Lookup} and \AD{Vec} are equivalent, in the sense that there is an \emph{isomorphism} between \AF{Lookup} and \AD{Vec}:\footnote{Since \AF{lookup} is an isomorphism with \AF{tabulate} as inverse, as we see from the relations \AF{lookup}\ \AF{∘}\ \AF{tabulate}\ \AD{≡}\ \AF{id} and  \AF{tabulate}\ \AF{∘}\ \AF{lookup}\ \AD{≡}\ \AF{id}. Without further assumptions, we cannot use the equality type \AD{≡} for this notion of equivalence of types: a type with a different name but exactly the same constructors as \AD{Vec} would not be equal to \AD{Vec}.}
\ExecuteMetaData[Tex/Descriptions/Numrep]{Iso}
An \AD{Iso} from \AV{A} to \AV{B} is a map from \AV{A} to \AV{B} with a (two-sided) inverse\footnote{In our situation (\AV{--with-K}), this is the same as the other notion of equivalence: there is a map $f : A \to B$, and for each \AV{b} in \AV{B} there is exactly one \AV{a} in \AV{A} for which $f(a) = b$.}. In terms of elements, this means that elements of \AV{A} and \AV{B} are in one-to-one correspondence.

Rather than deriving them ourselves, we can also establish properties like \AF{lookup-insert} from this equivalence. Instead of finding the properties of \AD{Vec} that were already there, let us view \AD{Vec} as a consequence of the definition of \bN{} and \AF{lookup}. By turning the \AD{Iso} on its head, and starting from the equation that \AD{Vec} is equivalent to \AD{Lookup}, we derive a definition of \AD{Vec} as if were solving an equation \cite{calcdata}.

As a warm-up, we can also derive \AD{Fin} from the fact that \AD{Fin}\ \AV{n} should contain \AV{n} elements, and thus be isomorphic to \AV{Σ[ m ∈ ℕ ] m < n}. To express such a definition by isomorphism, we define
\ExecuteMetaData[Tex/Descriptions/Numrep]{Def}
using:
\ExecuteMetaData[Tex/Descriptions/Numrep]{isigma}
The type \AD{Def}\ \AV{A} is deceptively simple, after all, there is (up to isomorphism) only one unique term in it! However, when using \AD{Def}initions, the implicit \AD{Σ'} extracts the right-hand side of a proof of an isomorphism, allowing us to reinterpret a proof as a definition.

To keep the isomorphisms we are trying to construct readable, we construct them as chains of simpler isomorphisms, using a variant of \emph{equational reasoning} \cite{agdastdlib, plfa}, letting us compose isomorphisms while displaying the intermediate steps. In the calculation of \AD{Fin}, we will use the following lemmas:
\ExecuteMetaData[Tex/Descriptions/Numrep]{Fin-lemmas}
If we allow reading isomorphisms as ``\emph{is}'', then the terminology of \autoref{sec:background-proving}, \AF{⊥-strict} states that ``if A is false, then A \emph{is} empty'', while \AF{<-split} states that the set of numbers below $n+1$ has one more element than the set of numbers below $n$. Using these, we can calculate:\footnote{Making non-essential use of \AF{cong} for type families. In the derivation of \AD{Vec} we use function extensionality, which has to be postulated, or can be obtained by using the cubical path types.}
\ExecuteMetaData[Tex/Descriptions/Numrep]{Fin-def}
This gives a different (but equivalent) definition of \AD{Fin} compared to \AF{FinD}; the description \AF{FinD} describes \AD{Fin} as an inductive family, whereas \AF{Fin-def} describes \AD{Fin} equivalently as a type-computing function \cite{progorn}. From this \AD{Def} we can extract a definition of \AD{Fin}:
\ExecuteMetaData[Tex/Descriptions/Numrep]{Fin}
To derive \AD{Vec}, we use the isomorphisms
\ExecuteMetaData[Tex/Descriptions/Numrep]{Vec-lemmas}
which one can compare to the familiar exponential laws. With these laws, we calculate the type of vectors
\ExecuteMetaData[Tex/Descriptions/Numrep]{Vec-def}
yielding a definition of vectors and the \AD{Iso} to \AF{Lookup} in one go:
\ExecuteMetaData[Tex/Descriptions/Numrep]{Vec}
In conclusion, we computed a type of finite maps (the numerical representation \AD{Vec}) from a number system (\bN{}), by cases on the number system and making use of the values represented by the number system.


\subsection{Room for Improvement}
We could now carry on and attempt to generalize this calculation to other number systems, but we would quickly run into dead ends for certain numerical representations. Let us give an overview of what bits of \AD{U-ix} are still missing if we are going to generically construct all numerical representations we promised.

\subsubsection{Number systems}\label{ssec:numbers}
In the calculation \AD{Vec} from \bN{}, we analyze and replicate the structure of \bN{}, for which we use the meaning of these constructors as numbers assigned to them by our explanation of \bN{} in words\footnote{More accurately, the meaning of \bN{} comes from \AF{Fin}, which gets its meaning from our definition of \AF{\_<\_}.}. Based on that interpretation of constructors as numbers, we deliberately choose to add zero fields in the case corresponding to \AIC{zero} and choose to add one field in the case of \AIC{one}.

However, if we want to compute numerical representations generically, we also have to convince Agda that our datatypes indeed represent number systems. As a first step, let us fix \bN{} as the primordial number system, so that we can compare other number systems by how they are mapped into \bN{}. Trivially, \bN{} can be interpreted as a number system via \AF{id}\ \AV{:}\ \bN{}\ \AV{→}\ \bN{}.

The binary numbers, as described in the introduction, can be mapped to \bN{} by:
\ExecuteMetaData[Tex/Descriptions/Number]{Bin}
As a more exotic example, we can describe a number system
\ExecuteMetaData[Tex/Descriptions/Number]{Carpal}
which consists of smaller ``number systems'':
\ExecuteMetaData[Tex/Descriptions/Number]{Phalanx}
We could now define a general number system as a type \AV{N} equipped with a map \AV{N →}\ \bN{}, but this would both be too general for our purpose and opaque to generic programs. On the other hand, allowing only traditional positional number systems excludes number systems like \AD{Carpal}, which would otherwise still have valid numerical representations, as we will see later.

Instead, we observe that across the above examples, the interpretation of a number is computed by a simple fold. In particular, leaves have associated constants, recursive fields correspond to multiplication and addition, while fields can defer to another function. If we encode number systems in \AD{U-sop}, we can thus encode each of these systems by associating a single number to each \AIC{𝟙} and \AIC{ρ}, and a function to each \AIC{σ}. This essentially encodes number systems as trees that evaluate nodes by linearly combining values of subnodes, generalizing \emph{dense} representations of positional number systems\footnote{As a consequence, this excludes the \emph{sparse} number systems, as we discuss in \autoref{sec:discussion-no-sparse}.}.

Using a modified version of \AD{U-sop}, we can encode the examples we gave as follows. Note that in \bN{}, we insert redundant fields of \AD{⊤} in order to express that the second constructor acts as $x \mapsto x + 1$
\ExecuteMetaData[Tex/Descriptions/Number]{Nat-num}
marking all leaves as zero. The binary numbers admit a similar encoding, but multiply their recursive fields by two instead:
\ExecuteMetaData[Tex/Descriptions/Number]{Bin-num}
Finally, the \AD{Carpal} system can be encoded by using the interpretation of \AD{Phalanx}
\ExecuteMetaData[Tex/Descriptions/Number]{Carpal-num}


\subsubsection{Nested types}
If our construction is going to cast \AD{Random}, as defined in \autoref{sec:introduction}, as the numerical representation associated to \AD{Bin}, then \AD{Random} needs to have a description to begin with. However, the recursive fields of \AD{Random} are not given the parameter \AV{A}, but \AV{A}\ \AD{×}\ \AV{A}. This makes \AD{Random} a nested type, as opposed to a uniformly recursive type, in which the parameters of the recursive fields would be identical to the top-level parameters. Consequently, \AD{Random} has no adequate description in \AD{U-ix}\footnote{Here, the ``inadequate'' descriptions either hardly resemble the user defined \AD{Random}, use indices to store the depth of a node (see \autoref{app:unnested}), or only have a complicated isomorphism to \AD{Random}.}. 

Due to the work of Johann and Ghani \cite{initenough}, we can model general nested types as fixpoints of \emph{higher-order functors} (i.e., endofunctors on the category of endofunctors):
\ExecuteMetaData[Tex/Descriptions/Nested]{HMu}
By placing the recursive field \AD{Mu}\ \AV{F} under \AV{F}, the functor \AV{F} can modify \AD{Mu}\ \AV{F} and \AV{A} to determine the type of the recursive field. We can encode \AD{Random} by a \AF{HFun} as:
\ExecuteMetaData[Tex/Descriptions/Nested]{HRandom}
However, this definition is unsafe\footnote{As you might have deduced from the pragma disabling the positivity checker. Consider \AF{HBad}\ \AV{F}\ \AV{A}\ \AV{=}\ \AV{F}\ \AV{A}\ \AV{→}\ \AD{⊥}.}, so we settle for the weaker but safe inner nesting instead. This kind of nesting is described by a simple modification to the recursive field \AIC{ρ} in \AD{U-ix}
\ExecuteMetaData[Tex/Descriptions/Nested]{rho-nest}
allowing a recursive field to specify a transformation \AF{Cxf} that is applied to the parameters before they are passed to the recursive field. Correspondingly, the interpretation of \AIC{ρ} applies \AV{f} before passing \AV{p} to the recursive field \AV{X}:
\ExecuteMetaData[Tex/Descriptions/Nested]{rho-nest-int}
With this modification, \AD{Random} can be directly transcribed 
\ExecuteMetaData[Tex/Descriptions/Nested]{Random}
using the map $A \mapsto A \times A$ to describe its nesting as usual.

Of course, the nesting which is useful here is only an inconvenience for uniformly recursive types, so we define a shorthand
\ExecuteMetaData[Tex/Descriptions/Nested]{rho0}
emulating the behaviour of a uniformly recursive field.

\subsubsection{Composite types}
In \autoref{ssec:numbers}, we defined the number system \AF{Carpal-num} as a \emph{composite type}, in the sense that its description references another concrete type \AD{Phalanx}. By the same argument as there, the description \AD{Carpal-num} which relies on \AF{toℕ-Phalanx} to describe the value of \AD{Phalanx}, turns out to be too imprecise to recover the complete numerical representation generically. 

In comparison, generic programming facilities like the deriving-mechanism in Haskell allow for code like
\begin{verbatim}
{-# LANGUAGE DeriveFunctor #-}

data Two a = Two a a deriving Functor
data Even a = Zero | More (Two a) (Even a) deriving Functor    
\end{verbatim}
In this example, we can define lists of even numbers of elements as lists of pairs of elements, and the Functor instance for \AD{Even} can be derived generically, using that \AD{Two} has a (derived) Functor instance. This would not work for \AD{U-ix} or \AD{U-num}, as a generic function would not be able to decide whether a field is of the form \AD{μ}\ \AV{D} to begin with. 

Inlining the constructors of \AD{Phalanx} into \AD{Carpal} does allow generic constructions to see the structure of \AD{Phalanx}, but is undesirable in this case and in general, as this yields a type with two of the original constructors of \AD{Carpal}, and 9 more constructors for each combination of constructors of \AD{Phalanx}\footnote{If working with 11 constructors sounds too feasible, consider that defining addition on types like \AD{Carpal} (or concatenation on its numerical representation) is not (yet) generic and, if fully written out, will instead demand 121 manually written cases.}.

In order to make the descriptions of fields that have them visible to generics, we simply add a more specific former of fields to the universe, specialized to adding \emph{composite fields} from provided descriptions
\ExecuteMetaData[Tex/Descriptions/Composite]{delta}
taking the functions \AV{d} and \AV{j} to determine the parameters and indices passed to \AV{R}. A composite field encoded by \AIC{δ} is then interpreted identically to how it would be if we used \AIC{σ} and \AD{μ} instead\footnote{The omission of \AD{μ}\ \AV{R} from the variable telescope is intentional. While adding it is workable, it also significantly complicates the treatment of ornaments.}:
\ExecuteMetaData[Tex/Descriptions/Composite]{delta-int}
Using \AIC{δ} rather than \AIC{σ} allows us to reveal the description of a field to a generic program. Rather than adding \AD{Phalanx} via a \AIC{σ}, we would use \AIC{δ} to directly add \AF{Phalanx-num} instead.

\subsubsection{Hiding variables}
With the modifications described above, we can describe all the structures we want. However, there is one peculiarity in the way \AD{U-ix} handles variables. Namely, each field added by a \AIC{σ} is treated as bound, and even if the value is then unused, all fields afterwards need to work around it. This only poses a minor inconvenience, but this does mean that two subsequent fields which refer to the same variable will have to be encoded differently. Furthermore, adding fields of complicated types can quickly clutter the context when writing or inspecting a generic program.

%\todo{With rather significant consequences on the definition of ornaments down the line\dots}
With a simple modification to the handling of telescopes in \AD{U-ix}, we can emulate both bound and unbound fields, without adding more formers to \AD{U-ix}. By accepting a transformation of variables \AF{Vxf}\ \AV{Γ}\ \AV{(V ▷ S)}\ \AV{W} after a \AIC{σ}\ \AV{S} in the context of \AV{V}, the remainder of the fields can be described in the context \AV{W}:
\ExecuteMetaData[Tex/Descriptions/Variable]{sigma-var}
Of course, it would be no use to redefine \AIC{σ} in an attempt to save the user some effort, while leaving them with the burden of manually adding these transformations. So, we define shorthands emulating precisely the bound field
\ExecuteMetaData[Tex/Descriptions/Variable]{sigma-plus}
and the unbound field
\ExecuteMetaData[Tex/Descriptions/Variable]{sigma-minus}


\subsection{A new Universe}\label{ssec:desc}
Using the modifications described above we define a new universe based on \AD{U-ix}, in which the descriptions are again lists of constructors:
\ExecuteMetaData[Ornament/Desc]{Desc}
The universe \AD{DescI} is also parametrized over the metadata \AD{Meta}, generalizing the annotations from \autoref{ssec:numbers} which we will use later to encode number systems in \AD{DescI}.

The constructors of described datatypes can be formed as follows:
\ExecuteMetaData[Ornament/Desc]{Con}
Remark that \AIC{𝟙} is the same as before, but \AIC{ρ} now accepts the transformation \AF{Cxf}\ \AV{Γ}\ \AV{Γ} to encode non-uniform parameters. Likewise, \AIC{σ} takes a transformation \AV{w} from \AV{V}\ \AIC{▷}\ \AV{S} to \AV{W}, allowing us to replace the context \AV{V}\ \AIC{▷}\ \AV{S} after a field with a context \AV{W} of our choice. Finally, \AIC{δ} is added to directly describe composite datatypes by giving a description \AV{R} to represent a field \AD{μ}\ \AV{R}.

Let us take a fresh look at some datatypes from before, now through the lens of \AD{DescI}. We will leave the metadata aside for now by using:
\ExecuteMetaData[Ornament/Desc]{Plain-synonyms}
Like before, we use the shorthands \AF{σ+}, \AF{σ-}, and \AF{ρ0} to avoid cluttering descriptions which do not make use of the corresponding features. 

In \AD{DescI}, we can \bN{} and \AD{List} as before, replacing \AIC{σ} with \AF{σ-} and \AIC{ρ} with \AIC{ρ0}:
\ExecuteMetaData[Ornament/Desc]{NatD-and-ListD}
If we define \AD{Vec}, we bind the length as a (implicit) field, for which we use \AF{σ+} instead, so we can extract the length \AV{n} in \AIC{ρ0} and \AIC{𝟙}:
\ExecuteMetaData[Ornament/Desc]{VecD}
By passing a recursive field \AIC{ρ} the function taking \AV{A} to \AV{A}\ \AD{×}\ \AV{A}, we can almost repeat the definition of \AD{Random} from \AD{U-nest}:
\ExecuteMetaData[Ornament/Desc]{RandomD}
Binary finger trees (as a simplification of 2-3 finger trees \cite{fingertrees}), which are a nested datatype like \AD{Random}, store elements in variably sized digits on both sides instead. Using the composite field \AIC{δ} we can then define digits and finger trees separately
\ExecuteMetaData[Ornament/Desc]{DigitD}
simply adding fields represented by \AF{DigitD} to \AF{FingerD}:
\ExecuteMetaData[Ornament/Desc]{FingerD}
These descriptions can be instantiated as before by taking the fixpoint
\ExecuteMetaData[Ornament/Desc]{fpoint}
of their interpretations as functors
\ExecuteMetaData[Ornament/Desc]{interpretation}
inserting the transformations of parameters \AV{g} in \AIC{ρ} and the transformations of variables \AV{w} in \AIC{σ}.

Like \AD{U-ix}, we can define a generic \AF{fold} for \AD{DescI}
\ExecuteMetaData[Ornament/Desc]{fold-type}
which comes in equally handy when using ornaments.

\subsubsection{Annotating Descriptions with Metadata}
We promised encodings of number systems in \AD{DescI}, so let us explain how number systems can be described as \emph{metadata} and how this lets use \AD{DescI} in the same way we used \AD{U-num} to describe type and numerical value in the same description.

By generalizing \AD{DescI} over \AD{Meta}, rather than coding the specification of number systems into the universe directly, we give ourselves the flexibility to both represent plain datatypes and number systems in the same universe. The specific \AD{Meta} passed to \AD{DescI} determines the types of information to be queried (in the implicit \AV{me} fields) at each of the type-formers. A term of \AD{Meta} simply lists the type of information to be queried at each type former\footnote{One can compare this to how generic representations of datatypes in Haskell can be (optionally) annotated with metadata making the names of datatypes, constructors and fields available on the type level.}:
\ExecuteMetaData[Ornament/Desc]{Meta}
In composite fields \AIC{δ}, the metadata on the other description is not necessarily the same as the top-level metadata. When this happens, we ask that both sides are related by a transformation
\ExecuteMetaData[Ornament/Desc]{MetaF}
making it possible to downcast (or upcast) between the different types of metadata. This, for example, allows one to include an annotated type \AD{DescI}\ \AV{Me} into an ordinary datatype \AD{Desc} without duplicating the former definition in \AD{Desc} first.

The encoding of number systems by associating numbers to \AIC{𝟙} and \AIC{ρ}, and functions to \AIC{σ}, can be summarized as:
\ExecuteMetaData[Ornament/Desc]{Number} 
We let the composite field \AIC{δ}, which was not described when we discussed encoding number systems in \AD{U-num}, act similar to \AIC{ρ}, also multiplying the value in its field by a constant; The equalities in the metadata of a \AIC{δ} ensure that number systems have no parameters or indices. 

Using \AF{Number}, we describe the binary numbers \AF{Bin-num} in \AD{DescI} as:
\ExecuteMetaData[Ornament/Desc]{BinND} 
The metadata transformations help us when we represent \AF{Carpal-num} in its more accurate form, by first defining 
\ExecuteMetaData[Ornament/Desc]{PhalanxND} 
and directly including it into \AD{Carpal}
\ExecuteMetaData[Ornament/Desc]{CarpalND} 
where we can use the identity function to indicate both sides have metadata of type \AF{Number}.

The metadata on a \AD{DescI}\ \AF{Number} can then be used to define a generic function sending terms of number systems to their \AF{value} in \bN{}
\ExecuteMetaData[Ornament/Desc]{toN-type}
which is defined by generalizing over the inner metadata and \AF{fold}ing using:
\ExecuteMetaData[Ornament/Desc]{toN-con}
Furthermore, also possible to use \AD{Meta} to encode conventionally useful metadata such as field names:
\ExecuteMetaData[Ornament/Desc]{Names}
On the other extreme, we can also declare that a description has no metadata at all by querying \AD{⊤} for all type-formers:
\ExecuteMetaData[Ornament/Desc]{Plain}
Because the queries for metadata are implicit in \AD{DescI}, descriptions from \AD{U-ix} can be imported into \AD{Desc}, without having to insert metadata anywhere.



\part{Ornaments}\label{part:ornaments}
In the framework of \AD{DescI} of the last section, we can write down a number system and its meaning as the starting point of the construction of a numerical representation. To write down the generic construction of those numerical representations, we will need a language in which we can describe modifications on the number systems.

In this section, we will describe the ornamental descriptions for the \AD{DescI} universe, and explain their working by means of examples. We omit the definition of the ornaments, since we will only construct new datatypes, rather than relate pre-existing types.
%\todo{Maybe, I will throw the ornaments into the appendix along with the conversion from ornamental description to ornament}.
%\todo{do we need to remark more?}

In the framework of \AD{DescI} in the last section, we can write down a number system and its meaning as the starting point of the construction of a numerical representation. To write down the generic construction of those numerical representations, we will need a language in which we can describe modifications on the number systems.

\changed{Somewhat final version above, draft/notes/rough comments/outline below.}
In this section, we will describe the ornamental descriptions for the \AD{DescI} universe, and explain their working by means of (plenty of examples). We omit the definition of the ornaments, since we will only construct new datatypes, rather than relate pre-existing types\footnote{Maybe, I will throw the ornaments into the appendix along with the conversion from ornamental description to ornament}.

% (Be alarmed, the implicits get out of hand pretty quickly.)
\todo{do we need to remark more?}


\section{Ornamental descriptions}
These ornamental descriptions take the same shape as those in \autoref{sec:background-ornamental-descriptions}, generalized to handle nested types, variable transformations, and composite types. Like the interpretation of a \AD{DescI}, ornaments also completely ignore the \AD{Info} of a \AD{DescI}.

Recall that a \AD{OrnDesc}\ \AV{If′ Δ c J i D} represents the ornament building on top of \AV{D}, which yields a description with information \AV{If′}, parameters \AV{Δ}, and indices \AV{J}. We use \AF{∼} to write down pointwise equality of functions, which in this case are all commutativity squares. Since \AD{ConI} allows the transformation of variable telescopes, we have to dedicate a lot of lines to writing down commutativity squares for variables, which along with the generally high number of arguments and implicits\footnote{Of which even more are hidden!} makes the definition rather dry and long.

One tip is to  ignore all squares involving a \AD{Vxf}, these are trivial when using the \AV{+-} variants of the \AIC{σ} and \AIC{δ} formers anyway! Due to the last constructor \AIC{δ•}, \AD{OrnDesc}, \AD{ConOrnDesc}, and \AF{toDesc}\footnote{We left out the variable square for \AIC{δ•}, because it is honestly just too long. If this was included, then we also would involve \AF{ornForget}.} become tightly connected, so the definition is given in one large mutual block:
\ExecuteMetaData[Ornament/OrnDesc]{OrnDesc}
Here the implicit \AV{If′} contains the information necessary to recover the \AD{DescI} from an \AD{OrnDesc}:\todo{line length}
\ExecuteMetaData[Ornament/OrnDesc]{toDesc}
The commutativity squares again ensure the existence of functions like \AF{ornForget}, and that these ornamental descriptions indeed induce ornaments.

Compared to the previous ornaments, we have the new constructors \AIC{δ}, \AIC{Δδ} and \AIC{δ•}, where the first two are analogues of \AIC{σ} and \AIC{Δσ}. The \AIC{δ•} constructor states that an ornamental description from a description \AV{R} and a (constructor) ornamental description from \AV{CD} can be composed to form an ornamental description from the composition (in the sense of the \AV{δ} type-former) of \AV{CD} with \AV{R}.

Let us make the uses of \AD{OrnDesc} more clear by means of examples, where we make use of the simpler variants: \todo{Oδ•+- needs ornForget to run}
\ExecuteMetaData[Ornament/OrnDesc]{O-sigma-pm}
With these we can give the now familiar ornamental description of \AD{Vec} from \AD{List}:
\ExecuteMetaData[Ornament/OrnDesc]{VecOD}
Using the new flexibility in \AIC{ρ}, we can now start from a description of binary numbers:
\ExecuteMetaData[Ornament/OrnDesc]{LeibnizD}
and give the random access lists from before as an ornamental description as well.
\ExecuteMetaData[Ornament/OrnDesc]{RandomOD}
Likewise, we can use \AIC{δ•} to start from the ``fingertree numbers'':
\todo{finger tree skeleton}
and compose this with the ornamental description of \AD{Digit}
\todo{DigitOD}
to obtain the ornamental description of finger trees:
\todo{FingerTreeOD}

%Again, ornForget, fold blabla.

\todo{Now we can compute everything generically.}


\begin{outline}
\ExecuteMetaData[Ornament/OrnDesc]{ConOrnDesc-type}
The definition of ornamental descriptions can be derived in a straightforward manner from ornaments, removing all mentions of the LHS and making all fields which then no longer appear in the indices explicit\footnote{One might expect to need less equalities, alas, this is difficult because of \autoref{rem:orn-lift}.}. We will show the leaf-preserving rule as an example, the others are derived analogously:
\ExecuteMetaData[Ornament/OrnDesc]{OrnDesc-1}
As we can see, the only change we need to make is that \AgdaBoundFontStyle{k} becomes explicit and fully annotated.

Almost by construction, we have that an ornamental description can be decomposed into a description of the new datatype
\ExecuteMetaData[Ornament/OrnDesc]{toDesc}
and an ornament between the starting description and this new description
\ExecuteMetaData[Ornament/OrnDesc]{toOrn}
\end{outline}


\begin{outline}    
\section{The ornaments}
we could ditch removal of fields: we don't use it. downside: ornament over ornament is the same as field removal for deltas

:warning: match everything, add/remove field, add/remove recursive field, add/remove description field, ornament over ornament

\todo{Nuke ornaments, keep ornamental descriptions}

\towrite{Put something that isn't yet in \autoref{ssec:bg-orn} here.}

\ExecuteMetaData[Ornament/Orn]{Orn-type}
\ExecuteMetaData[Ornament/Orn]{ornForget-type}

We will walk through the constructor ornaments
\ExecuteMetaData[Ornament/Orn]{ConOrn-type}
again, an ornament between datatypes is just a list of ornaments between their constructors
\ExecuteMetaData[Ornament/Orn]{Orn}
Note that all ornaments completely ignore information bundles! They cannot affect the existence of \AgdaFunction{ornForget} after all.

Copying parts from one description to another, up to parameter and index refinement, corresponds to reflexivity. Preservation of leaves follows the rule
\ExecuteMetaData[Ornament/Orn]{Orn-1}
We can see that this commuting square (\texttt{e (k p) ≡ j (over f p)}) is necessary: take a value of \texttt{E} at \texttt{p, i}, where \texttt{i} is given as \texttt{k p}. Then \AgdaFunction{ornForget} has to convert this to a value of \texttt{D} at \texttt{f p , e i}, but since \texttt{e i} must match \texttt{j (f p)}, this is only possible if \texttt{e (k p) = j (f p)}.

Preserving a recursive field similarly requires a square of indices and conversions to commute
\ExecuteMetaData[Ornament/Orn]{Orn-rho}
additionally requiring the recursive parameters to commute with the conversion. \todo{Does adding the derivations for the squares everywhere make this section clearler?}

Preservation of non-recursive fields and description fields is analogous
\ExecuteMetaData[Ornament/Orn]{Orn-sigma-delta}
differing only in that non-recursive fields appears transformed on the right hand, while description fields have their conversions modified instead. For this rule, we need that the variable transformations fit into a commuting square with the parameter conversions. The condition on indices for descriptions, which is a commuting triangle, is encoded in the return type\footnote{Should this become a problem like with \AgdaInductiveConstructor{ρ}, modifying the rule to require a triangle is trivial.}.

Ornaments would not be very interesting if they only related identical structures. The left-hand side can also have more fields than the right-hand side, in which case \AgdaFunction{ornForget} will simply drop the fields which have no counterpart on the right-hand side. As a consequence, the description extending rules have fewer conditions than the description preserving rules: 
\ExecuteMetaData[Ornament/Orn]{Orn-+-rho}
Note that this extension\footnote{Kind of breaking the ``ornaments relate types with similar recursive structure'' interpretation.} with a recursive field has no conditions.

Extending by a non-recursive field or a description field again only requires the variable transform to interact well with the parameter conversion
\ExecuteMetaData[Ornament/Orn]{Orn-+-sigma-delta}

In the other direction, the left-hand side can also omit a field which appears on the right-hand side, provided we can produce a default value
\ExecuteMetaData[Ornament/Orn]{Orn---sigma-delta}
These rules let us describe the basic set of ornaments between datatypes.

Intuitively we also expect a conversion to exist when two constructors have description fields which are not equal, but are only related by an ornament. Such a composition of ornaments takes two ornaments, one between the field, and one between the outer descriptions. This composition rule reads:\todo{The implicits kind of get out of control here, but the rule is also unreadable without them. I might hide the rule altogether and only run an example with it.}
\ExecuteMetaData[Ornament/Orn]{Orn-comp}
We first require two commuting squares, one relating the parameters of the fields to the inner and outer parameter conversions, and one relating the indices of the fields to the inner index conversion and the outer parameter conversion. Then, the last square has a rather complicated equation, which merely states that the variable transforms for the remainder respect the outer parameter conversion.

We will construct \AgdaFunction{ornForget} as a \AgdaFunction{fold}. Using
\ExecuteMetaData[Ornament/Orn]{erase-type}
we can define the algebra which forgets the added structure of the outer layer
\ExecuteMetaData[Ornament/Orn]{ornAlg}
Folding over this algebra gives the wanted function
\ExecuteMetaData[Ornament/Orn]{ornForget}

\todo{NatD was removed here}

We can also relate lists and vectors
\ExecuteMetaData[Ornament/Orn]{ListD-VecD}
Now the parameter conversion is the identity, since both have a single type parameter. The index conversion is \AgdaFunction{!}, since lists have no indices. Again, most structure is preserved, we only note that vectors have an added field carrying the length.

Instantiating \AgdaFunction{ornForget} to these ornaments, we now get the functions \AgdaFunction{length} and \AgdaFunction{toList} for free!

%\investigate{Having a function of the same type as \AgdaFunction{ornForget} is not sufficient to deduce an ornament. An obstacle is that the usual empty type (no constructors) and the non-wellfounded empty type (only a recursive field) don't have an ornament. Also, while the leaf-preservation case spells itself out, the substitutions obviously don't give us a way to recover the equalities.}
\end{outline}
    


\part{Generic Numerical Representations}\label{part:numrep}
The ornamental descriptions of the last section, together with the descriptions and number systems from before, complete the toolset we will use to construct numerical representations as ornaments.

To summarize, using \AD{DescI}\ \AF{Number} to represent number systems, we paraphrase the calculation of \autoref{sec:desc-numrep} as an ornament, rather than a direct definition. In fact, we have already seen ornaments to numerical representations before, such as \AF{ListOD} and \AF{RandomOD}. Generalizing those ornaments, we construct numerical representations by means of an ornament-computing function, sending number systems to the ornamental descriptions that describe their numerical representations. 

\begin{comment}
3 Calculating datastructures using Ornaments

In this part we return to the matter numerical representations. With 2.3 in mind, we can rephrase part our original question to ask

> Can numerical representations be described as ornaments on their number systems?

Let us look at a numerical representation presented as ornament in action.

\section{Numerical representations as ornaments}\label{sec:ornaments}
Reflecting on this derivation for \bN{}, we could perform the same computation for \bL{} to get Braun trees. However, we note that these computations proceed with roughly the same pattern: each constructor of the numeral system gets assigned a value, and is amended with a field holding a number of elements and subnodes using this value as a ``weight''. This kind of ``modifying constructors'' is formalized by ornamentation \cite{progorn}, which lets us formulate what it means for two types to have a ``similar'' recursive structure. This is achieved by interpreting (indexed inductive) datatypes from descriptions, between which an ornament is seen as a certificate of similarity, describing which fields or indices need to be introduced or dropped to go from one description to the other. \textit{Ornamental descriptions}, which act as one-sided ornaments, let us describe new datatypes by recording the modifications to an existing description.
\todo[inline]{Put some minimal definitions here.}

Looking back at \AgdaDatatype{Vec}, ornaments let us show that express that \AgdaDatatype{Vec} can be formed by introducing indices and adding a fields holding an elements to \bN{}.
However, deriving \AgdaDatatype{List} from \bN{} generalizes to \bL{} with less notational overhead, so we tackle that case first. We use the following description of \bN{}
\ExecuteMetaData[Tex/NumRepOrn]{NatD}
Here, \AgdaInductiveConstructor{σ} adds a field to the description, upon which the rest of the description can vary, and \AgdaInductiveConstructor{ṿ} lists the recursive fields and their indices (which can only be \AgdaInductiveConstructor{tt}).
We can now write down the ornament which adds fields to the \AgdaFunction{suc} constructor
\ExecuteMetaData[Tex/NumRepOrn]{ListO}
Here, the \AgdaInductiveConstructor{σ} and \AgdaInductiveConstructor{ṿ} are forced to match those of \AgdaDatatype{NatD},
but the \AgdaInductiveConstructor{Δ} adds a new field. Using the least fixpoint and description extraction, we can then define \AgdaDatatype{List} from this ornamental description. Note that we cannot hope to give an unindexed ornament from \bL{}
\ExecuteMetaData[Tex/NumRepOrn]{LeibnizD}
into trees, since trees have a very different recursive structure! Thus, we must keep track at what level we are in the tree so that we can ask for adequately many elements:
\ExecuteMetaData[Tex/NumRepOrn]{TreeO}
We use the \AgdaFunction{power} combinator to ensure that the digit at position $n$, which has weight $2^n$ in the interpretation of a binary number, also holds its value times $2^n$ elements. This makes sure that the number of elements in the tree shaped after a given binary number also is the value of that  binary number.
\end{comment}


\section{Generic numerical representations}\label{sec:trieo}
In this section, we will demonstrate how we can use ornamental descriptions to generically compute unindexed numerical representations. \todo{Explain why not indexed}
%The claim is that calculating a datastructure is actually an ornamental operation, so we might call our approach ``calculating ornaments''. 
% mark: more
The reasoning here proceeds differently from that in the calculation of \AD{Vec} from \bN{}. Indeed, we first construct a datatype and only prove it is the correct type after, as opposed to calculating the type by isomorphism reasoning. Nevertheless, the choices of fields depending on the analysis of a number system follow the same strategy.

Recall the ``natural numbers''-information \AF{Number}, which gets its semantics from the conversion to \bN{}:
\ExecuteMetaData[Ornament/Numerical]{toN-type}
which is defined by generalizing over the inner information bundle and folding using
\ExecuteMetaData[Ornament/Numerical]{toN-con}
The choice of interpretation restricts the numbers to the class of numbers which are evaluated as linear combinations of ``digits''\footnote{An arbitrary \AF{Number} system is not necessarily isomorphic to \bN{}, as the system can still be incomplete (i.e., it cannot express some numbers) or redundant (it has multiple representations of some numbers).}. This class certainly does not include all interesting number systems, but does include many systems that have associated arrays\footnote{Notably, arbitrary polynomials also have numerical representations, interpreting multiplication as precomposition.}. 

We let this interpretation into \bN{} guide the computation of the associated numerical representation, which will be a (nested) type of finite sequences. In each case, we follow the computation in \AF{value} by inserting vectors of sizes corresponding to the weights of the number system:
\ExecuteMetaData[Ornament/Numerical]{trieifyOD}
In the case of a leaf \AIC{𝟙} of weight \AV{k}, we insert a vector of size \AV{k}. Similarly, in a field \AIC{σ}, where the weight is determined by a value \AV{s} of \AV{S}, we insert a vector of the weight corresponding to the value of \AV{s}. Note that the actual value/number of elements a leaf or field contributes depends on the preceding multipliers of recursive fields: a recursive field of a number can have a weight \AV{k}, so we multiply the number of elements in a recursive sequence by wrapping the parameter in a vector of size \AV{k}. By roughly the same reasoning we pass the trieification of a subdescription \AV{R} the parameter wrapped in a vector, which we compose into the current numerical description by using the ornament \AIC{∙δ}. Since \AV{R} can have a different \AD{Info}, we generalized the whole construction over \AV{ϕ}\ \AV{:} \AD{InfoF}\ \AV{If}\ \AF{Number}.

As an example, let us define \AF{PhalanxD} as a number system and walk through the computation of its \AF{trieifyOD}. We define
\ExecuteMetaData[Ornament/Numerical]{PhalanxND}
Now, we see that applying \AF{trieifyOD} sends leaves with a value of \AV{k} to \AD{Vec}\ \AV{A}\ \AV{k}, so applying it to \AF{DigitND} yields
\ExecuteMetaData[Ornament/Numerical]{DigitOD-2}
which is equivalent to the \AF{DigitOD} from before, up to expanding a vector of \AV{k} elements into \AV{k} fields. The same happens for the first two constructors of \AF{PhalanxND}, replacing them with an empty vector and a vector of one element respectively. The \AF{ThreeND} in the last constructor gets trieified to \AF{DigitOD′} and composed by \AF{O∙δ+}, and the recursive field gets replaced by a recursive field nesting over vectors of length. Again, this is equivalent to \AF{FingerOD}, up to wrapping values in length one vectors, replacing \AD{Pair} with a length two vector, and inserting empty vectors.

% proving the size is correct is a bit difficult because of the vectors in the nesting.

\begin{outline}
This concludes a bunch of things, including this thesis.
\end{outline}



%\part{Related work}\label{part:related}
%summarizing why everything that is in my references is there

\part{Discussion}
\marker{End A}

\lowprio{Proof is left as exercise to the reader. Hint $\Sigma$-descriptions will come in handy.} 

\towrite{This concludes a bunch of things, including this thesis. Combine conclusion and discussion? ``We did X, but there still are many improvements that could be made''}


In conclusion, we formulated a universe encoding \AD{DescI} with which we can describe number systems in \AD{DescI}\ \AF{Number}. We adapted the language of ornamental descriptions \AD{OrnDesc} to \AD{DescI}, such that the numerical representations of the number systems could be seen as ornaments on top of the number systems. These ornaments and descriptions let us implement the generic programs \AF{TreeOD} and \AF{TrieOD} which, from a number system compute the associated (un)indexed numerical representation, of which we informally outlined proofs of correctness. 

While it is possible to write down a direct proof of correctness for \AF{TrieOD} by comparing it to \AF{Lookup} via \AF{value}, and from this extract a proof of correctness for \AF{TreeOD}, one might expect there to be a more useful and less laborious angle of attack. 

Namely, we expect that if we define \AF{PathOD} as a generic ornament from a \AD{DescI}\ \AF{Number} to the corresponding finite type (such that \AF{PathOD}\ \AV{ND}\ \AV{n} is equivalent to \AD{Fin}\ (\AF{value}\ \AV{n})), then we can show that \AF{TrieOD}\ \AV{ND}\ \AV{n} has a \AF{tabulate}/\AF{lookup} pair for \AF{PathOD}\ \AV{ND}\ \AV{n}, from which it follows that \AF{TrieOD}\ \AV{ND}\ \AV{n}\ \AV{A} is equivalent to \AF{PathOD}\ \AV{ND}\ \AV{n}\ \AV{→}\ \AV{A}, and in consequence \AF{TrieOD}\ \AV{ND} corresponds to \AD{Vec}.

Due to the \AF{remember}-\AF{forget} isomorphism \cite{algorn}, we have that \AF{TreeOD}\ \AV{ND} is equivalent to \AV{Σ}\ (\AD{μ} \AV{ND})\ (\AF{TrieOD}\ \AV{ND}), and in turn we also find that \AF{TreeOD}\ \AV{ND} is a normal functor (also referred to as Traversable). This yields traversability of \AF{TreeOD}\ \AV{ND}, with as corollaries \AF{toList}\footnote{Note that the foldable structure we get from the generic \AF{fold} is significantly harder to work with for this purpose.} and properties such as that \AF{toList} is a lifting of \AF{value} (again in the sense of \cite{orntrans}).

However, it turns out that \AF{PathOD} is not so easy to define, as we can see by the following.

\subsection{Σ-descriptions are more natural for expressing finite types}\label{sec:sigma-desc}
Due to our representation of types as sums of products, representing the finite types of larger number systems quickly becomes much more complex. Consider the binary numbers from before:
\ExecuteMetaData[Tex/Discussion/Sigma-Desc]{Leibniz}
The finite type associated to \AD{Leibniz} has more constructors than \AD{Leibniz}:
\ExecuteMetaData[Tex/Discussion/Sigma-Desc]{FinB}
Crucially, the constructors of the finite type at (e.g.) index \AIC{1b}\ \AV{n} do not share the same structure.

In general, given a description of a number system \AV{N}, the number of constructors of the finite type \AD{FinN} of \AV{N} depends directly on the interpretation of \AV{N}, preventing the construction \AD{FinN} by simple recursion on \AD{DescI} (that is, without passing around lists of constructors instead). Furthermore, since our definition of ornaments insists ornaments preserve the number of constructors, there cannot be an ornament from an arbitrary number system to its finite type. 

The apparent asymmetry between number systems and finite types stems from the definition of \AIC{σ} in \AD{DescI}. In \AD{DescI} and similar sums-of-products universes \cite{practgen,sijsling}, the remainder of a constructor \AV{C} after a \AIC{σ}\ \AV{S} simply has its context extended by \AV{S}. In contrast, a universe with \emph{Σ-descriptions} \cite{effectfully,progorn,algorn} (in the terminology of \cite{sijsling}) encodes a dependent field \AV{(s : S)} by asking for a function \AV{C} assigning values \AV{s} to descriptions.

Compared to Σ-descriptions, a sums-of-products universe keeps out some more exotic descriptions which do not have an obvious associated Agda datatype\footnote{Consider the constructor \AIC{σ}\ \bN{}\ \AV{λ}\ \AV{n}\ \AV{→}\ \AF{power}\ \AIC{ρ}\ \AV{n}\ \AIC{𝟙} which takes a number \AV{n} and asks for \AV{n} recursive fields (where \AF{power}\ \AV{f}\ \AV{n}\ \AV{x} applies \AV{f} \AV{n} times to \AV{x}). This description, resembling a rose tree, does not (trivially) lie in a sums-of-products universe.}.

However, this also prevents us from writing down the simpler form of finite types. If we instead started from Σ-descriptions, taking functions into \AD{DescI} to encode dependent fields, we could compute a ``type of paths'' in a number system by adding and deleting the appropriate fields. Consider the universe:
\ExecuteMetaData[Tex/Discussion/Sigma-Desc]{Sigma-Desc}
In this universe we can present the binary numbers as:
\ExecuteMetaData[Tex/Discussion/Sigma-Desc]{LeibnizD}
The finite type for these numbers can be described by:
\ExecuteMetaData[Tex/Discussion/Sigma-Desc]{FinBD}
Since this description of \AF{FinB} largely has the same structure as \AF{Leibniz}, and as a consequence also the numerical representation associated to \AF{Leibniz}, this would simplify proving that the indexed numerical representation is indeed equivalent to the representable representation (the maps out of \AF{FinB}). In a framework of ornaments for Σ-descriptions \cite{progorn,algorn}, we can even describe the finite type as an ornament on the number system.


\subsection{Branching numerical representations}
A numerical representation constructed by \AF{TrieOD} looks like a finger tree: the structure typically has a central chain, which rather than directly storing elements directly in nodes, stores the elements in trees of which the depth increases with the level of the node.

In contrast, structures like Braun trees, as Hinze and Swierstra \cite{calcdata} compute from binary numbers, represent the weight of a node by branching themselves. Because this kind of branching is uniform, i.e., each branch looks the same, we can still give an equivalent construction. By combining \AF{TreeOD} and \AF{TrieOD}, and using 
\ExecuteMetaData[Tex/Discussion]{power}
to apply \AIC{ρ} \AV{k}-fold in the case of \AIC{ρ}\ \AV{\{if = k\}}, rather than over \AV{k}-element vectors, we can replicate the structure of a Braun tree from \AF{BinND}. However, if we use the Σ-descriptions we discussed above, we can more elegantly present these structures by adding an internal branch over \AD{Fin}\ \AV{k}.

\subsection{Indices do not depend on parameters}\label{sec:no-dep-ix}
In \AD{DescI}, we represent the indices of a description as a single constant type, as opposed to an extension of the parameter telescope \cite{practgen}. This simplification keeps the treatment of ornaments and numerical representations more to the point, but rules out types like the identity type \AD{≡}. 

By replacing index computing functions \AV{Γ}\ \AF{\&}\ \AV{V}\ \AF{⊢}\ \AV{I} with dependent functions
\ExecuteMetaData[Tex/Discussion]{index-interpretation}
we can allow indices to depend on parameters in our framework. As a consequence, we have to modify nested recursive fields to ask for the index type \AF{⟦}\ \AV{I}\ \AF{⟧tel} precomposed with \AV{g :}\ \AF{Cxf}\ \AV{Γ Γ}, and we have to replace the square like \AV{i}\ \AF{∘}\ \AV{j′}\ \AF{∼}\ \AV{i′}\ \AF{∘}\ \AF{over}\ \AV{v} in the definition of ornaments with heterogeneous squares.

Other consequence of allowing indices to depend on parameters is that this allows \emph{algebraic ornaments} \cite{algorn} to be formulated in \AD{OrnDesc} in their fully general form, and also lets us describe \emph{singleton types}, which can, roughly speaking, be used to compute the additional information needed to invert \AF{ornForget}.


\subsection{No RoseTrees}
In \AD{DescI}, we encode nested types by allowing nesting over a function of parameters \AF{Cxf}\ \AV{Γ}\ \AV{Γ}. This is less expressive than full nested types, which may also nest a recursive field under a strictly positive functor. For example, rose trees
\ExecuteMetaData[Tex/Discussion]{RoseTree}
cannot be directly expressed as a \AD{DescI}\footnote{And, since \AD{DescI} does not allow for higher-order inductive arguments like Escot and Cockx \cite{practgen}, we can also not give an essentially equivalent definition.}.

If we were to describe full nested types, allowing applications of functors in the types of recursive arguments, we would have to convince Agda that these functors are indeed positive, possibly by using polarity annotations\footnote{See \url{https://github.com/agda/agda/pull/6385}.}. Alternatively, we could encode strictly positive functors in a separate universe, which only allows using parameters in strictly positive contexts \cite{sijsling}. Finally, we could modify \AD{DescI} in such a way that we can decide if a description uses a parameter strictly positively, for which we would modify \AIC{ρ} and \AIC{σ}, or add variants of \AIC{ρ} and \AIC{σ} restricted to strictly positive usage of parameters.


\subsection{No levitation}
Since our encoding does not support higher-order inductive arguments, let alone definitions by induction-recursion, there is no code for \AD{DescI} in itself. Such self-describing universes have been described by Chapman et al. \cite{levitation}, and we expect that the additional features of \AD{DescI}, i.e., parameters, nesting, and composition, would not obstruct a similar levitating variant of \AD{DescI}. Due to the work of Dagand and McBride \cite{orntrans}, ornaments might even be generalized to inductive-recursive descriptions.

If that is the case, then modifications of universes like \AD{Meta} could be expressed internally. In particular, rather than defining \AD{DescI} such that it can describe datatypes with the information of, e.g., number systems, \AD{DescI} should be expressible as an ornamental description on \AD{Desc}, in contrast to how \AD{Desc} is an instance of \AD{DescI} in our framework. This would allow treating information explicitly in \AD{DescI}, and not at all in \AD{Desc}.

Furthermore, constructions like \AF{TrieOD}, which have the recursive structure of a \AF{fold} over \AD{DescI}, could be expressed by instantiating \AF{fold} to \AD{DescI}.


\subsection{Metadata more tasteful externally than internally}
On the other hand, while incorporating general metadata into \AD{DescI} works out neatly in our case, and in general seems to work out if we think about one use-case at a time, it might not work so nicely in other situations. For example, if we are working with \AF{Number}, but we are given a \AD{DescI}\ \AF{Plain} (i.e., \AF{Desc}), then we would have to duplicate that description in \AD{DescI}\ \AV{Number} before we could use it. Even worse, if we want to give the constructors of a number system nice names using \AF{Names}, we would have to rewrite our code and descriptions to use something like the product of \AF{Number} and \AF{Names}. 

It might be more portable to take the same approach in handling metadata as True sums of products \cite{truesop}, where metadata is described externally to the universe and only combined again if needed by a generic function. From this point of view, a type of metadata can simply be a convenient function from \AD{Desc} to \AD{Type}, and if \AF{Number} was presented in this way, then \AF{TreeOD} would not have to ask for \AD{DescI}\ \AF{Number}, but rather for a \AV{D} of \AD{Desc} with \AF{Number}\ \AV{D}.


\subsection{δ is conservative}\label{sec:redundant-delta}
We define our universe \AD{DescI} with \AIC{δ} as a former of fields with known descriptions, and this makes it easier to write down \AF{TreeOD}, even though \AIC{δ} is redundant. If more concise universes and ornaments are preferable, we can actually get all the features of \AIC{δ} and ornaments like \AIC{∙δ} by describing them using \AIC{σ}, annotations, and other ornaments.

Indeed, rather than using \AIC{δ} to add a field from a description \AV{R}, we can simply use \AIC{σ} to add \AV{S}\ \AV{=}\ \AD{μ}\ \AV{R}, and remember that \AV{S} came from \AV{R} in the information:
\ExecuteMetaData[Tex/Discussion]{Delta-Meta}
We can then define \AIC{δ} as a pattern synonym matching on the \AIC{just} case, and \AIC{σ} matching on the \AIC{nothing} case.

Recall that, leaving out some details, the ornament \AIC{∙δ} lets us compose an ornament from \AV{D} to \AV{D'} with an ornament from \AV{R} to \AV{R'}, yielding an ornament from \AIC{δ}\ \AV{D}\ \AV{R} to \AIC{δ}\ \AV{D'}\ \AV{R'}. This ornament can equivalently be modelled by first adding a new field \AD{μ}\ \AV{R'}, and then deleting the original \AD{μ}\ \AV{R} field. The ornament \AIC{∇} \cite{kophd} allows one to provide a default value for a field, deleting it from the description. Hence, we can model \AIC{∙δ} by binding a value \AV{r'} of \AD{μ}\ \AV{R'} with \AF{OΔσ+} and deleting the field \AD{μ}\ \AV{R} using a default value computed by \AF{ornForget}.

This also partially explains why we did not refer to algebraic ornaments at all in our construction of \AF{TrieOD}; We can see that \AF{TrieOD} looks very similar to the algebraic ornament over \AF{TreeOD}, which sends ornaments from \AV{D} to \AV{E} to an ornament to a \AV{D}-indexed variant of \AV{E}. However, the case of \AIC{δ} requires \AF{TrieOD} to step in and re-index the subdescription. In contrast, the algebraic ornament would simply treat a \AIC{δ} like its equivalent \AIC{σ}, and even though this would produce a correct numerical representation, this amounts to presenting a \AD{Vec} as a tuple of a length \AV{n}, a \AD{List} \AV{xs}, and a proof that \AV{n} is equal to \AF{length}\ \AV{xs}.

Thus, while it would be possible to present \AF{TrieOD} as a kind of algebraic ornament, this would require redefining algebraic ornaments from algebras that are rather specific about how they treat a \AIC{σ}.


\subsection{No sparse numerical representations}\label{sec:discussion-no-sparse}
The encoding of number systems in a universe we explained in \autoref{ssec:numbers} corresponds to a generalization of dense number systems. Consequently, this excludes the skew binary numbers \cite{oka95b} in their useful sparse representation.

Representations of sparse number systems can be regained by allowing addition \emph{and} variable multiplication in a \AIC{σ}. In such a setup, skew binary numbers and other sparse representations could be described by adding their gaps as fields, and computing the appropriate multiplier from there. While not worked out in this thesis, this extension is compatible with the construction of numerical representations.

Another notable extension of \AF{Number} is to let some recursive and composite fields be interpreted by multiplication, with which we could equip \AF{U-fin} with its obvious interpretation into \bN{}. This can be compared to the last exponential law we did not use in \autoref{sec:desc-numrep}, which is that $A^{BC} = (A^B)^C$. Furthermore, any indexed numerical representation acts as a representable functor \AV{F}. If \AV{F} and \AV{G} are numerical representations corresponding to number systems \AV{N} and \AV{M}, then the multiplication of \AV{N} and \AV{M} just corresponds to composition \AV{F}\ \AF{∘}\ \AV{G}.



\printbibliography

\part{Appendix}
%\section{Appendices}

\begin{appendix}
\section*{Appendices}
\addcontentsline{toc}{section}{Appendices}
\renewcommand{\thesubsection}{\Alph{subsection}}
    
\todo{When finished, shuffle the appendices to the order they appear in}

\subsection{fold and mapFold in full}\label{app:gfold}

\todo{write me}


\subsection{Without K but with universe hierarchies}\label{app:withoutk}


\todo{write me}
See \cite{practgen} and the small blurb rewriting interpretations as datatypes.

\subsection{Random and friends \textit{do} live in U-ix}\label{app:unnested}
Some injustice to \AD{U-ix} has been done, and it can actually give equivalent encodings for some types we claimed it could not encode effectively or at all. This of course relies on the word ``equivalent'' to do most of the work.

Use \AF{power} and indices.

Can still do an encoding of rosetrees.

\todo{write me}

\begin{outline}
Kun je aannemelijk maken dat er geen dependently typed encoding bestaat van Finger Trees? Voor binary random access lijsten, perfect trees, en lambda termen bestaan die wel... Of is de constructie te omslachtig?
\end{outline}

\subsection{Index-first}

\subsection{Sigma descriptions}\label{app:large-sigma}
\autoref{app:withoutk}

\subsection{ornForget and ornErase in full}\label{app:ornforget}

\begin{comment}
    \subsection{Heterogenization}
    %The situation in which one wants to collect a variety of types is not uncommon, and is typically handled by tuples. However, if e.g., you are making a game in Haskell, you might feel the need to maintain a list of ``Drawables'', which may be of different types. Such a list would have to be a kind of ``heterogeneous list''. In Haskell, this can be resolved by using an existentially quantified list, which, informally speaking, can contain any type implementing a given constraint, but can only be inspected as if it contains the intersection of all types implementing this constraint. 

This ports directly to Agda, but becomes cumbersome quickly, and impractical if we want to be able to inspect the elements. The alternative is to split our heterogeneous list into two parts; one tracking the types, and one tracking the values. In practice, this means that we implement a heterogeneous list as a list of values indexed over a list of types. This approach and mainly its specialization to binary trees is investigated by Swierstra \cite{hetbin}.

We will demonstrate that we can express this ``lift a type over itself'' operation as an ornament. For this, we make a small adjustment to \AgdaDatatype{RDesc} to track a type parameter separately from the fields. Using this we define an ornament-computing function, which given a description computes an ornamental description on top of it:
\ExecuteMetaData[Tex/Heterogenize]{HetO}
This ornament relates the original unindexed type to a type indexed over it; we see that this ornament largely keeps all fields and structure identical, only performing the necessary bookkeeping in the index, and adding extra fields before parameters.

As an example, we adapt the list description
\ExecuteMetaData[Tex/Heterogenize]{List}
which is easily heterogenized to an \AgdaDatatype{HList}. In fact, \AgdaFunction{HetO} seems to act functorially; if we lift \AgdaDatatype{Maybe} like
\ExecuteMetaData[Tex/Heterogenize]{HMaybe}
then we can lift functions like \AgdaFunction{head} as
\ExecuteMetaData[Tex/Heterogenize]{hhead}

    
\footnote{If a foldable universe means nothing to you, there are simpler encodings for parameters and indices, which are recorded in \autoref{app:large-sigma}.}
    

\ExecuteMetaData[Ornament/OrnDesc]{toOrn} -> Appendix if at all

    \subsection{More equivalences for less effort}\label{sec:userfriendly}
    Noting that constructing equivalences directly or from isomorphisms as in \autoref{ssec:leibniz} can quickly become challenging when one of the sides is complicated, we work out a different approach making use of the initial semantics of W-types instead. We claim that the functions in the isomorphism of \autoref{ssec:leibniz} were partially forced, but this fact was unused there.
    
    First, we explain that if we assume that one of the two sides of the equivalence is a fixpoint or initial algebra of a polynomial functor (that is, the \AgdaDatatype{μ} of a \AgdaDatatype{Desc′}), this simplifies giving an equivalence to showing that the other side is also initial.
    
    We describe how we altered the original ornaments \cite{progorn} to ensure that \AgdaDatatype{μ} remains initial for its base functor in Cubical Agda, explaining why this fails otherwise, and how defining base functors as datatypes avoids this issue.
    
    In a subsection focussing on the categorical point of view, we show how we can describe initial algebras (and truncate the appropriate parts) in such a way that the construction both applies to general types (rather than only sets), and still produces an equivalence at the end. We explain how this definition, like the usual definition, makes sure that a pair of initial objects always induces a pair of conversion functions, which automatically become inverses. Finally, we explain that we can escape our earlier truncation by appealing to the fact that ``being an equivalence'' is a proposition.
    
    Next, we describe some theory, using which other types can be shown to be initial for a given algebra. This is compared to the construction in \autoref{ssec:leibniz}, observing that intuitively, initiality follows because the interpretation of the zero constructor is forced by the square defining algebra maps, and the other values are forced by repeatedly applying similar squares. This is clarified as an instance of recursion over a polynomial functor.
    
    To characterize when this recursion is allowed, we define accessibility with respect to polynomial functors as a mutually recursive datatype as follows. This datatype is constructed using the fibers of the algebra map, defining accessibility of elements of these fibers by cases over the description of the algebra. Then we remark that this construction is an atypical instance of well-founded recursion, and define a type as well-founded for an algebra when all its elements are accessible.
    
    We interpret well-foundedness as an upper bound on the size of a type, leading us to claim that injectivity of the algebra map gives a lower bound, which is sufficient to induce the isomorphism. We sketch the proof of the theorem, relating part of this construction to similar concepts in the formalization of well-founded recursion in the Standard Library. In particular, we prove an irrelevance and an unfolding lemma, which lets us show that the map into any other algebra induced by recursion is indeed an algebra map. By showing that it is also unique, we conclude initiality, and get the isomorphism as a corollary. 
    
    The theorem is applied and demonstrated to the example of binary natural numbers. We remark that the construction of well-foundedness looks similar to view-patterns. After this, we conclude that this example takes more lines that the direct derivation in \autoref{ssec:leibniz}, but we argue that most of this code can likely be automated.
    
\end{comment}

\end{appendix}


\end{document}
