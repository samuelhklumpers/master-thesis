\subsection{Agda}\label{sec:background-agda}
We formalize our work in the programming language Agda \cite{agda}. Agda is a total functional programming language with dependent types. Using dependent types we can use Agda as a proof assistant, reinterpreting types as formulas and functions as proofs, allowing us to state and prove theorems about our datastructures and programs. Since Agda is total, and hence all functions are total, all functions of a given type always terminate in a value of that type, ruling out invalid proofs as a bonus\footnote{On the other hand, we sometimes have to put in some effort to convince Agda that a function indeed terminates.}. While we will only occasionally reference Haskell, those more familiar with Haskell might understand (the reasonable part of) Agda as the subset of total Haskell programs \cite{agda2hs}.

In this section, we will explain and highlight some parts of Agda which we use in the later sections. Many of the types and functions we define in this section are also described and explained in most Agda tutorials (\cite{ulftutorial}, \cite{plfa}, etc.), and can be imported from the standard library \cite{agdastdlib}.


\subsection{Datatypes in Agda}\label{sec:background-data}
At the level of (generalized) algebraic datatypes Agda is close to Haskell. In both languages, one can define objects using data declarations, and interact with them using function declarations. For example, we can define the type of booleans by declaring:
\ExecuteMetaData[Tex/Background]{Bool}
The constructors of this type state that values of \AD{Bool} are produced in exactly two ways: \AIC{false} and \AIC{true}. We can then define functions on \AD{Bool} by \emph{pattern matching}, using that a variable of \AD{Bool} is necessarily either \AIC{false} or \AIC{true}. As an example, we can define the conditional operator as:
\ExecuteMetaData[Tex/Background]{conditional}
When we pattern match on a variable of type \AV{A}, in this case \AD{Bool}, the coverage checker ensures we define the function on all possible cases, and thus the function is completely defined.

We can also define a type representing the natural numbers:
\ExecuteMetaData[Tex/Background]{Nat}
Here, \bN{} always has a \AF{zero} element, and for each element $n$ the constructor \AIC{suc} expresses that there is also an element representing $n + 1$. Hence, \bN{} represents the natural numbers by encoding the existential axioms of the Peano axioms\footnote{The equality, injectivity, and induction axioms follow from the corresponding principles for arbitrary datatypes.}. By pattern matching and \emph{recursion} on \bN{}, we define the less-than operator:
\ExecuteMetaData[Tex/Background]{lt}
One of the cases contains a recursive instance of \bN{}, so termination checker also verifies that this recursion indeed terminates, ensuring that we still define \AV{n}\ \AF{<?} \AV{m} for all possible combinations of \AV{n} and \AV{m}. In this case the recursion is valid, since both arguments decrease before the recursive call, meaning that at some point \AV{n} or \AV{m} hits \AIC{zero} and the recursion terminates.

Like in Haskell, we can \emph{parametrize} a datatype over other types to make a \emph{polymorphic} type. By parameterizing a definition, the context of that definition is extended with a variable of the type parametrized over. Parametrizing lists over a type, we can define lists of values for all types:
\ExecuteMetaData[Tex/Background]{List} 
A list of \AV{A} can either be empty \AIC{[]}, or contain an element of \AV{A} and another list via \AIC{\_∷\_}. In other words, \AD{List} is a type of finite sequences in \AV{A}, in the sense of sequences as an abstract type \cite{purelyfunctional}.

Using polymorphic functions, we can manipulate and inspect lists by inserting or extracting elements. For example, we can define a function to look up the value at some position \AV{n} in a list:
\ExecuteMetaData[Tex/Background]{lookup-list}
However, this function is \emph{partial}, as we are relying on the type
\ExecuteMetaData[Tex/Background]{Maybe}
to handle the \AIC{[]} case, where the position does not lie in the list and we cannot return an element. If we know the length of the list \AV{xs}, then we also know for which positions \AF{lookup} will succeed, and for which it will not. We define 
\ExecuteMetaData[Tex/Background]{length}
so that we can test whether the position \AV{n} lies inside the list by checking \AV{n}\ \AF{<?}\ \AF{length}\ \AV{xs}. If we declare \AF{lookup} as a dependent function consuming a proof of \AV{n}\ \AF{<?}\ \AF{length}\ \AV{xs}, then \AF{lookup} always succeeds. This, however, merely replaces the check whether \AF{lookup} returns \AIC{nothing} with a check if \AV{n}\ \AF{<?}\ \AF{length}\ \AV{xs} is before applying \AF{lookup}.

More elegantly, we can combine natural numbers with an inequality by defining an \emph{indexed type}, representing numbers below an upper bound:
\ExecuteMetaData[Tex/Background]{Fin}
Like parameters, \emph{indices} add a variable to the context of a datatype, but unlike parameters, indices can influence the availability of constructors. The type \AD{Fin} is defined such that a variable of type \AD{Fin}\ \AV{n} represents a number less than \AV{n}. Since both constructors \AIC{zero} and \AIC{suc} dictate that the index is the \AIC{suc} of some natural number \AV{n}, we see that \AD{Fin}\ \AIC{zero} has no values. On the other hand, \AIC{suc} gives a value of \AD{Fin}\ (\AIC{suc}\ \AV{n}) for each value of \AD{Fin}\ \AV{n}, and \AIC{zero} gives exactly one additional value of \AD{Fin}\ (\AIC{suc}\ \AV{n}) for each \AV{n}. We can thus conclude that \AD{Fin}\ \AV{n} has exactly \AV{n} closed terms, each representing a number less than \AV{n}.

To complement \AD{Fin}, we define another indexed type representing lists of a known length, also known as vectors:
\ExecuteMetaData[Tex/Background]{Vec}
The \AIC{[]} constructor of this type produces the only term of type \AD{Vec}\ \AV{A}\ \AIC{zero}. The \AIC{\_∷\_} constructor ensures that a \AD{Vec}\ \AV{A}\ (\AIC{suc}\ \AV{n}) always consists of an element of \AV{A} and a \AD{Vec}\ \AV{A}\ \AV{n}. Similar to \AD{Fin}, we find that a \AD{Vec}\ \AV{A}\ \AV{n} contains exactly \AV{n} elements of \AV{A}. Thus, we conclude that \AD{Fin}\ \AV{n} is exactly the type of positions in a \AD{Vec}\ \AV{A}\ \AV{n}. In comparison to \AD{List}, we can say that \AD{Vec} is a type of arrays, in the sense of arrays as the abstract type of sequences of a fixed length. Furthermore, knowing the index of a term \AV{xs} of type \AV{Vec}\ \AV{A}\ \AV{n} uniquely determines the constructor it was formed by. Namely, if \AV{n} is \AIC{zero}, then \AV{xs} is \AIC{[]}, and if \AV{n} is \AIC{suc} of \AV{m}, then \AV{xs} is formed by \AIC{\_∷\_}. 

Using this, we define a variant of \AF{lookup} for \AD{Fin} and \AD{Vec}, taking a vector of length \AV{n} and a position below \AV{n}:
\ExecuteMetaData[Tex/Background]{lookup}
We can now omit the \AIC{[]} case, where \AF{lookup?} would return \AIC{nothing}. This happens because a variable of type \AD{Fin}\ \AV{n} is either \AIC{zero} or \AIC{suc}\ \AV{i}, and both cases imply that \AV{n} is \AIC{suc}\ \AV{m} for some \AV{m}. As we saw above, a \AD{Vec}\ \AV{A}\ (\AIC{suc} \AV{m}) is always formed by \AIC{\_∷\_}, which ensures that finding \AIC{[]} for \AV{xs} is impossible. Consequently, \AF{lookup} always succeeds for vectors. However, this does not yet prove that \AF{lookup} necessarily returns the right element, and we will need some more logic to verify this.

\subsection{Proving in Agda}\label{sec:background-proving}
To describe the equality of terms we define a new type:
\ExecuteMetaData[Tex/Background]{equiv}
If we have a value \AV{x} of \AV{a}\ \AD{≡}\ \AV{b}, then, as the only constructor of \AD{\_≡\_} is \AIC{refl}, we must have that \AV{a} is equal to \AV{b}. We can use \AD{\_≡\_} to describe the behaviour of functions like \AF{lookup}.

To test \AF{lookup}, we can insert elements into a vector with:
\ExecuteMetaData[Tex/Background]{insert}
If \AF{lookup} is correct, then we expect the following to hold
\ExecuteMetaData[Tex/Background]{lookup-insert-type}
which essentially states that we find elements where we insert them.

To prove \AF{lookup-insert-type}, we proceed as when defining any other function. By simultaneous induction on the position \AV{i} and vector \AV{xs}, we prove:
\ExecuteMetaData[Tex/Background]{lookup-insert}
In the first two cases, where we \AF{lookup} the first position, \AF{insert}\ \AV{xs}\ \AIC{zero}\ \AV{y} simplifies to \AV{y}\ \AF{∷}\ \AF{xs}, so the lookup immediately returns \AV{y} as wanted. In the last case, we have to prove that \AF{lookup} is correct for \AV{x}\ \AF{∷}\ \AF{xs}, so we use that the \AF{lookup} ignores the term \AV{x}, and appeal to the correctness of \AF{lookup} on the smaller list \AV{xs} to complete the proof.

Like \AD{\_≡\_}, we can encode many other logical operations into datatypes, which establishes a correspondence between types and formulas, known as the \emph{Curry-Howard correspondence}. For example, we can encode disjunctions (the logical `or' operation) as
\ExecuteMetaData[Tex/Background]{uplus}

Conjunction (logical `and') can be represented by:\footnote{We use a record here, rather than a datatype with a constructor \AV{A → B →}\ \AV{A}\ \AD{×}\ \AV{B}. The advantage of using a record is that this directly gives us projections like \ARF{fst}\ \AV{:}\ \AV{A}\ \AD{×}\ \AV{B}\ \AV{→ A}, and lets us use eta equality, making $(a, b) = (c , d) \iff a = c \land b = d$ holds automatically.}
\ExecuteMetaData[Tex/Background]{product}
True and false are respectively represented by
\ExecuteMetaData[Tex/Background]{true}
so that always \AIC{tt}\ \AV{:}\ \AD{⊤}, and:
\ExecuteMetaData[Tex/Background]{false}
The body of \AD{⊥} is intentionally empty: since \AD{⊥} has no constructors, there is no proof of false\footnote{If we did not use \AV{--type-in-type}, and even in that case I can only hope.}.

Because we identify function types with logical implications, we can also define the negation of a formula \AV{A} as ``\AV{A} implies false'':
\ExecuteMetaData[Tex/Background]{not}
The logical quantifiers $\forall$ and $\exists$ act on formulas with a free variable in a specific domain of discourse. We represent closed formulas by types, so we can represent a formula with one free variable of type \AV{A}, also known as a \emph{predicate}, by a function \AV{A}\ \AV{→}\ \AD{Type}, sending values of \AV{A} to types. The universal quantifier $\forall a P(a)$ is true when for all $a$ the formula $P(a)$ is true, so we represent the universal quantification of a predicate \AV{P} as a dependent function type \AV{(a : A) → P a}, producing for each \AV{a} of type \AV{A} a proof of \AV{P}\ \AV{a}. The existential quantifier $\exists a P(a)$ is true when there is some $a$ such that $P(a)$ is true, so we represent the existential quantification as
\ExecuteMetaData[Tex/Background]{exists}
so that we have \AD{Σ}\ \AV{A}\ \AV{P} if and only if we have an element \AV{fst} of \AV{A} and a proof \AV{snd} of \AV{P}\ \AV{a}. To avoid the need for lambda abstractions in existentials, we define the syntax
\ExecuteMetaData[Tex/Background]{sigma-syntax}
letting us write \AD{Σ[}\ \AV{a}\ \AD{∈}\ \AV{A}\ \AD{]}\ \AV{P a} for $\exists a P(a)$.

\subsection{Descriptions}\label{sec:background-descriptions}
In the previous sections we completed a quadruple of types (\bN{}, \AD{List}, \AD{Vec}, \AD{Fin}) equipped with the nice interactions \AF{length} and \AF{lookup}. Similar to the type of \AF{length}\ \AV{:}\ \AD{List}\ \AV{A}\ \AV{→}\ \bN{}, we can define
\ExecuteMetaData[Tex/Background]{toList}
converting vectors back to lists. In the other direction, we can also promote a list to a vector by recomputing its index:
\ExecuteMetaData[Tex/Background]{toVec}
This is no coincidence, but happens because \bN{}, \AD{List}, and \AD{Vec} are structurally similar.

But how can we prove that datatypes have similar structures? In this section, we will explain a framework of datatype \emph{descriptions} and ornaments, allowing us to describe datatypes as codes which can be compared directly, while also forming a foundation for generic programming in Agda \cite{ulftutorial, genericsamm, effectfully, practgen}. 

Recall that while polymorphism allows us to write one program for many types at once, those programs act parametrically \cite{reynolds1983types, wadlerfree}: polymorphic functions must work for all types, thus they cannot inspect values of their type argument. Generic programs, by design, do use the structure of a datatype, allowing for more complex functions that do inspect values\footnote{As examples, consider the generic JSON encoding of suitable datatypes \cite{truesop}, or the derivation of functor instances for a broad class of types \cite{haskellderiving}.}.

Using datatype descriptions we can then relate \bN{}, \AD{List} and \AD{Vec}, explaining how \AF{length} and \AF{toList} are instances of \emph{forgetful} functions. Let us walk through some ways of defining descriptions. We will start from simpler descriptions, building our way up to more general types, until we reach a framework in which we can describe \bN{}, \AD{List}, \AD{Vec} and \AD{Fin}. 


\subsubsection{Finite types}\label{ssec:background-fin}
An encoding of datatypes consists of two parts, a \emph{type of descriptions} \AV{U} of which the values are \emph{codes} representing other datatypes, and an \emph{interpretation} \AV{U}\ \AV{→}\ \AD{Type} decoding those codes to the represented types. In the terminology of Martin-L{\"{o}}f type theory (MLTT)\cite{levitation, mltt}, where types of types (e.g., \AD{Type}) are called \emph{universes}, we can think of a type of descriptions as an internal universe.

To start off, we define a basic universe with two codes \AIC{𝟘} and \AIC{𝟙}, respectively representing the types \AD{⊥} and \AD{⊤}, and the requirement that the universe is closed under sums and products:
\ExecuteMetaData[Tex/Background]{U-fin}
The meaning of the codes in this universe is then assigned by the interpretation:\footnote{One might recognize that \AF{⟦\_⟧fin} is a morphism between the rings (\AD{U-fin}, \AIC{⊕}, \AIC{⊗}) and (\AD{Type}, \AD{⊎}, \AD{×}). Similarly, \AD{Fin} also gives a ring morphism from \bN{} with \AF{+} and \AF{×} to \AD{Type}, and in fact \AF{⟦\_⟧fin} factors through \AD{Fin} via the map sending the expressions in \AD{U-fin} to their value in \bN{}.}
\ExecuteMetaData[Tex/Background]{int-fin}
In this universe, we can encode the type of booleans simply as:
\ExecuteMetaData[Tex/Background]{BoolD}
Since the types represented by \AIC{𝟘} and \AIC{𝟙} are finite, and sums and products of finite types are also finite, we refer to \AD{U-fin} as the universe of finite types. From this, one can immediately conclude that there is no code in \AD{U-fin} representing the (infinite) type of natural numbers \bN{}.

\subsubsection{Recursive types}\label{ssec:background-rec}
We saw before that \bN{} differs from \AD{Bool} by having a recursive field. So, in order to make a universe which can encode \bN{}, we begin by adding a code \AIC{ρ} to \AD{U-fin} representing recursive type occurrences:
\ExecuteMetaData[Tex/Background]{U-rec}
Then, we also have to redefine the interpretation: consider the interpretation of \AIC{𝟙}\ \AIC{⊕}\ \AIC{ρ}, for which we need to know that the whole type was \AIC{𝟙}\ \AIC{⊕}\ \AIC{ρ} while interpreting \AIC{ρ}. As a consequence, the interpretation splits in two phases. 

In the first, we use functions from \AD{Type} to \AD{Type}\footnote{We also refer to these functions as \emph{polynomial functors}, which are polynomial here because they consist of sums and products, and are functors because they have a (functorial) mapping operation, as we will see later.} to represent types with one free type variable. Interpreting a code \AV{D}, we use the free variable \AV{X} to represent ``the type D'':
\ExecuteMetaData[Tex/Background]{int-rec}
We can then model a recursive type by indeed setting the variable to the type itself, taking the \emph{least fixpoint} of the functor:
\ExecuteMetaData[Tex/Background]{mu-rec}
Recall the definition of \bN{}, which we can reinterpret as the declaration that \AD{ℕ} is the fixpoint of \AD{ℕ}\ \AD{≡}\ \AV{F}\ \AD{ℕ} for \AV{F X = ⊤ ⊎ X}. Hence, \bN{} can simply be encoded as:
\ExecuteMetaData[Tex/Background]{NatD}

\subsubsection{Sums of products}\label{ssec:background-sop}
A downside of \AD{U-rec} is that the definitions of types do not mirror their equivalents in user-written Agda very well. Using that polynomials can always be written as \emph{sums of products}\footnote{We do not require these to be reduced, as different representations of the same polynomial represent different datatypes for us.}, we can define a similar universe which more closely resembles handwritten code.

Unlike the arbitrarily shaped polynomials formed by \AIC{⊕} and \AIC{⊗}, a sum of products is analogous a datatype presented as a list of constructors. Thus, we split the descriptions into a stage in which we can form sums, equivalently datatypes
\ExecuteMetaData[Tex/Background]{U-sop}
on top of a stage where we can form products, equivalently constructors:
\ExecuteMetaData[Tex/Background]{Con-sop}
When doing this, we also let the left-hand side of a product be any type, which allows us to represent ordinary fields. The interpretation of this universe, while similar to the one in the previous section, is also split into a part interpreting datatypes
\ExecuteMetaData[Tex/Background]{int-U-sop}
and a part interpreting the constructors:
\ExecuteMetaData[Tex/Background]{int-Con-sop}
In this universe, we can define the type of lists as a description quantified over a type:
\ExecuteMetaData[Tex/Background]{ListD-bad}
Using this universe requires us to split functions on descriptions into multiple parts, but makes interconversion between representations and concrete types straightforward.

\subsubsection{Parametrized types}\label{ssec:background-par}
The encoding of fields in \AD{U-sop} makes the descriptions large in the following sense: by letting \AV{S} in \AIC{σ} be an infinite type, we can get a description referencing infinitely many other descriptions. As a consequence, we cannot inspect an arbitrary description in its entirety. At the same time, we could not express \AD{List} fully internally, and needed to handle the parameter externally. 

We can resolve both quirks simultaneously by introducing parameters and variables using a new gadget. In a naive attempt, we can represent the parameters of a type as \AD{List}\ \AD{Type}. However, this cannot represent many useful types; in the existential quantifier \AD{Σ\_}, the type \AV{A}\ \AV{→}\ \AD{Type} of second parameter \AV{B} references back to the first parameter \AV{A}, which is something that cannot be encoded in an ordinary list of types.

In a general parametrized type, parameters can refer to the values of all preceding parameters. The parameters of a type are thus a sequence of types depending on each other, which refer to as \emph{telescopes} \cite{telescopes} (also known as contexts in MLTT \cite{mltt}). We define telescopes using induction-recursion:
\ExecuteMetaData[Tex/Background]{Tel-simple}
A telescope can either be empty, or be formed from a telescope and a type in the context of that telescope, where we used the meaning of a telescope \AF{⟦\_⟧tel} to define types in the context of a telescope. This meaning represents valid assignments of values to parameters
\ExecuteMetaData[Tex/Background]{int-simple}
interpreting a telescope into the dependent product of all the parameter types. This definition of telescopes enables us to write down the type of \AD{Σ}:
\ExecuteMetaData[Tex/Background]{sigma-tel}
To encode \AD{Σ}, we will need to be able to bind the argument \AV{a} of \AV{A} and reference it in the field \AV{P}\ \AV{a}. While viable, a universe built around \AD{Tel′} would awkwardly confuse parameters and bound arguments.

By quantifying telescopes over a type \cite{practgen, sijsling}, we can distinguish parameters and bound arguments using almost the same setup:
\ExecuteMetaData[Tex/Background]{Tel-type}
A \AD{Tel}\ \AV{P} then represents a telescope for each value of \AV{P}, which we can view as a telescope in the context of \AV{P}. For readability, we redefine values in the context of a telescope as
\ExecuteMetaData[Tex/Background]{entails}
so we can define telescopes and their interpretations as:
\ExecuteMetaData[Tex/Background]{Tel-def}
By setting \AV{P}\ \AV{=}\ \AD{⊤}, we recover the previous definition of parameter telescopes. We can then define an \emph{extension} of a telescope as a telescope in the context of a parameter telescope
\ExecuteMetaData[Tex/Background]{ExTel}
representing a telescope of variables \AV{V} over the fixed parameter telescope \AV{Γ}, which can be extended independently of \AV{Γ}. An extension of \AV{Γ} can also be interpreted in the context of \AV{Γ}:
\ExecuteMetaData[Tex/Background]{int-ExTel}
To describe conversions of telescopes, we give names to maps of telescopes and extensions:
\ExecuteMetaData[Tex/Background]{tele-helpers}
We also defined two functions we will use extensively, \AF{var→par} states that a map of extensions extends to a map between the whole telescopes, and \AF{Vxf-▷} lets us extend a map of extensions by acting as the identity on a new variable. 

In the descriptions, the parameter telescopes are relayed directly to the constructors, but the variable telescope is reset to \AIC{∅} at the start of each constructor:
\ExecuteMetaData[Tex/Background]{U-par}
In the descriptions of constructors, we modify the \AIC{σ} code to request a type \AV{S} in the context of \AV{V}, which then also extends the context for the subsequent fields by \AV{S}:
\ExecuteMetaData[Tex/Background]{Con-par}
Replacing the function \AV{S →}\ \AD{U-sop} by \AD{Con-par}\ (\AV{V}\ \AIC{▷}\ \AV{S}) allows us to bind the value of \AV{S} while avoiding the higher order argument. The interpretation of the universe is then
\ExecuteMetaData[Tex/Background]{int-par}
where the \AIC{σ} case now provides the current parameters and variables to the field \AV{S}, and extends the variables by \AV{s} before passing them to the rest of the interpretation. In this universe, we can describe the parameters of lists with a one-type telescope:
\ExecuteMetaData[Tex/Background]{ListD}
This description declares that \AD{List} has two constructors, one with no fields, corresponding to \AIC{[]}, and the second with one field and a recursive field, representing \AIC{\_∷\_}. In the second constructor, we use pattern lambdas to deconstruct the telescope\footnote{Due to the interpretation of telescopes, the \AIC{∅} part always contributes a value \ARF{tt} we explicitly ignore, which also explicitly needs to be provided when passing parameters and variables.} and extract the type \AV{A}. 

Using the variable bound in \AIC{σ}, we can also give a description of the existential quantifier
\ExecuteMetaData[Tex/Background]{SigmaD}
having one constructor with two fields. The first field of type \AV{A} adds a value \AV{a} to the variable telescope, which we pass to \AV{B} in the second field by pattern matching on the variable telescope.


\subsubsection{Indexed types}\label{ssec:background-ix}
Lastly, we can integrate indexed types \cite{iir} into the universe by abstracting over indices:
\ExecuteMetaData[Tex/Background]{U-ix}
Recall that in native Agda datatypes, a choice of constructor can fix the indices of the recursive fields and the resultant type, so we encode:
\ExecuteMetaData[Tex/Background]{Con-ix}
Both \AIC{𝟙} and \AIC{ρ} now take a value of \AV{I} in context \AV{V}, where for \AIC{𝟙} this value represents the \emph{actual index}, and for \AIC{ρ} it represents the \emph{expected index} of the recursive field. Consider as an example \AD{Fin}, where ``\AIC{suc}\ \AV{n}'' bit of the constructor signatures tells us what the index of a constructor actually is, while the recursive type \AD{Fin}\ \AV{n} tells us which index the recursive field needs to have.

If we are constructing a term of some indexed type, then the previous choices of constructors and arguments build up the actual index of this term. This actual index must then match the expected index of the declaration of this term. Hence, in the case of a leaf, we replace the unit type with the necessary equality between the expected \AV{i} and actual index \AV{j} \cite{algorn}, while for a recursive field, the expected index \AV{j} is computed from the parameters and variables:
\ExecuteMetaData[Tex/Background]{int-ix}
With indexed types, we can describe finite types and vectors\footnote{Unlike some more elaborate encodings, we do not model implicit fields, so the descriptions of \AD{Fin} and \AD{Vec} look slightly different.} as
\ExecuteMetaData[Tex/Background]{FinD}
and:
\ExecuteMetaData[Tex/Background]{VecD}
In the first constructor of \AF{VecD} we report an actual index of \AIC{zero}. In the second, we have a field \bN{} to bring the index \AV{n} into scope, which is used to request a recursive field with index \AV{n}, and report the actual index of \AIC{suc}\ \AV{n}. For completeness, let us replicate the natural numbers and lists in \AD{U-ix}:
\ExecuteMetaData[Tex/Background]{new-Nat-List}
Writing the descriptions \AF{NatD}, \AF{ListD} and \AF{VecD} next to each other makes it easy to see the similarities: \AF{ListD} is the same as \AF{NatD} with a type parameter and an extra field \AIC{σ} of \AV{A}. Likewise, \AF{VecD} is the same as \AF{ListD}, but now indexing over \bN{} and with another extra field \AIC{σ} of \bN{}. In \autoref{sec:background-ornaments}, we will explain how this kind of analysis of descriptions can be performed formally inside Agda.

\subsubsection{Generic Programming}\label{ssec:generic-programming}
As a bonus, we can also use \AD{U-ix} for generic programming. For example, we can define the generic \AF{fold} operation\footnote{The full construction can be found in \ref{app:gfold}~Appendix A.}:
\ExecuteMetaData[Tex/Background]{fold-type}
From the point of view of category theory, we actually get \AF{fold} for free: since \AD{μ-ix}\ \AV{D} is the least fixpoint, or initial algebra, of \AF{⟦}\ \AV{D}\ \AF{⟧D}, \AF{fold}\ \AV{f} is simply the induced map to the algebra \AV{f :}\ \AF{⟦}\ \AV{D}\ \AF{⟧D}\ \AV{X}\ \AF{→₃}\ \AV{X}.

%An intuition for \AF{fold} will suffice here; \AF{fold} states that a map \AV{f} of \AF{⟦}\ \AV{D}\ \AF{⟧D}\ \AV{X}\ \AF{→₃}\ \AV{X} induces a map from \AD{μ-ix}\ \AV{D} to \AV{X}. 
More concretely, we can view \AF{⟦}\ \AV{D}\ \AF{⟧D}\ \AV{X} as a variant of \AF{μ-ix}\ \AV{D}, in which the recursive positions hold values of \AV{X} rather than other values of \AF{μ-ix}\ \AV{D}. For example, the type \AF{⟦}\ \AV{ListD}\ \AF{⟧D}\ \AV{X} reduces (up to equivalence) to \AD{⊤}\ \AD{⊎}\ (\AV{A}\ \AD{×}\ \AV{X}\ \AV{A}), substituting \AV{X}\ \AV{A} for what would usually be the recursive field.

In a sense, a term of \AF{⟦}\ \AV{D}\ \AF{⟧D}\ \AV{X} is a kind of \AV{D}-structured set of values of \AV{X}. From this perspective, \AF{fold} roughly states that an operation \AV{f}, collapsing \AV{D}-structured sets of \AV{X} into \AV{X}, extends to a function \AF{fold}\ \AV{f}. This function sends a whole value of \AF{μ-ix}\ \AV{D} into \AV{X}, recursively collapsing applications of \AIC{con} from the bottom up.

As an example, we can instantiate \AF{fold} to \AF{ListD}, which, by the aforementioned equivalence, corresponds to:
\ExecuteMetaData[Tex/Background]{foldr-type}
Much like the familiar \AF{foldr} operation lets us consume a \AD{List}\ \AV{A} to produce a value \AV{X A}; provided we give a value \AV{X A} for the \AIC{[]} case, and a means to convert a pair \AV{A}\ \AD{×}\ \AV{X A} to \AV{X A} for the \AIC{\_∷\_} case.

Do note that this version of \AF{fold} takes a polymorphic function as an argument, as opposed to the usual \AF{fold} which has the quantifiers on the outside:
\ExecuteMetaData[Tex/Background]{usual-fold}
Like a couple of constructions we will encounter in later sections, we can recover the usual \AF{fold} into a type \AV{C} by generalizing \AV{C} to the appropriate kind of maps into \AV{C}. For example, by letting \AV{X} be continuation-passing computations into \bN{}, we can recover:
\ExecuteMetaData[Tex/Background]{foldr-sum}


\subsection{Ornaments}\label{sec:background-ornaments}
In this section we will introduce a simplified definition of ornaments, and use it to compare descriptions. Simply looking at their descriptions, \bN{} and \AD{List} are rather similar, except that \AD{List} has some things \bN{} does not have. We could say that we can form the type of lists by starting from \bN{} and adding a parameter and field, while keeping everything else the same. In the other direction, we see that each list corresponds to a natural by stripping this information. Likewise, the type of vectors is almost identical to \AD{List}, can be formed from it by adding indices, and each vector corresponds to a list by dropping the indices.

Observations like these can be generalized using ornaments \cite{algorn, progorn, sijsling}, which define a binary relation describing which datatypes can be formed by ``decorating'' others. Conceptually, a type can be decorated by adding or modifying fields, extending its parameters, or refining its indices.

Essential to the concept of ornaments is the ability to convert back, forgetting the extra structure. After all, if there is an ornament from a description \AV{D} to \AV{E}, then \AV{E} should be \AV{D} with added fields, and more specific parameters and indices. This means that we should also be able to discard those extra fields, and revert to the less specific parameters and indices, obtaining a conversion from \AV{E} to \AV{D}. If \AV{D} is a \AD{U-ix}\ \AV{Γ}\ \AV{I} and \AV{E} is a \AD{U-ix}\ \AV{Δ}\ \AV{J}, then for a conversion from \AV{E} to \AV{D} to exist, there must also be functions \AV{re-par :}\ \AF{Cxf}\ \AV{Δ}\ \AV{Γ}  and \AV{re-index :}\ \AV{J}\ \AV{→}\ \AV{I} for re-parametrization and re-indexing.

In the same way that descriptions in \AD{U-ix} are lists of constructor descriptions, ornaments are lists of constructor ornaments. We define the type of ornaments re-parametrizing with \AV{re-par} and re-indexing with \AV{re-index} as a type indexed over \AD{U-ix}:
\ExecuteMetaData[Tex/Background]{Orn}
An ornament then induces a conversion between types via the forgetful map
\ExecuteMetaData[Tex/Background]{bimap}
\ExecuteMetaData[Tex/Background]{ornForget-type}
which reverts the modifications made by the constructor ornaments, and restores the original indices and parameters.

The definition of the constructor ornaments \AD{ConOrn} controls the kinds of modification ornaments allow. Each allowed modification, equivalently each constructor of \AD{ConOrn} also has to be reverted by \AF{ornForget}. Accordingly, some modifications will have preconditions, which are in this case always pointwise equalities:
\ExecuteMetaData[Tex/Background]{htpy}
Since constructors exist in the context of variables, we let constructor ornaments transform variables with \AV{re-var}, in addition to parameters and indices.

The first three constructors of \AD{ConOrn} represent the operations which copy the corresponding constructors of \AD{Con-ix}\footnote{Viewing \AD{ConOrn} as a binary relation on \AD{Con-ix}, these represent the preservation of \AD{ConOrn} by \AIC{𝟙}, \AIC{ρ}, and \AIC{σ}, up to parameters, variables, and indices.}. More precisely, as an example, the ornament \AIC{𝟙} states that if actual indices \AV{i} and \AV{j}, of potentially different types, are related, then the datatype constructors of the same names \AIC{𝟙}\ \AV{i} and \AIC{𝟙}\ \AV{j} are related.

By contrast, the \AIC{Δσ} ornament allows adding fields which are not present on the original datatype.:
\ExecuteMetaData[Tex/Background]{ConOrn}
The commuting square \AF{re-index}\ \AF{∘}\ \AV{j}\ \AF{∼}\ \AV{i}\ \AF{∘}\ \AF{var→par}\ \AV{re-var} in the first two constructors ensures that the indices on both sides are indeed related, up to \AV{re-index} and \AV{re-var}. As expected, we see that there can only be an ornament from a description \AV{D} to \AV{E} if there are constructor ornaments for all constructors; likewise, there can only be an ornament between constructors \AV{CD} and \AV{CE} if \AV{CE} consists wholly of added fields and fields copied from \AV{CD}, potentially refining parameters, variables, and indices.

Using these ornaments, we can make the claim that ``lists are indeed natural numbers decorated with fields'' more precise:
\ExecuteMetaData[Tex/Background]{NatD-ListD}
This ornament preserves most of the structure of \bN{}, and only adds a field \AV{A} using \AIC{∆σ}\footnote{Note that \AV{S}, and some later arguments we provide to ornaments, are implicit arguments: Agda would happily infer them from \AF{ListD} and then later from \AF{VecD}, had we omitted them.}. As \bN{} has no parameters or indices, \AD{List} has more specific parameters, namely a single type parameter. Consequently, all commuting squares factor through the unit type and can be satisfied with \AV{λ}\ \AV{\_}\ \AV{→}\ \AIC{refl}. 

We can also ornament lists to get vectors by re-indexing them over \bN{}:
\ExecuteMetaData[Tex/Background]{ListD-VecD}
We bind a new field of \bN{} with \AIC{∆σ}, extracting it in \AIC{𝟙} and \AIC{ρ} to declare that the constructor corresponding to \AIC{\_∷\_} takes a vector of length \AV{n} and returns a vector of length \AIC{suc}\ \AV{n}. 

The conversions from lists to natural numbers (\AF{length}), and from vectors to lists (\AF{toList}) arise as \AF{ornForget}, which we define using the \AF{fold} over an algebra that erases a single layer of decorations:
\ExecuteMetaData[Tex/Background]{ornForget}
Recursively applying this algebra, which reinterprets layers of \AV{E}-data as \AV{D}-data, lets us take apart a value in the fixpoint \AD{μ-ix}\ \AV{E} and rebuild it to a value of \AF{μ-ix}\ \AV{D}. This algebra
\ExecuteMetaData[Tex/Background]{ornAlg}
is a special case of the erasing function, which ``undecorates'' a single interpretation over an arbitrary type \AV{X}:
\ExecuteMetaData[Tex/Background]{ornErase}
By pattern matching on the ornament, \AF{conOrnErase} sees which bits of \AV{CE} are new, and which are copied from \AV{CD}. This tells us which parts of a term \AV{x} under an interpretation of \AV{CE} need to be forgotten, and which parts need to be copied or translated. Specifically, the first three cases of \AF{conOrnErase} correspond to the structure-preserving ornaments, and merely translate equivalent structures from \AV{CE} to \AV{CD}.

For example, the \AIC{𝟙}\ \AV{sq} case tells us that \AV{CD} is \AIC{𝟙}\ \AV{i'} and \AV{CE} is \AIC{𝟙} \AV{j'}. Recalling that a leaf \AIC{𝟙}\ \AV{j'} at parameter \AV{p} and expected index \AV{j} is interpreted as the equality \AV{j}\ \AD{≡}\ \AV{(j' p)}, we only need to produce the corresponding equality for  \AIC{𝟙}\ \AV{i'}, which is \AV{re-index j}\ \AD{≡}\ \AV{i' (}\AF{var→par}\ \AV{re-var p)}. This is precisely accomplished by applying \AF{re-index} to both sides and composing with the square \AV{sq} at \AV{p}. Likewise, in the case of \AIC{ρ} we have to show that \AV{x} can be converted from one \AIC{ρ} to the other \AIC{ρ} by translating its parameters, but in \AIC{σ} case, we can directly copy the field. Only the ornament \AIC{Δσ} adds a field, which is easily undone by removing that field.

In this way \AF{ornForget} reinforces the idea that \AV{E} is a decorated version of \AV{D} when there is an ornament from \AV{D} to \AV{E} by associating to each value of \AV{E} an underlying value in \AV{D}. This additionally makes it easier to relate functions between related types. For example, instantiating \AF{ornForget} for \AF{NatD-ListD} yields \AF{length}. Hence, the statement that \AF{length} sends concatenation \AF{\_++\_} to addition \AF{\_+\_}, that is \AF{length}\ \AV{(xs}\ \AF{++}\ \AV{ys)}\ \AD{≡}\ \AF{length}\ \AV{xs}\ \AF{+}\ \AF{length}\ \AV{ys}, is equivalent to the statement that \AF{\_++\_} and \AF{\_+\_} are related, or that \AF{\_++\_} is a lifting of \AF{\_+\_} \cite{orntrans}.
% remark, ornForget is not epi in general because of ∆σ ⊥

\subsection{Ornamental Descriptions}\label{sec:background-ornamental-descriptions}
By defining the ornaments \AF{NatD-ListD} and \AF{ListD-VecD} we demonstrated that lists are numbers with fields, and vectors are lists with fixed lengths. Even though we had to give \AF{ListD} before we could define \AF{NatD-ListD}, the value of \AF{NatD-ListD} actually forces the right-hand side to be \AF{ListD}.

If we somehow could leave out the right-hand side of ornaments, then we can also use ornaments to represent descriptions as patches on top of other descriptions. So, \emph{ornamental descriptions} are precisely defined as ornaments without the right-hand side, effectively bundling a description and an ornament to it\footnote{Consequently, \AD{OrnDesc}\ \AV{Δ}\ \AV{J}\ \AV{g}\ \AV{i}\ \AV{D} must simply be a convenient representation of \AD{Σ}\ (\AD{U-ix}\ \AV{Δ}\ \AV{J})\ (\AD{Orn}\ \AV{g}\ \AV{i}\ \AV{D}).}. Their definition is analogous to that of ornaments, making the arguments which would only appear in the new description explicit:
\ExecuteMetaData[Tex/Background]{OrnDesc}
\ExecuteMetaData[Tex/Background]{ConOrnDesc}
Using \AD{OrnDesc}, we can describe \AF{ListD} as a patch on \AF{NatD}, inserting a \AIC{σ} in the constructor corresponding to \AIC{suc}:
\ExecuteMetaData[Tex/Background]{ListOD}
Since an ornamental description simply represents a patch on top of a description, we can also extract the patched description and the ornament to it. To extract the description, we can use the projection applying the patch in an ornamental description
\ExecuteMetaData[Tex/Background]{toDesc}
which would extract \AF{ListD} out of \AF{ListOD}.

The other projection reconstructs the ornament to the extracted description
\ExecuteMetaData[Tex/Background]{toOrn}
and would extract \AF{NatD-ListD} from \AF{ListOD}. As a consequence, \AD{OrnDesc} enjoys the features of both \AD{Desc} and \AD{Orn}, such as interpretation into a datatype by \AF{μ} and the conversion to the underlying type by \AF{ornForget}, by factoring through these projections.

In later sections, we will routinely use \AD{OrnDesc} to view triples like (\AF{NatD}, \AF{ListD}, \AF{VecD}) as a base type equipped with two patches in sequence.
% exercise to reader: show OrnDesc AD ~ Exist[ BD in Desc ] Orn AD BD  