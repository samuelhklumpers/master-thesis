\section{Agda}\label{sec:background-agda}
\changed{Add more citations and then double check them below here}
We formalize our work in the programming language Agda \cite{agda}. While we will only occasionally reference Haskell, those more familiar with Haskell may look at Agda (in rough approximation) as an extension\todo{extension? reasonable agda is haskell, so agda kind of embeds into haskell} of Haskell\cite{?}. Agda is a total functional programming language with dependent types. Here, totality means that functions of a given type always terminate in a value of that type, ruling out non-terminating (and not obviously terminating) programs. Using dependent types we can use Agda as a proof assistant, allowing us to state and prove theorems about our datastructures and programs. 

In this section, we will explain and highlight some parts of Agda which we use in the later sections. 
Many of the types we use in this section are also described and explained in most Agda tutorials (\cite{ulftutorial}, \cite{plfa}, etc.), and can be imported from the standard library \cite{agdastdlib}.

We use \texttt{--type-in-type} %and \texttt{--with-K}
to keep the explanations more readable. % Even though the former makes Agda inconsistent, and the latter is not strictly necessary, we know that our work can be ported to a setting with neither option \autoref{app:withoutK}.


\section{Data in Agda}\label{sec:background-data}
At the level of generalized algebraic datatypes, Agda is close to Haskell. In both languages, one can define objects using data declarations, and interact with them using function declarations. For example, we can define the type of \emph{booleans}:
\ExecuteMetaData[Tex/Background]{Bool}
The constructors of this type state that we can make values of \AD{Bool} in exactly two ways: \AIC{false} and \AIC{true}. We can then define functions on \AD{Bool} by pattern matching. As an example, we can define the conditional operator as
\ExecuteMetaData[Tex/Background]{conditional}
When \emph{pattern matching}, the coverage checker ensures we define the function on all cases of the type matched on, and thus the function is completely defined. % mark: shuffle

We can also define a type representing the natural numbers
\ExecuteMetaData[Tex/Background]{Nat}
Here, \bN{} always has a \AF{zero} element, and for each element $n$ the constructor \AIC{suc} expresses that there is also an element representing $n + 1$. Hence, \bN{} represents the \textit{naturals} by encoding the existential axioms of the Peano axioms. By pattern matching and recursion on \bN{}, we define the less-than operator:
\ExecuteMetaData[Tex/Background]{lt}
One of the cases contains a recursive instance of \bN{}, so termination checker also verifies that this recursion indeed terminates, ensuring that we still define \AV{n}\ \AF{<?} \AV{m} for all possible combinations of \AV{n} and \AV{m}. %Essentially, the coverage and termination checker make sure that any valid definition by pattern matching corresponds to a valid proof by cases and induction.
In this case the recursion is valid, since both arguments decrease before the recursive call, meaning that at some point \AV{n} or \AV{m} hits \AIC{zero} and the recursion terminates.

Like in Haskell, we can \emph{parametrize} a datatype over other types to make \emph{polymorphic} type, which we can use to define lists of values for all types:
\ExecuteMetaData[Tex/Background]{List} 
A list of \AV{A} can either be empty \AIC{[]}, or contain an element of \AV{A} and another list via \AIC{\_∷\_}. In other words, \AD{List} is a type of \emph{finite sequences} in \AV{A} (in the sense of sequences as an abstract type \cite{purelyfunctional}).

Using polymorphic functions, we can manipulate and inspect lists by inserting or extracting elements. For example, we can define a function to look up the value at some position \AV{n} in a list
\ExecuteMetaData[Tex/Background]{lookup-list}
However, this function \emph{partial}, as we are relying on the type
\ExecuteMetaData[Tex/Background]{Maybe}
to handle the case where the position falls outside the list and we cannot return an element. 
If we know the length of the list \AV{xs}, then we also know for which positions \AF{lookup} will succeed, and for which it will not. We define 
\ExecuteMetaData[Tex/Background]{length}
so that we can test whether the position \AV{n} lies inside the list by checking \AV{n}\ \AF{<?}\ \AF{length}\ \AV{xs}. If we declare \AF{lookup} as a dependent function consuming a proof of \AV{n}\ \AF{<?}\ \AF{length}\ \AV{xs}, then \AF{lookup} always succeeds. However, this actually only moves the burden of checking whether the output was \AIC{nothing} afterwards to proving that \AV{n}\ \AF{<?}\ \AF{length}\ \AV{xs} beforehand.

We can avoid both by defining an \emph{indexed type} representing numbers below an upper bound
\ExecuteMetaData[Tex/Background]{Fin}
Like parameters, indices add a variable to the context of a datatype, but unlike parameters, indices can influence the availability of constructors. The type \AD{Fin} is defined such that a variable of type \AD{Fin}\ \AV{n} represents a number less than \AV{n}.
% \footnote{We can also see this in another way: an argument of type \AD{Fin}\ \AV{n} corresponds to a pair of an argument \AV{m} and an argument \AV{m}\ \AF{<?}\ \AV{n}.}
Since both constructors \AIC{zero} and \AIC{suc} dictate that the index is the \AIC{suc} of some natural \AV{n}, we see that \AD{Fin}\ \AIC{zero} has no values. On the other hand, \AIC{suc} gives a value of \AD{Fin}\ (\AIC{suc}\ \AV{n}) for each value of \AD{Fin}\ \AV{n}, and \AIC{zero} gives exactly one additional value of \AD{Fin}\ (\AIC{suc}\ \AV{n}) for each \AV{n}. By induction (externally), we find that \AD{Fin}\ \AV{n} has exactly \AV{n} closed terms, each representing a number less than \AV{n}.

To complement \AD{Fin}, we define another indexed type representing lists of a known length, also known as vectors:
\ExecuteMetaData[Tex/Background]{Vec}
The \AIC{[]} constructor of this type produces the only term of type \AD{Vec}\ \AV{A}\ \AIC{zero}. The \AIC{\_∷\_} constructor ensures that a \AD{Vec}\ \AV{A}\ (\AIC{suc}\ \AV{n}) always consists of an element of \AV{A} and a \AD{Vec}\ \AV{A}\ \AV{n}. By induction, we find that a \AD{Vec}\ \AV{A}\ \AV{n} contains exactly \AV{n} elements of \AV{A}. Thus, we conclude that \AD{Fin}\ \AV{n} is exactly the type of positions in a \AD{Vec}\ \AV{A}\ \AV{n}. In comparison to \AD{List}, we can say that \AD{Vec} is a type of arrays (in the sense of arrays as the abstract type of sequences of a fixed length). Furthermore, knowing the index of a term \AV{xs} of type \AV{Vec}\ \AV{A}\ \AV{n} uniquely determines the the constructor it was formed by. Namely, if \AV{n} is \AIC{zero}, then \AV{xs} is \AIC{[]}, and if \AV{n} is \AIC{suc} of \AV{m}, then \AV{xs} is formed by \AIC{\_∷\_}. 

Using this, we define a variant of \AF{lookup} for \AD{Fin} and \AD{Vec}, taking a vector of length \AV{n} and a position below \AV{n}:
\ExecuteMetaData[Tex/Background]{lookup}
The case in which we would return \AIC{nothing} for lists, which is when \AV{xs} is \AIC{[]}, is omitted. This happens because \AV{x} of type \AD{Fin}\ \AV{n} is either \AIC{zero} or \AIC{suc}\ \AV{i}, and both cases imply that \AV{n} is \AIC{suc}\ \AV{m} for some \AV{m}. As we saw above, a \AD{Vec}\ \AV{A}\ (\AIC{suc} \AV{m}) is always formed by \AIC{\_∷\_}, making the case in which \AV{xs} is \AIC{[]} impossible. Consequently, lookup always succeeds for vectors,
% demonstrating that vectors are correct-by-construction. 
however, this does not yet prove that \AF{lookup} necessarily returns the right element, we will need some more logic to verify this.

\section{Proving in Agda}\label{sec:background-proving}
To describe equality of terms we define a new type
\ExecuteMetaData[Tex/Background]{equiv}
If we have a value \AV{x} of \AV{a}\ \AD{≡}\ \AV{b}, then, as the only constructor of \AD{\_≡\_} is \AIC{refl}, we must have that \AV{a} is equal to \AV{b}. We can use this type to describe the behaviour of functions like \AF{lookup}: If we insert elements into a vector with
\ExecuteMetaData[Tex/Background]{insert}
we can express the correctness of \AF{lookup} as
\ExecuteMetaData[Tex/Background]{lookup-insert-type}
stating that we expect to find an element where we insert it.

% When we use pattern matching in a function, the coverage and termination checker ensure that the resulting function is total and defined by well-founded recursion\cite{?}. If we are proving some statement by constructing a function as a proof, this means that we can interpret a function definition by (dependent) pattern matching and well-founded recursion as a proof by well-founded induction\cite{?}.

%So, to 
To prove the statement, we proceed as when defining any other function. 
By simultaneous induction on the position and vector, we prove
\ExecuteMetaData[Tex/Background]{lookup-insert}
In the first two cases, where we \AF{lookup} the first position, \AF{insert}\ \AV{xs}\ \AIC{zero}\ \AV{y} simplifies to \AV{y}\ \AF{∷}\ \AF{xs}, so the lookup immediately returns \AV{y} as wanted. In the last case, we have to prove that \AF{lookup} is correct for \AV{x}\ \AF{∷}\ \AF{xs}, so we use that the \AF{lookup} ignores the term \AV{x} and we appeal to the correctness of \AF{lookup} on the smaller list \AV{xs} to complete the proof.

Like \AD{\_≡\_}, we can encode many other logical operations into datatypes, which establishes a correspondence between types and formulas, known as the Curry-Howard isomorphism. For example, we can encode disjunctions (the logical `or' operation) as
\ExecuteMetaData[Tex/Background]{uplus}

The other components of the isomorphism are as follows. Conjunction (logical `and') can be represented by\footnote{We use a record here, rather than a datatype with a constructor \AV{A → B →}\ \AV{A}\ \AD{×}\ \AV{B}. The advantage of using a record is that this directly gives us projections like \ARF{fst}\ \AV{:}\ \AV{A}\ \AD{×}\ \AV{B}\ \AV{→ A}, and lets us use eta equality, making $(a, b) = (c , d) \iff a = c \land b = d$ holds automatically.}
\ExecuteMetaData[Tex/Background]{product}
True and false are respectively represented by
\ExecuteMetaData[Tex/Background]{true}
so that always \AIC{tt}\ \AV{:}\ \AD{⊤}, and 
\ExecuteMetaData[Tex/Background]{false}
The body of \AD{⊥} is not accidentally left out: because \AD{⊥} has no constructors, there is no proof of false\footnote{If we did not use \AV{--type-in-type}, and even in that case I can only hope.}.

Because we identify function types with logical implications, we can also define the negation of a formula \AV{A} as ``\AV{A} implies false'':
\ExecuteMetaData[Tex/Background]{not}
The logical quantifiers $\forall$ and $\exists$ act on formulas with a free variable in a specific domain of discourse.
%\footnote{Whether this variable is actually used does not matter: every day the sun shines, $1 = 1$.}
We represent closed formulas by types, so we can represent a formula with a free variable of type \AV{A} by a function values of \AV{A} to types \AV{A}\ \AV{→}\ \AD{Type}, also known as a predicate. The universal quantifier $\forall a P(a)$ is true when for all $a$ the formula $P(a)$ is true, so we represent the universal quantification of a predicate \AV{P} as a dependent function type \AV{(a : A) → P a}, producing for each \AV{a} of type \AV{A} a proof of \AV{P}\ \AV{a}. The existential quantifier $\exists a P(a)$ is true when there is some $a$ such that $P(a)$ is true, so we represent the existential quantification as
\ExecuteMetaData[Tex/Background]{exists}
so that we have \AD{Σ}\ \AV{A}\ \AV{P} iff we have an element \AV{fst} of \AV{A} and a proof \AV{snd} of \AV{P}\ \AV{a}. To avoid the need for lambda abstractions in existentials, we define the syntax
\ExecuteMetaData[Tex/Background]{sigma-syntax}
letting us write \AD{Σ[}\ \AV{a}\ \AD{∈}\ \AV{A}\ \AD{]}\ \AV{P a} for $\exists a P(a)$.

\section{Descriptions}\label{sec:background-descriptions}
In the previous sections we completed a quadruple of types (\bN{}, \AD{List}, \AD{Vec}, \AD{Fin}), 
%, even computing the latter two from \bN{}.
which have nice interactions (\AF{length}, \AF{lookup}). Similar to the type of \AF{length}\ \AV{:}\ \AD{List}\ \AV{A}\ \AV{→}\ \bN{}, we can define
\ExecuteMetaData[Tex/Background]{toList}
converting vectors back to lists. In the other direction, we can also promote a list to a vector by recomputing its index:
\ExecuteMetaData[Tex/Background]{toVec}
We claim that is not a coincidence, but rather happens because \bN{}, \AD{List}, and \AD{Vec} have the same ``shape''.

But what is the shape of a datatype? In this section, we will explain a framework of datatype descriptions and ornaments, allowing us to describe the shapes of datatypes and use these for generic programming \cite{ulftutorial, mcbride, others}. Recall that while polymorphism allows us to write one program for many types at once, those programs act parametrically \cite{reynolds, forfree}: polymorphic functions must work for all types, thus they cannot inspect values of their type argument. Generic programs, in contrast, can use the structure of a datatype, allowing for more complex functions that do inspect values.

Using datatype descriptions we can then relate \bN{}, \AD{List} and \AD{Vec}, explaining how \AF{length} and \AF{toList} are instances of a generic construction. Let us walk through some ways of defining descriptions. We will start from simpler descriptions, building our way up to more general types, until we reach a framework in which we can describe \bN{}, \AD{List}, \AD{Vec} and \AD{Fin}. 
%, which, as a bonus, gives some insight into the meaning of datatypes.


\subsection{Finite types}\label{ssec:background-fin}
A datatype description, which are datatypes of which each value again represents a datatype, consist of two components. Namely, a type of descriptions \AV{U}, also referred to as codes, and an interpretation \AV{U}\ \AV{→}\ \AD{Type}, decoding descriptions to the represented types. In the terminology of Martin-L{\"{o}}f type theory\cite{levitation} \todo{No citation for MLTT? Agda is a rather loose extension, none of the original papers really match.}, where types of types like \AD{Type} are called universes, we can think of a type of descriptions as an internal universe.

As a start, we define a basic universe with two codes \AIC{𝟘} and \AIC{𝟙}, respectively representing the types \AD{⊥} and \AD{⊤}, and the requirement that the universe is closed under sums and products:
\ExecuteMetaData[Tex/Background]{U-fin}
The meaning of the codes in this universe is then assigned by the interpretation
\ExecuteMetaData[Tex/Background]{int-fin}
which indeed sends \AIC{𝟘} to \AD{⊥}, \AIC{𝟙} to \AD{⊤}, sums to sums and products to products\footnote{One might recognize that \AF{⟦\_⟧fin} is a morphism between the rings (\AD{U-fin}, \AIC{⊕}, \AIC{⊗}) and (\AD{Type}, \AD{⊎}, \AD{×}). Similarly, \AD{Fin} also gives a ring morphism from \bN{} with \AF{+} and \AF{×} to \AD{Type}, and in fact \AF{⟦\_⟧fin} factors through \AD{Fin} via the map sending the expressions in \AD{U-fin} to their value in \bN{}.}.

In this universe, we can encode the type of booleans simply as 
\ExecuteMetaData[Tex/Background]{BoolD}
The types \AIC{𝟘} and \AIC{𝟙} are finite, and sums and products of finite types are also finite, which is why we call \AD{U-fin} the universe of finite types. Consequently, the type of naturals \bN{} cannot fit in \AD{U-fin}.

\subsection{Recursive types}\label{ssec:background-rec}
To accommodate \bN{}, we need to be able to express recursive types. By adding a code \AIC{ρ} to \AD{U-fin} representing recursive type occurrences, we can express those types: 
\ExecuteMetaData[Tex/Background]{U-rec}
However, the interpretation cannot be defined like in the previous example: when interpreting \AIC{𝟙}\ \AIC{⊕}\ \AIC{ρ}, we need to know that the whole type was \AIC{𝟙}\ \AIC{⊕}\ \AIC{ρ} while processing \AIC{ρ}. As a consequence, we have to split the interpretation in two phases. First, we interpret the descriptions into polynomial functors
\ExecuteMetaData[Tex/Background]{int-rec}
Then, by viewing such a functor as a type with a free type variable, the functor can model a recursive type by setting the variable to the type itself:
\ExecuteMetaData[Tex/Background]{mu-rec}
Recall the definition of \bN{}, which can be read as the declaration that \AD{ℕ} is a fixpoint: \AD{ℕ}\ \AD{≡}\ \AV{F}\ \AD{ℕ} for \AV{F X = ⊤ ⊎ X}. This makes representing \bN{} as simple as:
\ExecuteMetaData[Tex/Background]{NatD}

\subsection{Sums of products}\label{ssec:background-sop}
A downside of \AD{U-rho} is that the definitions of types do not mirror their equivalent definitions in user-written Agda. We can define a similar universe using that polynomials can always be canonically written as sums of products. For this, we split the descriptions into a stage in which we can form sums, on top of a stage where we can form products.
\ExecuteMetaData[Tex/Background]{U-sop}
When doing this, we can also let the left-hand side of a product be any type, allowing us to represent ordinary fields:
\ExecuteMetaData[Tex/Background]{Con-sop}
The interpretation of this universe, while analogous to the one in the previous section, is also split into two parts:
\ExecuteMetaData[Tex/Background]{int-sop}
In this universe, we can define the type of lists as a description quantified over a type:
\ExecuteMetaData[Tex/Background]{ListD-bad}
Using this universe requires us to split functions on descriptions into multiple parts, but makes interconversion between representations and concrete types straightforward.

\subsection{Parametrized types}\label{ssec:background-par}
The encoding of fields in \AD{U-sop} makes the descriptions large in the following sense: by letting \AV{S} in \AIC{σ} be an infinite type, we can get a description referencing infinitely many other descriptions. As a consequence, we cannot inspect an arbitrary description in its entirety. We will introduce parameters in such a way that we recover the finiteness of descriptions as a bonus.

% move to discussion \footnote{If a foldable universe means nothing to you, there are simpler encodings for parameters and indices, which are recorded in \autoref{app:large-sigma}.}

In the last section, we saw that we could define the parametrized type \AD{List} by quantifying over a type. However, in some cases, we will want to be able to inspect or modify the parameters belonging to a type\footnote{For example, deriving Traversable for parametrized types as functions would not be possible (without macros), as one could not decide whether the signature of a type in a field is compatible.}. To represent the parameters of a type, we will need a new gadget.

In a naive attempt, we can represent the parameters of a type as \AD{List}\ \AD{Type}. However, this cannot represent many useful types, of which the parameters depend on each other. For example, in the existential quantifier \AD{Σ\_}, the type \AV{A}\ \AV{→}\ \AD{Type} of second parameter \AV{B} references back to the first parameter \AV{A}.

%\todo{Bij 6.4 list types - rules out existential -- waarom? (Ik weet het antwoord wel, maar misschien is het goed om er expliciet bij stil te staan)}.

In a general parametrized type, parameters can refer to the values of all preceding parameters. The parameters of a type are thus a sequence of types depending on each other, which we call telescopes \cite{practgen, sijsling, telescopes} (also known as contexts in \todo{MLTT}). We define telescopes using induction-recursion:
\ExecuteMetaData[Tex/Background]{Tel-simple}
A telescope can either be empty, or be formed from a telescope and a type in the context of that telescope. Here, we used the meaning of a telescope \AF{⟦\_⟧tel} to define types in the context of a telescope. This meaning represents the valid assignment of values to parameters:
\ExecuteMetaData[Tex/Background]{int-simple}
interpreting a telescope into the dependent product of all the parameter types.

This definition of telescopes would let us write down the type of \AD{Σ}:
\ExecuteMetaData[Tex/Background]{sigma-tel}
but is not sufficient to give its definition, as we need to be able to bind a value \AV{a} of \AV{A} and reference it in the field \AV{P}\ \AV{a}. By quantifying telescopes over a type, we can represent bound arguments using almost the same setup \cite{practgen}:
\ExecuteMetaData[Tex/Background]{Tel-type}
A \AD{Tel}\ \AV{P} then represents a telescope for each value of \AV{P}, which we can view as a telescope in the context of \AV{P}. For readability, we redefine values in the context of a telescope as:
\ExecuteMetaData[Tex/Background]{entails}
so we can define telescopes and their interpretations as:
\ExecuteMetaData[Tex/Background]{Tel-def}
By setting \AV{P}\ =\ \AD{⊤}, we recover the previous definition of parameter-telescopes. We can then define an extension of a telescope as a telescope in the context of a parameter telescope:
\ExecuteMetaData[Tex/Background]{ExTel}
representing a telescope of variables over the fixed parameter-telescope \AV{Γ}, which can be extended independently of \AV{Γ}. Extensions can be interpreted by interpreting the variable part given the interpretation of the parameter part:
\ExecuteMetaData[Tex/Background]{int-ExTel}
In the descriptions directly relay the parameter telescope to the constructors, resetting the variable telescope to \AIC{∅} for each constructor:
\ExecuteMetaData[Tex/Background]{U-par}
Of the constructors we only modify the \AIC{σ} to request a type \AV{S} in the context of \AV{V}, and to extend the context for the subsequent fields by \AV{S}:
\ExecuteMetaData[Tex/Background]{Con-par}
Replacing the function \AV{S →}\ \AD{U-sop} by \AD{Con-par}\ (\AV{V}\ \AIC{▷}\ \AV{S}) allows us to bind the value of \AV{S} while avoiding the higher order argument. 
We define a helper
\ExecuteMetaData[Tex/Background]{tele-helpers}
and interpret this universe as follows:
\ExecuteMetaData[Tex/Background]{int-par}
In particular, provide \AV{X} the parameters and variables in the \AIC{σ} case, and extend context by \AV{s} before passing to the rest of the interpretation.

In this universe, we can describe lists using a one-type telescope:
\ExecuteMetaData[Tex/Background]{ListD}
This description declares that \AD{List} has two constructors, one with no fields, corresponding to \AIC{[]}, and the second with one field and a recursive field, representing \AIC{\_∷\_}. In the second constructor, we used pattern lambdas to deconstruct the telescope and extract the type \AV{A}\footnote{Due to a quirk in the interpretation of telescopes, the \AIC{∅} part always contributes a value \ARF{tt} we explicitly ignore, which also explicitly needs to be provided when passing parameters and variables.}.
Using the variable bound in \AIC{σ}, we can also define the existential quantifier:
\ExecuteMetaData[Tex/Background]{SigmaD}
having one constructor with two fields. Here, the first field of type \AV{A} adds a value \AV{a} to the variable telescope, which we recover in the second field by pattern matching, before passing it to \AV{B}.


\subsection{Indexed types}\label{ssec:background-ix}
Lastly, we can integrate indexed types into the universe by abstracting over indices\todo{We could also use a telescope for indices, but we do not.}
\ExecuteMetaData[Tex/Background]{U-ix}
Recall that in native Agda datatypes, a choice of constructor can fix the indices of the recursive fields and the resultant type, so we encode:
\ExecuteMetaData[Tex/Background]{Con-ix}
%In most cases, the index is simply threaded through the interpretation, allowing for a choice in the relevant codes.
If we are constructing a term of some indexed type, then the previous choices of constructors and arguments build up the actual index of this term. This actual index must then match the index we expected in the declaration of this term. This means that in the case of a leaf, we have to replace the unit type with the necessary equality between the expected and actual indices:
\ExecuteMetaData[Tex/Background]{int-ix}
In a recursive field, the expected index can be chosen based on parameters and variables. % mark: wording

In this universe, we can define finite types and vectors as:
\ExecuteMetaData[Tex/Background]{FinD}
and
\ExecuteMetaData[Tex/Background]{VecD}
These are equivalent, but since we do not model implicit fields, they are slightly different in use compared to \AD{Fin} and \AD{Vec}. In the first constructor of \AF{VecD} we report an actual index of \AIC{zero}. In the second, we have a field \bN{} to bring the index \AV{n} into scope, which is used to request a recursive field with index \AV{n}, and report the actual index of \AIC{suc}\ \AV{n}.

We can now compare the structures in the quadruple (\bN{}, \AD{List}, \AD{Fin}, \AD{Vec}) by looking at their descriptions.

As a bonus, we can also use \AD{U-ix} for generic programming. For example, by a long construction which can be found in \autoref{app:gfold}, we can define the generic \AF{fold} operation\footnote{Do note that this version takes a polymorphic function as an argument, as opposed to the usual fold which has the quantifiers on the outside. The usual fold can be recovered by giving \AV{X} an argument to assert that the type variable is actually equal to some concrete type.}:
\ExecuteMetaData[Tex/Background]{fold-type}
Intuitively, \AF{fold} operation works as follows: Suppose the information of one constructor application of \AV{D}, where the recursive positions are valued in \AV{X}, can be collapsed to a value of \AV{X} again. Then, by recursively collapsing the constructors from the bottom up, we can collapse all values of \AD{μ-ix}\ \AV{D} to values of \AV{X}.

As a more concrete example, instantiating \AF{fold} to \AF{ListD}, we get (up to some type equivalences):
\ExecuteMetaData[Tex/Background]{foldr-type}
which, much like the familiar foldr operation lets us consume a list to a value \AV{X A}, provided a value \AV{X A} in the empty case, and a means to convert a pair of \AV{A} and \AV{X A} to \AV{X A}.

% deriving sum is still technical

\section{Ornaments}\label{sec:background-ornaments}
\todo{We also skip over field deletion and such because we don't need them, exercise to the reader.}
\todo{recheck}
In this section we will introduce the concept of an ornament, used to compare descriptions, and give a simplified definition. Since we settled on \AD{U-ix} as our universe (for now), we redefine and rename a couple of things for readability and reusability:
\ExecuteMetaData[Tex/Background]{new-Nat-List}
Purely looking at their descriptions, \bN{} and \AD{List} are rather similar, except that \AD{List} has a parameter and an extra field \bN{} does not have. We could say that we can form the type of lists by starting from \bN{} and adding this parameter and field, while keeping everything else the same. In the other direction, we see that each list corresponds to a natural by stripping this information. Likewise, the type of vectors is almost identical to \AD{List}, can be formed from it by adding indices, and each vector corresponds to a list by dropping the indices.

These and similar observations can be generalized using ornaments \cite{algorn, progorn, sijsling}, which define a language or binary relation, describing which datatypes can be formed by decorating others. Conceptually, an ornament from a type \AV{A} to a type \AV{B} represents that \AV{B} can be formed from \AV{A} by adding information or making the indices more specific. Consequently, for each ornament from \AV{A} to \AV{B}, we expect to get a function from \AV{B} to \AV{A} erasing this information and reverting to less specific indices.

If the indices \AV{J} and parameters \AV{Δ} of \AV{B} are more specific than the indices \AV{I} and parameters \AV{Γ} of \AV{A}, we require functions from \AV{J} to \AV{I} and from \AV{Δ} to \AV{Γ}. % mark: does not parse 
Our ornaments
\ExecuteMetaData[Tex/Background]{Orn-type}
should then come equipped with a function:
\ExecuteMetaData[Tex/Background]{ornForget-type}
where we define \AF{Cxf} as the type of functions between (the interpretations of) \AV{Δ} and \AV{Γ}.

Since we are working with sums of products descriptions, we can decide that ornaments cannot change the number or order of constructors, and the actual work happens in the constructor ornaments:
\ExecuteMetaData[Tex/Background]{ConOrn-type}
and we define ornaments as lists of ornaments for all constructors:
\ExecuteMetaData[Tex/Background]{Orn}
(Similarly to \AF{Cxf}, we use \AF{Cxf′} as the type of functions between variables, respecting \AV{g}.)

To (readably) write down \AD{ConOrn}, we use a couple of helpers to manipulate telescopes:
\ExecuteMetaData[Tex/Background]{ConOrn-helpers}
% these only serve to add noise to this picture and appease the type checker. conceptually, you can view them as the ``natural'' conversions between some slightly different telescopes
These mostly interconvert some values between similar telescopes. But notably, if \AV{S} is of type \AV{V ⊢ Type}, then \AV{S} is a type in the context of \AV{V}, while \AV{V ⊧ S} is the type of values of \AV{S} in the context of \AV{V}.

Now we can define \AD{ConOrn}. Of course, we expect that adding nothing gives the identity ornament, which is encoded in the first three constructors of \AD{ConOrn}.
\ExecuteMetaData[Tex/Background]{ConOrn}
However, since the parameters, indices, and variables need not be identical on both sides (in particular, the variables can diverge even more depending on the preceding ornament), we have to ask that for \AIC{𝟙} and \AIC{ρ}, these are related by a structure respecting conversion, or more graphically, a commuting square\footnote{While for \AIC{σ}, we bake the relatedness of the fields in by letting the resulting descriptions only differ by the conversion \AV{v}.}. In fact, we will soon see that these pieces of information are exactly what we need to complete \AF{ornForget}.

%The other two constructors, \AIC{∆} and \AIC{∇}, state that we can add fields, and remove fields if we provide a default value, respectively. Again, the constructor \AIC{∆} which adds a field depending on the variables requires some manner of commuting square.
The other constructors \AIC{∆} states that we can add fields.

We can now formulate the formation of \AD{List} from \bN{} as an ornament:
\ExecuteMetaData[Tex/Background]{NatD-ListD}
Using that \bN{} has no parameters or indices, we see that \AD{List} has more specific parameters, namely a single type parameter, and also no indices. Because of this, all commuting squares factor through the unit type and are hence (fortunately) trivial. This ornament preserves most structure of \bN{}, only adding a field of the type parameter of \AD{List} using \AIC{∆}.

We can also ornament \AD{List} to become \AD{Vec}, for which the index is more informative,
but the ornament does equally little:
\ExecuteMetaData[Tex/Background]{ListD-VecD}
Now the commuting square for the indices is equally trivial, but while the square for the parameters is still trivial, it is now an identity square, rather than a constant one.

We deferred the definition of \AF{ornForget}, so let us give it now. The process is split into two steps: first, we define a function to strip off a single layer of ornamentation:
\ExecuteMetaData[Tex/Background]{ornErase-type}
which uses the commutativity squares we required earlier to revert some values (and parameters, indices, and variables) to the unornamented type. For example, in the case of the \AIC{𝟙} preserving ornament\footnote{The other cases can be found in \autoref{app:ornforget}.}:
\ExecuteMetaData[Tex/Background]{ornErase}
This function defines an algebra for the functor associated to a description \AV{E}:
\ExecuteMetaData[Tex/Background]{ornAlg}
We can now make good use of the generic \AF{fold} we defined for \AD{U-ix}!
\ExecuteMetaData[Tex/Background]{ornForget}
The function \AF{ornForget} also makes it easy to generalize relations of functions between similar types. For example, if we instantiate \AF{ornForget} for \AF{ℕD-ListD}, then the statement that list concatenation preserves length can equivalently be expressed as the commutation of concatenation and \AF{ornForget}.

%[ also, recomputation, but appendix ]: #
%[ remark, ornForget is never epi because of \sigma 0 ]: #

%%outline:
%we explained why descriptions and ornaments are crucial to achieve our goals
%however, the descriptions we explained earlier are not powerful enough to house finger trees


%To capture finger trees as an ornament over a number system, we will need to describe ornaments over nested datatypes. In this section we will work out descriptions and ornaments suitable for nested datatypes.
If we are going to simplify working with complex containers %, such as finger trees,
by instantiating generic programs to them, we should first make sure that these types fit into the descriptions.

We construct descriptions for nested datatypes by extending the encoding of parametric and indexed datatypes from \autoref{ssec:bg-desc} with three features: information bundles, parameter transformation, and description composition. Also, to make sharing constructors easier, we introduce variable transformations. Transforming variables before they are passed to child descriptions allows both aggressively hiding variables and introducing values as if by let-constructs.

We base the encoding of off existing encodings \cite{sijsling,practgen}. The descriptions take shape as sums of products, enforce indices at leaf nodes, and have explicit parameter and variable telescopes. Unlike some encodings \cite{effectfully, practgen}, we do not allow higher-order inductive arguments. 

We use type-in-type and with-K to simplify the presentation, noting that these can be eliminated respectively by moving to Typeω and by implementing interpretations as datatypes.

\section{The descriptions}\label{ssec:desc}
We use telescopes identical to those in \autoref{ssec:bg-desc}:
\ExecuteMetaData[Ornament/Desc]{telescopes}
Recall that a \AgdaDatatype{Tel} represents a sequence of types, which can depend on the external type $P$. This lets us represent a telescope succeeding another using \AgdaDatatype{ExTel}. A term of the interpretation \AgdaFunction{⟦\_⟧tel} is then a sequence of terms of all the types in the telescope.

We use some shorthands
\ExecuteMetaData[Ornament/Desc]{tele-shorthands}
\ExecuteMetaData[Ornament/Desc]{shorthands}

As we will see in \autoref{sec:trieo}, some generics require descriptions augmented with more information. For example, a number system needs to describe both a datatype and its interpretation into naturals. This can be incorporated into a description by allowing description formers to query specific pieces of information. We will control where and when which pieces get queried by parametrizing descriptions over information bundles  
\ExecuteMetaData[Ornament/Desc]{Info}
Here a bundle declares for example that \AgdaField{𝟙i} is the type of information has to be provided at a \AgdaInductiveConstructor{𝟙} former. Remark that in \AgdaField{σi}, the bundle can ask for something depending on the type of the field. In \AgdaField{δi}, the bundle can ask something regarding the parameters and indices (e.g., it can force only unindexed subdescriptions.).

\begin{example}
    For example, we can encode a class of number systems using the information 
    \ExecuteMetaData[Ornament/Numerical]{Number}
    (refer to \autoref{sec:trieo}). If we then define the unit type, when viewed as a \AgdaFunction{Number}
    \ExecuteMetaData{Ornament/Numerical}{Unit}
    we have to provide the information that the only value of the unit type evaluates to 1.
\end{example}

We can recover the conventional descriptions by providing the plain bundle:
\ExecuteMetaData[Ornament/Desc]{Plain}
We define the ``down-casting'' of information as
\ExecuteMetaData[Ornament/Desc]{InfoF}
allowing us to reuse more specific descriptions in less specific ones, so that e.g., a number system can be used in a plain datatype.

We can now define the descriptions, which should represent a mapping between parametrized indexed functors
\ExecuteMetaData[Ornament/Desc]{PIType}
Recall that a description 
\ExecuteMetaData[Ornament/Desc]{DescI}
is simply a list of constructor descriptions
\ExecuteMetaData[Ornament/Desc]{Con}
The interpretations \hyperlink{desc-desc-interpretation}{\AgdaFunction{⟦\_⟧}} of the formers can be found below.

Leaves are formed by
\ExecuteMetaData[Ornament/Desc]{Con-1}
Here \AgdaBoundFontStyle{if} queries information according to \texttt{If}, and \AgdaBoundFontStyle{j} computes the index of the leaf from the parameters and variables.

A recursive field is formed by
\ExecuteMetaData[Ornament/Desc]{Con-rho}
where \AgdaBoundFontStyle{j} now determines the index of the recursive field. The function \AgdaBoundFontStyle{g} represents a parameter transform: the parameters of the recursive field can now changed at each recursive level, allowing us to describe nested datatypes. The remainder of the fields are described by \AgdaBoundFontStyle{C}. Note that a recursive field is intentionally not brought into scope: making use of it requires induction-recursion anyway!

A non-recursive field is formed similarly to a recursive field
\ExecuteMetaData[Ornament/Desc]{Con-sigma}
The type of the field is given by \AgdaBoundFontStyle{S}, which may depend on the values of the preceding fields. We bring the field into scope, so we continue the description in an extended context. However, we allow the remainder of the description to provide a conversion from \texttt{V ▷ S} into \texttt{W} to select a new context. This makes it possible to hide fields which are unused in the remainder.

Almost analogously, we make composition of descriptions internal by a variant of \AgdaInductiveConstructor{σ}
\ExecuteMetaData[Ornament/Desc]{Con-delta}
This takes a description \texttt{R}, and acts like the \AgdaInductiveConstructor{σ} of \texttt{μ R}, only with more ceremony. This will allow us to form descriptions by composing other descriptions, avoiding multiplying the number of constructors of composite datatypes.

Similar to \AgdaInductiveConstructor{ρ}, the functions \AgdaBoundFontStyle{j} and \AgdaBoundFontStyle{g} control indices and parameters, only now of the applied description. As we allow the description \AgdaBoundFontStyle{R} of the field to have a different kind of information bundle \AgdaBoundFontStyle{If′}, we must ask that we can down-cast it into \AgdaBoundFontStyle{If} via \AgdaBoundFontStyle{iff}. 

Descriptions and constructor descriptions can then be interpreted to appropriate kind of functor, constructor descriptions also taking variables
\hypertarget{desc-desc-interpretation}{}
\ExecuteMetaData[Ornament/Desc]{interpretation}
We see that a leaf becomes a constraint between expected index and the actual index. A recursive field passes down a transformation of the current parameters and the expected index computed from the variables, before interpreting the remainder of the description. Likewise, a non-recursive field adds a field with type depending on variables, but also adds this field to the variables, which are then transformed and passed on to the remainder. The composite field is analogous, only adding a field from a description rather than a type. Finally, the list of constructor descriptions are interpreted as alternatives.

The fixpoint can then be taken over the interpretation of a description
\ExecuteMetaData[Ornament/Desc]{fpoint}
giving the datatype represented by the description.

We can then give a generic fold for the represented datatypes
\ExecuteMetaData[Ornament/Desc]{fold}
which descends the description, mapping itself over all recursive fields before applying the folding function.
\begin{remark}
    The situation of \AgdaFunction{fold} is very common when dealing with different kinds of recursive interpretations: functions from the fixpoint are generally defined from functions out of the interpretation, generalizing over the inner description while pattern matching on the outer description. 
\end{remark}
Note that the fold requires a rather general function, limiting its usefulness: because of the parameter transformations, we cannot instantiate the fold to a single parameter. Defining, e.g., the vector sum, would require us to inspect the description, and ask that a vector of naturals can be converted into a vector of naturals, which is trivial in this case.

\todo{Sigma plus/minus}

Let's look at some examples. We can encode the naturals as a type parametrized by \AgdaInductiveConstructor{∅} and indexed by \AgdaDatatype{⊤}
\ExecuteMetaData[Ornament/Desc]{NatD}
Lists can be encoded similarly, but this time using the telescope
\ExecuteMetaData[Ornament/Desc]{ListTel}
declaring that lists have a single type parameter. Compared to the naturals, the description now also asks for a field in the second case
\ExecuteMetaData[Ornament/Desc]{ListD}
Since the type parameter is at the top of the parameter telescope, the type of the field is given as \AgdaBoundFontStyle{par top}.

Vectors are described using the same structure, but have indices in \bN{}.
\ExecuteMetaData[Ornament/Desc]{VecD}
In the first case, the index is fixed at 0. The second case declares that to construct a vector of length \AgdaBoundFontStyle{suc ∘ top}, the recursive field must have length \AgdaFunction{top}. Note that unlike index-first types, we cannot know the expected index from inside the description, so much like native indexed types, we must add a field choosing an index.

Recall the type of finger trees. Using parameter transformations and composition, we can give a description of full-fledges finger trees! First, we describe the digits
\ExecuteMetaData[Ornament/Desc]{DigitD}
and define the nodes\footnote{We could give the nodes as a description, but in this case we only use them in the recursive fields, so we would take the fixpoint without looking at their description anyway.}
\ExecuteMetaData[Ornament/Desc]{Node}
We encode finger trees as
\ExecuteMetaData[Ornament/Desc]{FingerD}
In the third case, we have digits which are passed the parameters on both sides in composite fields, and a recursive field in the middle. The recursive field has a parameter transformation, turning the type parameter \AgdaBoundFontStyle{A} into a \AgdaBoundFontStyle{Node A} in the recursive child.

%\investigate{Making \AgdaDatatype{Desc} coinductive would do a couple of things. First, recursion and composition become identical. Second, nesting of both types becomes easier to describe, but potentially impossible to prove strictly positive.}

%\investigate{We intentionally dodge having index telescopes (or having the index type depend on the parameters and values). Does this really change anything?}



\section{The ornaments}
\towrite{Put something that isn't yet in \autoref{ssec:bg-orn} here.}

\ExecuteMetaData[Ornament/Orn]{Orn-type}
\ExecuteMetaData[Ornament/Orn]{ornForget-type}
%Thus, the relation should be precise enough pairs of \AgdaBoundFontStyle{E} and \AgdaBoundFontStyle{D} for which we could not define \AgdaFunction{ornForget}.

We will walk through the constructor ornaments
\ExecuteMetaData[Ornament/Orn]{ConOrn-type}
again, an ornament between datatypes is just a list of ornaments between their constructors
\ExecuteMetaData[Ornament/Orn]{Orn}
Note that all ornaments completely ignore information bundles! They cannot affect the existence of \AgdaFunction{ornForget} after all.

Copying parts from one description to another, up to parameter and index refinement, corresponds to reflexivity. Preservation of leaves follows the rule
\ExecuteMetaData[Ornament/Orn]{Orn-1}
We can see that this commuting square (\texttt{e (k p) ≡ j (over f p)}) is necessary: take a value of \texttt{E} at \texttt{p, i}, where \texttt{i} is given as \texttt{k p}. Then \AgdaFunction{ornForget} has to convert this to a value of \texttt{D} at \texttt{f p , e i}, but since \texttt{e i} must match \texttt{j (f p)}, this is only possible if \texttt{e (k p) = j (f p)}.

Preserving a recursive field similarly requires a square of indices and conversions to commute
\ExecuteMetaData[Ornament/Orn]{Orn-rho}
additionally requiring the recursive parameters to commute with the conversion. \todo{Does adding the derivations for the squares everywhere make this section clearler?}

Preservation of non-recursive fields and description fields is analogous
\ExecuteMetaData[Ornament/Orn]{Orn-sigma-delta}
differing only in that non-recursive fields appears transformed on the right hand, while description fields have their conversions modified instead. For this rule, we need that the variable transformations fit into a commuting square with the parameter conversions. The condition on indices for descriptions, which is a commuting triangle, is encoded in the return type\footnote{Should this become a problem like with \AgdaInductiveConstructor{ρ}, modifying the rule to require a triangle is trivial.}.

Ornaments would not be very interesting if they only related identical structures. The left-hand side can also have more fields than the right-hand side, in which case \AgdaFunction{ornForget} will simply drop the fields which have no counterpart on the right-hand side. As a consequence, the description extending rules have fewer conditions than the description preserving rules: 
\ExecuteMetaData[Ornament/Orn]{Orn-+-rho}
Note that this extension\footnote{Kind of breaking the ``ornaments relate types with similar recursive structure'' interpretation.} with a recursive field has no conditions.

Extending by a non-recursive field or a description field again only requires the variable transform to interact well with the parameter conversion
\ExecuteMetaData[Ornament/Orn]{Orn-+-sigma-delta}

In the other direction, the left-hand side can also omit a field which appears on the right-hand side, provided we can produce a default value
\ExecuteMetaData[Ornament/Orn]{Orn---sigma-delta}
These rules let us describe the basic set of ornaments between datatypes.

Intuitively we also expect a conversion to exist when two constructors have description fields which are not equal, but are only related by an ornament. Such a composition of ornaments takes two ornaments, one between the field, and one between the outer descriptions. This composition rule reads:\todo{The implicits kind of get out of control here, but the rule is also unreadable without them. I might hide the rule altogether and only run an example with it.}
\ExecuteMetaData[Ornament/Orn]{Orn-comp}
We first require two commuting squares, one relating the parameters of the fields to the inner and outer parameter conversions, and one relating the indices of the fields to the inner index conversion and the outer parameter conversion. Then, the last square has a rather complicated equation, which merely states that the variable transforms for the remainder respect the outer parameter conversion.

We will construct \AgdaFunction{ornForget} as a \AgdaFunction{fold}. Using
\ExecuteMetaData[Ornament/Orn]{erase-type}
we can define the algebra which forgets the added structure of the outer layer
\ExecuteMetaData[Ornament/Orn]{ornAlg}
Folding over this algebra gives the wanted function
\ExecuteMetaData[Ornament/Orn]{ornForget}

\todo{NatD was removed here}

We can also relate lists and vectors
\ExecuteMetaData[Ornament/Orn]{ListD-VecD}
Now the parameter conversion is the identity, since both have a single type parameter. The index conversion is \AgdaFunction{!}, since lists have no indices. Again, most structure is preserved, we only note that vectors have an added field carrying the length.

Instantiating \AgdaFunction{ornForget} to these ornaments, we now get the functions \AgdaFunction{length} and \AgdaFunction{toList} for free!

%\investigate{Having a function of the same type as \AgdaFunction{ornForget} is not sufficient to deduce an ornament. An obstacle is that the usual empty type (no constructors) and the non-wellfounded empty type (only a recursive field) don't have an ornament. Also, while the leaf-preservation case spells itself out, the substitutions obviously don't give us a way to recover the equalities.}


\section{Ornamental descriptions}
A description can say ``this is how you make this datatype'', an ornament can say ``this is how you go between these types''. However, an ornament needs its left-hand side to be predefined before it can express the relation, while we might also interpret an ornament as a set of instructions to translate one description into another. A slight variation on ornaments can make this kind of usage possible: ornamental descriptions.

An ornamental description drops the left-hand side when compared to an ornament, and interprets the remaining right-hand side as the starting point of the new datatype:
\ExecuteMetaData[Ornament/OrnDesc]{ConOrnDesc-type}
The definition of ornamental descriptions can be derived in a straightforward manner from ornaments, removing all mentions of the LHS and making all fields which then no longer appear in the indices explicit\footnote{One might expect to need less equalities, alas, this is difficult because of \autoref{rem:orn-lift}.}. We will show the leaf-preserving rule as an example, the others are derived analogously:
\ExecuteMetaData[Ornament/OrnDesc]{OrnDesc-1}
As we can see, the only change we need to make is that \AgdaBoundFontStyle{k} becomes explicit and fully annotated.

Almost by construction, we have that an ornamental description can be decomposed into a description of the new datatype
\ExecuteMetaData[Ornament/OrnDesc]{toDesc}
and an ornament between the starting description and this new description
\ExecuteMetaData[Ornament/OrnDesc]{toOrn}


\section{Temporary: future work}
\begin{remark}
    Note that this allows us to express datatypes like finger trees, but not rose trees. Such datatypes would need a way to place a functor ``around the \AgdaInductiveConstructor{ρ}'', which then also requires a description of strictly positive functors. In our setup, this could only be encoded by separating general descriptions from strictly positive descriptions. The non-recursive fields of these strictly positive descriptions then need to be restricted to only allow compositions of strictly positive context functions. 
\end{remark} % \investigate{This setup does not allow nesting over recursive fields, which is necessary for structures like rose trees. This is actually kind of essential for enumeration. Nesting over a recursive field is problematic: we can incorporate it by adding ``this'' implicitly to a \AgdaInductiveConstructor{δ}, but then the \AgdaBoundFontStyle{R} needs to be strictly positive in its last argument, meaning we need to split \AgdaDatatype{Desc} into a strictly positive part and normal part. The strictly positive part should then only allow strictly positive parameter transforms in recursive and non-recursive fields, requiring an embedding of transforms.}

\begin{remark}
    Variable transforms are not essential in these descriptions, but there are a couple of reasons for keeping them. In particular, they make it possible to reuse a description in multiple contexts, and save us from writing complex expressions in the indices of our ornaments. On the other hand, the transforms still make defining ornaments harder (the majority of the commuting squares are from variables). Isolating them into a single constructor of \AgdaDatatype{Desc}, call it \AgdaInductiveConstructor{v}, seems like a good middle ground, but raises some odd questions, like ``why is there no ornament between \AgdaBoundFontStyle{v (g ∘ f) C} and \AgdaBoundFontStyle{v g (v f C)}''. (Furthermore, this also does not simplify the indices of ornaments).
\end{remark} %\investigate{Variable transforms are both less essential and less troublesome than I first thought. We can move variable transforms into a new former, and it probably simplifies the definition of ornaments a lot.}

\begin{remark}
    Rather, ornaments themselves could act as information bundles. If there was a description for \AgdaDatatype{Desc}, that is. Such a scheme of levitation would make it easier to switch between being able to actively manipulate information, and not having to interact with it at all. However, the complexity of our descriptions makes this a non-trivial task; since our \AgdaDatatype{Desc} is given by mutual recursion and induction-recursion, the descriptions, and the ornaments, would have to be amended to encode both forms of recursion as well.
\end{remark} % \investigate{If we levitate, then informed descriptions become ornaments over \AgdaDatatype{Desc}. This gives us the best of both worlds (modulo reflecting the description into a datatype): in plain descriptions, information does not even exist, and in informed descriptions, it is explicit. For levitation, we likely need induction-recursion and mutual recursion.}

\begin{remark}\label{rem:orn-lift}
    Rather than having the user provide two indices and show that the square commutes, we can ask for a ``lift'' $k$
    % https://q.uiver.app/#q=WzAsNCxbMCwwLCJcXGJ1bGxldCJdLFsxLDAsIlxcYnVsbGV0Il0sWzAsMSwiXFxidWxsZXQiXSxbMSwxLCJcXGJ1bGxldCJdLFswLDEsImUiXSxbMiwzLCJmIiwyXSxbMiwwLCJqIl0sWzMsMSwiaSIsMl0sWzMsMCwiayIsMV1d
    \[\begin{tikzcd}
        \bullet & \bullet \\
        \bullet & \bullet
        \arrow["e", from=1-1, to=1-2]
        \arrow["f"', from=2-1, to=2-2]
        \arrow["j", from=2-1, to=1-1]
        \arrow["i"', from=2-2, to=1-2]
        \arrow["k"{description}, from=2-2, to=1-1]
    \end{tikzcd}\]
    and derive the indices as $i = ek, j = kf$. However, this is more restrictive, unless $f$ is a split epi, as only then pairs $i,j$ and arrows $k$ are in bijection. In addition, this makes ornaments harder to work with, because we have to hit the indices definitionally, whereas asking for the square to commute gives us some leeway (i.e., the lift would require the user to transport the ornament). 
\end{remark}


%\investigate{Can these be simpler? Right now, these just construct the ornament and description on the fly, rather than actually asking for less.}


\section{Ornamental Descriptions}
\todo{And why we skip ornaments in the next part}

\begin{outline}
A description can say ``this is how you make this datatype'', an ornament can say ``this is how you go between these types''. However, an ornament needs its left-hand side to be predefined before it can express the relation, while we might also interpret an ornament as a set of instructions to translate one description into another. A slight variation on ornaments can make this kind of usage possible: ornamental descriptions.
    
An ornamental description drops the left-hand side when compared to an ornament, and interprets the remaining right-hand side as the starting point of the new datatype:
\ExecuteMetaData[Ornament/OrnDesc]{ConOrnDesc-type}
The definition of ornamental descriptions can be derived in a straightforward manner from ornaments, removing all mentions of the LHS and making all fields which then no longer appear in the indices explicit\footnote{One might expect to need less equalities, alas, this is difficult because of \autoref{rem:orn-lift}.}. We will show the leaf-preserving rule as an example, the others are derived analogously:
\ExecuteMetaData[Ornament/OrnDesc]{OrnDesc-1}
As we can see, the only change we need to make is that \AgdaBoundFontStyle{k} becomes explicit and fully annotated.

Almost by construction, we have that an ornamental description can be decomposed into a description of the new datatype
\ExecuteMetaData[Ornament/OrnDesc]{toDesc}
and an ornament between the starting description and this new description
\ExecuteMetaData[Ornament/OrnDesc]{toOrn}
\end{outline}